
\documentclass[11pt]{penrose}

\usepackage{mathsphystools}
\usepackage{thmstyles}

\newcommand{\cyclic}[1]{\mean{#1}}
\newcommand{\missing}[1]{{\color{red}#1}}
\newcommand{\keyword}[1]{\textsf{#1}}

\title{MATH 325: Group Theory I}
\subtitle{Brief lecture notes}
\author{Rashid M. Talha}
\affiliation{School of Natural Sciences, NUST}
\date{\today}
\begin{document}

\maketitle

\textbf{Textbook:} Contemporary Abstract Algebra, Joseph Gallian.

\textbf{Disclaimer:} This document most likely contains some errors --- use with caution. I have rephrased or paraphrased the content in most of the sections. Some examples may be missing. The numbering that I have used for sections, definitions, theorems, etc will not match the numbering given in the lectures.

\section{Introduction}
\begin{ndfn}
    A \keyword{binary operation} is a map $* : X \times X \to X$, $(a,b) \mapsto a * b$.
\end{ndfn}

By definition of $*$, $a * b \in X$ for all $a, b \in X$. This property is called \keyword{closure}.

\begin{ndfn}
    A binary operation $* : X \times X \to X$ is called \keyword{commutative} if
    \begin{equation*}
        \forall a, b \in X, \quad a * b = b * a
    \end{equation*}
\end{ndfn}

\begin{ndfn}
    Let $G$ be a non-empty set, and $* : G \times G \to G$ be a binary operation. The pair $(G, *)$ is called a \keyword{group} if it satisfies all of the following
    \begin{enumerate}[label=(\roman*)]
        \item $\forall a, b, c \in G$, $(a * b) * c = a * (b * c)$ \hfill(\keyword{Associativity})
        \item $\exists e \in G$ such that $\forall a \in G$, $a * e = e * a = a$ \hfill(\keyword{Identity})
        \item $\forall a \in G$, $\exists a^{-1} \in G$ such that $a * a^{-1} = a^{-1} * a = e$ \hfill(\keyword{Inverse})
    \end{enumerate}
\end{ndfn}

\begin{ndfn}
    A group $(G, *)$ is called \keyword{Abelian} if the binary operation $*$ is commutative. That is, for all $a, b \in G$, $a * b = b * a$ .
\end{ndfn}

\begin{remark}
Typically we write $a * b$ simply as $ab$ and call the binary operation multiplication. In the case where the binary operation is the usual addition, we write $a + b$ instead. Similarly, we often refer to $G$ as the group and don't explicitly mention the pair $(G, *)$. Moreover, we sometimes denote the identity element by $1$ for multiplicative binary operations, and by $0$ for additive binary operations.
\end{remark}

\begin{nthm}
    Each group has a unique identity element.
\end{nthm}
\begin{proof}
    Let $e, f \in G$ be identity elements. Then, for all $a \in G$
    \begin{equation*}
        ea = ae = a
        \qquad\text{and}\qquad
        fa = af = a.
    \end{equation*}
    In particular, (taking $a=f$ in the first case and $a=e$ in the second)
    \begin{equation*}
        ef = fe = f
        \qquad\text{and}\qquad
        fe = ef = e.
    \end{equation*}
    As a result, $e = ef = f$.
\end{proof}

\begin{nthm}
    Each $a \in G$ has a unique inverse element.
\end{nthm}
\begin{proof}
    Take any $a \in G$. Let $a^{-1}, b \in G$ be inverse elements of $a$. That means
    \begin{equation*}
        a a^{-1} = a^{-1} a = e
        \qquad\text{and}\qquad
        a b = b a = e.
    \end{equation*}
    As a result, $b = b e = b(a a^{-1}) = (b a) a^{-1} = e a^{-1} = a^{-1}$.
\end{proof}

\begin{nthm}
    Let $G$ be a group. Then, for all $a \in G$, $(a^{-1})^{-1} = a$.
\end{nthm}
\begin{proof}
    Take any $a \in G$. Then, it has an inverse $a^{-1} \in G$ such that $a a^{-1} = e$. Since, $a^{-1} \in G$ it also has an inverse $(a^{-1})^{-1}$ such that $a^{-1} (a^{-1})^{-1} = e$.

    Therefore,
    $a = a e = a \paren*{a^{-1} (a^{-1})^{-1}} = \paren*{a a^{-1}} (a^{-1})^{-1} = e (a^{-1})^{-1} = (a^{-1})^{-1}$.
\end{proof}

\begin{nthm}
    Let $G$ be a group. Then, for all $a, b \in G$, $(ab)^{-1} = b^{-1} a^{-1}$.
\end{nthm}
\begin{proof}
    Note that $(ab)^{-1} (ab) = (ab) (ab)^{-1} = e$ by definition of the inverse of $ab$.

    Now, $(b^{-1} a^{-1}) (a b) = b^{-1} ((a^{-1} a) b) = b^{-1} (e b) = b^{-1} b = e$.

    And, $(a b) (b^{-1} a^{-1}) = a ((b b^{-1}) a^{-1}) = a (e a^{-1}) = a a^{-1} = e$.

    So, $b^{-1} a^{-1}$ is also an inverse of $ab$. By the uniqueness of inverse, $b^{-1} a^{-1} = (ab)^{-1}$.
\end{proof}

\begin{nthm}
    Take $a, b, c \in G$. Then,
    \begin{enumerate}
        \item $ab = ac \implies b = c$.
        \item $ba = ca \implies b = c$.
    \end{enumerate}
\end{nthm}
\begin{proof}
    Since $a \in G$, we have $a^{-1} \in G$ such that $a a^{-1} = a^{-1} a = e$. Therefore,
    \begin{equation*}
        ab = ac
        \implies a^{-1} (ab) = a^{-1} (ac)
        \implies (a^{-1} a)b = (a^{-1} a)c
        \implies eb = ec
        \implies b = c
    \end{equation*}
    Similarly,
    \begin{equation*}
        ba = ca
        \implies (ba)a^{-1} = (ca)a^{-1}
        \implies b (a a^{-1}) = c (a a^{-1})
        \implies be = ce
        \implies b = c
        \qedhere
    \end{equation*}
\end{proof}

\begin{nthm}
    Let $G$ be a group, and $a, b \in G$. The equation $a x = b$ has a unique solution. Likewise, the equation $y a = b$ has a unique solution.
\end{nthm}
\begin{proof}
    Consider the equation $a x = b$.

    (Existence.) Since $a \in G$, we have $\inv{a} \in G$ such that $\inv{a} a = e$. So,
    \begin{equation*}
        ax = b
        \implies \inv{a} (a x) = \inv{a} b
        \implies (\inv{a} a) x = \inv{a} b
        \implies e x = \inv{a} b
        \implies x = \inv{a} b.
    \end{equation*}
    And $\inv{a} b \in G$ due to the closure property. So, $x = \inv{a} b \in G$.

    (Uniqueness.) Suppose there are $x_1, x_2 \in G$ that satisfy $a x = b$. Then, $a x_1 = b$ and $a x_2 = b$. So, by the cancellation property $a x_1 = a x_2 \implies x_1 = x_2$.
    
    The proof for $y a = b$ is analogous, with multiplications on the right hand side.
\end{proof}

\begin{ndfn}
    The \keyword{order} of a group $G$, denoted $\abs{G}$ or $O(G)$, is the number of elements in $G$. If $G$ has infinitely many elements then $\abs{G} = \infty$.
\end{ndfn}

\begin{egg}
    Some examples of groups are
    \begin{enumerate}
        \item $(\R,+)$, $(\C,+)$, $(\Z,+)$, $(\Q,+)$, $(\R^{*},\cdot)$, $(\C^{*},\cdot)$, $(\Q^{*},\cdot)$, $(\R^{+},\cdot)$. Here $\R^{*} = \R - \set{0}$, and $\R^{+} = \set{r \in \R \st r > 0}$.
        \item The set of $n$th roots of unity $U_n = \set{\exp(\frac{2\pi i}{n}) \in C \st n = 0, 1, \dots, n-1}$ forms a group under the multiplication of complex numbers.
        \item The set of $n \times n$ matrices with entries in $\R$ is denoted by $M_{n}(\R)$. This forms a group under the usual additional of matrices.
        \item $\GL_{n}(\R) = \set{A \in M_{n}(\R) \st \det A \neq 0}$ with the usual matrix multiplication is called the \keyword{general linear group} of order $n$.
        \item The usual matrix multiplication makes $\SL_{n}(\R) = \set{A \in M_{n}(\R) \st \det A = 1}$ into a group, called the the \keyword{special linear group} of order $n$.
    \end{enumerate}
\end{egg}

\begin{negg}
    Consider the set with a single element $G = \set{e}$ and the binary operation $e * e = e$. This forms a group, called the \keyword{trivial group}. Note that for the trivial group $\abs{G} = 1$.
\end{negg}

\begin{ndfn}
    A non-empty subset $H \subseteq G$ is called a \keyword{subgroup} of $G$ if it is a group under the same binary operation. We denote this as $H \subgroup G$.
\end{ndfn}

\begin{ndfn}
    $H \subgroup G$ is called a \keyword{proper subgroup} if $H \neq G$. This is sometimes emphasised by writing $H < G$. A proper subgroup is called non-trivial if $H \neq \set{e}$.
\end{ndfn}

\begin{egg}
    Some examples of subgroups are
    \begin{enumerate}
        \item $\Z \subgroup \R$.
        \item $\R^{+} \subgroup \R^{*}$.
        \item $2\Z \subgroup \Z$, with $2\Z = \set{2k \st k \in \Z}$.
        \item $\SL_{n}(\R) \subgroup \GL_{n}(\R)$.
    \end{enumerate}
\end{egg}

\begin{nthm}
    Let $G$ be a group. A non-empty subset $H \subseteq G$ is a subgroup of $G$ if and only if
    \begin{enumerate}
        \item $a, b \in H \implies ab \in H$.
        \item $a \in H \implies a^{-1} \in H$.
    \end{enumerate}
\end{nthm}
\begin{proof}
    Suppose $H \subgroup G$. Then, $H$ is a group under the same binary operation. In particular, both the closure property and the existence of inverse property holds in $H$.

    Conversely, the closure property is explicitly given. Associativity is inherited from the binary operation on $G$. Also, the existence of inverse property is explicitly given. Finally, since $H$ is non-empty, take $a \in H$. Then, $a^{-1} \in H$. By the closure property, $a a^{-1} = e \in H$. Therefore, $H$ also contains the identity element. As a result, $H$ is a group with respect to the same binary operation. That is, $H \subgroup G$.
\end{proof}

\begin{nthm}
    Let $G$ be a group. A non-empty subset $H \subseteq G$ is a subgroup of $G$ if and only if $a, b \in H \implies ab^{-1} \in H$.
\end{nthm}
\begin{proof}
    Suppose $H \subgroup G$. Then, $H$ is a group under the same binary operation. Take $a, b \in H$. Then, by the previous subgroup test $b^{-1} \in H$. Again, by the previous subgroup test, $a b^{-1} \in H$.

    For the converse, we check that $H$ satisfies all the group axioms.

    Firstly, $H$ has the same binary operation as $G$, so associativity is inherited from $G$. Next, since $H$ is non-empty, take any $a \in H$. Then, $a a^{-1} \in H$ implies $e \in H$. So, $H$ contains the identity element. Similarly, take $e, a \in H$. Then, $e a^{-1} \in H$ implies $a^{-1} \in H$. Therefore, each element of $H$ has an inverse within $H$. Lastly, take $a, b \in H$. Then, $b^{-1} \in H$. So, $a \left(b^{-1}\right)^{-1} \in H$ implies $ab \in H$, since $\left(b^{-1}\right)^{-1} = b$.

    Therefore, $H$ is a group under the same binary operation as $G$. So, $H \subgroup G$.
\end{proof}

\begin{nthm}
    Let $H_i \subgroup G$, for all $i \in I$. Then, $H = \cap_{i \in I} H_i$ is a subgroup of $G$.
\end{nthm}
\begin{proof}
    Firstly, $e \in H_i$ for all $i \in I$ because each $H_i$ is a subgroup of $G$. As a result, $e \in H$. So, $H$ is non-empty.

    Take $a, b \in H$. Then, $a, b \in H_i$ for all $i \in I$. As $H_i$ are subgroups, $ab^{-1} \in H_i$ for all $i \in I$. Therefore, $ab^{-1} \in H$. By the subgroup criteria, this shows that $H \subgroup G$.
\end{proof}

\begin{nthm}
    Let $G$ be a group and take $a \in G$. The set $H = \set{a^n \st n \in \Z}$ is a subgroup of $G$. Here, $a^0 = e$ and $a^{-n} = \left(a^{-1}\right)^{n}$.
\end{nthm}
\begin{proof}
    Firstly, $H$ is non-empty because $a^0 = e \in H$.

    Next, take any $a^{n}, a^{m} \in H$.. So, $a^{n} (a^{m})^{-1} = a^{n} a^{-m} = a^{n - m} \in H$ since $n - m \in \Z$ and $\inv{(a^{m})} = a^{-m}$.

    By the subgroup criteria, this shows that $H \subgroup G$.
\end{proof}

\begin{ndfn}
    The order of an element $a \in G$ is the least positive integer $k$ such that $a^k = e$. We denote this as $O(a)$ or $\abs{a}$.
\end{ndfn}

\begin{nthm}
    Let $H, K \subgroup G$. Then, $H \cup K$ is a subgroup of $G$ if and only if $H \subseteq K$ or $K \subseteq H$.
\end{nthm}
\begin{proof}
    If $H \subseteq K$ then, $H \cup K = H \subgroup G$. Instead, if $K \subseteq H$ then, $H \cup K = K \subgroup G$. In either case, $H \cup K \subgroup G$.

    Conversely, suppose $H \cup K \subgroup G$. For a contradiction assume $H \nsubseteq K$ and $K \nsubseteq H$. Then we can pick $h \in H - K$ and $k \in K - H$. So, $h, k \in H \cup K$. Since $H \cup K$ is a group of $G$, we have $hk \in H \cup K$. So, either $hk \in H$ or $hk \in K$ (or both).

    If $hk \in H$, then $k \in H$ because $h \in H$ (and so, $\inv{h} \in H$). Alternatively, if $hk \in K$, then $h \in K$ because $k \in K$ (and so, $\inv{k} \in K$). Both of these are contradictions. Therefore the assumption $H \nsubseteq K$ and $K \nsubseteq H$ is wrong, and either $H \subseteq K$ or $K \subseteq H$.
\end{proof}

\section{Modular Addition}

\begin{ndfn}
    Let $a, b$ be integers and fix a positive integer $n$. We say that $a$ is congruent to $b$ modulo $n$ if $n$ divide $a-b$. That is $n | (a-b)$. This is denoted as $a \equiv b \mod n$.
\end{ndfn}

\begin{ndfn}
    Take $a \in \Z$, and fix some integer $n \geq 2$. The set of all the integers that are equivalent to $a$ modulo $n$ is called the \keyword{residue class} of $a$ modulo $n$. We write this as
    \begin{equation*}
        [a]_n = \set{b \in \Z \st a \equiv b \mod n}.
    \end{equation*}
\end{ndfn}

\begin{remark}
    $[a]_n$ and $[b]_n$ are either equal of disjoint. (This was skipped.)
\end{remark}

\begin{ndfn}
    Fix some integer $n \geq 2$. The set of all the residue classes modulo $n$ in $\Z$ is denoted as
    \begin{equation*}
        \Z_n = \set*{[a]_n \st a \in \Z}.
    \end{equation*}
\end{ndfn}

We can define a binary operation $+_n$, called \keyword{modular addition} ($\mod n$) on this set,
\begin{equation*}
    [a]_n +_n [b]_n \coloneq [a + b]_n.
\end{equation*}
It needs to be shown that this is well-defined; that is if $[a]_n = [c]_n$ and $[b]_n = [d]_n$, then, $[a]_n +_n [b]_n = [c]_n +_n [d]_n$. (This was skipped.)

\begin{nthm}
    The set $\Z_n$ forms a group with respect to $+_n$.
\end{nthm}

We sometimes drop the subscript and simply write $[a]_n$ as $a$ and $+_n$ as $+$.

\section{\texorpdfstring{Klein $4$-Group}{Klein 4 Group}}
Consider a set $G = \set{e, a, b, c}$ with a binary operation that satisfies $a^2 = b^2 = c^2 = e$, $xy = yx$ for all $x, y \in G$. It can easily be checked that this forms a group. We also find that some of the conditions imposed on the binary operation are redundant, and instead this group can be expressed more compactly. The following theorem states this observation.

\begin{nthm}
    The set $K_4 = \set{1, a, b, ab}$ where the order of each non-identity element is $2$ forms a group.
\end{nthm}

This group $K_4$ is called the \keyword{Klein $4$-group}. It is a group of order $4$. Its multiplication rule can be represented as table, called a Cayley table.
\begin{center}
    \ra{1.2}
    \begin{tabularx}{0.3\textwidth}{|R||R|R|R|R|}
        \hline
             & $ 1$ & $ a$ & $ b$ & $ab$ \\ \hline\hline
        $ 1$ & $ 1$ & $ a$ & $ b$ & $ab$ \\ \hline
        $ a$ & $ a$ & $ 1$ & $ab$ & $ b$ \\ \hline
        $ b$ & $ b$ & $ab$ & $ 1$ & $ a$ \\ \hline
        $ab$ & $ab$ & $ b$ & $ a$ & $ 1$ \\ \hline
    \end{tabularx}

    {Multiplication table for $K_4$.}
\end{center}

Here, the entry in the the $(i,j)$-th entry is the result of multiplying the element in the $j$th column with the element in the $i$th row in the `column-on-the-left' order.
\begin{equation*}
    (\text{col}_j) * (\text{row}_i) = (i,j)\text{-th entry}
\end{equation*}

There are precisely three non-trivial subgroups of $K_4$: $\set{1, a}$, $\set{1, b}$ and $\set{1, ab}$.

Consider the set
\begin{equation*}
    G = \set*{
    \begin{bmatrix*}[r]1 & 0\\ 0 & 1\end{bmatrix*},
    \begin{bmatrix*}[r]-1 & 0\\ 0 &-1\end{bmatrix*},
    \begin{bmatrix*}[r]-1 & 0\\0 & 1\end{bmatrix*},
    \begin{bmatrix*}[r]1 & 0\\0 & -1\end{bmatrix*} }
\end{equation*}
of matrices with the usual matrix multiplication. We find that it forms a group. Clearly, this is a group of order $4$. Moreover, each non-identity element in this group has order $2$. This coincides exactly with the group structure of $K_4$. We say that this group is the same as $K_4$ (this notion will be made precise when we define \keyword{isomorphisms} later), and that it is simply a matrix \keyword{representation} of $K_4$.

Another group of order $4$ that we have already seen is $\Z_4$. This group has an element of order $4$, namely $[1]_4$. Therefore, it cannot be the `same' as $K_4$. (Again, this observation will be made precise through the use of isomorphisms.) This shows that not all groups of order $4$ are the same as $K_4$.

\section{Group of Quaternions}
Consider the subset
\begin{equation*}
    H = \set*{
    \pm\begin{bmatrix*}[r]1 & 0\\ 0 & 1\end{bmatrix*},
    \pm\begin{bmatrix*}[r]1 & 0\\ 0 &-1\end{bmatrix*},
    \pm\begin{bmatrix*}[r]0 & 1\\-1 & 0\end{bmatrix*},
    \pm\begin{bmatrix*}[r]0 & i\\ i & 0\end{bmatrix*} }
\end{equation*}
of the group $\GL_2(\C)$ of $2 \times 2$ invertible matrices with complex entries. It is easy to check that this is a subgroup of $\GL_2(\C)$ and therefore a group in its own right. We note that this is a group of order $8$.

Using this as a template, we can define an abstract group of order $8$ as follows.
\begin{nthm}
    The set $Q_8 = \set{\pm 1, \pm i, \pm j, \pm k}$ forms a group with the multiplication rule $i^2 = j^2 = k^2 = ijk = -1, (-1)^2 = 1$.
\end{nthm}

This is called the group of \keyword{quaternions}. Its full multiplication table is given below.
\begin{center}
\ra{1.2}
\begin{tabularx}{0.62\textwidth}{|R||R|R|R|R|R|R|R|R|}
    \hline
         & $ 1$ & $ i$ & $ j$ & $ k$ & $-1$ & $-i$ & $-j$ & $-k$ \\ \hline\hline
    $ 1$ & $ 1$ & $ i$ & $ j$ & $ k$ & $-1$ & $-i$ & $-j$ & $-k$ \\ \hline
    $ i$ & $ i$ & $-1$ & $-k$ & $ j$ & $-i$ & $ 1$ & $ k$ & $-j$ \\ \hline
    $ j$ & $ j$ & $ k$ & $-1$ & $-i$ & $-j$ & $-k$ & $ 1$ & $ i$ \\ \hline
    $ k$ & $ k$ & $-j$ & $ i$ & $-1$ & $-k$ & $ j$ & $-i$ & $ 1$ \\ \hline
    $-1$ & $-1$ & $-i$ & $-j$ & $-k$ & $ 1$ & $ i$ & $ j$ & $ k$ \\ \hline
    $-i$ & $-i$ & $ 1$ & $ k$ & $-j$ & $ i$ & $-1$ & $-k$ & $ j$ \\ \hline
    $-j$ & $-j$ & $-k$ & $ 1$ & $ i$ & $ j$ & $ k$ & $-1$ & $-i$ \\ \hline
    $-k$ & $-k$ & $ j$ & $-i$ & $ 1$ & $ k$ & $-j$ & $ i$ & $-1$ \\ \hline
\end{tabularx}

{Multiplication table for $Q_8$.}
\end{center}

\section{Dihedral Group (of Order 6)}
Consider the set $D_3  =\set{1, a, a^2, b, ba, ba^2}$ with the conditions $a^3 = 1$, $b^2 = 1$ and $ab = ba^2$. We can check that this forms a non-abelian group. This is called the \keyword{dihedral group} of order $6$.

\begin{center}
\ra{1.2}
\begin{tabularx}{0.62\textwidth}{|R||R|R|R|R|R|R|}
    \hline
           & $   1$ & $   a$ & $ a^2$ & $   b$ & $  ba$ & $ba^2$ \\ \hline\hline
    $   1$ & $   1$ & $   a$ & $ a^2$ & $   b$ & $  ba$ & $ba^2$ \\ \hline
    $   a$ & $   a$ & $ a^2$ & $   1$ & $  ba$ & $ba^2$ & $   b$ \\ \hline
    $ a^2$ & $ a^2$ & $   1$ & $   a$ & $ba^2$ & $   b$ & $  ba$ \\ \hline
    $   b$ & $   b$ & $ba^2$ & $  ba$ & $   1$ & $ a^2$ & $ a^2$ \\ \hline
    $  ba$ & $  ba$ & $   b$ & $ba^2$ & $   a$ & $   1$ & $   a$ \\ \hline
    $ba^2$ & $ba^2$ & $  ba$ & $   b$ & $ a^2$ & $   a$ & $   1$ \\ \hline
\end{tabularx}

{Multiplication table for $D_3$.}
\end{center}

\section{Cyclic Group}
\begin{ndfn}
    A group $G$ is called \keyword{cyclic} if there is an element $a \in G$ such that all elements of $G$ can be written as powers of $a$. More precisely, $\forall g \in G$, $\exists m \in \Z$ such that $g = a^m$.

    Such as element $a$ is called a \keyword{generator} of $G$, and we say that $G$ is the group generated by $a$ and denote this as $G = \cyclic{a}$. Cyclic group of order $n$ is sometimes denoted as $C_n$.
\end{ndfn}

Generators are not unique. Indeed if $a \in G$ is a generator then so is $\inv{a}$.

\begin{notation}
    For $m > 0$, $a^m$ means $a * \dots * a$, where $m$ factors of $a$ are multiplied together. Similarly, $a^0 \equiv e$, the identity element. And, $a^{-m}$ means $\paren*{\inv{a}}^m$.
\end{notation}

\begin{nthm}\label{thm:cyclic-finite-form}
    Let $G$ be a group, and take $a \in G$ such that $a^n = e$. Then, the cyclic group $\cyclic{a}$ has the form $\set{e, a, a^2, \dots, a^{n-1}}$.
\end{nthm}
\begin{proof}
    By definition $\cyclic{a} = \set{a^k \st k \in \Z}$. We can write $k = qn + r$ for $q, r \in \Z$ and $0 \leq r < n$. So, for all $k \in \Z$, $a^k = a^{qn+r} = a^{qn} a^r = (a^{n})^{q} = e a^r = a^r$.

    Therefore, $\cyclic{a} = \set{a^r \st 0 \leq r < n} = \set{e, a, a^2, \dots, a^{n-1}}$.
\end{proof}

\begin{nthm}\label{thm:cyclic-group-order}
    Let $G = \cyclic{a}$ be a cyclic group. Then, $|G| = O(a)$.
\end{nthm}
\begin{proof}
    If $O(a)$ is infinite then $a^n$ and $a^m$ are distinct for all $n \neq m$, because otherwise,
    \begin{equation*}
        a^n = a^m \implies a^{n - m} = 1 \implies O(a) \leq \abs{n-m}
    \end{equation*}
    which is a contradiction. Now, since each $a^n$ is different from $a^{n+1}$, we have infinitely many elements in the set $\cyclic{a} = \set{a^n \st n \in \Z}$. Consequently, $G = \cyclic{a}$ has infinitely many elements; i.e. $|G|$ is also infinite.

    Next, consider the case when $O(a) = n$ is finite. Then, $G = \cyclic{a} = \set{e, a, a^2, \dots, a^{n-1}}$ by theorem~(\ref{thm:cyclic-finite-form}). So, $|G| = n$ also.
\end{proof}

\begin{nthm}
    Let $G$ be a group and $a \in G$ with $O(a) = n$. If $a^m = e$ then $n \mid m$.
\end{nthm}
\begin{proof}
    By the division algorithm, we have $m = qn + r$ for $q, r \in \Z$ and $0 \leq r < n$. So,
    \begin{equation*}
        a^m = a^{nq + r} = \paren*{{a^n}^q} \paren*{a^r} = \paren*{e^q} \paren*{a^r} = e \paren*{a^r} = a^r.
    \end{equation*}

    Now, $a^m = e \implies a^r = e$. However, $r < n$ and $n$ is the least positive integer for which $a^n = e$. Therefore, $r$ must be zero (if it was positive then it would contradict the minimality of $n$). Therefore, $m = qn$, or $n \mid m$
\end{proof}

\begin{nthm}
    Every cyclic group is abelian.
\end{nthm}
\begin{proof}
    Let $G = \cyclic{g}$ be a cyclic group. Take $a, b \in G$. Then, $a = g^m$, $b = g^n$ for some $m, n \in \Z$. As a result, $ab = g^m g^n = g^{m+n} = g^{n+m} = g^n g^m = ba$.
\end{proof}

\begin{nthm}
    Every subgroup of a cyclic group is cyclic.
\end{nthm}
\begin{proof}
    Consider the cyclic group $G = \cyclic{a}$. Let $H$ be a subgroup of $G$. If $H = \set{e}$ then it is generated by $e$. So, suppose $H$ is not the trivial subgroup. Then, every element in $H$ can be written as $a^k$ for some $k \in \Z$. Let $m$ be the least positive integer such that $a^m \in H$. Therefore, $a^{-m} \in H$ also.

    Take any $a^t \in H$. Then, we can write $t = mq + r$ for some $q, r \in \Z$ with $0 \leq r < m$. Equivalently, $r = t - mq$. So,
    \begin{equation*}
        a^r = a^{t - mq} = a^{t} a^{-mq} = a^{t} (a^{-m})^{q} \in H
    \end{equation*}
    by closure. If $r \neq 0$, then this contradicts the requirement that $m$ is the least positive integer with $a^m \in H$. Therefore, $r = 0$. So, $t = mq$, and every arbitrary element of $H$ has the form $a^t = a^{mq} = (a^{m})^{q}$. Therefore, $H = \cyclic{a^m}$.
\end{proof}

\begin{nthm}
    Let $G = \cyclic{a}$ be a finite cyclic group of order $n$. Then an element $a^k$ is a generator of $G$ if and only if $\gcd(k,n) = 1$.
\end{nthm}
\begin{proof}
    Suppose $a^k$ is a generator of $G$. Then, we can write $a$ as a power of $a^k$, say $a = \paren{a^k}^m = a^{km}$ for some $m \in \Z$.

    Then, $a = \paren{a^k}^m \implies a a^{-km} = e \implies a^{1-km} = e$.

    So, $n \mid 1 - km$. That is $\exists q \in \Z$ such that $1 - km = qn$. We can re-arrange this to get $qn + km = 1$. From number theory (Bezout's lemma) we know that this implies $\gcd(n,k)=1$.

    Conversely, suppose $\gcd(n,k)=1$. Then, there exist integers $x, y$ such that $xk + yn = 1$. So, $a = a^{xk + yn} \implies a = \paren{a^{k}}^{x} \paren{a^{n}}^{y} \implies a = \paren{a^{k}}^{x} e^{y} \implies a = \paren{a^{k}}^{x}$.

    Now for all $b \in \cyclic{a}$, we have $b = a^r$ for some $r \in Z$. Therefore, we can write it as a power of $a^k$ as $b = a^r = \paren{\paren{a^{k}}^{x}}^{r} = \paren{a^{k}}^{xr}$. So, $a^k$ also generates $\cyclic{a}$.
\end{proof}

\begin{remark}
    The number of generators for a finite cyclic group of order $n$ is $\varphi(n)$, the Euler's $\varphi$ function.
\end{remark}

\begin{nthm}
    An infinite cyclic group $G = \cyclic{a}$ has exactly two generators.
\end{nthm}
\begin{proof}
    Firstly, since $|\cyclic{a}| = \infty$, the order of $a$ is infinite. And $a^n = e$ is only possible when $n = 0$.

    Let $b \in G$ be another generator of $G$. Then, we can write $b = a^s$ and $a = b^t$ for some $s, t \in \Z$. Therefore, $a = \paren*{a^{s}}^{t} = a^{st} \implies a^{st-1} = e \implies st-1 = 0$.
    
    The only solutions to this diophantine equation are $s=t=1$ and $s=t=-1$. So, $b = a$ or $b = \inv{a}$.
\end{proof}

\section{Equivalence Relations}
\begin{ndfn}
    A \keyword{partition} of a non-empty set $S$ is a collection of non-empty disjoint subsets $S_i \subseteq S$ such that $\cup_{i \in I} S_i = S$.
\end{ndfn}

\begin{ndfn}
    A \keyword{relation} $R$ on a set $S$ is a subset of $S \times S$. We say that $x$ is related to $y$ if $(x,y) \in R$. This is denoted as $xRy$.
\end{ndfn}

\begin{ndfn}
    A relation $R$ on $S$ is called an \keyword{equivalence relation} if it satisfies
    \begin{enumerate}[label=(\roman*)]
        \item For all $x \in S$, $xRx$. \hfill(\keyword{reflexive})
        \item For all $x, y \in S$, $xRy \implies yRx$. \hfill(\keyword{symmetric})
        \item For all $x, y, z \in S$, if $xRy$ and $yRz$ then $xRz$. \hfill(\keyword{transitive})
    \end{enumerate}
    An equivalence relation is typically denoted by the symbol $\sim$ instead of $R$.
\end{ndfn}

\begin{negg}
    $a \sim b$ if $n \mid (a-b)$. This is an equivalence relation.
\end{negg}

\begin{negg}
    $a \sim b$ if $a \leq b$ is not an equivalence relation because it is not symmetric.
\end{negg}

\begin{ndfn}
    Let $\sim$ be an equivalence relation on $S$. The equivalence class of $a \in S$ is the set
    \begin{equation*}
        [a] = \set{b \in S \st b \sim a}.
    \end{equation*}
    Some authors use the notation $\bar{a}$ or $cl(a)$ to denote the equivalence class of $a$.
\end{ndfn}


\begin{nthm}
    Let $\sim$ be an equivalence relation on $S$. The collection of equivalence classes $\set{[a] \st a \in S}$ partitions $S$. More precisely, each $[a]$ is non-empty, and $S = \bigcup_{a \in S} [a]$, and if $[a] \neq [b]$ then $[a] \cap [b] = \emptyset$.
\end{nthm}
\begin{proof}\phantom{}
    \begin{enumerate}
        \item By reflexivity, $a \sim a$ we have $a \in [a]$. Therefore, $[a] \neq \emptyset$.

        \item By definition $[a] \subseteq S$, so $\bigcup_{a \in S} [a] \subseteq S$.

        Take any $a \in S$. Then, $a \in [a] \subseteq \bigcup_{a \in S} [a]$. So, $S = \bigcup_{a \in S} [a]$.

        \item We prove the contrapositive statement.

        Suppose $[a] \cap [b] \neq \emptyset$. So, there is some $c \in [a] \cap [b]$. By definition, this means $c \sim a$ and $c \sim b$. Then, by symmetry, $a \sim c$. So, the transitivity of $\sim$ gives
        \begin{equation*}
            a \sim c \quad\text{and}\quad c \sim b \implies a \sim b.
        \end{equation*}
        Also, by symmetry, $b \sim a$.

        Now, if $x \in [a]$, then $x \sim a$. By transitivity, $x \sim a$ and $a \sim b$ implies $x \sim b$. That is, $x \in [b]$. So, $[a] \subseteq [b]$.

        Similarly, if $y \in [b]$, then $y \sim b$. Again, by transitivity, $y \sim b$ and $b \sim a$ implies $y \sim a$. That is, $y \in [a]$. So, $[b] \subseteq [a]$. Overall, $[a] = [b]$.\qedhere
    \end{enumerate}
\end{proof}

\begin{nthm}
    Let $H \subgroup G$ and $\sim$ be a relation on $G$ such that $a \sim b \coloneq \inv{a} b \in H$. Then, $\sim$ is an equivalence relation.
\end{nthm}
\begin{proof}
    We check that $\sim$ satisfies the three conditions of being an equivalence relation.
    
    (Reflexive.) Take $a \in G$. Then, $\inv{a} a = e$ and $e \in H$ because $H \subgroup G$. So, $a \sim a$.

    (Symmetric.) Suppose $a \sim b$. That means $\inv{a} b \in H$. As $H \subgroup G$, the element $\inv{\paren*{\inv{a} b}} \in H$. And $\inv{\paren*{\inv{a} b}} = \inv{b} a$. That is, $\inv{b} a \in H$. So, $b \sim a$.

    (Transitive.) Suppose $a \sim b$ and $b \sim c$. That is $\inv{a} b \in H$ and $\inv{b} c \in H$. As $H$ is a subgroup, closure and associativity gives $\paren*{\inv{a} b} \paren*{\inv{b} c} = \inv{a} c \in H$. So, $a \sim c$. \qedhere
\end{proof}

\begin{nthm}
    Let $H \subgroup G$ and $\sim$ be a relation on $G$ such that $a \sim b \coloneq a \inv{b} \in H$. Then, $\sim$ is an equivalence relation.
\end{nthm}

\section{Cosets}
\begin{ndfn}
    Let $H \subgroup G$. Take some $a \in G$. The subsets
    \begin{equation*}
        aH = \set{ah \st h \in H}
        \quad\text{and}\quad
        Ha = \set{ha \st h \in H}
    \end{equation*}
    are called the left and right \keyword{cosets} of $H$ containing $a \in G$, respectively.
\end{ndfn}

For the coset $aH$, the element $a$ is called a \keyword{representative} of the coset. We note that any element of $aH$ can act as its representative. A coset always contains its representative element; because $a = a e = e a$ and $e \in H$ for every subgroup.

\begin{nthm}
    Let $H \subgroup G$ and $\sim$ be an equivalence relation on $G$ such that $a \sim b$ if $\inv{a} b \in H$. Then, $[a] = aH$.
\end{nthm}
\begin{proof}
    We know that $[a] = \set{b \in G \st b \sim a} = \set{b \in G \st a \sim b} = \set{b \in G \st \inv{a} b \in H}$.

    Take any $x \in aH$. Then, $x = ah$ for some $h \in H$. So, $\inv{a} x = h \in H$. Therefore, $x \in [a]$. So, $aH \subseteq [a]$.

    Likewise, take any $x \in [a]$. Then, $\inv{a} x \in H$. So, there is some $h \in H$ such that $\inv{a} x = h$. Therefore, $x = ah \in aH$. So, $[a] \subseteq aH$. Overall, $[a] = aH$.
\end{proof}

We have an analogous result for the right cosets.
\begin{nthm}
    Let $H \subgroup G$ and $\sim$ be an equivalence relation on $G$ such that $a \sim b$ if $a \inv{b} \in H$. Then, $[a] = Ha$.
\end{nthm}

\begin{nthm}\label{thm:coset-properties-a}
    Let $H \subgroup G$ and $a \in G$.
    \begin{enumerate}[label=(\roman*)]
        \item $a \in aH$
        \item $aH = bH$ iff $a \in bH$
        \item $aH = bH$ or $aH \cap bH = \emptyset$
    \end{enumerate}
\end{nthm}
\begin{proof}
    \begin{enumerate}[label=(\roman*)]
        \item Since $H \subgroup G$, $e \in H$. So, $a = ae \in H$.

        \item Suppose $aH = bH$. By part (i) $a \in aH$. So, $a \in aH = bH$. Therefore, $a \in bH$.

        For the converse, suppose $a \in bH$. Then, $a = bh_0$ for some $h_0 \in H$. So,
        \begin{equation*}
            aH
            = \set{ah \st h \in H}
            = \set{(bh_0)h \st h \in H}
            = \set{b(h_0 h) \st h \in H}
            = \set{bk \st k \in H}
            = bH
        \end{equation*}

        \item Suppose $aH \cap bH \neq \emptyset$. Then, $\exists c \in G$ such that $c \in aH \cap bH$. So, $c = ah_1$ and $c = bh_2$ for some $h_1, h_2 \in H$. Thus,
        \begin{equation*}
            ah_1 = bh_2
            \implies a = bh_2 \inv{h_1}
            \implies a = bh
            \implies a \in bH,
        \end{equation*}
        with $h = h_2 \inv{h_1} \in H$. By part (ii) $a \in bH$ implies $aH = bH$.

        Therefore, either $aH \cap bH = \emptyset$ of $aH = bH$.\qedhere
    \end{enumerate}
\end{proof}

\begin{nthm}
    Let $H$ be a subgroup of $G$ and let $a, b \in G$. Then,
    \begin{enumerate}[label=(\roman*)]
        \item $aH = H$ if and only if $a \in H$.
        \item $|aH| = |bH|$
        \item $aH = bH$ if and only if $\inv{a} b \in H$.
    \end{enumerate}
\end{nthm}
\begin{proof}
    \begin{enumerate}[label=(\roman*)]
        \item This is a special case of part (ii) from theorem~(\ref{thm:coset-properties-a}) with $b = e$.

        \item Consider the map $\varphi : aH \to bH$ with $\varphi(ah) = bh$. This is injective because
        \begin{equation*}
            \varphi(ah_1) = \varphi(ah_2)
            \implies bh_1 = bh_2
            \implies h_1 = h_2
            \implies ah_1 = ah_2.
        \end{equation*}
        It is also surjective by construction. Therefore, $\varphi$ is a bijection between the sets $aH$ and $bH$. So, $|aH| = |bH|$.

        \item By theorem~(\ref{thm:coset-properties-a}) part(ii),
        \begin{equation*}
            aH = bH
            \iff b \in aH
            \iff b = ah \text{ for some } h \in H
            \iff \inv{a} b = h \in H.
            \qedhere
        \end{equation*}
    \end{enumerate}
\end{proof}

Again, we have an analogous version of the previous two theorems for right cosets.

\begin{remark}
    Since $|aH| = |bH|$ for all $a, b \in G$, we have that $|aH| = |H|$ for all $a \in G$. Therefore, the cardinality of each coset of $H$ is the same as the order of $H$.
\end{remark}

\begin{nthm}[Lagrange]
    Let $G$ be a finite group and $H$ be its subgroup. Then $|H|$ divides $|G|$. Moreover, the number of distinct left (right) cosets of $H$ in $G$ is $|G|/|H|$.
\end{nthm}
\begin{proof}
    Let $a_1 H, \dots, a_k H$ be all the distinct cosets of $H$ in $G$. Then, $G = \bigcup_{j=1}^{k} a_j H$ because for all $g \in G$, $g \in a_j H$ for some $j$.

    Moreover, $|aH| = |bH|$, for all $a, b \in G$. In particular, $|a_j H| = |e H| = |H|$ for all $j$.

    Now, since $G$ is written as a union of distinct sets, we have
    \begin{equation*}
        |G| = |a_1 H| \cup \dots \cup |a_k H| = |H| \cup \dots \cup |H| = k |H|.
    \end{equation*}
    So, $|H|$ divides $|G|$ and the number of distinct left (right) cosets is $k = |G|/|H|$.
\end{proof}

This proof shows that number of distinct left cosets is the same as the number of distinct right cosets. Therefore, the following statement is well-defined.

\begin{ndfn}
    Let $H \subgroup G$. The number of distinct left (right) cosets of $H$ in $G$ is called the index of $H$ in $G$. It is denoted as $[G:H]$.
\end{ndfn}

When $G$ is a finite group, Lagrange's theorem states that $[G:H] = |G|/|H|$.

\begin{ncor}\label{cor:lagrange-element-order}
    Let $G$ be a finite group and $a \in G$. Then $O(a)$ divides the order of $G$.
\end{ncor}
\begin{proof}
    Consider the subgroup $\cyclic{a}$ generated by $a \in G$. By theorem~(\ref{thm:cyclic-group-order}) we know that $|\cyclic{a}| = O(a)$. By Lagrange's theorem, $|\cyclic{a}|$ divides $|G|$. Therefore, $O(a)$ divides the order of $|G|$.
\end{proof}

\begin{ncor}
    If $G$ is a finite group and $a \in G$, then $a^{|G|} = e$.
\end{ncor}
\begin{proof}
    Suppose $|G| = n$ and $O(a) = m$. By corollary~(\ref{cor:lagrange-element-order}), $m \mid n$. Therefore, $n = mk$ for some $k \in \Z$. So, $a^{|G|} = a^n = a^{mk} = \paren*{a^{m}}^{k} = e^k = e$.
\end{proof}

\begin{ncor}
    Every group of prime order is cyclic.
\end{ncor}
\begin{proof}
    Let $|G| = p$, where $p$ is prime. Take $a \in G-\set{e}$, with $O(a) = m$. Since, $O(a) \mid p$, we have either $O(a) = 1$ or $O(a) = p$. Since, $a \neq e$, we have $O(a) \neq 1$. Therefore, $O(a) = p$.

    Let $H = \cyclic{a}$. Then $|H| = O(a) = p$. Now, $H \subseteq G$ and $|H| = |G|$ (finite). Therefore, $G = H = \cyclic{a}$.
\end{proof}

\begin{remark}
    If $|G|$ is infinite then $[G:H]$ may or may not be finite. Therefore, we cannot write $[G:H] = |G|/|H|$ when $|G|$ is infinite. For example, $[\Z:n\Z]=n$, while $[\R:\Z] = \infty$.
\end{remark}

\end{document}