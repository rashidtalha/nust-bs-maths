
\documentclass[11pt]{penrose}

\usepackage{mathsphystools}
\usepackage{thmstyles}

\newcommand{\cyclic}[1]{\mean{#1}}

\title{MATH 325: Group Theory I}
\subtitle{Brief lecture notes}
\author{Rashid M. Talha}
\affiliation{School of Natural Sciences, NUST}
\date{\today}
\begin{document}

\maketitle

\textbf{Textbook:} Contemporary Abstract Algebra, Joseph Gallian

\section{Introduction}
\begin{ndfn}
    A binary operation is a map $* : X \times X \to X$, $(a,b) \mapsto a * b$.
\end{ndfn}

By definition, a binary operation ensure that $a * b \in X$ for all $a, b \in X$. This property is called closure.

\begin{ndfn}
    A binary operation $* : X \times X \to X$ is called commutative if
    \begin{equation*}
        \forall a, b \in X, \quad a * b = b * a
    \end{equation*}
\end{ndfn}

\begin{ndfn}
    Let $G$ be a non-empty set, and $* : G \times G \to G$ be a binary operation. The pair $(G, *)$ is called a group if it satisfies all of the following
    \begin{enumerate}
        \item $\forall a, b, c \in G$, $(a * b) * c = a * (b * c)$ \hfill(Associativity)
        \item $\exists e \in G$ such that $\forall a \in G$, $a * e = e * a = a$ \hfill(Identity)
        \item $\forall a \in G$, $\exists a^{-1} \in G$ such that $a * a^{-1} = a^{-1} * a = e$ \hfill(Inverse)
    \end{enumerate}
\end{ndfn}

\begin{ndfn}
    A group $(G, *)$ is called Abelian if the binary operation $*$ is commutative. That is
    \begin{equation*}
        \forall a, b \in G, \quad a * b = b * a
    \end{equation*}
\end{ndfn}

Typically we write $a * b$ simply as $ab$ and call the binary operation a multiplication. In the case where the binary operation is the usual addition, we write $a + b$ instead.

Similarly, we often refer to $G$ as the group and don't explicitly mention the pair $(G, *)$.

\begin{nthm}
    Each group has a unique identity element.
\end{nthm}
\begin{proof}
    Let $e, f \in G$ be identity elements. Then, for all $a \in G$
    \begin{equation*}
        ea = ae = a
        \qquad\text{and}\qquad
        fa = af = a.
    \end{equation*}
    In particular, (taking $a=f$ in the first case and $a=e$ in the second)
    \begin{equation*}
        ef = fe = f
        \qquad\text{and}\qquad
        fe = ef = e.
    \end{equation*}
    As a result, $e = ef = f$.
\end{proof}

\begin{nthm}
    Each $a \in G$ has a unique inverse element.
\end{nthm}
\begin{proof}
    Take any $a \in G$. Let $a^{-1}, b \in G$ be inverse elements of $a$. That means
    \begin{equation*}
        a a^{-1} = a^{-1} a = e
        \qquad\text{and}\qquad
        a b = b a = e.
    \end{equation*}
    As a result, $b = b e = b(a a^{-1}) = (b a) a^{-1} = e a^{-1} = a^{-1}$.
\end{proof}

\begin{nthm}
    Let $G$ be a group. Then, for all $a \in G$, $(a^{-1})^{-1} = a$.
\end{nthm}
\begin{proof}
    Take any $a \in G$. Then, it has an inverse $a^{-1} \in G$ such that $a a^{-1} = e$. Since, $a^{-1} \in G$ it also has an inverse $(a^{-1})^{-1}$ such that $a^{-1} (a^{-1})^{-1} = e$.

    Therefore,
    $a = a e = a \paren*{a^{-1} (a^{-1})^{-1}} = \paren*{a a^{-1}} (a^{-1})^{-1} = e (a^{-1})^{-1} = (a^{-1})^{-1}$.
\end{proof}

\begin{nthm}
    Let $G$ be a group. Then, for all $a, b \in G$, $(ab)^{-1} = b^{-1} a^{-1}$.
\end{nthm}
\begin{proof}
    Note that $(ab)^{-1} (ab) = (ab) (ab)^{-1} = e$ by definition of the inverse of $ab$.

    Now,
    \begin{equation*}
        (b^{-1} a^{-1}) (a b)
        = b^{-1} ((a^{-1} a) b)
        = b^{-1} (e b)
        = b^{-1} b
        = e.
    \end{equation*}
    And,
    \begin{equation*}
        (a b) (b^{-1} a^{-1})
        = a ((b b^{-1}) a^{-1})
        = a (e a^{-1})
        = a a^{-1}
        = e.
    \end{equation*}

    So, $b^{-1} a^{-1}$ is also an inverse of $ab$. By uniqueness of inverse, $b^{-1} a^{-1} = (ab)^{-1}$.
\end{proof}

\begin{nthm}
    Take $a, b, c \in G$. Then,
    \begin{enumerate}
        \item $ab = ac \implies b = c$.
        \item $ba = ca \implies b = c$.
    \end{enumerate}
\end{nthm}
\begin{proof}
    Since $a \in G$, we have $a^{-1} \in G$ such that $a a^{-1} = a^{-1} a = e$. Therefore,
    \begin{equation*}
        ab = ac
        \implies a^{-1} (ab) = a^{-1} (ac)
        \implies (a^{-1} a)b = (a^{-1} a)c
        \implies eb = ec
        \implies b = c
    \end{equation*}
    Similarly,
    \begin{equation*}
        ba = ca
        \implies (ba)a^{-1} = (ca)a^{-1}
        \implies b (a a^{-1}) = c (a a^{-1})
        \implies be = ce
        \implies b = c
    \end{equation*}
\end{proof}

\begin{nthm}
    Let $G$ be a group. Take $a, b \in G$. Then, the equation $a x = b$ has a unique solution. Likewise, the equation $y a = b$ has a unique solution.
\end{nthm}
\begin{proof}
    (Existence.) Since $a \in G$, we have $a^{-1} \in G$ such that $a^{-1} a = e$. So,
    \begin{equation*}
        ax = b
        \implies a^{-1} (a x) = a^{-1} b
        \implies (a^{-1} a) x = a^{-1} b
        \implies e x = a^{-1} b
        \implies x = a^{-1} b.
    \end{equation*}
    And $a b^{-1} \in H$ due to the closure property. So, $x = ab^{-1} \in H$.

    (Uniqueness.) Suppose there are $x_1, x_2 \in H$ that satisfy $a x = b$. Then, $a x_1 = b$ and $a x_2 = b$. So, by the cancellation property
    \begin{equation*}
        a x_1 = a x_2
        \implies x_1 = x_2
    \end{equation*}
    
    The proof for $y a = b$ is analogous, with multiplications on the right hand side.
\end{proof}

\begin{ndfn}
    The order of a group $G$, denoted $\abs{G}$ or $O(G)$, is the number of elements in $G$. If $G$ has infinitely many elements then $\abs{G} = \infty$.
\end{ndfn}

\begin{egg}
    Some examples of groups are
    \begin{enumerate}
        \item $(\R,+)$, $(\C,+)$, $(\Z,+)$, $(\Q,+)$, $(\R^{*},\cdot)$, $(\C^{*},\cdot)$, $(\Q^{*},\cdot)$, $(\R^{+},\cdot)$. Here $\R^{*} = \R - \set{0}$, and $\R^{+} = \set{r \in \R \st r > 0}$.
        \item The set of $n$th roots of unity $U_n = \set{\exp(\frac{2\pi i}{n}) \in C | n = 0, 1, \dots, n-1}$ forms a group under the multiplication of complex numbers.
        \item The set of $n \times n$ matrices with entries in $\R$ is denoted as $M_{n}(\R)$. This forms a group under the usual additional of matrices.
        \item $\GL_{n}(\R) = \set{A \in M_{n}(\R) \st \det A \neq 0}$ with the usual matrix multiplication is called the general linear group of order $n$.
        \item The usual matrix multiplication makes $\SL_{n}(\R) = \set{A \in M_{n}(\R) \st \det A = 1}$ into a group, called the the special linear group of order $n$.
    \end{enumerate}
\end{egg}

\begin{negg}
    Consider the set with a single element $G = \set{e}$ and the binary operation $e * e = e$. This forms a group, called the trivial group. Note that for the trivial group $\abs{G} = 1$.
\end{negg}

\begin{ndfn}
    A non-empty subset $H \subseteq G$ is called a subgroup of $G$ if it is a group under the same binary operation. We denote this as $H \subgroup G$.
\end{ndfn}

\begin{ndfn}
    $H \subgroup G$ is called a proper subgroup if $H \neq G$. This is sometimes emphasised by writing $H < G$. A proper subgroup is called non-trivial if $H \neq \set{e}$.
\end{ndfn}

\begin{egg}
    Some examples of subgroups are
    \begin{enumerate}
        \item $\Z \subgroup \R$.
        \item $\R^{+} \subgroup \R^{*}$.
        \item $2\Z \subgroup \Z$, with $2\Z = \set{2k \st k \in \Z}$.
        \item $\SL_{n}(\R) \subgroup \GL_{n}(\R)$.
    \end{enumerate}
\end{egg}

\begin{nthm}
    Let $G$ be a group. A non-empty subset $H \subseteq G$ is a subgroup of $G$ if and only if
    \begin{enumerate}
        \item $a, b \in H \implies ab \in H$.
        \item $a \in H \implies a^{-1} \in H$.
    \end{enumerate}
\end{nthm}
\begin{proof}
    Suppose $H \subgroup G$. Then, $H$ is a group under the same binary operation. In particular, both the closure property and the existence of inverse property holds in $H$.

    Conversely, the closure property is explicitly given. Associativity is inherited from the binary operation on $G$. Also, the existence of inverse property is explicitly given. Finally, since $H$ is non-empty, take $a \in H$. Then, $a^{-1} \in H$. By the closure property, $a a^{-1} = e \in H$. Therefore, $H$ also contains the identity element. As a result, $H$ is a group with respect to the same binary operation. That is, $H \subgroup G$.
\end{proof}

\begin{nthm}
    Let $G$ be a group. A non-empty subset $H \subseteq G$ is a subgroup of $G$ if and only if $a, b \in H \implies ab^{-1} \in H$.
\end{nthm}
\begin{proof}
    Suppose $H \subgroup G$. Then, $H$ is a group under the same binary operation. Take $a, b \in H$. Then, by the previous subgroup test $b^{-1} \in H$. Again, by the previous subgroup test, $a b^{-1} \in H$.

    For the converse, we check that $H$ satisfies all the group axioms.

    Firstly, $H$ has the same binary operation as $G$, so associativity is inherited from $G$.

    Next, since $H$ is non-empty, take any $a \in H$. Then, $a a^{-1} \in H$ implies $e \in H$. So, $H$ contains the identity element.

    Similarly, take $e, a \in H$. Then, $e a^{-1} \in H$ implies $a^{-1} \in H$. Therefore, each element of $H$ has an inverse within $H$.

    Lastly, take $a, b \in H$. Then, $b^{-1} \in H$. So, $a \left(b^{-1}\right)^{-1} \in H$ implies $ab \in H$, since $\left(b^{-1}\right)^{-1} = b$.

    Therefore, $H$ is a group under the same binary operation as $G$. So, $H \subgroup G$.
\end{proof}

\begin{nthm}
    Let $H_i \subgroup G$, for all $i \in I$. Then, $H = \cap_{i \in I} H_i$ is a subgroup of $G$.
\end{nthm}
\begin{proof}
    Firstly, $e \in H_i$ for all $i \in I$ because each $H_i$ is a subgroup of $G$. As a result, $e \in H$. So, $H$ is non-empty.

    Take $a, b \in H$. Then, $a, b \in H_i$ for all $i \in I$. As $H_i$ are subgroups, $ab^{-1} \in H_i$ for all $i \in I$. Therefore, $ab^{-1} \in H$. By the subgroup criteria, this shows that $H \subgroup G$.
\end{proof}

\begin{nthm}
    Let $G$ be a group and take $a \in G$. The set $H = \set{a^n \st n \in \Z}$ is a subgroup of $G$. Here, $a^0 = e$ and $a^{-n} = \left(a^{-1}\right)^{n}$.
\end{nthm}
\begin{proof}
    Firstly, $H$ is non-empty because $a^0 = e \in H$.

    Next, take any $a^{n}, a^{m} \in H$. Then, $(a^{m})^{-1} = a^{-m}$. So,
    \begin{equation*}
        a^{n} (a^{m})^{-1}
        = a^{n} a^{-m}
        = a^{n - m} \in H,
    \end{equation*}
    since $n - m \in \Z$.

    By the subgroup criteria, this shows that $H \subgroup G$.
\end{proof}

\begin{ndfn}
    The order of an element $a \in G$ is the least positive integer $k$ such that $a^k = e$. We denote this as $O(a)$ or $\abs{a}$.
\end{ndfn}

\begin{nthm}
    $H_1, H_2 \subgroup G$. Then, $H_1 \cup H_2$ is a subgroup of $G$ if and only if $H_1 \subseteq H_2$ or $H_2 \subseteq H_1$.
\end{nthm}
\begin{proof}
    Suppose (WLOG) $H_1 \subseteq H_2$. Then, $H_1 \cup H_2 = H_2$. And $H_2 \subgroup G$. Therefore, $H_1 \cup H_2 \subgroup G$.

    Conversely, suppose $H_1 \subsetneq H_2$ and $H_2 \subsetneq H_1$ but $H_1 \cup H_2$ is a subgroup of $G$. Then, \dots
\end{proof}

\section{Modular Addition}

\begin{ndfn}
    Let $a, b$ be integers and fix a positive integer $n$. We say that $a$ is congruent to $b$ modulo $n$ if $n$ divide $a-b$. That is $n | (a-b)$. This is denoted as $a \equiv b \mod n$.
\end{ndfn}

\begin{ndfn}
    Take $a \in \Z$, and fix some integer $n \geq 2$. The set of all the integers that are equivalent to $a$ modulo $n$ is called the residue class of $a$ modulo $n$. We write this as
    \begin{equation}
        [a]_n = \set{b \in \Z \st a \equiv b \mod n}.
    \end{equation}
\end{ndfn}

\begin{remark}
    $[a]_n$ and $[b]_n$ are either equal of disjoint. (This was skipped.)
\end{remark}

\section{\texorpdfstring{$\Z_3$}{Z3}}
\begin{ndfn}
    Fix some integer $n \geq 2$. The set of all the residue classes modulo $n$ in $\Z$ is denoted as
    \begin{equation}
        \Z_n = \set*{[a]_n \st a \in \Z}.
    \end{equation}
\end{ndfn}

We can define a binary operation $+_n$, called modular addition (mod $n$) on this set,
\begin{equation}
    [a]_n +_n [b]_n \coloneq [a + b]_n.
\end{equation}
It needs to be shown that this is well-defined; that is if $[a]_n = [c]_n$ and $[b]_n = [d]_n$, then, $[a]_n +_n [b]_n = [c]_n +_n [d]_n$. (This was skipped.)

\begin{nthm}
    The set $\Z_n$ forms a group with respect to $+_n$.
\end{nthm}
\begin{proof}
    TBC.
\end{proof}

When it is clear from the context, we drop the subscript and simply write $[a]_n$ as $a$ and $+_n$ as $+$.

\section{\texorpdfstring{Klein $4$-Group}{Klein 4 Group}}
Consider a set $G = \set{e, a, b, c}$ with a binary operation that satisfies $a^2 = b^2 = c^2 = e$, $xy = yx$ for all $x, y \in G$. It can easily be checked that this forms a group. We also find that some of the conditions imposed on the binary operation are redundant, and instead this group can be expressed more compactly. The following theorem states this observation.

\begin{nthm}
    The set $K_4 = \set{1, a, b, ab}$, where the order of each non-identity element is $2$ forms a group.
\end{nthm}
\begin{proof}
    TBC.
\end{proof}

This group $K_4$ is called the Klein $4$-group. It is a group of order $4$. Its multiplication rule can be represented as table, called a Cayley table.
\begin{center}
    \ra{1.2}
    \begin{tabularx}{0.3\textwidth}{|R|R|R|R|R|}
        \hline
             & $ 1$ & $ a$ & $ b$ & $ab$ \\ \hline
        $ 1$ & $ 1$ & $ a$ & $ b$ & $ab$ \\ \hline
        $ a$ & $ a$ & $ 1$ & $ab$ & $ b$ \\ \hline
        $ b$ & $ b$ & $ab$ & $ 1$ & $ a$ \\ \hline
        $ab$ & $ab$ & $ b$ & $ a$ & $ 1$ \\ \hline
    \end{tabularx}

    {Multiplication table for $K_4$.}
\end{center}

Here, the entry in the the $(i,j)$-th entry is the result of multiplying the element in the $j$th column with the element in the $i$th row in the `column-on-the-left' order.
\begin{equation*}
    (\text{col}_j) * (\text{row}_i) = (i,j)\text{-th entry}
\end{equation*}

There are precisely three non-trivial subgroups of $K_4$: $\set{1, a}$, $\set{1, b}$ and $\set{1, ab}$.

Consider the set
\begin{equation}
    G = \set*{
    \begin{bmatrix}1 & 0 \\ 0 & 1\end{bmatrix},
    \begin{bmatrix}1 & 0 \\ 0 & 1\end{bmatrix},
    \begin{bmatrix}1 & 0 \\ 0 & 1\end{bmatrix},
    \begin{bmatrix}1 & 0 \\ 0 & 1\end{bmatrix}}
\end{equation}
of matrices with the usual matrix multiplication. We find that it forms a group. Clearly, this is a group of order $4$. Moreover, each non-identity element in this group has order $2$. This coincides exactly with the group structure of $K_4$. We say that this group is the same as $K_4$ (this notion will be made precise when we define isomorphisms later), and that it is simply a matrix representation of $K_4$.

Another group of order $4$ that we have already seen in $\Z_4$. This group has an element of order $4$, namely $[1]_4$. Therefore, it cannot be the `same' as $K_4$. (Again, this observation will be made precise through the use of isomorphisms.) This shows that not all groups of order $4$ are the same as $k_4$.

\section{Group of Quaternions}
Consider the group $\GL_2(\C)$ of $2 \times 2$ invertible matrices with complex entries. Consider its subset
\begin{equation}
    H = \set*{
    \pm\begin{bmatrix*}[r]1 & 0\\ 0 & 1\end{bmatrix*},
    \pm\begin{bmatrix*}[r]1 & 0\\ 0 &-1\end{bmatrix*},
    \pm\begin{bmatrix*}[r]0 & 1\\-1 & 0\end{bmatrix*},
    \pm\begin{bmatrix*}[r]0 & i\\ i & 0\end{bmatrix*} }.
\end{equation}
It is easy to check that this is a subgroup of $\GL_2(\C)$, and therefore a group. We note that this is a group of order $8$.

Using the multiplication of $H$ as a template, we can define an abstract group of order $8$ as follows.
\begin{nthm}
    The set $Q_8 = \set{\pm 1, \pm i, \pm j, \pm k}$ forms a group with the multiplication rule
    \begin{gather}
        i^2 = j^2 = k^2 = 1\\
        ij = k, jk = i, ki = j\\
        ji = -k, kj = -i, ik = -j.
    \end{gather}
\end{nthm}

This group is called the group of quaternions.
\begin{center}
    \ra{1.2}
    \begin{tabularx}{0.62\textwidth}{|R|R|R|R|R|R|R|R|R|}
        \hline
             & $ 1$ & $ i$ & $ j$ & $ k$ & $-1$ & $-i$ & $-j$ & $-k$ \\ \hline
        $ 1$ & $ 1$ & $ i$ & $ j$ & $ k$ & $-1$ & $-i$ & $-j$ & $-k$ \\ \hline
        $ i$ & $ i$ & $ 1$ & $-k$ & $ j$ & $-i$ & $-1$ & $ k$ & $-j$ \\ \hline
        $ j$ & $ j$ & $ k$ & $ 1$ & $ -$ & $ -$ & $ -$ & $ -$ & $ -$ \\ \hline
        $ k$ & $ k$ & $-j$ & $ -$ & $ 1$ & $ -$ & $ -$ & $ -$ & $ -$ \\ \hline
        $-1$ & $-1$ & $-i$ & $ -$ & $ -$ & $ 1$ & $ -$ & $ -$ & $ -$ \\ \hline
        $-i$ & $-i$ & $-1$ & $ -$ & $ -$ & $ -$ & $ 1$ & $ -$ & $ -$ \\ \hline
        $-j$ & $-j$ & $-k$ & $ -$ & $ -$ & $ -$ & $ -$ & $ 1$ & $ -$ \\ \hline
        $-k$ & $-k$ & $ j$ & $ -$ & $ -$ & $ -$ & $ -$ & $ -$ & $ 1$ \\ \hline
    \end{tabularx}

    {Multiplication table for $Q_4$.}
\end{center}

\section{Dihedral Group}


\section{Cyclic Group}
\begin{ndfn}
    A group $G$ is called cyclic if there is an element $a \in G$ such that all elements of $G$ can be written as powers of $a$. More precisely, $\forall g \in G$, $\exists m \in \Z$ such that $g = a^m$.

    Such as element $a$ is called a generator of $G$, and we say that $G$ is the group generated by $a$ and denote this as $G = \cyclic{a}$. Cyclic group of order $n$ is sometimes denoted as $C_n$.
\end{ndfn}

Generators are not unique. Indeed if $a \in G$ is a generator then so is $\inv{a}$.

\begin{notation}
    For $m > 0$, $a^m$ means $a * \dots * a$, where $m$ factors of $a$ are multiplied together. Similarly, $a^0 \equiv 1$, the identity element. And, $a^{-m}$ means $\paren*{\inv{a}}^m$.
\end{notation}

\begin{nthm}
    Let $G$ be a group, and take $a \in G$ such that $a^n = 1$. Then, the cyclic group $\cyclic{a}$ has the form $\set{1, a, a^2, \dots, a^{n-1}}$.
\end{nthm}

\begin{nthm}
    Let $G = \cyclic{a}$ be a cyclic group. Then, $O(G) = O(a)$.
\end{nthm}

\begin{nthm}
    Let $G$ be a group, and take $a \in G$ such that $a^n = 1$. If $a^m = 1$ then $n | m$.
\end{nthm}

\begin{nthm}
    Every cyclic group is abelian.
\end{nthm}
\begin{proof}
    Let $G = \cyclic{g}$ be a cyclic group. Take $a, b \in G$. Then, $a = g^m$, $b = g^n$ for some $m, n \in \Z$. As a result, $ab = g^m g^n = g^{m+n} = g^{n+m} = g^n g^m = ba$.
\end{proof}

\begin{nthm}
    Every subgroup of a cyclic group is cyclic.
\end{nthm}


\end{document}