\documentclass[11pt]{penrose}

\usepackage{mathsphystools}
\usepackage{thmstyles}

\title{MATH 452: Ordinary Differential Equations II}
\subtitle{Brief lecture notes}
\author{Rashid M. Talha}
\affiliation{School of Natural Sciences, NUST}
\date{\today}
\begin{document}

\maketitle

\textbf{Textbook:} Differential Equations with Boundary Value Problems, Dennis Zill and Michael Cullen. Linear Partial Differential Equations for Scientists and Engineers, Tyn Myint-U and Lokenath Debnath.

\textbf{Disclaimer:} This document most likely contains some errors --- use with caution. I have rephrased or paraphrased the content in many of the sections. Some examples may be missing. The numbering that I have used for sections, definitions, theorems, etc will not match the numbering given in the lectures.

\section{Introduction}

\section{Boundary-Value Problems Involving Ordinary Differential Equations}
In this section, we take
\begin{equation*}
    L[y] = y'' + p(x)y' + q(x)y,
    \quad
    U_1 [y] = a_1 y(a) + a_2 y'(a),
    \quad
    U_2 [y] = b_1 y(b) + b_2 y'(b).
\end{equation*}

\begin{nthm}
    Let $p(x), q(x)$ and $f(x)$ be continuous functions on $[a,b]$ and let $\alpha, \beta \in \R$. Then, either the boundary value problem
    \begin{equation*}
        L[y] = f,
        \quad
        U_1[y] = \alpha,
        \quad
        U_2[y] = \beta
    \end{equation*}
    has a unique solution, or the associated homogenous boundary value problem
    \begin{equation*}
        L[y] = 0,
        \quad
        U_1[y] = 0,
        \quad
        U_2[y] = 0
    \end{equation*}
    has a non-trivial solution.    
\end{nthm}

\section{Green's Functions for Ordinary Differential Equations}


\end{document}