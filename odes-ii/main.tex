\documentclass[11pt]{penrose}

\usepackage{mathsphystools}
\usepackage{thmstyles}

\title{MATH 452: Ordinary Differential Equations II}
\subtitle{Brief lecture notes}
\author{Rashid M. Talha}
\affiliation{School of Natural Sciences, NUST}
\date{\today}
\begin{document}

\maketitle

\textbf{Textbook:} Differential Equations with Boundary Value Problems, Dennis Zill and Michael Cullen. Linear Partial Differential Equations for Scientists and Engineers, Tyn Myint-U and Lokenath Debnath.

\textbf{Disclaimer:} This document most likely contains some errors --- use with caution. I have rephrased or paraphrased the content in most of the sections. Some examples may be missing. The numbering that I have used for sections, definitions, theorems, etc will not match the numbering given in the lectures.

\section{Introduction}
% \begin{ndfn}
%     A \keyword{binary operation} is a map $* : X \times X \to X$, $(a,b) \mapsto a * b$.
% \end{ndfn}

% By definition of $*$, $a * b \in X$ for all $a, b \in X$. This property is called \keyword{closure}.

% \begin{ndfn}
%     A binary operation $* : X \times X \to X$ is called \keyword{commutative} if
%     \begin{equation*}
%         \forall a, b \in X, \quad a * b = b * a
%     \end{equation*}
% \end{ndfn}

% \begin{ndfn}
%     Let $G$ be a non-empty set, and $* : G \times G \to G$ be a binary operation. The pair $(G, *)$ is called a \keyword{group} if it satisfies all of the following
%     \begin{enumerate}[label=(\roman*)]
%         \item $\forall a, b, c \in G$, $(a * b) * c = a * (b * c)$ \hfill(\keyword{Associativity})
%         \item $\exists e \in G$ such that $\forall a \in G$, $a * e = e * a = a$ \hfill(\keyword{Identity})
%         \item $\forall a \in G$, $\exists a^{-1} \in G$ such that $a * a^{-1} = a^{-1} * a = e$ \hfill(\keyword{Inverse})
%     \end{enumerate}
% \end{ndfn}

% \begin{ndfn}
%     A group $(G, *)$ is called \keyword{Abelian} if the binary operation $*$ is commutative. That is, for all $a, b \in G$, $a * b = b * a$ .
% \end{ndfn}

% \begin{remark}
% Typically we write $a * b$ simply as $ab$ and call the binary operation multiplication. In the case where the binary operation is the usual addition, we write $a + b$ instead. Similarly, we often refer to $G$ as the group and don't explicitly mention the pair $(G, *)$. Moreover, we sometimes denote the identity element by $1$ for multiplicative binary operations, and by $0$ for additive binary operations.
% \end{remark}

% \begin{nthm}
%     Each group has a unique identity element.
% \end{nthm}
% \begin{proof}
%     Let $e, f \in G$ be identity elements. Then, for all $a \in G$
%     \begin{equation*}
%         ea = ae = a
%         \qquad\text{and}\qquad
%         fa = af = a.
%     \end{equation*}
%     In particular, (taking $a=f$ in the first case and $a=e$ in the second)
%     \begin{equation*}
%         ef = fe = f
%         \qquad\text{and}\qquad
%         fe = ef = e.
%     \end{equation*}
%     As a result, $e = ef = f$.
% \end{proof}


\end{document}