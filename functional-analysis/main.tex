\documentclass[11pt]{penrose}

\usepackage{mathsphystools}
\usepackage{thmstyles}

\setlist[enumerate]{label=(\roman*)}

\title{MATH 446: Functional Analysis}
\subtitle{Brief lecture notes}
\author{Rashid M. Talha}
\affiliation{School of Natural Sciences, NUST}
\date{\today}
\begin{document}

\maketitle

\textbf{Textbooks \& Reference books:}
\begin{itemize}
    \item \dots
    \item \dots
    \item \dots
\end{itemize}

\textbf{Disclaimer:} This document most likely contains some errors. Use with caution. I have rephrased or paraphrased the content in many of the sections. Some examples may be missing. The numbering that I have used for sections, definitions, theorems, etc will not match the numbering given in the lectures.

\section{Metric Spaces}
\begin{ndfn}
    Let $X$ be a set. A metric on $X$ is a function $d : X \times X \to [0,\infty)$ such that for all $x, y, z \in X$
    \begin{enumerate}
        \item $d(x,y) = 0$ if and only if $x = y$ \hfill (positive definiteness)
        \item $d(x,y) = d(y,x)$ \hfill (symmetry)
        \item $d(x,y) \leq d(x,z) + d(z,y)$ \hfill (triangle inequality)
    \end{enumerate}
    The pair $(X,d)$ is called a metric space.
\end{ndfn}

\begin{ndfn}
    Given a metric space $(X,d)$, let $Y \subseteq X$ and $\tilde{d} = d \at{Y \times Y}$. Then, $(Y,\tilde{d})$ is called a metric subspace of $(X,d)$, and $\tilde{d}$ is called the metric induced on $Y$ by $d$.
\end{ndfn}

Discrete metric.

EUCLIDEAN METRIC on $\R^n$ and $\C^n$.

Sequence space $l^{\infty}$ metric.

Function space $C[a,b]$.

Sequence space $s$.


\end{document}