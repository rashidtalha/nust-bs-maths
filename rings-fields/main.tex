\documentclass[11pt]{penrose}

\usepackage{mathsphystools}
\usepackage{thmstyles}

\DeclareMathOperator{\charac}{char}
\setlist[enumerate,1]{label=(\roman*)}

\title{MATH 423: Rings and Fields}
\subtitle{Brief lecture notes}
\author{Rashid M. Talha}
\affiliation{School of Natural Sciences, NUST}
\date{\today}
\begin{document}

\maketitle

\textbf{Textbook:} Contemporary Abstract Algebra, Joseph Gallian.

\textbf{Disclaimer:} This document most likely contains some errors --- use with caution. I have rephrased or paraphrased the content in many of the sections. Some examples may be missing. The numbering that I have used for sections, definitions, theorems, etc will not match the numbering given in the lectures.

\section{Introduction}
\begin{ndfn}
    A \keyword{ring} is a non-empty set $R$ with two \keyword{binary operations}
    \begin{equation*}
        + : R \times R \to R
        \quad\text{and}\quad
        \cdot : R \times R \to R
    \end{equation*}
    such that
    \begin{enumerate}
        \item[(R1)] $(R,+)$ is an Abelian group with $O \in R$ as the identity element.
        \item[(R2)] For all $a, b, c \in R$, $a \cdot (b \cdot c) = (a \cdot b) \cdot c$.
        \item[(R3)] There exists $1 \in R$ such that for all $a \in R$, $a \cdot 1 = 1 \cot a = a$.
        \item[(R4)] For all $a, b, c \in R$, $a \cdot (b + c) = a \cdot b + a \cdot c$ and $(b+c) \cdot a = b \cdot a + c \cdot a$.
    \end{enumerate}
\end{ndfn}

\begin{notation}
    We call `$+$' \keyword{addition} and `$\cdot$' \keyword{multiplication}. Typically $a \cdot b$ is abbreviated as $ab$, but $a+b$ is always written in full. The element $0 \in R$ is called the \keyword{additive identity}; this is unique because $(R,+)$ is a group and each group has a unique identity element. The element $1 \in R$ is called a \keyword{multiplicative identity}; later we shall show that this is also unique.
\end{notation}

\begin{remark}
    The definition of a ring does not require the existence of multiplicative inverses, nor does it require the multiplication to be commutative.

    Some authors exclude axiom (R3) about the existence of a multiplicative identity. We shall not adopt this approach. A ring without a multiplicative identity is called a \keyword{rng}.

    Property (R4) is called that the distribution of multiplication over addition.
\end{remark}

\begin{ndfn}
    A ring $R$ is called \keyword{Abelian} or \keyword{commutative} if its multiplication is commutative. That is, if for all $a, b \in R$, we have $ab = ba$.
\end{ndfn}

\begin{nthm}
    Let $R$ be a ring and $a, b, c \in R$. Then
    \begin{enumerate}[label=(\roman*)]
        \item $a0 = 0a = a$
        \item $a(-b) = (-a)b = -(ab)$
        \item $(-a)(-b) = ab$
        \item $a(b-c) = ab - ac$
        \item $(b-c)a = ba - ca$
        \item $(-1)a = -a$
        \item $(-1)(-1) = 1$
    \end{enumerate}
\end{nthm}

\begin{nthm}
    Each group $R$ has a unique multiplicative identity.
\end{nthm}

\begin{nlemma}
    Let $R$ be a ring. $0 = 1$ if and only if $R = \set{0}$.
\end{nlemma}

\begin{ndfn}
    The ring $R = \set{0}$ is called the \keyword{zero ring}. It is usually written as $R = 0$.
\end{ndfn}


\section{Subrings}
\begin{ndfn}
    A subset $S \subseteq R$ is called a \keyword{subring} of $R$ if it is a ring with respect to the same binary operations and the same identity elements.
\end{ndfn}

\begin{remark}
    We shall see later that having the same identity elements needs to be specified as a separate condition, because $S$ might form a ring with the same binary operations of $R$ but with a different multiplicative identity. In such a case, $S$ does not form a subring of $R$. Some authors drop the condition that $1_R \in S$.
\end{remark}

\begin{nlemma}
    Let $R$ be a ring. $S \subseteq R$ is a subring of $R$ if and only if
    \begin{enumerate}[label=(\roman*)]
        \item $0, 1 \in S$
        \item For all $a, b \in S$, $a-b \in S$
        \item For all $a, b \in S$, $ab \in S$
    \end{enumerate}
    (So, $S \subgroup R$, $S$ is closed under multiplication and $1 \in S$.)
\end{nlemma}

\begin{negg}
    The prototypical examples of a ring is the set of integers $\Z$ with the usual addition and multiplication. This is a commutative ring.
\end{negg}

\begin{negg}
    The sets $\Q$, $\R$, $\C$ and $\Z_n$ are all commutative rings with their standard addition and multiplication operations. Here, $\Z_n$ is the set of integers mod $n$.

    Let $R \neq 0$. The set of $n \times n$ matrices with entries in $R$, denoted $M_n (\R)$ is ring. It is non-commutative for $n \geq 2$.
\end{negg}

\begin{negg}
    The set $S = \set*{\begin{psmallmatrix*} a & b \\ 0 & c \end{psmallmatrix*} \st a, b, c \in \Z}$ is a subring of $M_2 (\Z)$.

    We know that $2\Z \subgroup \Z$ and that it is closed under multiplication, but it is not a subring of $\Z$ because $1 \notin 2\Z$.
\end{negg}

\begin{negg}
    We can give $\R^2$ a ring structure by defining element-wise addition and multiplication
    \begin{equation}
        (a_1, a_2) + (b_1, b_2) = (a_1 + b_1, a_2 + b_2)
        \quad\text{and}\quad
        (a_1, a_2) \cdot (b_1, b_2) = (a_1 b_1, a_2 b_2),
    \end{equation}
    where $a_1, a_2, b_1, b_2 \in \R$. We find that $0_{\R^2} = (0,0)$ and $1_{\R^2} = (1,1)$.

    Consider $S = \set{(a,0) \set a \in \R} \subseteq \R^2$. This is not a subring of $R$ since $1_{\R^2} = (1,1) \notin S$. This is despite the fact that $S$ does form a ring with the same element-wise addition and multiplication as defined above; its multiplicative identity is $(1,0)$.
\end{negg}

\section{Division Rings and Integral Domains}
\begin{ndfn}
    Let $R \neq 0$ be a ring. We say that $u \in R$ is a \keyword{unit} if there exists some $v \in R$ such that $uv = vu = 1$.
\end{ndfn}

\begin{remark}
    If $u \in R$ is a unit due to $v \in R$, then $v$ is also a unit (due to $u$). The element $v$ is then called the multiplicative inverse of $u$. We shall later show that this is unique.
\end{remark}

\begin{nthm}
    Take $R \neq 0$ and consider $R^* = \set{u \in R \st u \text{ is a unit}}$. Then, $(R^*, \cdot)$ forms a group. It is called the \keyword{group of units} of $R$.
\end{nthm}

\begin{remark}
    We note that if $R \neq 0$, then $R^* \neq \emptyset$ because $1 \in R$ is a unit.

    Since $(R^*, \cdot)$ is a group, we have that if $u$ is a unit with the inverse $v$, then $v$ is the unique element satisfying $uv = vu = 1$ due to the uniqueness of inverses in a group.
\end{remark}

\begin{ndfn}
    A \keyword{division ring} is a non-zero ring $R$ where every non-zero element is a unit. That is, $R \neq 0$ and $R^* = R - \set{0}$. Such a ring is also called a \keyword{skew-field}.
\end{ndfn}

\begin{ndfn}
    A \keyword{field} $\F$ is a commutative division ring. That is, $\F \neq 0$, $\F^* = \F - \set{0}$ and for all $a, b \in \F$ we have $ab = ba$.
\end{ndfn}

\begin{nthm}
    Every finite division ring is a field.
\end{nthm}

\begin{ndfn}
    The \keyword{characteristic} of a field $\F$ is the least positive integer $n$ such that $n \cdot 1 = 0$. If no such number exists then the characteristic is defined to be $0$. In symbols, we write $\charac \F = n$.
\end{ndfn}

\begin{negg}
    $\Q$, $\R$ and $\C$ are fields. We find that $\charac \Q = \charac \R = \charac \C = 0$.

    $\Z$ is a commutative ring but not a division ring (and therefore not a field) because
    \begin{equation*}
        \Z^* = \set{\pm 1} \neq \Z - \set{0}.
    \end{equation*}
\end{negg}

\begin{negg}
    It is easy to check that $M_2 (\R)^* = \GL_2 (\R)$. Similarly,
    \begin{equation*}
        M_2 (\Z)^*
        = \set{A \in M_2 (\Z) \st \det A = \pm 1}
        = \GL_2 (\Z).
    \end{equation*}
    In fact we find that, for any $R \neq 0$, $M_n (R)^* = \GL_n (R)$, where
    \begin{equation*}
        \GL_n (R) = \set{A \in M_n (R) \st \det A \in R^*}.
    \end{equation*}
\end{negg}

\begin{nthm}
    If $S$ is a subring of $R$, then $S^*$ is a subgroup of $R^*$.
\end{nthm}

\begin{nthm}
    The ring $\Z_n$ is a field if and only if $n$ is prime.
\end{nthm}

\begin{remark}
    When $p$ is prime, we write $\F_p$ in place of $\Z_p$. It is easy to see that $\charac \F_p = p$.
\end{remark}

\begin{ndfn}
    Let $R$ be a ring. A non-zero element $a \in R$ is called a \keyword{zero divisor} if there is a non-zero element $b \in R$ such that either $ab = 0$ or $ba = 0$.
\end{ndfn}

\begin{ndfn}
    An \keyword{integral domain} is a non-zero commutative ring $R \neq 0$ that has no zero divisors.
\end{ndfn}

\begin{nthm}
    Let $R$ be a ring and take $a, b, c \in R$. Suppose $a$ is not a zero divisor. If $ab = ac$, then either $a = 0$ or $b = c$.
\end{nthm}
\begin{proof}
    If $ab = ac$ then $a (b - c) = 0$ by distributivity. Since, $a$ is not a zero divisor, we must have either $a = 0$ or $b-c = 0$. Consequently, either $a = 0$ or $b = c$.
\end{proof}
\begin{remark}
    In particular, if $R$ is an integral domain then it has no zero divisors. Therefore, for all $a, b, c \in R$, $ab = ac \implies a = 0$ or $b = c$.
\end{remark}

\begin{nthm}
    Every field is an integral domain.
\end{nthm}

\begin{nthm}
    Every finite integral domain is a field.
\end{nthm}

\begin{negg}
    We know that $\Z_p$ is a finite commutative ring, and that it is a field if and only if $p$ is prime. So, $\Z_p$ is an integral domain exactly when $p$ is prime.
\end{negg}

\section{Polynomial Rings}

\begin{negg}
    Let $R \neq 0$ be a ring. The set of polynomials in powers of $X$ and with coefficients in $R$ is denoted by $R[X]$. In other words, the general element of $R[X]$ is a polynomial
    \begin{equation*}
        r(X) = a_0 + a_1 X + a_2 X^2 + \dots + a_n X^n \equiv \sum_{i=0}^n a_i X^i.
    \end{equation*}
    Note that only finite summations are allowed.

    We find that $R[X]$ forms are ring with the standard polynomial addition and multiplication. The additive identity is the zero polynomial $f(X) = 0$ and the multiplicative identity is $f(X) = 1$.

    $R[X]$ does not form a ring if we use composition of functions as the multiplicative operation because composition does not distribute over addition.
\end{negg}

\begin{negg}
    Consider the set $\Z[i]$ with $i \in \C$. This is
    \begin{equation*}
        \Z[i] = \set{a + bi \st a, b \in \Z} \subseteq \C.
    \end{equation*}
    Elements of this set are called \keyword{Gaussian integers}. We note that $\Z[i]$ is a ring under the usual polynomial addition and multiplication. Moreover, it is a subgroup of $\C$. Also,
    \begin{equation*}
        \Z[i]^* = \set{1, i, -1, -i} \neq \Z[i] - \set{0}.
    \end{equation*}
    So, this is not a division ring (and therefore not a field).
\end{negg}

\begin{nlemma}
    Let $\F$ be a field. Then, $\F[X]^* = \F^*$.
\end{nlemma}

\section{Homomorphisms}

\begin{ndfn}
    Let $R, S$ be rings. A ring \keyword{homomorphism} is a map $\varphi : R \to S$ such that for all $a, b \in R$
    \begin{itemize}
        \item $\varphi(a + b) = \varphi(a) + \varphi(b)$
        \item $\varphi(ab) = \varphi(a)\varphi(b)$
        \item $\varphi(1_R) = 1_S$
    \end{itemize}
    In other words, $\varphi$ is a group homomorphism $R \to S$, that preserves the multiplication and the multiplicative identity.
\end{ndfn}

\begin{ndfn}
    A bijective ring homomorphism is called an \keyword{isomorphism}.
\end{ndfn}

\begin{remark}
    If there is an isomorphism between two rings then they are called isomorphic to each other. Being isomorphic is an equivalence relation.
\end{remark}

\begin{nthm}
    Let $\varphi : R \to S$ be a ring homomorphism. Then, for all $a, b \in R$
    \begin{equation*}
        \varphi(-a) = -\varphi(a)
        \quad\text{and}\quad
        \varphi(0_R) = 0_S.
    \end{equation*}
\end{nthm}

\section{Ideals}

\begin{ndfn}
    Let $R$ be a ring. A left ideal of $R$ is a subset $I \subset R$ such that
    \begin{itemize}
        \item $(I,+) \subgroup (R,+)$
        \item For all $r \in R$ and $x \in I$, we have $rx \in I$
    \end{itemize}
    A right ideal of $R$ is a subset $I \subset R$ such that
    \begin{itemize}
        \item $(I,+) \subgroup (R,+)$
        \item For all $r \in R$ and $x \in I$, we have $xr \in I$
    \end{itemize}
    A 2-sided ideal is a subset that is both a left and a right ideal.
\end{ndfn}

\begin{remark}
    The zero ring $0$ and the full ring $R$ are always 2-sided ideals for any ring $R$. A proper ideal is one which does not equal $R$. In a commutative ring, every ideal is a 2-sided ideal.
\end{remark}

\begin{negg}
    We saw earlier that $2\Z$ is not a subring of $\Z$. It is easy to check that $2\Z$ is an ideal of $\Z$. Here, $I + J = \set{i+j \st i \in I, j \in J}$.
\end{negg}

\begin{nlemma}
    Let $I, J$ be left ideals of $R$. Then, $I \cap J$ and $I + J$ are both left ideals of $R$.
\end{nlemma}

\begin{remark}
    If $I, J$ are left ideals of $R$ then, $I \cup J$ need not be an ideal of $R$.
\end{remark}

\begin{nlemma}
    The only ideals of a field $\F$ are $0$ and $\F$ itself. (Here, $0$ is the zero ring).
\end{nlemma}

\begin{ndfn}
    Let $A, B$ be subsets of a ring $R$. We define the product $AB$ to be the set of all finite sums
    \begin{equation*}
        AB = \set*{ \sum_{i=1}^{n} a_i b_i \st a_i \in A, b_i \in B, n \geq 0 }.
    \end{equation*}
    We interpret the empty sum with $n = 0$ as the $0$ element. Thus $0 \in AB$.
\end{ndfn}

\begin{nlemma}
    Let $I, J$ be left ideals of $R$. Then, $IJ$ is a left ideal of $R$, and $IJ \subseteq I \cap J$.
\end{nlemma}

\begin{negg}
    Let $R$ be a ring and $a \in R$. We define $aR$ to be the set $\set{a}R$. That is,
    \begin{equation*}
        aR = \set*{\sum_{i=1}^{n} a r_i \st r_i \in R}.
    \end{equation*}
    However, $ar_1 + \dots + ar_n = a (r_1 + \dots + r_n) = ar$ for some $r \in R$. Therefore, we have
    \begin{equation*}
        aR = \set{ar \st r \in R}.
    \end{equation*}
    Similarly, $Ra = \set{ra \st r \in R}$.
\end{negg}

\begin{nlemma}
    Let $R$ be a ring and $a \in R$. Then, $aR$ is a left ideal of $R$ and $Ra$ is a right ideal of $R$.
\end{nlemma}

\begin{ndfn}
    Let $R$ be a commutative ring. An ideal $I$ of $R$ is called \keyword{principal} if there is some $a \in R$ such that $I = Ra = \set{ra \st r \in R}$. (Note that $Ra = aR$ in this case because $R$ is a commutative ring.)
\end{ndfn}

\begin{ndfn}
    Let $R$ be a commutative ring and $a \in R$. The ideal $Ra$ is called the \keyword{ideal generated by $a$}. We write this as $(a) = Ra$. More generally, if $a_1, \dots a_n \in R$ then we write
    \begin{equation*}
        (a_1, \dots a_n) = Ra_1 + \dots + Ra_n
    \end{equation*}
    and call it the ideal generated by $a_1, \dots a_n$.
\end{ndfn}

\begin{remark}
    For ideal $(a)$ is a principal ideal since it has the form $Ra$ by definition. The set $Ra_1 + \dots + Ra_n$ is an ideal because it is a sum of ideals. The ideal $(a_1, \dots a_n)$ may or may not be principal.
\end{remark}

\begin{negg}
    For any non-zero ring $R$, we have $(1) = R$. So, $R$ is a principle ideal of itself, generated by $1 \in R$.
\end{negg}

\section{Prime and Maximal Ideals}
\begin{ndfn}
    Let $R$ be a commutative ring. A prime ideal $I$ of $R$ is a proper ideal where if $a, b \in R$ and $ab \in I$ then either $a \in I$ or $b \in I$.
\end{ndfn}

\begin{negg}
    $2\Z$ is a prime ideal of $\Z$. More generally, a positive integer $n$ is a prime number if and only if $n\Z$ is a prime ideal of $\Z$.
\end{negg}

\begin{ndfn}
    Let $R$ be a non-zero ring and $I$ a proper ideal of $R$. We say that $I$ is maximal in $R$ if for every other ideal $J$ with $I \subseteq J$, either $J = I$ or $J = R$.
\end{ndfn}

\begin{nlemma}
    The only maximal ideal of a field is $\set{0}$.
\end{nlemma}

\section{Euclidean Domain and Principal Ideal Domain}
\begin{ndfn}
    Let $R$ be an integral domain. We say that $R$ is a \keyword{Euclidean domain} (ED) if there is a function $\pd : R - \set{0} \to \N$ such that
    \begin{enumerate}
        \item $\pd(ab) \geq \pd(b)$ for all $a, b \in R - \set{0}$.
        \item For all $a, b \in R$ with $b \neq 0$, there exist $q, r \in R$ such that $a = qb + r$, and, either $r = 0$ or $\pd(r) < \pd(b)$.
    \end{enumerate}
    We call $q$ the quotient and $r$ the remainder obtained on dividing $a$ by $b$.
\end{ndfn}

\begin{ndfn}
    A \keyword{principal ideal domain} (PID) is an integral domain in which every ideal in principal.
\end{ndfn}

\begin{nthm}
    Every Euclidean domain is a principal ideal domain
\end{nthm}

\begin{remark}
    The previous theorem states that if $R$ is a Euclidean domain and $I$ is an ideal of $R$ then $I = Ra$ for some $a \in R$.
\end{remark}

\begin{negg}
    Not every PID is a Euclidean domain.

    Consider $\Z[w]$ with $w = \frac{1}{2}(1 + i\sqrt{19})$. It is easy to check that this is an integral domain. We find that $\Z[w]$ is a PID but not an ED.
\end{negg}

\begin{negg}
    Let $K$ be a field. The rings $\Z$ and $K[X]$ are Euclidean domains. Therefore, they are also PIDs. It is easy to check that $K[X,Y]$ is not a PID.
\end{negg}

\end{document}