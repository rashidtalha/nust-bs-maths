\documentclass[11pt]{penrose}

\usepackage{mathsphystools}
\usepackage{thmstyles}

\setlist[enumerate]{label=(\roman*)}
\geometry{left=26mm, right=26mm}
\fancyhfoffset{0pt}

\title{MATH 421: Group Theory II}
\subtitle{Brief lecture notes}
\author{Rashid M. Talha}
\affiliation{School of Natural Sciences, NUST}
\date{\today}
\begin{document}

\maketitle

\textbf{Disclaimer:} This document most likely contains some errors. Use with caution. I have rephrased or paraphrased the content in many of the sections. Some examples may be missing. The numbering that I have used for sections, definitions, theorems, etc will not match the numbering given in the lectures.

\section{Recap of Group Theory I}
\begin{ndfn}[Group]
    Let $G$ be a non-empty set, and $* : G \times G \to G$ be a \keyword{binary operation}. The pair $(G, *)$ is called a \keyword{group} if it satisfies all of the following
    \begin{enumerate}
        \item $\forall a, b, c \in G$, $(a * b) * c = a * (b * c)$ \hfill(\keyword{Associativity})
        \item $\exists e \in G$ such that $\forall a \in G$, $a * e = e * a = a$ \hfill(\keyword{Identity})
        \item $\forall a \in G$, $\exists a^{-1} \in G$ such that $a * a^{-1} = a^{-1} * a = e$ \hfill(\keyword{Inverse})
    \end{enumerate}
\end{ndfn}

Often we don't explicitly specify the binary operation and simply refer to $G$ as the group. Likewise, $a * b$ is abbreviated to $ab$. When the binary operation behaves like addition, we denote it by $+$, and the use $0$ as the identity element and $-a$ as the inverse of $a \in G$.

\begin{nthm}
    Every group $G$ has a unique identity element. Each $a \in G$ has a unique inverse.
\end{nthm}

\begin{nthm}
    Let $G$ be a group. For all $a, b \in G$, $(a^{-1})^{-1} = a$ and $(ab)^{-1} = b^{-1} a^{-1}$.
\end{nthm}

\begin{ndfn}
    The \keyword{order of a group} $G$, denoted $\abs{G}$ or $O(G)$, is the number of elements in $G$. If $G$ has infinitely many elements then $\abs{G} = \infty$. The \keyword{order of an element} $a \in G$ is the least positive integer $k$ such that $a^k = e$. We denote this as $O(a)$ or $\abs{a}$. If there is no such positive integer then $\abs{a} = \infty$.
\end{ndfn}

\begin{ndfn}[Subgroup]
    A non-empty subset $H \subseteq G$ is called a \keyword{subgroup} of $G$ if it is a group under the same binary operation. We denote this as $H \subgroup G$. If $H \neq G$ then $H$ is called a \keyword{proper subgroup} of $G$. This is sometimes emphasised by writing $H \lneqq G$. A subgroup is called \keyword{trivial} if $H = \set{e}$, and non-trivial otherwise.
\end{ndfn}

\begin{nthm}[Subgroup test]
    Let $G$ be a group. A non-empty subset $H \subseteq G$ is a subgroup of $G$ if and only if for all $a, b \in H$, $ab^{-1} \in H$.
\end{nthm}

\begin{ndfn}[Cyclic group]
    $G$ is called \keyword{cyclic} if there is an element $a \in G$ such that all elements of $G$ can be written as powers of $a$. More precisely, $\forall g \in G$, $\exists m \in \Z$ such that $g = a^m$.

    Such an element $a$ is called a \keyword{generator} of $G$, and we say that $G$ is the group generated by $a$ and denote this as $G = \cyclic{a}$. Cyclic group of order $n$ is sometimes denoted as $C_n$.
\end{ndfn}

Generators are not unique. Indeed if $a \in G$ is a generator then so is $a^{-1}$.

\begin{nthm}
    Let $G$ be a group and take $a \in G$ such that $a^n = e$. The cyclic subgroup $\cyclic{a}$ has the form $\set{e, a, a^2, \dots, a^{n-1}}$.
\end{nthm}

\begin{nthm}
    Every cyclic group is abelian. Every subgroup of a cyclic group is cyclic. And, the order of a cyclic group is equal to the order of its generator.
\end{nthm}

\begin{nthm}
    Let $G = \cyclic{a}$ be a finite cyclic group of order $n$. Then an element $a^k$ is a generator of $G$ if and only if $\gcd(k,n) = 1$.
\end{nthm}

\begin{nthm}[Lagrange]
    Let $G$ be a finite group and $H$ be its subgroup. Then $\abs{H}$ divides $\abs{G}$. Moreover, the number of distinct left (right) cosets of $H$ in $G$ is $\abs{G}/\abs{H}$.
\end{nthm}

\begin{ncor}
    Let $G$ be a finite group and $a \in G$. Then $O(a)$ divides $\abs{G}$, and $a^{|G|} = e$.
\end{ncor}

\begin{ncor}
    Every group of prime order is cyclic.
\end{ncor}

\begin{ndfn}[Cosets]
    Let $H \subgroup G$ and $a \in G$. The subset $aH = \set{ah \st h \in H}$ is called a left \keyword{coset} of $H$ containing $a \in G$. The element $a$ is called a \keyword{representative} of the coset.
\end{ndfn}

We note that any element of $aH$ can act as its representative.

\begin{ndfn}[Index]
    Let $H \subgroup G$. The number of distinct left (right) cosets of $H$ in $G$ is called the index of $H$ in $G$. It is denoted as $[G:H]$.
\end{ndfn}

\begin{nthm}
    Let $H \subgroup G$ and $a, b \in G$.
    \begin{enumerate}
        \item $a \in aH$.
        \item $aH = H$ if and only if $a \in H$.
        \item $aH = bH$ if and only if $a \in bH$ if and only if $a^{-1} b \in H$.
        \item $aH = bH$ or $aH \cap bH = \emptyset$.
        \item $|aH| = |bH|$.
    \end{enumerate}
\end{nthm}

\begin{ndfn}[Normal subgroup]
    Let $H \subgroup G$. If $aH = Ha$ for all $a \in G$ then $H$ is called a normal subgroup of $G$. We write this as $H \nsubgroup G$.
\end{ndfn}

\begin{nthm}
    Let $H \subgroup G$. Then, the following are equivalent:
    \begin{enumerate}
        \item $H \nsubgroup G$.
        \item $\forall x \in H$ and $g \in G$, $g x g^{-1} \in H$.
        \item $\forall g \in G$, $g H g^{-1} = H$.
    \end{enumerate}
\end{nthm}

\begin{nthm}
    Let $H \subgroup G$. If $[H:G] = 2$, then $H \nsubgroup G$.
\end{nthm}

\begin{nthm}
    Let $H \nsubgroup G$. Then, $G/H = \set{gH : g \in G}$ is a group with respect to $(aH)(bH) = (ab)H$, for $a, b \in G$. Here, the identity element is $H$ and $(aH)^{-1} = a^{-1} H$. This is called the quotient group or factor group of $H$ in $G$.
\end{nthm}

\begin{ndfn}[Homomorphism]
    A \keyword{homomorphism} between two groups $G$ and $H$ is a map $\varphi : G \to H$ such that $\varphi(ab) = \varphi(a)\varphi(b)$ for all $a,b \in G$. If $\varphi$ is bijective then it is called an \keyword{isomorphism}. If there is an isomorphism between $G$ and $H$ then we say that they are \keyword{isomorphic} groups and denoted this by $G \isom H$.
\end{ndfn}

\begin{nthm}
    Let $\varphi : G \to H$ be a homomorphism. Then, $\ker\varphi \nsubgroup G$ and $\img\varphi \subgroup H$. Moreover, $\varphi$ is injective iff $\ker\varphi = \set{e}$.
\end{nthm}

\begin{nthm}[First Isomorphism Theorem]
    Let $\varphi : G \to H$ be a surjective homomorphism. Then, $G / \ker\varphi \isom H$.
\end{nthm}

\section{Subgroups generated by Subsets}

\begin{nthm}
    Let $G$ be a group and $S$ be a non-empty subset of $G$. Let
    \begin{equation*}
        H = \set*{ x_{1}^{a_{1}} x_{2}^{a_{2}} \dots x_{n}^{a_{n}} \st n \in \N, x_{i} \in S, a_{i} = \pm 1, \forall i}
    \end{equation*}
    be the set of all finite products of elements of $S$ and the inverses of elements of $S$. Then, $H \subgroup G$ and $S \subseteq H$. This is called the subgroup generated by $S$, and we write $H = \cyclic{S}$.
\end{nthm}
\begin{proof}
    As $S \neq \emptyset$, we have $H \neq \emptyset$. Take any $\alpha, \beta \in H$. Then, we can write
    \begin{equation*}
        \alpha = x_{1}^{a_{1}} x_{2}^{a_{2}} \dots x_{n}^{a_{n}}
        \quad\text{and}\quad
        \beta = y_{1}^{b_{1}} y_{2}^{b_{2}} \dots y_{k}^{b_{k}}
    \end{equation*}
    where $n, k \in \N$, $x_{i}, y_{j} \in S$ and $a_{i}, b_{j} \in \set{\pm 1}$ for all $i, j$. So,
    \begin{equation*}
        \alpha \beta^{-1}
        = \paren*{ x_{1}^{a_{1}} x_{2}^{a_{2}} \dots x_{n}^{a_{n}} } \paren*{ y_{1}^{b_{1}} y_{2}^{b_{2}} \dots y_{k}^{b_{k}} }^{-1}
        = x_{1}^{a_{1}} x_{2}^{a_{2}} \dots x_{n}^{a_{n}} y_{k}^{-b_{k}} \dots y_{2}^{-b_{2}} \dots y_{1}^{-b_{1}}.
    \end{equation*}
    Clearly then, $\alpha\beta^{-1} \in H$ because it is a finite product of elements of $S$ and the inverses of elements of $S$. Moreover, $S \subseteq H$ by definition of $H$.
\end{proof}

\begin{remark}\phantom{}
    \begin{enumerate}
        \item $H = \cyclic{S}$ is the smallest subgroup of $G$ that contains $S$. In other words, if $K \subgroup G$ with $S \subseteq K$ then $H \subseteq K$.
        \item If $S = \set{x_{1}, \dots, x_{n}}$ then we write $\cyclic{S} = \cyclic{x_{1}, \dots, x_{n}}$, instead of $\cyclic{\set{x_{1}, \dots, x_{n}}}$.
        \item If $S = \set{a}$ is a singleton set, then $H = \cyclic{a}$ is a cyclic subgroup.
        \item More than one generating sets may give rise to the same subgroup.
        \item Every group $G$ has at least one generating set, namely $S = G$.
        \item If $G = \cyclic{S}$ and $S$ is finite, then we say that $G$ is finitely generated. If no such finite set exists then we say that $G$ is infinitely generated.
    \end{enumerate}
\end{remark}


\section{Commutator Subgroup}
\begin{ndfn}
    Let $G$ be a group and take $a,b \in G$. We define $\comm{a,b} = aba^{-1}b^{-1}$ as the commutator of $a$ and $b$.
\end{ndfn}

Note that $\comm{a,b}^{-1} = \paren*{aba^{-1}b^{-1}}^{-1} = bab^{-1}a^{-1} = \comm{b,a}$.

\begin{ndfn}
    Let $G$ be a group. Consider the set $S = \set*{ \comm{a,b} \st a,b \in G }$. The subgroup generated by $S$ is called the commutator subgroup of the derived subgroup of $G$. We denote this as $G'$ or $\comm{G,G}$.
\end{ndfn}

By definition, a general element of $\cyclic{S}$ is a finite product of commutators and their inverses. However, since the inverse of a commutator is again a commutator, we get that a general element of $\cyclic{S}$ is simply a product of commutators;
\begin{equation*}
    x \in \cyclic{S} = \comm{G,G}
    \implies
    x = \comm{a_{1}, b_{1}} \dots \comm{a_{k}, b_{k}}
    \quad\text{for}\quad a_{1}, \dots, a_{k}, b_{1}, \dots, b_{k} \in G.
\end{equation*}

\begin{nthm}
    A group $G$ is abelian if and only if $G'$ is trivial.
\end{nthm}
\begin{proof}
    If $G'$ is trivial, then for all $a, b \in G$, $\comm{a,b} = 1 \implies aba^{-1}b^{-1} = 1 \implies ab = ba$. Therefore, $G$ is abelian.

    Conversely, if $G$ is abelian, then for all $a,b \in G$, $\comm{a,b} = aba^{-1}b^{-1} = aa^{-1}bb^{-1} = 1$. Therefore, $G' = \cyclic{1} = \set{1}$.
\end{proof}

\begin{negg}
    It is easy to check that $V_{4}' = \set{e}$ and $D_{3}' = \set{1, a, a^{2}}$.
\end{negg}

\begin{nthm}
    Let $G'$ be the commutator subgroup of a group $G$. Then,
    \begin{enumerate}
        \item $G' \nsubgroup G$ (i.e. $G'$ is normal in $G$).
        \item $G / G'$ is abelian.
        \item $G'$ is the smallest subgroup of $G$ such that $G/G'$ is abelian; more precisely, if $G / K$ is abelian for some $K \nsubgroup G$, then $G' \subseteq K$.
    \end{enumerate}
\end{nthm}
\begin{proof}
    \phantom{}
    \begin{enumerate}
        \item We need to show that for all $g \in G$ and $x \in G'$, $g x g^{-1} \in G'$.

        Take any $g \in G$ and $x \in G' \subseteq G$. Then $g x g^{-1} x^{-1} = \comm{g, x} \in G'$ because it is a commutator. Now, by closure in $G'$, we have
        $\paren*{ g x g^{-1} x^{-1} } x \in G' \implies g x g^{-1} \in G'$. As a result, $G' \nsubgroup G$.

        \item Take any $a, b \in G$. Then, $(ba)^{-1} (ab) \in G'$ because $(ba)^{-1} (ab) = a^{-1} b^{-1} a b = [a^{-1}, b^{-1}]$ is a commutator of $a^{-1}, b^{-1} \in G$. And, $(ba)^{-1} (ab) \in G' \iff ab G' = ba G'$.

        So, $G/G'$ is abelian because we obtain $(aG')(bG') = ab G' = ba G' = (bG')(aG')$.

        \item Suppose $G/K$ is abelian for some $K \nsubgroup G$. This means, $\forall a,b \in G$,
        \begin{equation*}
            abK = (aK)(bK) = (bK)(aK) = baK.
        \end{equation*}
        Now, $abK = baK \iff (ba)^{-1}(ab) \in K$. And, $(ba)^{-1} (ab) = a^{-1} b^{-1} a b = [a^{-1}, b^{-1}]$. Therefore, $K$ contains all the commutators of $G$. Thus, $G' \subseteq K$.
        \qedhere
    \end{enumerate}
\end{proof}

\section{Direct Product of Groups}

\begin{ndfn}
    Let $H, K \subgroup G$. Define $HK = \set{hk \st h \in H, k \in K}$ .
\end{ndfn}

In general, $HK$ may or may not be a subgroup of $G$.

\begin{nthm}
    Let $H, K \subgroup G$. Then, $HK$ is a subgroup of $G$ if and only if $HK = KH$.
\end{nthm}
\begin{proof}
    TBC
\end{proof}

\begin{nthm}\label{thm:normal-product-subgroup}
    Let $H \nsubgroup G$ and $K \subgroup G$. Then, $HK \subgroup G$.
\end{nthm}
\begin{proof}
    Since $H \nsubgroup G$, we have $kH = Hk$, for all $k \in K \subseteq G$. And,
    \begin{equation*}
        kH = Hk \implies khk^{-1} = h \implies kh = hk
    \end{equation*}
    for all $k \in K$ and $h \in H$. Therefore, $HK = \set{hk \st h \in H, k \in K} = \set{kh \st h \in H, k \in K} = KH$.
    Thus, $HK \subgroup G$.
\end{proof}

\begin{nthm}
    If $H, K$ are two finite subgroups of a group $G$, then
    \begin{equation*}
        \abs{H} = \frac{\abs{H} \abs{K}}{\abs{H \cap K}}.
    \end{equation*}
\end{nthm}
\begin{proof}
    TBC
\end{proof}

\begin{ndfn}[External direct product]
    The \keyword{external direct product} of a collection of groups $G_{1}, \dots, G_{n}$ is the set $G_{1} \oplus \dots \oplus G_{n} = \set{ (g_{1}, \dots, g_{n}) \st g_{i} \in G_{i} }$.
\end{ndfn}

\begin{nthm}
    The external direct product $G_{1} \oplus \dots \oplus G_{n}$ forms a group under the binary operation
    \begin{equation*}
        (g_{1}, \dots, g_{n}) (g'_{1}, \dots, g'_{n}) = (g_{1}g'_{1}, \dots, g_{n}g'_{n})
    \end{equation*}
    where $(g_{1}, \dots, g_{n}), (g'_{1}, \dots, g'_{n}) \in G_{1} \oplus \dots \oplus G_{n}$.
\end{nthm}
\begin{proof}
    Easy exercise. We simply check all the group axioms.
\end{proof}

\begin{remark}
    It is easy to see that if $\abs{G_{i}} = r_{i}$, then $\abs{ G_{1} \oplus \dots \oplus G_{n} } = r_{1} r_{2} \dots r_{n}$.
\end{remark}

\begin{nthm}
    Let $G = G_{1} \oplus \dots \oplus G_{n}$ be a finite product of group and $g = (g_{1}, \dots, g_{n}) \in G$. Then, $\abs{g} = \lcm \set{\abs{g_{1}}, \dots, \abs{g_{n}}}$.
\end{nthm}
\begin{proof}
    Take $(g_{1}, \dots, g_{n}) \in G_{1} \oplus \dots \oplus G_{n}$. Let $s = \lcm \set{\abs{g_{1}}, \dots, \abs{g_{n}}}$. Then, $s$ divides $\abs{g_{k}}$ for all $k$. Therefore,
    \begin{equation*}
        (g_{1}, \dots, g_{n})^{s}
        = (g_{1}^{s}, \dots, g_{n}^{s})
        = (e_{1}, \dots, e_{n})
    \end{equation*}
    As a result, $t = \abs{(g_{1}, \dots, g_{n})}$ divides $s$. In particular, $t \leq s$.

    Also, by construction,
    \begin{equation*}
        (g_{1}^{t}, \dots, g_{n}^{t})
        = (g_{1}, \dots, g_{n})^{t}
        = (e_{1}, \dots, e_{n}).
    \end{equation*}
    So, $g_{k}^{t} = e_{k}$ for all $k$. Consequently, $\abs{e_{k}}$ divides $t$ for all $k$, and we obtain that $t$ divides $\lcm \set{\abs{g_{1}}, \dots, \abs{g_{n}}}$.
\end{proof}

\begin{nthm}
    Let $G, H$ be finite cyclic groups. Then, $G \oplus H$ is cyclic if and only if $\abs{G}$ and $\abs{H}$ are relatively prime.
\end{nthm}
\begin{proof}
    Let $\abs{G} = m$ and $\abs{H} = n$. So, $\abs{G \oplus H} = mn$.

    Suppose $G \oplus H$ is cyclic with the generator $(g,h)$. Then, $O((g,h)) = mn$. However, $O((g,h)) = \lcm\set{O(g), O(h)}$. Now, $O(g) \mid m$ and $O(h) \mid n$ by Lagrange's theorem. This implies $\lcm\set{O(g), O(h)} \mid \lcm\set{m,n}$.

    So, $mn = \lcm\set{O(g), O(h)} \mid \lcm\set{m,n}$. Thus, $mn \leq \lcm\set{m,n}$. However, $\lcm\set{m,n} \mid mn$. Therefore, $\lcm\set{m,n} = mn$. In other words, $\gcd\set{m,n} = 1$.
\end{proof}

\begin{ncor}
    An external direct product $G_{1} \oplus \dots \oplus G_{n}$ of finite cyclic groups is cyclic if and only if $\abs{G_{i}}$ and $\abs{G_{j}}$ are relatively prime whenever $i \neq j$.
\end{ncor}

\begin{ncor}
    Let $m = n_{1} n_{2} \dots n_{k}$. Then, $\Z_{m}$ is isomorphic to $\Z_{n_{1}} \oplus \dots \oplus \Z_{n^{k}}$ if and only if $n_{i}$ and $n_{j}$ are relatively prime whenever $i \neq j$.
\end{ncor}

\begin{nlemma}
    The group $G \oplus H$ is abelian if and only if and $G$ and $H$ are abelian.
\end{nlemma}
\begin{proof}
    Suppose $G$ and $H$ are abelian. Then, for any $(g,h), (g',h') \in G \oplus H$ we have
    \begin{equation*}
        (g,h) \oplus (g',h')
        = (gg', hh')
        = (g'g, h'h)
        = (g',h') \oplus (g,h).
    \end{equation*}
    Therefore, $G \oplus H$ is abelian.

    Conversely, suppose $G \oplus H$ is abelian. Then, for all $g, g' \in G$ and $h, h' \in H$ we have
    \begin{equation*}
        (gg', hh')
        = (g,h) \oplus (g',h')
        = (g',h') \oplus (g,h)
        = (g'g, h'h).
    \end{equation*}
    Therefore, $gg' = g'g$ and $hh' = h'h$. So, both $G$ and $H$ are abelian.
\end{proof}

\begin{ndfn}[Internal direct product]
    We say that a group $G$ is the internal direct product of $H$ and $K$ and write $G = H \times K$ if $H$ and $K$ are normal subgroups of $G$ and $H \cap K = \set{e}$ and $G = HK$.
\end{ndfn}

\begin{ndfn}[Internal direct product]
    Let $H_{1}, H_{2}, \dots, H_{n}$ be a finite collection of normal subgroups of $G$. We say that $G$ is the internal direct product of $H_{1}, H_{2}, \dots, H_{n}$ and write $G = H_{1} \times H_{2} \times \dots \times H_{n}$ if
    \begin{enumerate}
        \item $G = H_{1} H_{2} \dots H_{n} = {h_{1} h_{2} \dots h_{n} \st h_{i} \in H_{i}}$.
        \item $(H_{1} H_{2} \dots H_{i}) \cap H_{i+1} = \set{e}$, for $i = 1,2,\dots,n-1$
    \end{enumerate}
\end{ndfn}

% MISSING: USEFUL EXAMPLE OF AN INTERNAL DIRECT PRODUCT

\begin{nthm}
    If a group $G$ is the internal direct product of subgroups $H_{1}, \dots, H_{n}$, then $G$ is isomorphic to the external direct product of $H_{1}, \dots, H_{n}$. That is,
    \begin{equation*}
        H_{1} \times \dots \times H_{n} \isom H_{1} \oplus \dots \oplus H_{n}.
    \end{equation*}
\end{nthm}
\begin{proof}
    STUFF

    Consider the isomorphism $\varphi : H_{1} \times \dots \times H_{n} \to H_{1} \oplus \dots \oplus H_{n}$ by        

    TBC
\end{proof}

\begin{negg}
    The group $S^{3}$ cannot be written as the (internal) direct product of two proper subgroups.

    For a contradiction, suppose that we have two such proper subgroups $H, K \subgroup S^{3}$. Then, $\abs{H} = 2$ and $\abs{K} = 3$ or vice versa. However, every group of order $2$ and $3$ is abelian. So the product $H \times K \isom H \oplus K$ will also be abelian, while $S_{3}$ is not abelian. \eggqed
\end{negg}

\section{Fundamental Theorem of Finite Abelian Groups}
In this section, we shall classify the finite abelian groups based on their order.

We already know that every cyclic group of order $n$ is isomorphic to $\Z_{n}$. The fundamental theorem of finite abelian groups states that every such group can be written as a direct product of cyclic groups whose order is a power of a prime. In other words, every finite abelian group $G$ is isomorphic to a group of the form
\begin{equation*}
    \Z_{p_{1}^{n_{1}}} \oplus \dots \oplus \Z_{p_{k}^{n_{k}}}
\end{equation*}
where $p_{i}$ and $n_{i}$ are uniquely determined by $G$, and $p_{i}$ (primes) need not be distinct.

\begin{ndfn}
    Let $n \geq 1$ be an integer. A way of writing $n$ as a sum of positive integers is called a partition of $n$.
\end{ndfn}

Two partitions that differ only in the order of the summands are considered as the same partitions. For example, $3+1$ and $1+3$ are the same partitions. Since the order of summand does not matter therefore we write the summands in the descending order.

Note that the relation $\sim$ on the set of all groups, defined by $G_{1} \sim G_{2}$ iff $G_{1} \isom G_{2}$ is an equivalence relation. The class $[G] = \set{H : H \isom G}$ is called the isomorphism class of $G$.

\begin{nthm}[Fundamental Theorem]
    Let $G$ be a finite abelian group of order $n > 1$. Consider the unique factorisation of $n$ into distinct prime powers $n = p_{1}^{\alpha_{1}} p_{2}^{\alpha_{2}} \dots p_{k}^{\alpha_{k}}$. Then,
    \begin{equation*}
        G \isom A_{1} \oplus \dots \oplus A_{k}
    \end{equation*}
    where $\abs{A_{i}} = p_{i}^{\alpha^{i}}$. For each $A \in \set{A_{1}, \dots, A_{k}}$ with $\abs{A} = p^{\alpha}$,
    \begin{equation*}
        A \isom \Z_{p^{\beta_{1}}} \oplus \dots \oplus \Z_{p^{\beta_{t}}}
    \end{equation*}
    where $\beta_{1} \geq \beta_{2} \geq \dots \geq \beta_{t}$ and $\beta_{1} + \beta_{2} + \dots + \beta_{t} = \alpha$.
\end{nthm}
(Proof of this theorem is non-examinable.)

\begin{negg}
    Find all the non-isomorphic abelian groups of order $12 = 3^{4}$.
    \begin{center}
        \begin{tabularx}{0.66\textwidth}{p{8mm}XX}
            \toprule
            $p^{\beta}$ & Partitions of $\beta$ & Abelian Groups\\
            \midrule
            $3^{4}$ & $4$ & $\Z_{3^{4}}$ \\
                    & $3+1$ & $\Z_{3^{3}} \oplus \Z_{3}$ \\
                    & $2+2$ & $\Z_{3^{2}} \oplus \Z_{3^{2}}$ \\
                    & $2+1+1$ & $\Z_{3^{2}} \oplus \Z_{3} \oplus \Z_{3}$ \\
                    & $1+1+1+1$ & $\Z_{3} \oplus \Z_{3} \oplus \Z_{3} \oplus \Z_{3}$ \\
            \bottomrule
        \end{tabularx}
    \end{center}
    So, up to isomorphism, there are are exactly $5$ abelian groups of order $12$:
    \begin{equation*}
        \Z_{3^{4}}, \Z_{3^{3}} \oplus \Z_{3}, \Z_{3^{2}} \oplus \Z_{3^{2}}, \Z_{3^{2}} \oplus \Z_{3} \oplus \Z_{3}, \Z_{3} \oplus \Z_{3} \oplus \Z_{3} \oplus \Z_{3}.
    \end{equation*}
\end{negg}

\begin{negg}
    Find all the non-isomorphic abelian groups of order $60 = 2^{2} \times 3 \times 5$.
    \begin{center}
        \begin{tabularx}{0.66\textwidth}{p{8mm}XX}
            \toprule
            $p^{\beta}$ & Partitions of $\beta$ & Abelian Groups\\
            \midrule
            $2^{2}$ & $2$, $1+1$ & $\Z_{2^{2}}$, $\Z_{2} \oplus \Z_{2}$ \\
            $3^{1}$ & $1$        & $\Z_{3}$ \\
            $5^{1}$ & $1$        & $\Z_{5}$ \\
            \bottomrule
        \end{tabularx}
    \end{center}
    Now, we consider all possible combinations of these constituent groups (taking one group from each row). So, up to isomorphism, there are are exactly $2$ abelian groups of order $12$:
    \begin{equation*}
        \Z_{2^{2}} \oplus \Z_{3} \oplus \Z_{5}
        \quad\text{and}\quad 
        \Z_{2} \oplus \Z_{2} \oplus \Z_{3} \oplus \Z_{5}.
    \end{equation*}
\end{negg}

In general, if consider the unique prime factorisation $n = p_{1}^{\alpha_{1}} p_{2}^{\alpha_{2}} \dots p_{k}^{\alpha_{k}}$ and let $q_{i}$ be the number of partitions of $\alpha_{i}$ then the number of non-isomorphic groups of order $n$ is equal to $q_{1} q_{2} \dots q_{k}$.

\begin{negg}
    Take $n = 1800 = 2^{3} 3^{2} 5^{2}$. Then, there are $3$ partitions of $3$ and $2$ partitions of $2$. So, there are $3 \times 2 \times 2 = 12$ non-isomorphic abelian groups of order $1800$.
\end{negg}


\section{Isomorphism Theorems}

Throughout this section, let $G$ be a group.

\begin{nlemma}
    Let $H \subgroup G$ and $N \nsubgroup G$. Then,
    \begin{enumerate}
        \item $HN$ is a subgroup of $G$.
        \item $N$ is a normal subgroup of $HN$.
        \item $H \cap N$ is a normal subgroup of $H$.
    \end{enumerate}
\end{nlemma}
\begin{proof}\phantom{}
    \begin{enumerate}
        \item This was proved in theorem (\ref{thm:normal-product-subgroup}).

        \item Firstly, note that $N, HN \subgroup G$ and $N \subseteq HN$ means that $N \subgroup HN$.

        Now, $N \subgroup HN \subgroup G$ along with $N \nsubgroup G$ means that $N \nsubgroup HN$.

        \item Firstly, note that $H, N \subgroup G \implies (H \cap N) \subgroup G$. Also, $H, (H \cap N) \subgroup G$ with $(H \cap N) \subseteq H$ means that $(H \cap N) \subgroup H$.

        Now, take any $a \in (H \cap N)$ and $h \in H$. Then,
        \begin{align*}
            a \in (H \cap N) \implies a \in N &\implies h a h^{-1} \in N \qquad\text{(as $N \nsubgroup G$)},\\
            a \in (H \cap N) \implies a \in N &\implies h a h^{-1} \in H \qquad\text{(by closure in  $H$)}.
        \end{align*}
        So, $h a h^{-1} \in (H \cap N)$, for all $h \in H$ and $a \in (H \cap N)$. Therefore, $H \cap N \nsubgroup N$.
        \qedhere
    \end{enumerate}
\end{proof}

\begin{nthm}[Second Isomorphism Theorem]
    Let $H \subgroup G$ and $N \nsubgroup G$. Then,
    \begin{equation*}
        H / (H \cap N) \isom HN / N.
    \end{equation*}
\end{nthm}
\begin{proof}
    Firstly, note that elements of $HN/N$ have the form $hN$ because $(hn)N = hN$ as $n \in N$.

    Consider the map $\varphi : H \to HN/N$ defined as $\varphi(h) = hN$. This is clearly surjective because the pre-image of $hN \in HN/N$ is $h \in H$. Moreover, it is a homomorphism because
    \begin{equation*}
        \varphi( h_{1} h_{2} )
        = (h_{1} h_{2}) N
        = (h_{1}N) (h_{2}N)
        = \varphi(h_{1}) \varphi(h_{2}).
    \end{equation*}
    Therefore, by the 1st isomorphism theorem, $H / \ker\varphi \isom HN / N$.

    However, $\ker\varphi = \set{h \in H \st \varphi(h) = N} = \set{h \in H \st hN = N} = \set{h \in H \st h \in N} = H \cap N$. Therefore,
    \begin{equation*}
        H / (H \cap N) \isom HN / N. \qedhere
    \end{equation*}
\end{proof}

\begin{nlemma}
    Let $K, N \nsubgroup G$ with $K \subseteq N$. Then, $N/K \nsubgroup G/K$.
\end{nlemma}
\begin{proof}
    Note that $N/K$ is a non-empty subset of $G/K$ because $1 \in N \implies 1K = K \in N/K$. Take any $n_{1}K, n_{2}K \in N/K$, then
    \begin{equation*}
        (n_{1}K) (n_{2}K)^{-1}
        = (n_{1}K) (n_{2}^{-1} K)
        = n_{1} n_{2}^{-1} K
        \in N/K
    \end{equation*}
    since $n_{1} n_{2}^{-1} \in N$. Therefore, $N/K \subgroup G/K$.

    Next, for every $nK \in N/K$ and $gK \in G/K$, we have
    \begin{equation*}
        (gK) (nK) (gK)^{-1}
        = gK nK g^{-1}K
        = g n g^{-1 } K
        \in N/K
    \end{equation*}
    since $N \nsubgroup G \implies g n g^{-1 } \in N$. Thus, $N/K \nsubgroup G/K$.
\end{proof}

\begin{nthm}[Third Isomorphism Theorem]
    Let $K, N \nsubgroup G$ with $K \subseteq N$. Then,
    \begin{equation*}
        (G/K) / (N/K) \isom G/N.
    \end{equation*}
\end{nthm}
\begin{proof}
    Consider the map $\varphi : G/K \to G/N$ with $\varphi(gK) = gN$.

    This is well-defined because
    \begin{equation*}
        g_{1}K = g_{2}K
        \implies
        g_{1}^{-1}g_{2} \in K \subseteq N
        \implies
        g_{1}^{-1}g_{2} \in N
        \implies
        g_{1}N = g_{2}N
        \implies
        \varphi(g_{1}K) = \varphi(g_{2}K).
    \end{equation*}
    Also, $\varphi$ is clearly surjective, since $gK$ is the preimage of $gN$. Next, note that
    \begin{equation*}
        \varphi( (g_{1}K)(g_{2}K) )
        = \varphi( g_{1} g_{2} K )
        = g_{1} g_{2} N
        = (g_{1}N) (g_{2}N)
        = \varphi(g_{1}K) \varphi(g_{2}K).
    \end{equation*}
    So, $\varphi$ is a surjective homomorphism. Moreover,
    \begin{equation*}
        \ker\varphi = \set{gK \st \varphi(gK) = N} = \set{gK \st gN = N} = \set{gK \st g \in N} = N/K.
    \end{equation*}
    Therefore, by the first isomorphism theorem, $(G/K) / (N/K) \isom G/N$.
\end{proof}

\section{Centraliser and Normaliser}

Throughout this section let $A$ be a non-empty subset of $G$.

\begin{ndfn}
    Let $A \subseteq G$. The centraliser of $A$ in $G$ is the set
    \begin{equation*}
        C_{G}(A) = \set{ g \in G \st g a g^{-1} = a, \forall a \in A }.
    \end{equation*}
\end{ndfn}

\begin{nthm}
    Let $A \subseteq G$. Then, $C_{G}(A) \subgroup G$.
\end{nthm}
\begin{proof}
    The centraliser $C_{G}(A) \subseteq G$ is non-empty because $1 \in C_{G}(A)$. Take any $x, y \in C_{G}(A)$. Then, $xax^{-1} = a$ and $yay^{-1} = a$, for all $a \in A$. Then, $yay^{-1} = a \implies ya = ay \implies a = y^{-1}ay$.

    So, $(xy^{-1}) a (xy^{-1})^{-1} = x y^{-1} a y x^{-1} = x a x^{-1} = a$. Therefore, $xy^{-1} \in C_{G}(A)$, and we conclude that $C_{G}(A) \subgroup G$.
\end{proof}

\begin{remark}\phantom{}
    \begin{enumerate}
        \item If $G$ is abelian, then $C_{G}(A) = G$ for all non-empty $A \subseteq G$.
        \item When $A = \set{a}$ is a singleton, we write $C_{G}(a)$ instead of $C_{G}(\set{a})$.
        \item For every $a \in G$, $Z(G) \subgroup C_{G}(a)$.
        \item The centraliser of $G$ in $G$ is the centre of $G$; i.e. $Z(G) = C_{G}(G)$.
    \end{enumerate}
\end{remark}

\begin{ndfn}
    Let $A \subseteq G$. The normaliser of $A$ in $G$ is the set
    \begin{equation*}
        N_{G}(A) = \set{ g \in G \st g A g^{-1} = A}.
    \end{equation*}
    Here, $g A g^{-1} = \set{g a g^{-1} \st a \in A}$.
\end{ndfn}

\begin{nprop}
    Let $A \subseteq G$. Then, $N_{G}(A) \subgroup G$.
\end{nprop}
\begin{proof}
    The normaliser $N_{G}(A) \subseteq G$ is non-empty because $1 \in N_{G}(A)$.

    Take any $x, y \in N_{G}(A)$. Then, $xAx^{-1} = A$ and $yAy^{-1} = A$. Since,
    \begin{equation*}
        yAy^{-1} = A
        \implies
        yA = Ay
        \implies
        A = y^{-1}Ay.
    \end{equation*}
    So, $(xy^{-1}) A (xy^{-1})^{-1} = x y^{-1} A y x^{-1} = x A x^{-1} = A$. Therefore, $xy^{-1} \in N_{G}(A)$, and we conclude that $N_{G}(A) \subgroup G$.
\end{proof}

\begin{remark}
    \phantom{}
    \begin{enumerate}
        \item We have $C_{G}(A) \subgroup N_{G}(A)$, for every non-empty $A \subseteq G$.
        \item If $G$ is abelian, then $N_{G}(A) = G$, for every non-empty $A \subseteq G$.
    \end{enumerate}
\end{remark}

\begin{nprop}
    Let $A \subseteq G$. Then, $C_{G}(A) \nsubgroup N_{G}(A)$.
\end{nprop}
\begin{proof}
    We already have $C_{G}(A) \subgroup N_{G}(A)$.

    We need to show that for all $c \in C_{G}(A)$ and $n \in N_{G}(A)$, the element $n c n^{-1} \in C_{G}(A)$. So, let $c \in C_{G}(A)$ and $n \in N_{G}(A)$. Then, for every $a \in A$,
    \begin{equation*}
        (ncn^{-1}) a (ncn^{-1})^{-1}
        = nc(n^{-1} a n)c^{-1}n^{-1}
        = n(cbc^{-1})n^{-1}
        = nbn^{-1}
        = (nn^{-1}) a (nn^{-1})
        = a,
    \end{equation*}
    where $b = n^{-1} a n \in A$. Thus, $n c n^{-1} \in C_{G}(A)$, and we conclude that $C_{G}(A) \nsubgroup N_{G}(A)$.
\end{proof}

\begin{nprop}
    Let $H \subgroup G$. Then,
    \begin{enumerate}
        \item $N_{G}(H)$ is the largest subgroup of $G$ in which $H$ is normal; more precisely, $H$ is normal in $N_{G}(H)$, and if $K \subgroup G$ with $H \nsubgroup K$ then $K \subseteq N_{G}(H)$.
        \item $H \nsubgroup G$ if and only if $N_{G}(H) = G$.
    \end{enumerate}
\end{nprop}
\begin{proof}\phantom{}
    \begin{enumerate}
        \item $H \subgroup G$, $N_{G}(H) \subgroup G$ and $H \subseteq N_{G}(H)$ collectively imply that $H \subgroup N_{G}(H)$.

        Now, take any $x \in N_{G}(H)$. Then, by definition, $xHx^{-1} = H$. So, $H$ is normal in $N_{G}(H)$.

        \item If $H \nsubgroup G$, then $\forall g \in G$, $gHg^{-1} = H$. Therefore, $G \subseteq N_{G}(H) \subseteq G$. So, $N_{G}(H) = G$.

        Conversely, if $N_{G}(H) = G$, then $gHg^{-1} = H$ for all $g \in G$. So, $H \nsubgroup G$ ($H \subgroup G$ already).
    \end{enumerate}
\end{proof}

\begin{nex}
    If $H, K \subgroup G$ and $H \subgroup N_{G}(K)$. Then, $HK \subgroup G$. (Hint: it is enough to show that $HK = KH$.)
\end{nex}


\section{Conjugacy Relation \& the Class Equation}
\begin{ndfn}
    An element $b \in G$ is called conjugate of $a \in G$ if for some $g \in G$, we have $b = g a g^{-1}$.
\end{ndfn}

\begin{nthm}
    Given $a,b \in G$, define the relation $\sim$ on $G$ such that $a \sim b$ iff $b = g a g^{-1}$ for some $g \in G$. Then, this is an equivalence relation, called the conjugacy relation.
\end{nthm}
\begin{proof}
    Easy exercise. We simply show that $\sim$ is reflexive, symmetric and transitive.

    Take $a, b, c \in G$. Then, $a = e a e^{-1}$, so $a \sim a$. Also,
    \begin{align*}
        a \sim b
        \implies
        a = g b g^{-1}
        \implies
        g^{-1} a g = b
        \implies
        (g^{-1}) a (g^{-1})^{-1} = b
        \implies
        b \sim a,
    \end{align*}
    where $g \in G$. Moreover, suppose $a \sim b$ and $b \sim c$. This means, $a = g b g^{-1}$ and $b = h c h^{-1}$ for some $g, h \in G$. Then, $a = g b g^{-1} = g h c h^{-1} g^{-1} = (gh) c (gh)^{-1}$ because $gh \in G$. So, $a \sim c$.

    Overall, $\sim$ is an equivalence relation.
\end{proof}

\begin{ndfn}
    Let $\sim$ be the conjugacy relation on $G$. For every $a \in G$, the equivalence class $[a] = \set{ g a g^{-1} \st g \in G }$ is called the conjugacy class of $a$ in $G$.
\end{ndfn}

Note that in every group the identity element $1$ is only conjugate to itself. Therefore, $[1] = \set{1}$. Moreover, as equivalence classes, the conjugacy classes partition the group.

\begin{nthm}
    Let $G$ be a group and $a \in G$ be any element. Then, $\abs{[a]} = [G : C_{G}(a)]$.
\end{nthm}
\begin{proof}
    Let $H = C_{G}(a) \subgroup G$ and $\mathcal{H} = \set{xH \st x \in G}$ be the set of all left cosets of $H$ in $G$. Consider the map $\varphi : \mathcal{H} \to [a]$ defined as $\varphi(xH) = x a x^{-1} \in [a]$.

    We shall show that $\varphi$ is a well-defined bijective map.

    To this end, suppose $xH = yH$, then $y^{-1} x \in H = C_{G}(a)$. So, $(y^{-1} x) a (y^{-1} x)^{-1} = a$. And,
    \begin{align*}
        (y^{-1} x) a (y^{-1} x)^{-1} = a
        \implies
        (y^{-1} x) a = a (y^{-1} x)
        \implies
        xa = yay^{-1} x
        \implies
        xax^{-1} = yay^{-1}
    \end{align*}
    Therefore, $\varphi(xH) = \varphi(yH)$. So, $\varphi$ is well-defined.

    Now, $\varphi$ is clearly surjective because the pre-image of any $x a x^{-1} \in [a]$ is simply $xH \in \mathcal{H}$. Also, $\varphi$ is also injective because
    \begin{align*}
        \varphi(xH) = \varphi(yH)
        \implies
        xax^{-1} = yay^{-1}
        \implies
        xa = yay^{-1} x
        &\implies
        y^{-1} x a = a y^{-1} x\\
        &\implies
        (y^{-1} x) a (y^{-1} x)^{-1} = a\\
        &\implies
        y^{-1} x \in C_{G}(a) = H\\
        &\implies
        xH = yH.
    \end{align*}
    Overall, $\varphi$ is a well-defined bijective map. Therefore, $\abs{ \mathcal{H} } = \abs{ [a] }$ as sets.
\end{proof}

\begin{nthm}
    Let $g_{1}, \dots, g_{n} \in G$ be representatives of distinct conjugacy classes. Then,
    \begin{equation*}
        \abs{G} = \sum_{i=1}^{n} [G : C_{G}(g_{i})].
    \end{equation*}
\end{nthm}
\begin{proof}
    The set of all the equivalence classes partitions the group $G$. So,
    \begin{equation*}
        G = \bigcup_{i = 1}^{n} [g_{i}]
        \implies \abs{G} = \sum_{i = 1}^{n} \abs{[g_{i}]} = \sum_{i = 1}^{n} [G : C_{G}(a)]
        \qedhere
    \end{equation*}
\end{proof}

\begin{nlemma}
    Let $G$ be a group and $a \in G$. Then, $[a] = \set{a}$ if and only if $a \in Z(G)$.
\end{nlemma}
\begin{proof}
    If $a \in Z(G)$, then $g a g^{-1} = a, \forall g \in G$. So, $[a] = \set{g a g^{-1} \st g \in G} = \set{a \st g \in G} = \set{a}$.

    Conversely, if $[a] = \set{a}$, then $\set{g a g^{-1} \st g \in G} = \set{a}$. So, $g a g^{-1} = a, \forall g \in G$. Equivalently, $ga = ag$, for all $g \in G$. Therefore, $a \in Z(G)$.
\end{proof}

\begin{ncor}
    If $a \in Z(G)$, then $[G : C_{G}(a)] = \abs{ [a] } = 1$.
\end{ncor}
\begin{proof}
    If $a \in Z(G)$, then $[a] = \set{a}$. So, $[G : C_{G}(a)] = \abs{[a]} = \abs{\set{a}} = 1$.
\end{proof}

\begin{nthm}[Class Equation]
    Let $G$ be a finite group and $g_{1}, \dots, g_{r}$ be representatives of the non-central conjugacy classes of $G$. Then,
    \begin{equation*}
        \abs{G} = \abs{Z(G)} + \sum_{i=1}^{r} [G : C_{G}(g_{i})].
    \end{equation*}
\end{nthm}
\begin{proof}
    TBC
\end{proof}

\begin{nthm}
    Let $G$ be a group, $p$ be a prime. If $\abs{G} = p^{n}$ for $n \geq 1$, then $Z(G) \neq \set{e}$.
\end{nthm}
\begin{proof}
    If $G$ is abelian, then $Z(G) = G \neq \set{e}$. So, we are done.

    Let $G$ be non-abelian. Let $g_{1}, \dots, g_{r}$ be the representatives of the non-central conjugacy classes of $G$. Since, $C_{G}(g_{i}) \subgroup G$ and $\abs{G} = p^{n}$, by Lagrange's theorem $\abs{C_{G}(g_{i})} = p^{k}$ for some $k < n$. Thus,
    \begin{equation*}
        [G : C_{G}(g_{i})] = \frac{\abs{G}}{\abs{C_{G}(g_{i})}} = p^{n-k}.
    \end{equation*}
    So, $p$ divides $[G : C_{G}(g_{i})]$ for all $i$. Therefore, $p$ divides $\sum_{i=1}^{r} [G : C_{G}(g_{i})]$.

    By the class equation,
    \begin{equation*}
        \abs{G} = \abs{Z(G)} + \sum_{i=1}^{r} \comm{G : C_{G}(g_{i})}
        \implies
        \abs{Z(G)} = \abs{G} - \sum_{i=1}^{r} \comm{G : C_{G}(g_{i})}.
    \end{equation*}
    As $p$ divides both $\abs{G}$ and $\sum_{i=1}^{r} \comm{G : C_{G}(g_{i})}$ we conclude that $p$ divides $\abs{Z(G)}$. Therefore, $\abs{Z(G)} > 1$. Consequently, $Z(G) \neq \set{e}$.
\end{proof}

\begin{ncor}
    Let $G$ be a group and $p$ be a prime. If $\abs{G} = p^{2}$ then $G$ is abelian.
\end{ncor}
\begin{proof}
    Since $\abs{G}$ is of the form $p^{n}$, we have $Z(G) \neq \set{e}$. Moreover, $Z(G) \subgroup G$, so by Lagrange's theorem $\abs{Z(G)} = p$ or $p^{2}$.

    Suppose $\abs{Z(G)} = p$, and pick any $a \in G$ such that $a \notin Z(G)$. Then,
    \begin{equation*}
        Z(G) \lneqq C_{G}(a) \lneqq G
        \implies
        \abs{Z(G)} < \abs{C_{G}(a)} < \abs{G}
        \implies
        p < \abs{C_{G}(a)} < p^{2}.
    \end{equation*}
    This is not possible because $\abs{C_{G}(a)}$ divides $p^{2}$ by Lagrange's theorem and $p$ is prime.

    Therefore, $\abs{Z(G)} = p^{2}$. In other words, $Z(G) = G$. So, $G$ is abelian.
\end{proof}

\section{Sylow's Theorem}
\begin{nthm}[Cauchy]
    Let $G$ be a finite abelian group and $p$ be a prime such that $p$ divides $\abs{G}$. Then, there is an element $x \in G$ with $O(x) = p$.
\end{nthm}
\begin{proof}
    We shall use strong induction on $\abs{G} = n$.

    If $\abs{G} = 1$, then there is nothing to prove. If $\abs{G} = p$, then every non-identity element has order $p$. Assume that the statement is true for all groups with order less than $\abs{G} = n$.

    Take any non-identity element $a \in G$ and consider the subgroup $H = \cyclic{a}$. As $G$ is abelian, $H$ is normal in $G$. Moreover, $a \neq e$ implies $\abs{H} > 1$.

    If $p$ divides $\abs{H}$, then $a^{\abs{H}/p}$ is an element of order $p$ in $G$; this is because $(a^{\abs{H}/p})^{p} = e$, so $O(a^{\abs{H}/p})$ divides $p$, but $O(a^{\abs{H}/p}) \neq 1$ because $O(a) = \abs{H} > \abs{H}/p \implies a^{\abs{H}/p} \neq e$. Therefore, $O(a^{\abs{H}/p}) = p$.

    Otherwise, consider the group $G/H$. This is abelian because $G$ is abelian, and
    \begin{equation*}
        \abs{G/H} = \abs{G} / \abs{H} < \abs{G}
        \quad\text{and}\quad
        \abs{G} = \abs{G/H} \abs{H}.
    \end{equation*}
    Now, $p$ divides $\abs{G}$ but not $\abs{H}$. So, $p$ divides $\abs{G/H}$. Then, by the inductive hypothesis, there exists $yH \in G/H$ with $O(yH) = p$. Now,
    \begin{equation*}
        (yH)^{p} = H
        \implies
        y^{p} H = H
        \implies
        y^{p} \in H
        \implies
        (y^{p})^{\abs{H}} = e
        \implies
        (y^{\abs{H}})^{p} = e.
    \end{equation*}
    Thus, $O(y^{\abs{H}})$ divides $p$. So, $O(y^{\abs{H}}) = 1$ or $p$.

    If $O(y^{\abs{H}}) = 1$, then $y^{\abs{H}} = e$. And,
    \begin{equation*}
        y^{\abs{H}} = e
        \implies
        y^{\abs{H}} H = H
        \implies
        (yH)^{\abs{H}} = H
        \implies
        O(yH) \mid \abs{H}
        \implies
        p \mid \abs{H}.
    \end{equation*}
    This is a contradiction. Therefore, we must have $O(y^{\abs{H}}) = p$ instead, and we are done because $y^{\abs{H}} \in G$ is an element of order $p$ in this case.
\end{proof}

(END OF SYLABUS FOR THE MIDTERM EXAM)

\begin{nthm}[Sylow's 1st Theorem]
    Let $G$ be a finite group and $p$ a prime. If $p^{k}$ divides $\abs{G}$, then $G$ has at least one subgroup of order $p^{k}$.
\end{nthm}
\begin{proof}
    We proceed by induction on $\abs{G}$. If $\abs{G} = 1$, then there is nothing to prove.

    Assume that the statement is true for all groups of order less than $\abs{G}$.

    If $G$ has a proper subgroup $H$ such that $p^{k}$ divides $\abs{H}$, then by our induction hypothesis, $H$ (and therefore $G$) has at least one subgroup of order $p^{k}$ and we are done. Thus, we may assume that $p^{k}$ does not divide the order of any proper subgroup of $G$.

    Now, let $a_{1}, \dots, a_{r}$ be the representatives of the distinct non-central conjugacy classes of $G$.

    Now, for all $i$, $p^{k}$ divides $\abs{G} = [G : C_{G}(a_{i})] \abs{C_{G}(a_{i})}$ and does not divide $\abs{C_{G}(a_{i})}$ (since $C_{G}(a_{i})$ is a proper subgroup of $G$). Therefore, we must have that $p$ must divide $[G : C_{G}(a_{i})]$ for all $a_{i} \notin Z(G)$. Then, it follow from the class equation
    \begin{equation*}
        \abs{G} = \abs{Z(G)} + \sum_{i=1}^{r} \comm{G : C_{G}(a_{i})}
    \end{equation*}
    that $p$ divides $\abs{Z(G)}$.

    Thus, by Cauchy's theorem for abelian groups, there is some $x \in Z(G)$ such that $O(x) = p$. Then, $\cyclic{x} \subgroup Z(G) \nsubgroup G$ implies that $\cyclic{x} \nsubgroup G$.

    Consider $G / \cyclic{x}$. We see that $p^{k-1}$ divides $\abs{G / \cyclic{x}}$. Thus, by the inductive hypothesis, $G / \cyclic{x}$ has a subgroup of order $p^{k-1}$. Denote this subgroup as $H / \cyclic{x}$, where $H \subgroup G$. Finally, note that $\abs{ H / \cyclic{x} } = p^{k-1}$ and $\abs{\cyclic{x}} = p$ imply that $\abs{H} = p^{k}$. This complete the proof.
\end{proof}

\begin{ndfn}[Sylow $p$-subgroup]
    Let $G$ be a finite group and let $p$ be a prime dividing $\abs{G}$. If $p^{k}$ divides $\abs{G}$ and $p^{k+1} $ does not divide $\abs{G}$, then any subgroup of $G$ of order $p^{k}$ is called a Sylow $p$-subgroup.
\end{ndfn}

\begin{ndfn}[Conjugate groups]
    Let $H, K \subgroup G$. We say that $H$ and $K$ are conjugates in $G$, if $H = g K g^{-1}$ for some $g \in G$.
\end{ndfn}

\begin{nprop}
    Let $H, K \subgroup G$ and $a, b \in G$. Define a relation $\sim$ on $G$ by $a \sim b$ if and only if there exists $h \in H$ and $k \in K$ such that $a = hbk$. Then, $\sim$ is an equivalence relation on $G$.
\end{nprop}
\begin{proof}
    Since $e \in H, K$, then we have $a = e a e$, so $a \sim a$.

    Also, $a \sim b$ means that $a = hbk$ for some $h \in H$ and $k \in K$. So, $b = h^{-1} a k^{-1}$ with $h^{-1} \in H$ and $k^{-1} \in K$. Thus, $b \sim a$.

    And, if $a \sim b$ and $b \sim c$, then $a = hbk$ and $b = h'ck'$ for $h, h' \in H$ and $k, k' \in K$. So, $a \sim c$ because $a = h b k = (h h') c (k' k)$ and $hh' \in H$ and $kk' \in K$.
\end{proof}

\begin{remark}
    Since $\sim$ is an equivalence relation on $G$, it divides $G$ into disjoint equivalence classes. The equivalence class for $a$ is given by
    \begin{equation*}
        Cl(a)
        = \set{x \in G \st x \sim a}
        = \set{hak \st h \in H, k \in K}
        \equiv HaK.
    \end{equation*}

    In other words, $G = \bigsqcup_{a} Cl(a) = \bigsqcup_{a} HaK$.
\end{remark}

\begin{ndfn}[Double cosets]
    Let $H, K \subgroup G$ and $a \in G$. Then, $HaK$ is called a double coset of $H$ and $K$ in $G$.
\end{ndfn}

\begin{nlemma}
    Let $G$ be a group and $H$ be a subgroup of $G$. Then, for $g \in G$, $f : H \to g H g^{-1}$ defined by $f(h) = ghg^{-1}$ is a bijection. (It is an isomorphism)
\end{nlemma}
\begin{proof}
    Easy exercise.
\end{proof}

\begin{nex}
    Let $H \subgroup G$. Show that $gHg^{-1} \subgroup G$ for all $g \in G$.
\end{nex}

\end{document}