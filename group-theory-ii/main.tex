\documentclass[11pt]{penrose}

\usepackage{mathsphystools}
\usepackage{thmstyles}

\setlist[enumerate]{label=(\roman*)}
\geometry{left=26mm, right=26mm}
\fancyhfoffset{0pt}

\title{MATH 421: Group Theory II}
\subtitle{Brief lecture notes}
\author{Rashid M. Talha}
\affiliation{School of Natural Sciences, NUST}
\date{\today}
\begin{document}

\maketitle

\textbf{Disclaimer:} This document most likely contains some errors. Use with caution. I have rephrased or paraphrased the content in many of the sections. Some examples may be missing. The numbering that I have used for sections, definitions, theorems, etc will not match the numbering given in the lectures.

Throughout these notes, let $G$ denote a group, unless stated otherwise.

\section{Recap of Group Theory I}
\begin{ndfn}[Group]
    Let $G$ be a non-empty set, and $* : G \times G \to G$ be a \keyword{binary operation}. The pair $(G, *)$ is called a \keyword{group} if it satisfies all of the following
    \begin{enumerate}
        \item $\forall a, b, c \in G$, $(a * b) * c = a * (b * c)$ \hfill(\keyword{Associativity})
        \item $\exists e \in G$ such that $\forall a \in G$, $a * e = e * a = a$ \hfill(\keyword{Identity})
        \item $\forall a \in G$, $\exists a^{-1} \in G$ such that $a * a^{-1} = a^{-1} * a = e$ \hfill(\keyword{Inverse})
    \end{enumerate}
\end{ndfn}

Often we don't explicitly specify the binary operation and simply refer to $G$ as the group. Likewise, $a * b$ is abbreviated to $ab$. When the binary operation behaves like addition, we denote it by $+$, and the use $0$ as the identity element and $-a$ as the inverse of $a \in G$.

\begin{nthm}
    Every group $G$ has a unique identity element. Each $a \in G$ has a unique inverse.
\end{nthm}

\begin{nthm}
    Let $G$ be a group. For all $a, b \in G$, $(a^{-1})^{-1} = a$ and $(ab)^{-1} = b^{-1} a^{-1}$.
\end{nthm}

\begin{ndfn}
    The \keyword{order of a group} $G$, denoted $\abs{G}$ or $O(G)$, is the number of elements in $G$. If $G$ has infinitely many elements then $\abs{G} = \infty$. The \keyword{order of an element} $a \in G$ is the least positive integer $k$ such that $a^k = e$. We denote this as $O(a)$ or $\abs{a}$. If there is no such positive integer then $\abs{a} = \infty$.
\end{ndfn}

\begin{ndfn}[Subgroup]
    A non-empty subset $H \subseteq G$ is called a \keyword{subgroup} of $G$ if it is a group under the same binary operation. We denote this as $H \subgroup G$. If $H \neq G$ then $H$ is called a \keyword{proper subgroup} of $G$. This is sometimes emphasised by writing $H \lneqq G$. A subgroup is called \keyword{trivial} if $H = \set{e}$, and non-trivial otherwise.
\end{ndfn}

\begin{nthm}[Subgroup test]
    Let $G$ be a group. A non-empty subset $H \subseteq G$ is a subgroup of $G$ if and only if for all $a, b \in H$, $ab^{-1} \in H$.
\end{nthm}

\begin{ndfn}[Cyclic group]
    $G$ is called \keyword{cyclic} if there is an element $a \in G$ such that all elements of $G$ can be written as powers of $a$. More precisely, $\forall g \in G$, $\exists m \in \Z$ such that $g = a^m$.

    Such an element $a$ is called a \keyword{generator} of $G$, and we say that $G$ is the group generated by $a$ and denote this as $G = \cyclic{a}$. Cyclic group of order $n$ is sometimes denoted as $C_n$.
\end{ndfn}

Generators are not unique. Indeed if $a \in G$ is a generator then so is $a^{-1}$.

\begin{nthm}
    Let $G$ be a group and take $a \in G$ such that $a^n = e$. The cyclic subgroup $\cyclic{a}$ has the form $\set{e, a, a^2, \dots, a^{n-1}}$.
\end{nthm}

\begin{nthm}
    Every cyclic group is abelian. Every subgroup of a cyclic group is cyclic. And, the order of a cyclic group is equal to the order of its generator.
\end{nthm}

\begin{nthm}
    Let $G = \cyclic{a}$ be a finite cyclic group of order $n$. Then an element $a^k$ is a generator of $G$ if and only if $\gcd(k,n) = 1$.
\end{nthm}

\begin{nthm}[Lagrange]
    Let $G$ be a finite group and $H$ be its subgroup. Then $\abs{H}$ divides $\abs{G}$. Moreover, the number of distinct left (right) cosets of $H$ in $G$ is $\abs{G}/\abs{H}$.
\end{nthm}

\begin{ncor}
    Let $G$ be a finite group and $a \in G$. Then $O(a)$ divides $\abs{G}$, and $a^{|G|} = e$.
\end{ncor}

\begin{ncor}
    Every group of prime order is cyclic.
\end{ncor}

\begin{ndfn}[Cosets]
    Let $H \subgroup G$ and $a \in G$. The subset $aH = \set{ah \st h \in H}$ is called a left \keyword{coset} of $H$ containing $a \in G$. The element $a$ is called a \keyword{representative} of the coset. We note that any element of $aH$ can act as its representative.
\end{ndfn}

\begin{ndfn}[Index]
    Let $H \subgroup G$. The number of distinct left (right) cosets of $H$ in $G$ is called the \keyword{index} of $H$ in $G$. It is denoted as $[G:H]$.
\end{ndfn}

\begin{nthm}
    Let $H \subgroup G$ and $a, b \in G$.
    \begin{enumerate}
        \item $a \in aH$.
        \item $aH = H$ if and only if $a \in H$.
        \item $aH = bH$ if and only if $a \in bH$ if and only if $a^{-1} b \in H$.
        \item $aH = bH$ or $aH \cap bH = \emptyset$.
        \item $|aH| = |bH|$.
    \end{enumerate}
\end{nthm}

\begin{ndfn}[Normal subgroup]
    Let $H \subgroup G$. If $aH = Ha$ for all $a \in G$ then $H$ is called a \keyword{normal subgroup} of $G$. We write this as $H \nsubgroup G$.
\end{ndfn}

\begin{nthm}
    Let $H \subgroup G$. Then, the following are equivalent:
    \begin{enumerate}
        \item $H \nsubgroup G$.
        \item $\forall x \in H$ and $g \in G$, $g x g^{-1} \in H$.
        \item $\forall g \in G$, $g H g^{-1} = H$.
    \end{enumerate}
\end{nthm}

\begin{nthm}
    Let $H \subgroup G$. If $[H:G] = 2$, then $H \nsubgroup G$.
\end{nthm}

\begin{nthm}
    Let $H \nsubgroup G$. Then, $G/H = \set{gH : g \in G}$ is a group with respect to $(aH)(bH) = (ab)H$, for $a, b \in G$. Here, the identity element is $H$ and $(aH)^{-1} = a^{-1} H$. This is called the \keyword{quotient group} or \keyword{factor group} of $H$ in $G$.
\end{nthm}

\begin{ndfn}[Homomorphism]
    A \keyword{homomorphism} between two groups $G$ and $H$ is a map $\varphi : G \to H$ such that $\varphi(ab) = \varphi(a)\varphi(b)$ for all $a,b \in G$. If $\varphi$ is bijective then it is called an \keyword{isomorphism}. If there is an isomorphism between $G$ and $H$ then we say that they are \keyword{isomorphic} groups and denoted this by $G \isom H$.
\end{ndfn}

\begin{nthm}
    Let $\varphi : G \to H$ be a homomorphism. Then, $\ker\varphi \nsubgroup G$ and $\img\varphi \subgroup H$. Moreover, $\varphi$ is injective if and only if $\ker\varphi = \set{e}$.
\end{nthm}

\begin{nthm}[First Isomorphism Theorem]
    Let $\varphi : G \to H$ be a surjective homomorphism. Then, $G / \ker\varphi \isom H$.
\end{nthm}

\begin{remark}
    Some of the commonly used groups are
    \begin{itemize}
        \item $\Z, \Q, \R, \C$ under usual addition, and $\Q^{*}, \R^{*}, \C^{*}$ under usual multiplication. Here, the symbol $A^{*}$ denotes a set $A$ with its additive identity $0$ removed.
        \item $\Z_{n} = \set{0, 1, \dots, n-1}$ under modular addition.
        \item $D_{n} = \set{1,a,a^{2},\dots,a^{n-1}, b, ba, \dots, ba^{n-1}}$ with $a^{n} = b^{2} = 1$ and $ab = ba^{n-1}$ denotes the dihedral group of order $2n$.
        \item $V_{4} = \set{1, a, b, ab}$ with $a^{2} = b^{2} = (ab)^{2} = 1$ is the Klein-$4$ group.
        \item $S_{n}$ is called the symmetric group. It is the group of permutations of $\set{1,2,\dots,n}$.
        \item $C_{n} = \set{1, a, a^{2}, \dots, a^{n-1}}$ is the cyclic group of order $n$, with $a$ as a generator.
        \item $Q_{8}$ is the group of quaternions.
    \end{itemize}
\end{remark}

\section{Subgroups generated by Subsets}

\begin{nthm}
    Let $S$ be a non-empty subset of $G$, and
    \begin{equation*}
        H = \set*{ x_{1}^{a_{1}} x_{2}^{a_{2}} \dots x_{n}^{a_{n}} \st n \in \N, x_{i} \in S, a_{i} = \pm 1, \forall i}
    \end{equation*}
    be the set of all finite products of elements of $S$ and the inverses of elements of $S$. Then, $H \subgroup G$ and $S \subseteq H$. This is called the \keyword{subgroup generated by $S$}, and we write $H = \cyclic{S}$.
\end{nthm}
\begin{proof}
    As $S \neq \emptyset$, we have $H \neq \emptyset$. Take any $\alpha, \beta \in H$. Then, we can write
    \begin{equation*}
        \alpha = x_{1}^{a_{1}} x_{2}^{a_{2}} \dots x_{n}^{a_{n}}
        \quad\text{and}\quad
        \beta = y_{1}^{b_{1}} y_{2}^{b_{2}} \dots y_{k}^{b_{k}}
    \end{equation*}
    where $n, k \in \N$, $x_{i}, y_{j} \in S$ and $a_{i}, b_{j} \in \set{\pm 1}$ for all $i, j$. So,
    \begin{equation*}
        \alpha \beta^{-1}
        = \paren*{ x_{1}^{a_{1}} x_{2}^{a_{2}} \dots x_{n}^{a_{n}} } \paren*{ y_{1}^{b_{1}} y_{2}^{b_{2}} \dots y_{k}^{b_{k}} }^{-1}
        = x_{1}^{a_{1}} x_{2}^{a_{2}} \dots x_{n}^{a_{n}} y_{k}^{-b_{k}} \dots y_{2}^{-b_{2}} y_{1}^{-b_{1}}.
    \end{equation*}
    Clearly then, $\alpha\beta^{-1} \in H$ because it is a finite product of elements of $S$ and the inverses of elements of $S$. Moreover, $S \subseteq H$ by definition of $H$.
\end{proof}

\begin{remark}\phantom{}
    \begin{enumerate}
        \item $H = \cyclic{S}$ is the smallest subgroup of $G$ that contains $S$. In other words, if $K \subgroup G$ with $S \subseteq K$ then $H \subseteq K$.
        \item If $S = \set{x_{1}, \dots, x_{n}}$ then we write $\cyclic{S} = \cyclic{x_{1}, \dots, x_{n}}$, instead of $\cyclic{\set{x_{1}, \dots, x_{n}}}$.
        \item If $S = \set{a}$ is a singleton set, then $H = \cyclic{a}$ is a cyclic subgroup.
        \item More than one generating sets may give rise to the same subgroup.
        \item Every group $G$ has at least one generating set, namely $S = G$.
        \item If $G = \cyclic{S}$ and $S$ is finite, then we say that $G$ is finitely generated. If no such finite set exists then we say that $G$ is infinitely generated.
    \end{enumerate}
\end{remark}


\section{Commutator Subgroup}
\begin{ndfn}
    Take $a,b \in G$. We define $\comm{a,b} = aba^{-1}b^{-1}$ as the \keyword{commutator} of $a$ and $b$.
\end{ndfn}

Note that $\comm{a,b}^{-1} = \paren*{aba^{-1}b^{-1}}^{-1} = bab^{-1}a^{-1} = \comm{b,a}$.

\begin{ndfn}
    Consider the set $S = \set*{ \comm{a,b} \st a,b \in G }$. The subgroup generated by $S$ is called the \keyword{commutator subgroup} or the \keyword{derived subgroup} of $G$. We denote this as $G'$ or $\comm{G,G}$.
\end{ndfn}

By definition, a general element of $G'=\cyclic{S}$ is a finite product of commutators and their inverses. However, since the inverse of a commutator is again a commutator, every general element of $G'$ is simply a product of commutators;
\begin{equation*}
    x \in G' = \comm{G,G}
    \implies
    x = \comm{a_{1}, b_{1}} \dots \comm{a_{k}, b_{k}}
    \quad\text{for}\quad a_{1}, \dots, a_{k}, b_{1}, \dots, b_{k} \in G.
\end{equation*}

\begin{nthm}
    A group $G$ is abelian if and only if $G'$ is trivial.
\end{nthm}
\begin{proof}
    If $G'$ is trivial, then for all $a, b \in G$, $\comm{a,b} = 1 \implies aba^{-1}b^{-1} = 1 \implies ab = ba$. Therefore, $G$ is abelian.

    Conversely, if $G$ is abelian, then for all $a,b \in G$, $\comm{a,b} = aba^{-1}b^{-1} = aa^{-1}bb^{-1} = 1$. Therefore, $G' = \cyclic{1} = \set{1}$.
\end{proof}

\begin{negg}
    It is easy to check that $V_{4}' = \set{e}$ and $D_{3}' = \set{1, a, a^{2}}$.
\end{negg}

\begin{nthm}
    Let $G'$ be the commutator subgroup of $G$. Then,
    \begin{enumerate}
        \item $G' \nsubgroup G$ (i.e. $G'$ is normal in $G$).
        \item $G / G'$ is abelian.
        \item $G'$ is the smallest subgroup of $G$ such that $G/G'$ is abelian; more precisely, if $G / K$ is abelian for some $K \nsubgroup G$, then $G' \subseteq K$.
    \end{enumerate}
\end{nthm}
\begin{proof}
    \phantom{}
    \begin{enumerate}
        \item We need to show that for all $g \in G$ and $x \in G'$, $g x g^{-1} \in G'$.

        Take any $g \in G$ and $x \in G' \subseteq G$. Then $g x g^{-1} x^{-1} = \comm{g, x} \in G'$ because it is a commutator. Now, by closure in $G'$, we have
        $\paren*{ g x g^{-1} x^{-1} } x \in G' \implies g x g^{-1} \in G'$. As a result, $G' \nsubgroup G$.

        \item Take any $a, b \in G$. Then, $(ba)^{-1} (ab) \in G'$ because $(ba)^{-1} (ab) = a^{-1} b^{-1} a b = [a^{-1}, b^{-1}]$ is a commutator of $a^{-1}, b^{-1} \in G$. And, $(ba)^{-1} (ab) \in G' \iff ab G' = ba G'$.

        So, $G/G'$ is abelian because we obtain $(aG')(bG') = ab G' = ba G' = (bG')(aG')$.

        \item Suppose $G/K$ is abelian for some $K \nsubgroup G$. This means, $\forall a,b \in G$,
        \begin{equation*}
            abK = (aK)(bK) = (bK)(aK) = baK.
        \end{equation*}
        Now, $abK = baK \iff (ba)^{-1}(ab) \in K$. And, $(ba)^{-1} (ab) = a^{-1} b^{-1} a b = [a^{-1}, b^{-1}]$. Therefore, $K$ contains all the commutators of $G$. Thus, $G' \subseteq K$.
        \qedhere
    \end{enumerate}
\end{proof}

\section{Direct Product of Groups}

\begin{ndfn}
    Let $H, K \subgroup G$. Define $HK = \set{hk \st h \in H, k \in K}$ .
\end{ndfn}

In general, $HK$ may or may not be a subgroup of $G$.

\begin{nthm}
    Let $H, K \subgroup G$. Then, $HK$ is a subgroup of $G$ if and only if $HK = KH$.
\end{nthm}
\begin{proof}
    Suppose $HK \subgroup G$. Then, for all $x \in HK$ we have $x^{-1} \in HK$ because $HK$ is a subgroup of $G$. Thus, $x^{-1} = hk$ for some $h \in H$ and $k \in K$. So, $x = (x^{-1})^{-1} = (hk)^{-1} = k^{-1} h^{-1} \in KH$. As a result, $HK \subseteq KH$.

    Next, let $x \in KH$. Then, $x = kh$ for some $h \in H$, $k \in K$. So, $x^{-1} = (kh)^{-1} = h^{-1} k^{-1} \in HK$. Thus, $x \in HK$ because $HK \subgroup G$. As a result, $KH \subseteq HK$.

    Overall, $HK = KH$.

    Conversely, suppose $HK = KH$. Take any $a, b \in HK$. Then, $a = h_{1} k_{1}$ and $b = h_{2} k_{2}$ for some $h_{1}, h_{2} \in H$ and $k_{1}, k_{2} \in K$. So, $a b^{-1} = (h_{1} k_{1}) (h_{2} k_{2})^{-1} = h_{1} k_{1} k_{2}^{-1} h_{2}^{-1}$.

    Now, $(k_{1} k_{2}^{-1}) h_{2}^{-1} \in KH = HK$. So, $(k_{1} k_{2}^{-1}) h_{2}^{-1} = hk$ for some $h \in H$ and $k \in K$. Therefore,
    \begin{equation*}
        a b^{-1}
        = h_{1} k_{1} k_{2}^{-1} h_{2}^{-1}
        = h_{1} (h k)
        = (h_{1} h) k
        \in HK.
    \end{equation*}
    And we conclude that $HK \subgroup G$.
\end{proof}

\begin{nthm}\label{thm:normal-product-subgroup}
    Let $H \nsubgroup G$ and $K \subgroup G$. Then, $HK \subgroup G$.
\end{nthm}
\begin{proof}
    Since $H \nsubgroup G$, we have $khk^{-1} = h$, $\forall h \in H, k \in K \subseteq G$. And, $khk^{-1} = h \implies kh = hk$.

    Therefore, $HK = \set{hk \st h \in H, k \in K} = \set{kh \st h \in H, k \in K} = KH$.
    Thus, $HK \subgroup G$.
\end{proof}

\begin{nthm}
    If $H, K$ are two finite subgroups of $G$, then
    \begin{equation*}
        \abs{H} = \frac{\abs{H} \abs{K}}{\abs{H \cap K}}.
    \end{equation*}
\end{nthm}
\begin{proof}
    We begin by noting that
    \begin{equation*}
        HK = \bigcup_{h \in H} hK,
    \end{equation*}
    and $\abs{hK} = \abs{K}$ for all $h \in H$. So, $\abs{HK} = r \abs{K}$, where $r$ is the number of distinct left cosets of the form $hK$ with $h \in H$. Now, for all $h_{1}, h_{2} \in H$,
    \begin{equation*}
        h_{1} K = h_{2} K
        \iff h_{2}^{-1} h_{1} \in K
        \iff h_{2}^{-1} h_{1} \in H \cap K
        \iff h_{1} (H \cap K) = h_{2} (H \cap K).
    \end{equation*}
    Therefore, the number of distinct left cosets of the form $hK$ and those of the form $h (H \cap K)$ is the same. Since, $(H \cap K) \subgroup H$, we have $r = [H : H \cap K]$. Thus,
    \begin{equation*}
        \abs{HK} = r \abs{K} = [H : H \cap K] \abs{K} = \frac{\abs{H}\abs{K}}{\abs{H \cap K}}.
        \qedhere
    \end{equation*}
\end{proof}

\begin{ndfn}[External direct product]
    The \keyword{external direct product} of a collection of groups $G_{1}, \dots, G_{n}$ is the set $G_{1} \oplus \dots \oplus G_{n} = \set{ (g_{1}, \dots, g_{n}) \st g_{i} \in G_{i} }$.
\end{ndfn}

\begin{nthm}
    The external direct product $G_{1} \oplus \dots \oplus G_{n}$ forms a group under the binary operation $(g_{1}, \dots, g_{n}) (g'_{1}, \dots, g'_{n}) = (g_{1}g'_{1}, \dots, g_{n}g'_{n})$,
    where $(g_{1}, \dots, g_{n}), (g'_{1}, \dots, g'_{n}) \in G_{1} \oplus \dots \oplus G_{n}$.
\end{nthm}
\begin{proof}
    Easy exercise. We simply check all the group axioms.
\end{proof}

\begin{remark}
    It is easy to see that if $\abs{G_{i}} = r_{i}$, then $\abs{ G_{1} \oplus \dots \oplus G_{n} } = r_{1} r_{2} \dots r_{n}$.
\end{remark}

\begin{nthm}
    Let $G = G_{1} \oplus \dots \oplus G_{n}$ be a finite (external) direct product of groups and $(g_{1}, \dots, g_{n}) \in G$. Then, $O( (g_{1}, \dots, g_{n}) ) = \lcm( O(g_{1}), \dots, O(g_{n}) )$.
\end{nthm}
\begin{proof}
    Let $t = O( (g_{1}, \dots, g_{n}) )$ and $s = \lcm( O(g_{1}), \dots, O(g_{n}) )$.

    Now, $s = \lcm( O(g_{1}), \dots, O(g_{n}) )$ means that $s$ divides $O(g_{k})$ for all $k$. Therefore,
    \begin{equation*}
        (g_{1}, \dots, g_{n})^{s} = (g_{1}^{s}, \dots, g_{n}^{s}) = (e, \dots, e)
    \end{equation*}
    As a result, $t \mid s$. In particular, $t \leq s$.

    Also, by construction, $(g_{1}^{t}, \dots, g_{n}^{t}) = (g_{1}, \dots, g_{n})^{t} = (e, \dots, e)$. So, $g_{k}^{t} = e_{k}$ for all $k$. Thus, $O(g_{k})$ divides $t$ for all $k$, and we obtain that $s = \lcm( O(g_{1}), \dots, O(g_{n}) )$ divides $t$; i.e. $s \leq t$.

    Overall, $t = s$.
\end{proof}

\begin{nthm}
    Let $G, H$ be finite cyclic groups. Then, $G \oplus H$ is cyclic if and only if $\abs{G}$ and $\abs{H}$ are relatively prime.
\end{nthm}
\begin{proof}
    Let $\abs{G} = m$ and $\abs{H} = n$. So, $\abs{G \oplus H} = mn$.

    Suppose $G \oplus H$ is cyclic with a generator $(g,h)$. Then, $O((g,h)) = mn$. Also, by Lagrange's theorem $O(g) \mid m$ and $O(h) \mid n$. So, $\lcm(O(g), O(h))$ divides $\lcm(m,n)$. Thus,
    \begin{equation*}
        mn = O((g,h)) = \lcm(O(g), O(h)) \mid \lcm(m,n)
    \end{equation*}
    i.e. $mn \leq \lcm\set{m,n}$. However, $\lcm(m,n) \mid mn$. So, $\lcm(m,n) = mn$, and $\gcd(m,n) = 1$.

    Conversely, suppose $\abs{G}$ and $\abs{H}$ are relatively prime; i.e. $\gcd(m,n) = 1$. Let $G = \cyclic{g}$ and $H = \cyclic{h}$ for some $g \in G$ and $h \in H$. Then, $O(g) = \abs{G} = m$ and $O(h) = \abs{H} = n$. So,
    \begin{equation*}
        O( (g,h) )
        = \lcm(O(g), O(h))
        = \frac{O(g)O(h)}{\lcm(O(g), O(h))}
        = \frac{mn}{\gcd(m,n)}
        = mn
        = \abs{G \oplus H}.
    \end{equation*}
    Therefore, $\cyclic{(g,h)} = G \oplus H$, and we conclude that $G \oplus H$ is cyclic.
\end{proof}

\begin{ncor}
    An external direct product $G_{1} \oplus \dots \oplus G_{n}$ of finite cyclic groups is cyclic if and only if $\abs{G_{i}}$ and $\abs{G_{j}}$ are relatively prime whenever $i \neq j$.
\end{ncor}

\begin{ncor}
    Let $m = n_{1} n_{2} \dots n_{k}$. Then, $\Z_{m}$ is isomorphic to $\Z_{n_{1}} \oplus \dots \oplus \Z_{n^{k}}$ if and only if $n_{i}$ and $n_{j}$ are relatively prime whenever $i \neq j$.
\end{ncor}

\begin{nlemma}
    The group $G \oplus H$ is abelian if and only if and $G$ and $H$ are abelian.
\end{nlemma}
\begin{proof}
    Suppose $G$ and $H$ are abelian. Then, for any $(g,h), (g',h') \in G \oplus H$ we have
    \begin{equation*}
        (g,h) \oplus (g',h')
        = (gg', hh')
        = (g'g, h'h)
        = (g',h') \oplus (g,h).
    \end{equation*}
    Therefore, $G \oplus H$ is abelian.

    Conversely, suppose $G \oplus H$ is abelian. Then, for all $g, g' \in G$ and $h, h' \in H$ we have
    \begin{equation*}
        (gg', hh')
        = (g,h) \oplus (g',h')
        = (g',h') \oplus (g,h)
        = (g'g, h'h).
    \end{equation*}
    Therefore, $gg' = g'g$ and $hh' = h'h$. So, both $G$ and $H$ are abelian.
\end{proof}

\begin{ndfn}[Internal direct product]
    We say that a group $G$ is the \keyword{internal direct product} of $H$ and $K$ and write $G = H \times K$ if $H$ and $K$ are normal subgroups of $G$ and $H \cap K = \set{e}$ and $G = HK$.
\end{ndfn}

\begin{ndfn}[Internal direct product]
    Let $H_{1}, H_{2}, \dots, H_{n}$ be a finite collection of normal subgroups of $G$. We say that $G$ is the internal direct product of $H_{1}, H_{2}, \dots, H_{n}$ and write $G = H_{1} \times H_{2} \times \dots \times H_{n}$ if
    \begin{enumerate}
        \item $G = H_{1} H_{2} \dots H_{n} = {h_{1} h_{2} \dots h_{n} \st h_{i} \in H_{i}}$.
        \item $(H_{1} H_{2} \dots H_{i}) \cap H_{i+1} = \set{e}$, for $i = 1,2,\dots,n-1$
    \end{enumerate}
\end{ndfn}

\begin{nthm}
    If $G$ is the internal direct product of subgroups $H_{1}, \dots, H_{n}$, then $G$ is isomorphic to the external direct product of $H_{1}, \dots, H_{n}$. That is, $H_{1} \times \dots \times H_{n} \isom H_{1} \oplus \dots \oplus H_{n}$.
\end{nthm}
\begin{proof}
    Firstly note that for all $i \neq j$,
    \begin{equation*}
        H_{i} \nsubgroup G \implies h_{j} h_{i}^{-1} h_{j}^{-1} \in H_{i}
        \qquad\text{and}\qquad
        H_{j} \nsubgroup G \implies h_{i} h_{j} h_{i}^{-1} \in H_{j}.
    \end{equation*}
    Thus, $h_{i} (h_{j} h_{i}^{-1} h_{j}^{-1}) \in H_{i}$ and $(h_{i} h_{j} h_{i}^{-1}) h_{j}^{-1} \in H_{j}$ by closure. This means $h_{i} h_{j} h_{i}^{-1} h_{j}^{-1} \in H_{i} \cap H_{j}$. However, $H_{i} \cap H_{j} = \set{e} \implies h_{i} h_{j} h_{i}^{-1} h_{j}^{-1} = e \implies h_{i} h_{j} = h_{j} h_{i}$, whenever $i \neq j$.

    Now, since $G = H_{1} H_{2} \dots H_{n}$, every $g \in G$ has the form $g = h_{1} h_{2} \dots h_{n}$ with $h_{i} \in H_{i}$. To see that this representation is unique, suppose $h_{1} h_{2} \dots h_{n}  = h'_{1} h'_{2} \dots h'_{n}$ for some $h'_{i} \in H_{i}$. Then,
    \begin{align*}
        h_{1} h_{2} \dots h_{n-1} h_{n}  = h'_{1} h'_{2} \dots h'_{n-1} h'_{n}
        &\implies h_{1} h_{2} \dots h_{n-1} h_{n} (h'_{n})^{-1}  = h'_{1} h'_{2} \dots h'_{n-1}\\
        &\implies h_{n} (h'_{n})^{-1} = h'_{1} h'_{2} \dots h'_{n-1} (h_{1} h_{2} \dots h_{n-1})^{-1}\\
        &\implies h_{n} (h'_{n})^{-1} = h'_{1} h'_{2} \dots h'_{n-1} h_{n-1}^{-1} \dots h_{2}^{-1} h_{1}^{-1}\\
        &\implies h_{n} (h'_{n})^{-1} = h'_{1}h_{1}^{-1} h'_{2}h_{2}^{-1} \dots h'_{n-1}h_{n-1}^{-1}\\
        &\implies h_{n} (h'_{n})^{-1} \in H_{1}H_{2} \dots H_{n-1}.
    \end{align*}
    Also, $h_{n} (h'_{n})^{-1} \in H_{n}$. Therefore, $h_{n} (h'_{n})^{-1} \in (H_{1}H_{2} \dots H_{n-1}) \cap H_{n} = \set{e}$. In other words, $h_{n} (h'_{n})^{-1} = e$. Thus, $h_{n} = h'_{n}$. This allows us to cancel $h_{n}$ from both sides and write
    \begin{equation*}
        h_{1} h_{2} \dots h_{n-1} h_{n}  = h'_{1} h'_{2} \dots h'_{n-1} h'_{n}
        \implies
        h_{1} h_{2} \dots h_{n-1}  = h'_{1} h'_{2} \dots h'_{n-1}.
    \end{equation*}
    By repeating the same argument $n-1$ times, we conclude that $h_{i} = h'_{i}$ for all $i$.

    Next, consider the map $\varphi : H_{1} \times \dots \times H_{n} \to H_{1} \oplus \dots \oplus H_{n}$ with $\varphi(h_{1} h_{2} \dots h_{n}) = (h_{1}, h_{2}, \dots, h_{n})$. This is a homomorphism because
    \begin{align*}
        \varphi((h_{1} h_{2} \dots h_{n}) (h'_{1} h'_{2} \dots h'_{n}))
        = \varphi(h_{1}h'_{1} h_{2}h'_{2} \dots h_{n}h'_{n})
        &= (h_{1}h'_{1}, h_{2}h'_{2}, \dots, h_{n}h'_{n})\\
        &= (h_{1}, h_{2}, \dots, h_{n}) (h'_{1}, h'_{2}, \dots, h'_{n})\\
        &= \varphi(h_{1} h_{2} \dots h_{n} ) \varphi(h'_{1} h'_{2} \dots h'_{n}).
    \end{align*}
    Moreover, $\varphi$ is clearly surjective because the pre-image of $(h_{1}, h_{2}, \dots, h_{n})$ is simply $h_{1} h_{2} \dots h_{n}$. Additionally, $\varphi(h_{1} h_{2} \dots h_{n}) = (e, e, \dots, e) \implies (h_{1}, h_{2}, \dots, h_{n}) = (e, e, \dots, e) \implies h_{i} = e, \forall i$.
    Consequently, $\ker\varphi$ is trivial and we conclude that $\varphi$ is injective.

    Overall, $\varphi$ is an isomorphism. Thus, $H_{1} \times \dots \times H_{n} \isom H_{1} \oplus \dots \oplus H_{n}$.
\end{proof}

\begin{negg}
    The group $S_{3}$ cannot be written as the (internal) direct product of two proper subgroups.

    For a contradiction, suppose that we have two such proper subgroups $H, K \subgroup S_{3}$. Then, $\abs{H} = 2$ and $\abs{K} = 3$ or vice versa. However, every group of order less than $6$ is abelian. Thus, $H \oplus K$ is also abelian. As a result, $H \times K \isom H \oplus K$ will be abelian. This is a contradiction because $S_{3} = H \times K$ is not abelian. \eggqed
\end{negg}

\section{Fundamental Theorem of Finite Abelian Groups}
In this section, we shall classify the \keyword{finite abelian groups} based on their order.

We already know that every cyclic group of order $n$ is isomorphic to $\Z_{n}$. The fundamental theorem of finite abelian groups states that every such group can be written as a direct product of cyclic groups whose order is a power of a prime. In other words, every finite abelian group $G$ is isomorphic to a group of the form
\begin{equation*}
    \Z_{p_{1}^{n_{1}}} \oplus \dots \oplus \Z_{p_{k}^{n_{k}}}
\end{equation*}
where $p_{i}$ and $n_{i}$ are uniquely determined by $G$, and $p_{i}$ (primes) need not be distinct.

\begin{ndfn}
    Let $n \geq 1$ be an integer. A way of writing $n$ as a sum of positive integers is called a partition of $n$.
\end{ndfn}

Two partitions that differ only in the order of the summands are considered the same. For example, $3+1$ and $1+3$ are the same partitions. Since the order of summand does not matter, we write the summands in descending order.

Note that the relation $\sim$ on the set of all groups, defined by $G_{1} \sim G_{2}$ if and only if $G_{1} \isom G_{2}$ is an equivalence relation. The class $[G] = \set{H : H \isom G}$ is called the isomorphism class of $G$.

\begin{nthm}[Fundamental Theorem]
    Let $G$ be a finite abelian group of order $n \geq 2$. Consider the unique factorisation of $n$ into distinct prime powers $n = p_{1}^{\alpha_{1}} p_{2}^{\alpha_{2}} \dots p_{k}^{\alpha_{k}}$. Then,
    \begin{equation*}
        G \isom A_{1} \oplus \dots \oplus A_{k}
    \end{equation*}
    where $\abs{A_{i}} = p_{i}^{\alpha^{i}}$. For each $A \in \set{A_{1}, \dots, A_{k}}$ with $\abs{A} = p^{\alpha}$,
    \begin{equation*}
        A \isom \Z_{p^{\beta_{1}}} \oplus \dots \oplus \Z_{p^{\beta_{t}}}
    \end{equation*}
    where $\beta_{1} \geq \beta_{2} \geq \dots \geq \beta_{t}$ and $\beta_{1} + \beta_{2} + \dots + \beta_{t} = \alpha$.
\end{nthm}
(Proof of this theorem is non-examinable.)

\begin{negg}
    Find all the non-isomorphic abelian groups of order $12 = 3^{4}$.
    \begin{center}
        \begin{tabularx}{0.66\textwidth}{p{8mm}XX}
            \toprule
            $p^{\beta}$ & Partitions of $\beta$ & Abelian Groups\\
            \midrule
            $3^{4}$ & $4$ & $\Z_{3^{4}}$ \\
                    & $3+1$ & $\Z_{3^{3}} \oplus \Z_{3}$ \\
                    & $2+2$ & $\Z_{3^{2}} \oplus \Z_{3^{2}}$ \\
                    & $2+1+1$ & $\Z_{3^{2}} \oplus \Z_{3} \oplus \Z_{3}$ \\
                    & $1+1+1+1$ & $\Z_{3} \oplus \Z_{3} \oplus \Z_{3} \oplus \Z_{3}$ \\
            \bottomrule
        \end{tabularx}
    \end{center}
    So, up to isomorphism, there are exactly $5$ abelian groups of order $12$:
    \begin{equation*}
        \Z_{3^{4}}, \Z_{3^{3}} \oplus \Z_{3}, \Z_{3^{2}} \oplus \Z_{3^{2}}, \Z_{3^{2}} \oplus \Z_{3} \oplus \Z_{3}, \Z_{3} \oplus \Z_{3} \oplus \Z_{3} \oplus \Z_{3}.
    \end{equation*}
\end{negg}

\begin{negg}
    Find all the non-isomorphic abelian groups of order $60 = 2^{2} \times 3 \times 5$.
    \begin{center}
        \begin{tabularx}{0.66\textwidth}{p{8mm}XX}
            \toprule
            $p^{\beta}$ & Partitions of $\beta$ & Abelian Groups\\
            \midrule
            $2^{2}$ & $2$, $1+1$ & $\Z_{2^{2}}$, $\Z_{2} \oplus \Z_{2}$ \\
            $3^{1}$ & $1$        & $\Z_{3}$ \\
            $5^{1}$ & $1$        & $\Z_{5}$ \\
            \bottomrule
        \end{tabularx}
    \end{center}
    Now, we consider all possible combinations of these constituent groups (taking one group from each row). So, up to isomorphism, there are exactly $2$ abelian groups of order $12$:
    \begin{equation*}
        \Z_{2^{2}} \oplus \Z_{3} \oplus \Z_{5}
        \quad\text{and}\quad 
        \Z_{2} \oplus \Z_{2} \oplus \Z_{3} \oplus \Z_{5}.
    \end{equation*}
\end{negg}

In general, for the unique prime factorisation $n = p_{1}^{\alpha_{1}} p_{2}^{\alpha_{2}} \dots p_{k}^{\alpha_{k}}$ let $q_{i}$ be the number of partitions of $\alpha_{i}$ then the number of non-isomorphic groups of order $n$ is equal to $q_{1} q_{2} \dots q_{k}$.

\begin{negg}
    Take $n = 1800 = 2^{3} 3^{2} 5^{2}$. There are $3$ partitions of $3$ and $2$ partitions of $2$ and $5$ each. So, there are $3 \times 2 \times 2 = 12$ non-isomorphic abelian groups of order $1800$.
\end{negg}


\section{Isomorphism Theorems}

\begin{nlemma}
    Let $H \subgroup G$ and $N \nsubgroup G$. Then,
    \begin{enumerate}
        \item $HN$ is a subgroup of $G$.
        \item $N$ is a normal subgroup of $HN$.
        \item $H \cap N$ is a normal subgroup of $H$.
    \end{enumerate}
\end{nlemma}
\begin{proof}\phantom{}
    \begin{enumerate}
        \item This was proved in theorem (\ref{thm:normal-product-subgroup}).

        \item Firstly, note that $N, HN \subgroup G$ and $N \subseteq HN$ means that $N \subgroup HN$.

        Now, $N \subgroup HN \subgroup G$ along with $N \nsubgroup G$ means that $N \nsubgroup HN$.

        \item Firstly, note that $H, N \subgroup G \implies (H \cap N) \subgroup G$. Also, $H, (H \cap N) \subgroup G$ with $(H \cap N) \subseteq H$ means that $(H \cap N) \subgroup H$.

        Now, take any $a \in (H \cap N)$ and $h \in H$. Then,
        \begin{align*}
            a \in (H \cap N) \implies a \in N &\implies h a h^{-1} \in N \qquad\text{(as $N \nsubgroup G$)},\\
            a \in (H \cap N) \implies a \in H &\implies h a h^{-1} \in H \qquad\text{(by closure in  $H$)}.
        \end{align*}
        So, $h a h^{-1} \in (H \cap N)$, for all $h \in H$ and $a \in (H \cap N)$. Therefore, $(H \cap N) \nsubgroup H$.
        \qedhere
    \end{enumerate}
\end{proof}

\begin{nthm}[Second Isomorphism Theorem]
    Let $H \subgroup G$ and $N \nsubgroup G$. Then,
    \begin{equation*}
        H / (H \cap N) \isom HN / N.
    \end{equation*}
\end{nthm}
\begin{proof}
    Firstly, note that elements of $HN/N$ have the form $hN$ because $(hn)N = hN$ as $n \in N$.

    Consider the map $\varphi : H \to HN/N$ defined as $\varphi(h) = hN$. This is clearly surjective because the pre-image of $hN \in HN/N$ is $h \in H$. Moreover, it is a homomorphism because
    \begin{equation*}
        \varphi( h_{1} h_{2} )
        = (h_{1} h_{2}) N
        = (h_{1}N) (h_{2}N)
        = \varphi(h_{1}) \varphi(h_{2}).
    \end{equation*}
    Therefore, by the 1st isomorphism theorem, $H / \ker\varphi \isom HN / N$.

    However, $\ker\varphi = \set{h \in H \st \varphi(h) = N} = \set{h \in H \st hN = N} = \set{h \in H \st h \in N} = H \cap N$. Therefore, $H / (H \cap N) \isom HN / N$.
\end{proof}

\begin{nlemma}
    Let $K, N \nsubgroup G$ with $K \subseteq N$. Then, $N/K \nsubgroup G/K$.
\end{nlemma}
\begin{proof}
    Note that $N/K$ is a non-empty subset of $G/K$ because $1 \in N \implies 1K = K \in N/K$. Take any $n_{1}K, n_{2}K \in N/K$, then $(n_{1}K) (n_{2}K)^{-1} = (n_{1}K) (n_{2}^{-1} K) = (n_{1} n_{2}^{-1}) K \in N/K$
    since $n_{1} n_{2}^{-1} \in N$. Therefore, $N/K \subgroup G/K$.

    Next, for every $nK \in N/K$ and $gK \in G/K$, we have
    \begin{equation*}
        (gK) (nK) (gK)^{-1}
        = gK nK g^{-1}K
        = g n g^{-1 } K
        \in N/K
    \end{equation*}
    since $N \nsubgroup G \implies g n g^{-1 } \in N$. Thus, $N/K \nsubgroup G/K$.
\end{proof}

\begin{nthm}[Third Isomorphism Theorem]
    Let $K, N \nsubgroup G$ with $K \subseteq N$. Then,
    \begin{equation*}
        (G/K) / (N/K) \isom G/N.
    \end{equation*}
\end{nthm}
\begin{proof}
    Consider the map $\varphi : G/K \to G/N$ with $\varphi(gK) = gN$.

    This is well-defined because
    \begin{equation*}
        g_{1}K = g_{2}K
        \implies
        g_{1}^{-1}g_{2} \in K \subseteq N
        \implies
        g_{1}^{-1}g_{2} \in N
        \implies
        g_{1}N = g_{2}N
        \implies
        \varphi(g_{1}K) = \varphi(g_{2}K).
    \end{equation*}
    Also, $\varphi$ is clearly surjective, since $gK$ is the preimage of $gN$. Next, note that
    \begin{equation*}
        \varphi( (g_{1}K)(g_{2}K) )
        = \varphi( g_{1} g_{2} K )
        = g_{1} g_{2} N
        = (g_{1}N) (g_{2}N)
        = \varphi(g_{1}K) \varphi(g_{2}K).
    \end{equation*}
    So, $\varphi$ is a surjective homomorphism. Moreover,
    \begin{equation*}
        \ker\varphi = \set{gK \st \varphi(gK) = N} = \set{gK \st gN = N} = \set{gK \st g \in N} = N/K.
    \end{equation*}
    Therefore, by the first isomorphism theorem, $(G/K) / (N/K) \isom G/N$.
\end{proof}

\section{Centraliser and Normaliser}

Throughout this section let $A$ be a non-empty subset of $G$.

\begin{ndfn}[Centraliser]
    Let $A \subseteq G$. The \keyword{centraliser} of $A$ in $G$ is the set
    \begin{equation*}
        C_{G}(A) = \set{ g \in G \st g a g^{-1} = a, \forall a \in A }.
    \end{equation*}
\end{ndfn}

\begin{nthm}
    Let $A \subseteq G$. Then, $C_{G}(A) \subgroup G$.
\end{nthm}
\begin{proof}
    The centraliser $C_{G}(A) \subseteq G$ is non-empty because $1 \in C_{G}(A)$. Take any $x, y \in C_{G}(A)$. Then, $xax^{-1} = a$ and $yay^{-1} = a$, for all $a \in A$. Then,
    \begin{equation*}
        yay^{-1} = a \implies ya = ay \implies a = y^{-1}ay.
    \end{equation*}

    So, $(xy^{-1}) a (xy^{-1})^{-1} = x y^{-1} a y x^{-1} = x a x^{-1} = a$. Therefore, $xy^{-1} \in C_{G}(A)$, and we conclude that $C_{G}(A) \subgroup G$.
\end{proof}

\begin{remark}\phantom{}
    \begin{enumerate}
        \item If $G$ is abelian, then $C_{G}(A) = G$ for all non-empty $A \subseteq G$.
        \item When $A = \set{a}$ is a singleton, we write $C_{G}(a)$ instead of $C_{G}(\set{a})$.
        \item For every $a \in G$, $Z(G) \subgroup C_{G}(a)$.
        \item The centraliser of $G$ in $G$ is the centre of $G$; i.e. $Z(G) = C_{G}(G)$.
    \end{enumerate}
\end{remark}

\begin{ndfn}[Normaliser]
    Let $A \subseteq G$. The \keyword{normaliser} of $A$ in $G$ is the set
    \begin{equation*}
        N_{G}(A) = \set{ g \in G \st g A g^{-1} = A}.
    \end{equation*}
    Here, $g A g^{-1} = \set{g a g^{-1} \st a \in A}$.
\end{ndfn}

\begin{nprop}
    Let $A \subseteq G$. Then, $N_{G}(A) \subgroup G$.
\end{nprop}
\begin{proof}
    The normaliser $N_{G}(A) \subseteq G$ is non-empty because $1 \in N_{G}(A)$.

    Take any $x, y \in N_{G}(A)$. Then, $xAx^{-1} = A$ and $yAy^{-1} = A$. Since,
    \begin{equation*}
        yAy^{-1} = A
        \implies
        yA = Ay
        \implies
        A = y^{-1}Ay.
    \end{equation*}
    So, $(xy^{-1}) A (xy^{-1})^{-1} = x y^{-1} A y x^{-1} = x A x^{-1} = A$. Therefore, $xy^{-1} \in N_{G}(A)$, and we conclude that $N_{G}(A) \subgroup G$.
\end{proof}

\begin{remark}
    \phantom{}
    \begin{enumerate}
        \item We have $C_{G}(A) \subgroup N_{G}(A)$, for every non-empty $A \subseteq G$.
        \item If $G$ is abelian, then $N_{G}(A) = G$, for every non-empty $A \subseteq G$.
    \end{enumerate}
\end{remark}

\begin{nprop}
    Let $A \subseteq G$. Then, $C_{G}(A) \nsubgroup N_{G}(A)$.
\end{nprop}
\begin{proof}
    We already have $C_{G}(A) \subgroup N_{G}(A)$.

    We need to show that for all $c \in C_{G}(A)$ and $n \in N_{G}(A)$, the element $n c n^{-1} \in C_{G}(A)$. So, let $c \in C_{G}(A)$ and $n \in N_{G}(A)$. Then, for every $a \in A$,
    \begin{equation*}
        (ncn^{-1}) a (ncn^{-1})^{-1}
        = nc(n^{-1} a n)c^{-1}n^{-1}
        = n(cbc^{-1})n^{-1}
        = nbn^{-1}
        = (nn^{-1}) a (nn^{-1})
        = a,
    \end{equation*}
    where $b = n^{-1} a n \in A$. Thus, $n c n^{-1} \in C_{G}(A)$, and we conclude that $C_{G}(A) \nsubgroup N_{G}(A)$.
\end{proof}

\begin{nprop}
    Let $H \subgroup G$. Then,
    \begin{enumerate}
        \item $N_{G}(H)$ is the largest subgroup of $G$ in which $H$ is normal; more precisely, $H$ is normal in $N_{G}(H)$, and if $K \subgroup G$ with $H \nsubgroup K$ then $K \subseteq N_{G}(H)$.
        \item $H \nsubgroup G$ if and only if $N_{G}(H) = G$.
    \end{enumerate}
\end{nprop}
\begin{proof}\phantom{}
    \begin{enumerate}
        \item $H \subgroup G$, $N_{G}(H) \subgroup G$ and $H \subseteq N_{G}(H)$ collectively imply that $H \subgroup N_{G}(H)$.

        Now, take any $x \in N_{G}(H)$. Then, by definition, $xHx^{-1} = H$. So, $H$ is normal in $N_{G}(H)$.

        Next, suppose $H \nsubgroup K \subgroup G$. Then, $H \nsubgroup K \implies kHk^{-1} = H, \forall k \in K$. So, $K \subseteq N_{G}(H)$.

        \item Firstly, $H \nsubgroup G \implies gHg^{-1} = H, \forall g \in G$. Therefore, $G \subseteq N_{G}(H) \subseteq G$. So, $N_{G}(H) = G$.

        Conversely, if $N_{G}(H) = G \implies gHg^{-1} = H, \forall g \in G$. So, $H \nsubgroup G$ ($H \subgroup G$ already).\qedhere
    \end{enumerate}
\end{proof}

\begin{nex}
    If $H, K \subgroup G$ and $H \subgroup N_{G}(K)$. Then, $HK \subgroup G$. (Hint: it is enough to show that $HK = KH$.)
\end{nex}


\section{Conjugacy Relation \& the Class Equation}
\begin{ndfn}
    An element $b \in G$ is called \keyword{conjugate} of $a \in G$ if $b = g a g^{-1}$ for some $g \in G$.
\end{ndfn}

\begin{nthm}
    Given $a,b \in G$, define the relation $\sim$ on $G$ such that $a \sim b$ if and only if $b = g a g^{-1}$ for some $g \in G$. Then, this is an equivalence relation, called the \keyword{conjugacy relation}.
\end{nthm}
\begin{proof}
    Easy exercise. We simply show that $\sim$ is reflexive, symmetric and transitive.

    Take $a, b, c \in G$. Then, $a = e a e^{-1}$, so $a \sim a$. Also,
    \begin{align*}
        a \sim b
        \implies
        a = g b g^{-1}
        \implies
        g^{-1} a g = b
        \implies
        (g^{-1}) a (g^{-1})^{-1} = b
        \implies
        b \sim a,
    \end{align*}
    where $g \in G$. Moreover, suppose $a \sim b$ and $b \sim c$. This means, $a = g b g^{-1}$ and $b = h c h^{-1}$ for some $g, h \in G$. Then, $a = g b g^{-1} = g h c h^{-1} g^{-1} = (gh) c (gh)^{-1}$ because $gh \in G$. So, $a \sim c$.
    % Overall, $\sim$ is an equivalence relation.
\end{proof}

\begin{ndfn}
    Let $\sim$ be the conjugacy relation on $G$. For every $a \in G$, the equivalence class $[a] = \set{ g a g^{-1} \st g \in G }$ is called the \keyword{conjugacy class} of $a$ in $G$.
\end{ndfn}

Note that in every group the identity element $e$ is only conjugate to itself. Therefore, $[e] = \set{e}$. Moreover, as equivalence classes, the conjugacy classes partition the group.

\begin{nthm}
    Let $a \in G$ be any element. Then, $\abs{[a]} = [G : C_{G}(a)]$.
\end{nthm}
\begin{proof}
    Let $H = C_{G}(a) \subgroup G$ and $\mathcal{H} = \set{xH \st x \in G}$ be the set of all left cosets of $H$ in $G$. Consider the map $\varphi : \mathcal{H} \to [a]$ defined as $\varphi(xH) = x a x^{-1}$.

    We shall show that $\varphi$ is a well-defined bijective map.

    To this end, suppose $xH = yH$, then $y^{-1} x \in H = C_{G}(a)$. So, $(y^{-1} x) a (y^{-1} x)^{-1} = a$. And,
    \begin{align*}
        (y^{-1} x) a (y^{-1} x)^{-1} = a
        \implies
        (y^{-1} x) a = a (y^{-1} x)
        \implies
        xa = yay^{-1} x
        \implies
        xax^{-1} = yay^{-1}
    \end{align*}
    Therefore, $\varphi(xH) = \varphi(yH)$. So, $\varphi$ is well-defined.

    Now, $\varphi$ is clearly surjective because the pre-image of any $x a x^{-1} \in [a]$ is simply $xH \in \mathcal{H}$. Also, $\varphi$ is injective because $\varphi(xH) = \varphi(yH) \implies xax^{-1} = yay^{-1} \implies y^{-1} x a = a y^{-1} x$. And,
    \begin{equation*}
        y^{-1} x a = a y^{-1} x
        \implies
        (y^{-1} x) a (y^{-1} x)^{-1} = a
        \implies
        y^{-1} x \in C_{G}(a) = H
        \implies
        xH = yH.
    \end{equation*}
    Overall, $\varphi$ is a well-defined bijective map. Therefore, $\abs{ [a] } = \abs{ \mathcal{H} } = [G : C_{G}(a)]$ as sets.
\end{proof}

\begin{nthm}
    Let $g_{1}, \dots, g_{n} \in G$ be representatives of distinct conjugacy classes. Then,
    \begin{equation*}
        \abs{G} = \sum_{i=1}^{n} [G : C_{G}(g_{i})].
    \end{equation*}
\end{nthm}
\begin{proof}
    The set of all the equivalence classes partitions the group $G$. So,
    \begin{equation*}
        G = \bigcup_{i = 1}^{n} [g_{i}]
        \implies \abs{G} = \sum_{i = 1}^{n} \abs{[g_{i}]} = \sum_{i = 1}^{n} [G : C_{G}(g_{i})].
        \qedhere
    \end{equation*}
\end{proof}

\begin{nlemma}
    Let $a \in G$. Then, $[a] = \set{a}$ if and only if $a \in Z(G)$.
\end{nlemma}
\begin{proof}
    If $a \in Z(G)$, then $g a g^{-1} = a, \forall g \in G$. So, $[a] = \set{g a g^{-1} \st g \in G} = \set{a \st g \in G} = \set{a}$.

    Conversely, if $[a] = \set{a}$, then $\set{g a g^{-1} \st g \in G} = \set{a}$. So, $g a g^{-1} = a, \forall g \in G$. Equivalently, $ga = ag$, for all $g \in G$. Therefore, $a \in Z(G)$.
\end{proof}

\begin{ncor}
    If $a \in Z(G)$, then $[G : C_{G}(a)] = \abs{ [a] } = 1$.
\end{ncor}
\begin{proof}
    If $a \in Z(G)$, then $[a] = \set{a}$. So, $[G : C_{G}(a)] = \abs{[a]} = \abs{\set{a}} = 1$.
\end{proof}

\begin{nthm}[Class Equation]
    Let $G$ be a finite group and $g_{1}, \dots, g_{r}$ be representatives of the non-central conjugacy classes of $G$. Then,
    \begin{equation*}
        \abs{G} = \abs{Z(G)} + \sum_{i=1}^{r} [G : C_{G}(g_{i})].
    \end{equation*}
\end{nthm}
\begin{proof}
    Let $k = \abs{Z(G)}$ and note that $a_{i} \in Z(G) \implies \abs{[a_{i}]} = 1, \forall i=1,\dots,k$.

    Since $G$ can be partitioned by conjugacy classes of elements from $Z(G)$, and the non-central conjugacy classes with representative $g_{1}, \dots, g_{r}$, we obtain   
    \begin{equation*}
        \abs{G}
        = \underbrace{1 + \dots + 1}_{k-\text{times}} + \sum_{i=1}^{r} [G : C_{G}(g_{i})]
        = \abs{Z(G)} + \sum_{i=1}^{r} [G : C_{G}(g_{i})].\qedhere
    \end{equation*}
\end{proof}

\begin{nthm}
    Let $p$ be a prime. If $\abs{G} = p^{n}$ for $n \geq 1$, then $Z(G) \neq \set{e}$.
\end{nthm}
\begin{proof}
    If $G$ is abelian, then $Z(G) = G \neq \set{e}$. So, we are done.

    Let $G$ be non-abelian, and $g_{1}, \dots, g_{r}$ be representatives of its non-central conjugacy classes. Since, $C_{G}(g_{i}) \subgroup G$ and $\abs{G} = p^{n}$, by Lagrange's theorem $\abs{C_{G}(g_{i})} = p^{k}$ for some $k < n$. Thus,
    \begin{equation*}
        [G : C_{G}(g_{i})] = \frac{\abs{G}}{\abs{C_{G}(g_{i})}} = p^{n-k}.
    \end{equation*}
    So, $p$ divides $[G : C_{G}(g_{i})]$ for all $i$. Therefore, $p$ divides $\sum_{i=1}^{r} [G : C_{G}(g_{i})]$. By the class equation,
    \begin{equation*}
        \abs{G} = \abs{Z(G)} + \sum_{i=1}^{r} \comm{G : C_{G}(g_{i})}
        \implies
        \abs{Z(G)} = \abs{G} - \sum_{i=1}^{r} \comm{G : C_{G}(g_{i})}.
    \end{equation*}
    As $p$ divides both $\abs{G}$ and $\sum_{i=1}^{r} \comm{G : C_{G}(g_{i})}$ we conclude that $p$ divides $\abs{Z(G)}$. Therefore, $\abs{Z(G)} \geq p > 1$. Consequently, $Z(G) \neq \set{e}$.
\end{proof}

\begin{ncor}
    Let $p$ be a prime. If $\abs{G} = p^{2}$ then $G$ is abelian.
\end{ncor}
\begin{proof}
    Since $\abs{G}$ is of the form $p^{n}$, we have $Z(G) \neq \set{e}$. Moreover, $Z(G) \subgroup G$, so by Lagrange's theorem $\abs{Z(G)} = p$ or $p^{2}$.

    Suppose $\abs{Z(G)} = p$, and pick any $a \in G$ such that $a \notin Z(G)$. Then,
    \begin{equation*}
        Z(G) \lneqq C_{G}(a) \lneqq G
        \implies
        \abs{Z(G)} < \abs{C_{G}(a)} < \abs{G}
        \implies
        p < \abs{C_{G}(a)} < p^{2}.
    \end{equation*}
    This is not possible because $\abs{C_{G}(a)}$ divides $p^{2}$ by Lagrange's theorem and $p$ is prime.

    Therefore, $\abs{Z(G)} = p^{2}$. In other words, $Z(G) = G$. So, $G$ is abelian.
\end{proof}

% \missing{(END OF SYLABUS FOR THE MIDTERM EXAM)}

\section{Sylow's Theorem}
Throughout this section, let $G$ be a finite group.

\begin{nthm}[Cauchy]
    Let $G$ be a finite abelian group and $p$ be a prime such that $p$ divides $\abs{G}$. Then, there is an element $x \in G$ with $O(x) = p$.
\end{nthm}
\begin{proof}
    We shall use strong induction on $\abs{G}$.

    % If $\abs{G} = 1$, then there is nothing to prove.
    If $\abs{G} = p$, then every non-identity element has order $p$. So, assume $\abs{G} > p$.

    Assume that the statement is true for all groups with order less than $\abs{G}$.

    Take any non-identity element $a \in G$ and consider the subgroup $H = \cyclic{a}$. As $G$ is abelian, $H$ is normal in $G$. Moreover, $a \neq e$ implies $\abs{H} > 1$.

    % If $p$ divides $\abs{H}$, then $a^{\abs{H}/p}$ is an element of order $p$ in $G$; this is because $(a^{\abs{H}/p})^{p} = e$, so $O(a^{\abs{H}/p})$ divides $p$, but $O(a^{\abs{H}/p}) \neq 1$ because $O(a) = \abs{\cyclic{a}} = \abs{H} > \abs{H}/p \implies a^{\abs{H}/p} \neq e$. Therefore, $O(a^{\abs{H}/p}) = p$.

    If $p$ divides $\abs{H}$, then $\abs{H}/p \in \Z$. So, $(a^{\abs{H}/p})^{p} = a^{\abs{H}} = e$. Thus, $O(a^{\abs{H}/p})$ divides $p$. We claim that $O(a^{\abs{H}/p}) \neq 1$. This is because $O(a) = \abs{\cyclic{a}} = \abs{H} > \abs{H}/p$. So, $a^{\abs{H}/p} \neq e$. Therefore, $O(a^{\abs{H}/p}) = p$; i.e. $a^{\abs{H}/p} \in G$ is an element of order $p$.

    Otherwise, if $p \nmid \abs{H}$, then consider the group $G/H$. This is abelian because $G$ is abelian. And, $\abs{G/H} = \abs{G} / \abs{H} < \abs{G}$ and $\abs{G} = \abs{G/H} \abs{H}$.

    As, $p$ divides $\abs{G}$ but not $\abs{H}$, we get that $p$ divides $\abs{G/H}$. Then, by the inductive hypothesis, there exists $yH \in G/H$ with $O(yH) = p$. Now,
    \begin{equation*}
        (yH)^{p} = H
        \implies
        y^{p} H = H
        \implies
        y^{p} \in H
        \implies
        (y^{p})^{\abs{H}} = e
        \implies
        (y^{\abs{H}})^{p} = e.
    \end{equation*}
    Thus, $O(y^{\abs{H}})$ divides $p$. So, $O(y^{\abs{H}}) = 1$ or $p$. But, $O(y^{\abs{H}}) = 1 \implies y^{\abs{H}} = e$. So,
    \begin{equation*}
        (yH)^{\abs{H}} = y^{\abs{H}}H = eH = H
        \implies
        O(yH) \mid \abs{H}
        \implies
        p \mid \abs{H}
    \end{equation*}
    which is a contradiction (since $p \nmid \abs{H}$). Therefore, we must have $O(y^{\abs{H}}) = p$, and we are done because $y^{\abs{H}} \in G$ is an element of order $p$ in this case.
\end{proof}

\begin{nthm}[Sylow's 1st Theorem]
    Let $p$ be a prime and $k \geq 1$. If $p^{k}$ divides $\abs{G}$, then $G$ has at least one subgroup of order $p^{k}$.
\end{nthm}
\begin{proof}
    If $\abs{G} = p^{k}$, then $G \subgroup G$ satisfies the statement of the theorem. So, assume $\abs{G} > p^{k}$.

    Assume that the statement is true for all groups of order less than $\abs{G}$.

    If $G \neq Z(G)$, then let $a_{1}, \dots, a_{r}$ be the representatives of the distinct non-central conjugacy classes of $G$. If $p^{k}$ divides $\abs{C_{G}(a_{i})}$ for any $i$, then we are done because $C_{G}(a_{i}) \subgroup G$ satisfies the statement of the theorem. If this is not the case, then $\forall i$, $p$ divides $[G : C_{G}(a_{i})]$ because $p^{k}$ divides $\abs{G} = [G : C_{G}(a_{i})] \abs{C_{G}(a_{i})}$ and we have assumed that $p^{k}$ doesn't divide $\abs{C_{G}(a_{i})}$. From the class equation $\abs{Z(G)} = \abs{G} - \sum_{i=1}^{r} \comm{G : C_{G}(a_{i})}$ it follows that, again, $p$ divides $\abs{Z(G)}$.

    On the other hand, if $G = Z(G)$, then again $p^{k} \mid \abs{G} \implies p \mid \abs{G} \implies p$ divides $\abs{Z(G)}$.

    Now, by Cauchy's theorem for abelian groups, there exists $x \in Z(G)$ such that $O(x) = p$. Also, $\cyclic{x} \subgroup Z(G) \nsubgroup G$ implies $\cyclic{x} \nsubgroup G$. So, consider $G / \cyclic{x}$. Here $\abs{G / \cyclic{x}} = \abs{G} / \abs{\cyclic{x}} = p^{k-1}$.

    Since, $p^{k-1}$ divides $\abs{G / \cyclic{x}}$ and $\abs{G / \cyclic{x}} < \abs{G}$, by the inductive hypothesis, there exists a subgroup $H / \cyclic{x} \subgroup G / \cyclic{x}$ with $\abs{H / \cyclic{x}} = p^{k-1}$. Here $H \subgroup G$.

    However, $p^{k-1} = \abs{H / \cyclic{x}} = \abs{H} / \abs{\cyclic{x}}$ gives $\abs{H} = p^{k}$. Again, we are done because $H \subgroup G$ satisfies the requirements of the theorem. This covers all the possible cases.
\end{proof}

\begin{ndfn}[Sylow $p$-subgroup]
    Let $p$ be a prime dividing $\abs{G}$. If $p^{k}$ divides $\abs{G}$ but $p^{k+1} $ does not divide $\abs{G}$, then any subgroup of $G$ of order $p^{k}$ is called a \keyword{Sylow $p$-subgroup}.
\end{ndfn}

\begin{ndfn}[Conjugate groups]
    Let $H, K \subgroup G$. We say that $H$ and $K$ are \keyword{conjugates} in $G$, if $H = g K g^{-1}$ for some $g \in G$.
\end{ndfn}

\begin{nprop}
    Let $H, K \subgroup G$ and $a, b \in G$. Define a relation $\sim$ on $G$ by $a \sim b$ if and only if there exists $h \in H$ and $k \in K$ such that $a = hbk$. Then, $\sim$ is an equivalence relation on $G$.
\end{nprop}
\begin{proof}
    Since $e \in H, K$, then we have $a = e a e$, so $a \sim a$.

    Also, $a \sim b$ means that $a = hbk$ for some $h \in H$ and $k \in K$. So, $b = h^{-1} a k^{-1}$ with $h^{-1} \in H$ and $k^{-1} \in K$. Thus, $b \sim a$.

    And, if $a \sim b$ and $b \sim c$, then $a = hbk$ and $b = h'ck'$ for $h, h' \in H$ and $k, k' \in K$. So, $a \sim c$ because $a = h b k = (h h') c (k' k)$ and $hh' \in H$ and $kk' \in K$.
\end{proof}

\begin{remark}
    Since $\sim$ is an equivalence relation on $G$, it divides $G$ into disjoint equivalence classes. The equivalence class for $a$ is given by $\mathrm{Cl}(a) = \set{x \in G \st x \sim a} = \set{hak \st h \in H, k \in K} \equiv HaK$. In other words, $\displaystyle G = \bigsqcup_{a} \mathrm{Cl}(a) = \bigsqcup_{a} HaK$.

\end{remark}

\begin{ndfn}[Double cosets]
    Let $H, K \subgroup G$ and $a \in G$. Then, $HaK = \set{hak \st h \in H, k \in K}$ is called a \keyword{double coset} of $H$ and $K$ in $G$.
\end{ndfn}

\begin{nlemma}
    Let $H \subgroup G$. For every fixed $a \in G$, $f : H \to a H a^{-1}$ defined by $f(h) = aha^{-1}$ is a bijection. (It is an isomorphism)
\end{nlemma}
\begin{proof}
    This map is clearly surjective, because the pre-image of every $aha^{-1} \in aHa^{-1}$ is $h \in H$. Also, $f$ is injective because $\forall h_{1}, h_{2} \in H$,
    \begin{equation*}
        f(h_{1}) = f(h_{2}) \implies ah_{1}a^{-1} = ah_{2}a^{-1} \implies h_{1} = h_{2}.
    \end{equation*}
    Therefore, $f$ is a bijection.
\end{proof}

\begin{nex}
    Let $H \subgroup G$. Show that $gHg^{-1} \subgroup G$ for all $g \in G$.

    Note that $e \in gHg^{-1}$, so it is non-empty. Next, take $gag^{-1}, gbg^{-1} \in gHg^{-1}$. Then,
    \begin{equation*}
        \paren*{gag^{-1}} \paren*{gbg^{-1}}^{-1} = \paren*{gag^{-1}} \paren*{gb^{-1}g^{-1}} = g \paren*{ab^{-1}} g^{-1}.
    \end{equation*}
    Now, $a, b \in H \subgroup G \implies ab^{-1} \in H$. So, $g \paren*{ab^{-1}} g^{-1} \in gHg^{-1}$. Therefore, $gHg^{-1} \subgroup G$.
\end{nex}

\begin{nlemma}
    Let $H \subgroup G$. Then, for every fixed $a \in G$, $f : HaK \to H a K a^{-1}$ defined by $f(hak) = haka^{-1}$ is a bijection.
\end{nlemma}
\begin{proof}
    This map is clearly surjective, because the pre-image of every $haka^{-1} \in HaKa^{-1}$ is $hak \in HaK$. Also, $f$ is injective because $\forall h_{1}ak_{1}, h_{2}ak_{2} \in H$,
    \begin{equation*}
        f(h_{1}ak_{1}) = f(h_{2}ak_{2}) \implies h_{1}ak_{1}a^{-1} = h_{2}ak_{2}a^{-1} \implies h_{1}ak_{1} = h_{2}ak_{2}.
    \end{equation*}
    Therefore, $f$ is a bijection.
\end{proof}

\begin{ncor}
    Let $H, K$ be subgroups of a finite group $G$. Then, for all $a \in G$
    \begin{equation*}
        \abs{H} = \abs{aHa^{-1}}
        \quad\text{and}\quad
        \abs{HaK} = \abs{HaKa^{-1}}.
    \end{equation*}
\end{ncor}

\begin{remark}
    Let $H, K$ be subgroups of a finite group $G$, and fix $a \in G$. Then,
    \begin{equation*}
        \abs{HaK} = \abs{HaKa^{-1}} = \frac{\abs{H}\abs{K}}{\abs{H \cap aKa^{-1}}}
        \quad\text{and}\quad
        \abs{G} = \sum_{a \in G} \frac{\abs{H}\abs{K}}{\abs{H \cap aKa^{-1}}},
    \end{equation*}
    where the sum runs over distinct double cosets.
\end{remark}

\begin{nthm}[Sylow's 2nd Theorem]
    Any two Sylow $p$-subgroups of $G$ are conjugate to each other. More precisely, if $P$ and $Q$ are Sylow $p$-subgroups of $G$, then $Q = gPg^{-1}$.
\end{nthm}
\begin{proof}
    Let $P, Q$ be Sylow $p$-subgroups of $G$ of order $p^{n}$. Then, $p^{n+1} \nmid G$. Decompose $G$ as $\displaystyle G = \bigsqcup_{x} PxQ$. Here the union runs over disjoint double cosets.

    Now, if $P \neq xQx^{-1}$ then, $\abs{P \cap x Q x^{-1}} = p^{m}$ for $m < n$. Thus,
    \begin{equation*}
        \abs{PxQ}
        = \frac{\abs{P}\abs{Q}}{\abs{P \cap x Q x^{-1}}}
        = \frac{p^{2n}}{p^{m}}
        = p^{2n-m}.
    \end{equation*}
    And, $2n-m \geq n+1$. Therefore, $p^{n+1} \mid p^{2n-m}$. So, $p^{n+1} \mid \abs{PxQ}, \forall x \in G$. Moreover,
    \begin{equation*}
        \abs{G} = \sum_{x} \abs{PxQ}
        \implies
        p^{n+1} \mid \abs{G}.
    \end{equation*}
    This is a contradiction. Therefore, $P = xQx^{-1}$ for some $x \in G$.
\end{proof}

\begin{ncor}
    $P$ is the unique Sylow $p$-subgroup of $G$ if and only if $P \nsubgroup G$.
\end{ncor}
\begin{proof}
    Suppose $P$ is the unique Sylow $p$-subgroup of $G$. This means, $\forall g \in G$, we have $gPg^{-1} = P$, otherwise, $gPg^{-1}$ would be another Sylow $p$-subgroup by Sylow's 2nd theorem. And, $\forall g \in G$, $gPg^{-1} = P \implies P \nsubgroup G$.

    For the converse, suppose $P \nsubgroup G$. Now, if $Q$ is another Sylow $p$-subgroup, then by Sylow's 2nd theorem, we can write $Q = gPg^{-1}$ for some $g \in G$. But, $P \nsubgroup G \implies gPg^{-1} = P$, for all $g \in G$. Therefore, $Q = P$.
\end{proof}

\begin{ndfn}
    Let $\abs{G} \geq 2$. We say $G$ is \keyword{simple} iff $N \nsubgroup G \implies N = \set{1}$ or $N = G$; i.e. there are no proper non-trivial normal subgroups.
\end{ndfn}

\begin{nlemma}
    Let $P$ be a Sylow $p$-subgroup of $G$, then the number of Sylow $p$-subgroups of $G$ is
    \begin{equation*}
        n_{p} = [G : N_{G}(P)] = \frac{ \abs{G} }{ \abs{N_{G}(P)} }
    \end{equation*}
\end{nlemma}

\begin{nthm}[Sylow's 3rd Theorem]
    The number of Sylow $p$-subgroups of $G$ is of the form $n_{p} = 1 + kp$ with $k \in \Z_{+}$. Moreover, $n_{p} \mid \abs{G}$.
\end{nthm}
\begin{proof}
    Let $P$ be a Sylow $p$-subgroup with $\abs{P} = p^{n}$. Then, $p^{n+1} \nmid G$. Partition $G$ as
    \begin{equation*}
        G
        = \bigsqcup_{x} PxP
        = \paren*{\bigsqcup_{x \in N_{G}(P)} PxP} \bigcup \paren*{\bigsqcup_{x \notin N_{G}(P)} PxP}
    \end{equation*}
    where the union runs over the distinct equivalence classes.

    Now, $x \in N_{G}(P) \implies xPx^{-1} = P \implies xP = Px \implies PxP = PPx \implies PxP = Px$. So,
    \begin{equation*}
        \bigsqcup_{x \in N_{G}(P)} PxP
        = \bigsqcup_{x \in N_{G}(P)} Px
        = N_{G}(P).
    \end{equation*}
    This is because $P \subgroup N_{G}(P)$, and the union of all distinct right cosets of $P$ in $N_{G}(P)$ forms the full group $N_{G}(P)$.

    On the other hand, $x \notin N_{G}(P) \implies xPx^{-1} \neq P$. But, $P \cap xPx^{-1} \subgroup G$. So, $\abs{P \cap xPx^{-1}} = p^{m}$ with $0 \leq m < n$. As a result,
    \begin{equation*}
        \sum_{x} \abs{PxP}
        = \sum_{x} \frac{\abs{P}^{2}}{\abs{P \cap xPx^{-1}}}
        = \sum_{x} p^{2n-m}
        = p^{n+1} \sum_{x} p^{n-m-1}
        = p^{n+1} t.
    \end{equation*}
    Here, the summation runs over distinct equivalence classes outside of $N_{G}(P)$, and $t = \sum_{x} p^{n-m-1}$. Now, $P \subgroup N_{G}(P)$ and $p^{n} \mid \abs{P}$ implies that $p^{n} \mid \abs{N_{G}(P)}$. Thus, $\abs{N_{G}(P)} = p^{n}b$ for some $b \in \Z_{+}$.
    \begin{equation*}
        \abs{G}
        = \abs{N_{G}(P)} + \sum_{x} \abs{PxP}
        = \abs{N_{G}(P)} + p^{n+1} t
        \implies
        \frac{\abs{G}}{\abs{N_{G}(P)}} = 1 + \frac{p^{n+1}}{\abs{N_{G}(P)}} t
    \end{equation*}
    Using $n_{p} = \abs{G}/\abs{N_{G}(P)}$ gives $\displaystyle n_{p} = 1 + \frac{pt}{b}$.

    Now, $pt/b = n_{p} - 1 \in \Z$. So, $pt/b = r$ for some $r \in \Z$. And, $pt/b = r \implies pt = br \implies p \mid br$.

    If $p \mid b$, then $p^{n+1} \mid \abs{N_{G}(P)}$, and thus $p^{n+1} \mid \abs{G}$. This is a contradiction. Therefore, $p \mid r$, and we obtain $t/b = r/p \in \Z_{+}$. Thus, $t/b = k$ for some $k \in \Z_{+}$. Overall, $\displaystyle n_{p} = 1 + \frac{pt}{b} = 1 + pk$.

    Lastly, by Lagrange's theorem $n_{p} = [G : N_{G}(P)] \implies n_{p} \mid \abs{G}$.
\end{proof}

\begin{nex}\phantom{}
    \begin{enumerate}
        \item $A_{4}$ is not simple. (Because it has a single Sylow $2$-subgroup.)
        \item Let $\abs{G} = 2 p^{2}$ with $p$ prime and $p \neq 2$. Then, $G$ is not simple.
        \item Groups of the following order are not simple: $33, 51, 56, 65, 69$ and $100$.
    \end{enumerate}
\end{nex}

\section{Automorphisms}

\begin{ndfn}[Automorphism]
    An isomorphism from $G$ to $G$ is called an \keyword{automorphism}. The set of all automorphism of $G$ is denoted by $\Aut(G)$.
\end{ndfn}

Let $S$ be a non-empty set and $A(S)$ denote the set of all bijective functions from $S$ to $S$. Then, $A(S)$ forms a group under the composition of functions.

\begin{nprop}
    $\Aut(G) \subgroup A(G)$.
\end{nprop}
\begin{proof}
    Every isomorphism is bijective, and the identity map is an isomorphism from $G$ to $G$. Therefore, $\Aut(G)$ is a non-empty subset of $A(G)$.

    Take $\varphi, \psi \in \Aut(G)$. By definition, these are isomorphisms from $G$ to $G$. Therefore, $\psi^{-1}$ exists and it is also an isomorphism from $G$ to $G$. Composition of two isomorphisms is again an isomorphism, so $\varphi \circ \psi^{-1}$ is an isomorphism from $G$ to $G$. In particular, $\varphi \circ \psi^{-1} \in \Aut(G)$. Therefore, $\Aut(G) \subgroup A(G)$ by the subgroup criteria.
\end{proof}

\begin{nprop}
    Let $G$ be abelian. The map $\varphi: G \to G$ with $\varphi(x) = x^{-1}$ is an automorphism.
\end{nprop}
\begin{proof}
    This map is bijective because it has an inverse function, namely itself; $\forall x \in G$,
    \begin{equation*}
        (\varphi \circ \varphi) (x) = \varphi( \varphi(x) ) = \varphi( x^{-1} ) = x = I(x).
    \end{equation*}
    Moreover, for all $x, y \in G$, $\varphi(xy) = (xy)^{-1} = y^{-1} x^{-1} = x^{-1} y^{-1} = \varphi(x) \varphi(y)$.

    Therefore, $\varphi$ is a bijective homomorphism from $G$ to $G$, i.e. it is an automorphism of $G$.
\end{proof}

\begin{nprop}
    Fix $g \in G$. The map $T_{g} : G \to G$ with $T_{g}(x) = g x g^{-1}$ is an automorphism.
\end{nprop}
\begin{proof}
    This map is bijective because it has an inverse function, given by $T_{g^{-1}}$; $\forall x \in G$,
    \begin{gather*}
        (T_{g^{-1}} \circ T_{g}) (x) = T_{g^{-1}}( T_{g}(x) ) = T_{g^{-1}}( g x g^{-1} ) = g^{-1} ( g x g^{-1} ) g = x = I(x),\\
        (T_{g} \circ T_{g^{-1}}) (x) = T_{g}( T_{g^{-1}}(x) ) = T_{g}( g^{-1} x g ) = g ( g^{-1} x g ) g^{-1} = x = I(x).
    \end{gather*}

    Moreover, for all $x, y \in G$, $T_{g} (xy) = g (xy) g^{-1} = g x (g^{-1}g) y g^{-1} = (g x g^{-1})(g y g^{-1}) = T_{g}(x) T_{g}(y)$.

    Therefore, $T_{g}$ is a bijective homomorphism from $G$ to $G$, i.e. it is an automorphism of $G$.
\end{proof}

\begin{ndfn}
    For every fixed $g \in G$, the automorphism $x \mapsto g x g^{-1}$ is called an inner automorphism corresponding to $g$. We denote it by $T_{g}$. The set of all inner automorphisms is denoted by $\Inn(G)$. Thus, $\Inn(G) = \set{T_{g} \in \Aut(G) \st g \in G} \subseteq \Aut(G)$.
\end{ndfn}

\begin{nprop}
    Let $g \in G$. We have $T_{g} = I$ (identity) if and only if $g \in Z(G)$.
\end{nprop}
\begin{proof}
    Firstly, note that $T_{g}(x) = I(x) \iff g x g^{-1} = x \iff g x = x g$, for all $x \in G$.

    Now, if $g \in Z(G)$, then $gx = xg$ for all $x \in G$. Thus, $T_{g} = I$. On the other hand, if $T_{g} = I$, then $gx = xg$ for all $x \in G$. Therefore, $g \in Z(G)$.
\end{proof}

\begin{ncor}
    If $G$ is abelian then $\Inn(G) = \set{I}$.
\end{ncor}
\begin{proof}
    If $G$ is abelian, then $G = Z(G)$. So, $G/Z(G)$ is trivial. Therefore, $\Inn(G) = \set{I}$.
\end{proof}

\begin{nthm}
    $\Inn(G) \nsubgroup \Aut(G)$.
\end{nthm}
\begin{proof}
    Every inner automorphism is an automorphism, and the identity automorphism is an inner automorphism corresponding to $T_{1}$. Therefore, $\Inn(G)$ is a non-empty subset of $\Aut(G)$.

    Furthermore, note that for every $T_{h} \in \Inn(G)$, we have $T_{h}^{-1} = T_{h^{-1}}$ because for all $x \in G$,
    \begin{gather*}
        \paren*{T_{h^{-1}} \circ T_{h}} (x)
        = T_{h^{-1}} \paren{ T_{h}(x) }
        = h^{-1} \paren{ h x h^{-1} } h
        = \paren{h^{-1} h} x \paren{h^{-1} h}
        = x,\\
        \paren{ T_{h} \circ T_{h^{-1}} } (x)
        = T_{h} \paren{ T_{h^{-1}}(x) }
        = h \paren{ h^{-1} x h } h^{-1}
        = \paren{h h^{-1}} x \paren*{h h^{-1}}
        = x.
    \end{gather*}

    Now, take $T_{g}, T_{h} \in \Inn(G)$. Then, for all $x \in G$
    \begin{equation*}
        (T_{g} \circ T_{h}^{-1}) (x)
        = (T_{g} \circ T_{h^{-1}})(x)
        = g (h^{-1} x h) g^{-1}
        = (g h^{-1}) x (g h^{-1})^{-1}
        = T_{gh^{-1}}(x).
    \end{equation*}
    Thus, $T_{g} \circ T_{h}^{-1} = T_{gh^{-1}} \in \Inn(G)$. By the subgroup test, this shows that $\Inn(G) \subgroup \Aut(G)$.

    Next, take $\varphi \in \Aut(G)$ and $T_{g} \in \Inn(G)$. Then, for all $x \in G$
    \begin{equation*}
        (\varphi \circ T_{g} \circ \varphi^{-1})(x)
        = \varphi( T_{g}( \varphi^{-1}(x) ) )
        = \varphi( g \varphi^{-1}(x) g^{-1} )
        = \varphi( g ) \varphi( \varphi^{-1}(x) ) \varphi ( g^{-1} )
        = \varphi( g ) x \varphi ( g^{-1} ).
    \end{equation*}
    Since $\varphi(g) \in G$, we can write $\varphi( g ) x \varphi ( g^{-1} ) = \varphi( g ) x \varphi ( g )^{-1} = T_{\varphi(g)}(x)$.

    Thus, $\varphi \circ T_{g} \circ \varphi^{-1} = T_{\varphi(g)} \in \Inn(G)$. This shows that $\Inn(G) \nsubgroup \Aut(G)$.
\end{proof}

\begin{nthm}
    The map $\varphi: G \to \Inn(G)$ defined by $\varphi(g) = T_{g}$ is a surjective homomorphism.
\end{nthm}
\begin{proof}
    Firstly, note that for all $g,h \in G$ and $x \in G$,
    \begin{equation*}
        (T_{g} \circ T_{h})(x)
        = T_{g} ( T_{h}(x) )
        = g ( hxh^{-1} ) g^{-1}
        = (gh) x (h^{-1}g^{-1})
        = (gh) x (gh)^{-1}
        = T_{gh} (x).
    \end{equation*}
    Thus, $\varphi$ is a homomorphism $\varphi(gh) = T_{gh} = T_{g} \circ T_{h} = \varphi(g) \circ \varphi(h)$. Next, $\varphi$ is clearly a surjection because the pre-image of $T_{g} \in \Inn(G)$ is $g \in G$. So, $\varphi$ is a surjective homomorphism.
\end{proof}

\begin{nthm}
    $G/Z(G) \isom \Inn(G)$.
\end{nthm}
\begin{proof}
    We know that the map $\varphi : G \to \Inn(G)$ with $\varphi(g) = T_{g}$ is a surjective homomorphism.

    Also, $\ker\varphi = \set{g \in G \st \varphi(g) = I,\, \text{identity}} = \set{g \in G \st T_{g} = I} = \set{g \in G \st g \in Z(G)} = Z(G)$.

    Therefore, by the 1st isomorphism theorem, $G / \ker\varphi \isom \Inn(G) \implies G / Z(G) \isom \Inn(G)$.
\end{proof}

\begin{nex}
    Find $\Inn(D_{4})$. Here, $D_{4} = \set{1, a, a^{2}, a^{3}, ba, ba^{2}, ba^{3}}$.
\end{nex}

\begin{nthm}
    Let $G = \cyclic{a}$ be an infinite cyclic group. Then, $\Aut(G) = \set{I,T}$, where $I$ is the identity map and $T:G \to G$ is given by $T(x) = x^{-1}$.
\end{nthm}
\begin{proof}
    Let $\psi \in \Aut(G)$ be an arbitrary automorphism of $G$.

    Since $\psi$ is surjective, for every $x \in G$, we have some $y \in G$ such that $\psi(y) = x$. Since $G = \cyclic{a}$, we can write $x = a^{k}$ and $y = a^{r}$ for some $k, r \in \Z$. Thus, $x = \psi(y) = \psi(a^{r}) = (\psi(a))^{r}$.

    In other words, every element $x \in G$ can be written as a power of $\psi(a)$. Therefore, $\psi(a)$ generates $G$; $G = \cyclic{\psi(a)}$. However, every infinite cyclic group has exactly two generators, $a$ and $a^{-1}$. Therefore, either $\psi(a) = a$ or $\psi(a) = a^{-1}$.

    In the first case, $\psi(x) = \psi(a^{k}) = \psi(a)^{k} = a^{k} = x$. This is the identity map $I$.

    While, in the second case, $\psi(x) = \psi(a^{k}) = \psi(a)^{k} = \paren*{a^{-1}}^{k} = \paren*{a^{k}}^{-1} = x^{-1}$. This is the inverse map $T(x) = x^{-1}$.

    Therefore, $\Aut(G)$ has exactly two elements, $I$ and $T$.
\end{proof}

\begin{remark}
    we observe that if $G$ is an infinite cyclic group then $\Aut(G) \isom \Z_{2}$.
\end{remark}

\begin{nthm}
    Let $G$ be a cyclic group with $\abs{G} = n$. Then,
    \begin{equation*}
        \Aut(G) = \set{T_{m} \st T_{m}(x) = x^{m},\, 1 \leq m < n,\, \gcd(m,n) = 1}.
    \end{equation*}
    Consequently, $\abs{\Aut(G)} = \varphi(n)$, where $\varphi$ is the Euler totient function.
\end{nthm}
\begin{proof}
    Let $G = \cyclic{a}$. Since $G$ is a finite cyclic group, it has exactly $\varphi(n)$ generators, each of the form $a^{m}$ where $1 \leq m < n$ with $\gcd(m,n) = 1$.

    Take $\psi \in \Aut(G)$ and $x \in G$. Since $\psi$ is surjective, we have $x = \psi(y)$ for some $y \in G$. But, we can write $y = a^{r}$ for some $1 \leq r < n$. So, $x = \psi(y) = \psi(a^{r}) = \psi(a)^{r}$. Thus, every $x \in G$ can be written as a power of $\psi(a)$. In other words, $\psi(a)$ is a generator of $G$. Consequently, $\psi(a) = T_{m}(a) \equiv a^{m}$ for some $1 \leq m < n$ with $\gcd(m,n) = 1$.

    Next, suppose $1 \leq m_{1} < m_{2} < n$ with $\gcd(m_{1}, n) = \gcd(m_{2}, n) = 1$. Then,
    \begin{equation*}
        T_{m_{1}}(a) = T_{m_{2}}(a)
        \implies a^{m_{1}} = a^{m_{2}}
        \implies a^{m_{2} - m_{1}} = 1
        \implies n \mid (m_{2} - m_{1}).
    \end{equation*}
    This is a contradiction because $(m_{2} - m_{1}) < n$. Therefore, $T_{m_{1}} \neq T_{m_{2}}$. This shows that every element of $\Aut(G)$ has the form $T_{m}$ and that there are exactly $\varphi(n)$ such elements.

    Finally, if $x \in G$, then $x = a^{k}$ for some $1 \leq k \leq n$. So,
    \begin{equation*}
        T(x) = T(a^{k}) = T(a)^{k} = (a^{m})^{k} = (a^{k})^{m} = x^{m}.
    \end{equation*}
    Overall, $\Aut(G) = \set{T_{m} \st T_{m}(x) = x^{m},\, 1 \leq m < n,\, \gcd(m,n) = 1}$ and $\abs{\Aut(G)} = \varphi(n)$.
\end{proof}

\begin{ndfn}
    Let $n \geq 2$ be an integer. Define $U(n) = \set{m \in \Z \st 1 \leq m \leq n,\, \gcd(m, n) = 1}$ as the set of positive integers less than or equal to $n$ that are relatively prime with $n$.
\end{ndfn}

\begin{nthm}
    $U(n)$ is a group under multiplication modulo $n$.
\end{nthm}
\begin{proof}
    Firstly, note that $U(n)$ is non-empty because $\forall n$, $\gcd(1,n) = 1 \implies 1 \in U(n)$. Moreover, modular multiplication is associative and commutative.

    Take $x,y \in U(n)$. Then, $ax+bn = 1$ and $cy + dn = 1$ for $a,b,c,d \in \Z$. Then,
    \begin{equation*}
        1
        = (1)(1)
        = (ax+bn)(cy+dn)
        = acxy + adxn + bcny + bdn^{2}
        = (ac)(xy) + (adx + bcy + bdn)n.
    \end{equation*}
    Thus, $\gcd(xy, n) = 1$ because $ac, adx + bcy + bdn \in \Z$. So, the closure property also holds.

    Next, $\forall x \in U(n)$, $1 x = x \pmod n$. Therefore, $1 \in U(n)$ is the identity element.

    Also, $\gcd(x,n) = 1 \implies ax + bn = 1$ for some $a, b \in \Z$. So, $ax \equiv 1 \pmod n$. Therefore, the inverse of $x \in U(n)$ is $a \pmod n \in U(n)$.


    Overall, $U(n)$ satisfies all of the group axioms.
\end{proof}

\begin{remark}
    The Euler's totient function $\varphi(n)$ counts the number of positive integers less than or equal to $n$ that are relatively prime with $n$. So, clearly, $\abs{U(n)} = \varphi(n)$.
\end{remark}

\begin{nthm}
    Let $G$ be a cyclic group with $\abs{G} = n$. Then, $\Aut(G) \isom U(n)$.
\end{nthm}
\begin{proof}
    Define the map $\psi : \Aut(G) \to U(n)$ by $\psi(T_{m}) = m$, where
    \begin{gather*}
        \Aut(G) = \set{T_{m} \st T_{m}(x) = x^{m},\, 1 \leq m < n,\, \gcd(m,n) = 1},\\
        U(n) = \set{m \in \Z \st 1 \leq m \leq n,\, \gcd(m, n) = 1}.
    \end{gather*}

    Now, for all $T_{x}, T_{y} \in \Aut(G)$, $\psi(T_{x}) = \psi(T_{y}) \implies x = y \implies T_{x} = T_{y}$.

    Therefore, $\psi$ is injective. Since $\abs{\Aut(G)} = \varphi(n)$ and $\psi$ is injective, we conclude that $\psi$ is surjective. Next, take $T_{x}, T_{y} \in \Aut(G)$ with $1 \leq x, y < n$ and $\gcd(x,n) = \gcd(y,n) = 1$. Then, $\forall g \in G$ we obtain
    \begin{equation*}
        (T_{x} \circ T_{y})(g)
        = T_{x}( T_{y}(g) )
        = T_{x}( g^{y} )
        = (g^{y})^{x}
        = g^{xy \,(\mathrm{mod}\, n)}
        = T_{xy \,(\mathrm{mod}\, n)}(g).
    \end{equation*}
    So, $\psi(T_{x} T_{y}) = \psi( T_{xy \,(\mathrm{mod}\, n)} ) = xy \,(\mathrm{mod}\, n) = \psi(T_{x}) \psi(T_{y})$.

    Therefore, $\psi$ is an isomorphism and we obtain $\Aut(G) \isom U(n)$.
\end{proof}

\section{Characteristic Subgroups}

\begin{ndfn}
    We call $H \subgroup G$ a characteristic subgroup of $G$ if, $\forall T \in \Aut(G)$, $T(H) \subseteq H$.
\end{ndfn}

\begin{negg}
    Consider $C_{4}$, the cyclic group of order $4$. We have $\Aut(C_{4}) \isom U(4) = \set{1,3}$. Thus, $\Aut(C_{4}) = \set{T_{1}, T_{3}}$ with $T_{1}(x) = x = I(x)$ and $T_{3}(x) = x^{3}$.

    Let $H = \set{1, a^{2}} \subgroup G$. Then,
    \begin{gather*}
        T_{1}(H) = I(H) = \set{I(1), I(a^{2})} = \set{1, a^{2}} = H,\\
        T_{3}(H) = \set{T_{3}(1), T_{3}(a^{2})} = \set{1, a^{6}} = \set{1, a^{2}} = H.
    \end{gather*}
    Therefore, $H$ is a characteristic subgroup of $H$.
\end{negg}

\begin{remark}
    Both $\set{1} \subgroup G$ and $G \subgroup G$ are always characteristic subgroups of $G$.
\end{remark}

\begin{nprop}
    If $H \subgroup G$ is a characteristic subgroup of $G$, then $H \nsubgroup G$.
\end{nprop}
\begin{proof}
    Let $H \subgroup G$ be a characteristic subgroup of $G$. This means, $T(H) \subseteq H$ for all $T \in \Aut(G)$. In particular, $\forall g \in G$, $T_{g}(H) \subseteq H$ because $T_{g} \in \Inn(G) \subseteq \Aut(G)$.

    Then, for all $g \in G$ and $h \in H$, $T_{g}(H) \subseteq H \implies T_{g}(h) \in T_{g}(H) \subseteq H \implies g h g^{-1} \in H$.

    Therefore, $H \nsubgroup G$.
\end{proof}

\begin{remark}
    The converse of this proposition is not true. For example, consider the Klein-$4$ group $V_{4} = \set{1, a, b, ab}$ and take $H = \set{1, a}$. Consider the automorphism $T: V_{4} \to V_{4}$ such that
    \begin{equation*}
        T(1) = 1, \quad
        T(a) = b, \quad
        T(b) = a, \quad
        T(ab) = ab.
    \end{equation*}
    Then, $T(H) = \set{T(1), T(a)} = \set{1, b} \nsubseteq H$.
\end{remark}

\begin{nprop}
    Suppose for some $g \in G$, $g^{2} \neq 1$. Then, $G$ has a non-trivial automorphism.
\end{nprop}
\begin{proof}
    Suppose $G$ is abelian. Consider the map $T: G \to G$ with $T(g) = g^{-1}$. This is an automorphism. Moreover, it is not the identity automorphism because otherwise
    \begin{equation*}
        T(g) = I(g)
        \implies
        g^{-1} = g
        \implies
        g^{2} = 1,
    \end{equation*}
    for all $g \in G$. This is false by hypothesis. Therefore, $\Aut(G)$ has a non-trivial element.

    If $G$ is non-abelian, then $Z(G) \neq G$. This means, $\abs{G / Z(G)} > 1$. Thus,
    \begin{equation*}
        \Inn(G) \isom G / Z(G)
        \implies
        \abs{\Inn(G)} = \abs{G / Z(G)} > 1
        \implies
        \abs{\Aut(G)} \geq \abs{\Inn(G)} > 1.
    \end{equation*}
    Here we used $\Inn(G) \subseteq \Aut(G)$. Therefore, again, $\Aut(G)$ has a non-trivial element.
\end{proof}

\section{Group Actions \& Cayley's Theorem}
\begin{nthm}[Cayley's Theorem]
    Every group $G$ is isomorphic to a subgroup of a permutation group. In particular, if $\abs{G} = n$, then $G \isom H$ with $H \subgroup S_{n}$.
\end{nthm}
\begin{proof}
    Let $S_{G}$ be the group of permutations of $G$. Define the map $\varphi : G \to S_{G}$ by $\varphi(x) = t_{x}$, where $t_{x} : G \to G$ is a map such that $t_{x}(g) = xg$. We need to check that $\forall x \in G$, $t_{x}$ is indeed a bijection (permutation of $G$).

    To this end, note that $t_{x}$ is injective because for all $a, b \in G$ we have
    \begin{equation*}
        t_{x}(a) = t_{x}(b) \implies xa = xb \implies a = b.
    \end{equation*}
    Also, $\forall b \in G$ we have $x^{-1}b \in G$ such that $t_{x}\paren*{x^{-1} b} = x \paren*{x^{-1} b} = b$. So, $t_{g}$ is also surjective. Overall, $t_{g}$ is a bijection (permutation of $G$).

    Next, note that for all $x, y \in G$, $t_{x} \circ t_{y} = t_{xy}$ because $\forall g \in G$
    \begin{equation*}
        t_{x y}(g)
        = \paren*{x y} g
        = x \paren*{ yg }
        = t_{x} \paren*{ yg }
        = t_{x} \paren*{ t_{y}(g) }
        = \paren*{t_{x} \circ t_{y}} (g).
    \end{equation*}
    As a result, $\varphi$ is a homomorphism because $\varphi(xy) = t_{xy} = t_{x} \circ t_{y} = \varphi(x) \circ \varphi(y)$. Also,
    \begin{align*}
        \ker\varphi
        = \set{x \in G \st t_{x} = I}
        = \set{x \in G \st t_{x}(g) = I(g), \forall g \in G}
        = \set{x \in G \st xg = g, \forall g \in G}
        = \set{1}.
    \end{align*}

    Now, by the 1st isomorphism theorem, $G / \ker\varphi \isom \img\varphi \implies G \isom H$,
    where $H = \img\varphi$ is some subgroup of $S_{G}$. In other words, $G$ is isomorphic to a subgroup of a permutation group. If $\abs{G} = n$, then $S_{G}$ reduces to $S_{n}$.
\end{proof}

\begin{ndfn}
    A group action of $G$ on a non-empty set $A$ is a map $G \times A \to A$ with $(g, a) \mapsto g \cdot a$ such that $\forall g_{1}, g_{2} \in G$ and $\forall a \in A$, $g_{1} \cdot \paren*{ g_{2} \cdot a } = (g_{1} g_{2}) \cdot a$ and $1 \cdot a = a$.
\end{ndfn}

\begin{negg}
    Let $A$ be a non-empty set. Consider the map $\cdot : G \times A \to A$ with $g \cdot a = a$, for all $a \in A$. It is easy to check that it satisfies the conditions of being a group action. This is called the trivial action of $G$ on $A$.
\end{negg}

\begin{negg}
    Let $V$ be a vector space over $\R$ and consider the group $\R^{*}$ under the usual multiplication of real numbers. The map $\cdot : \R^{*} \times V \to V$ defined as $\alpha \cdot \vec{v} = \alpha \vec{v}$ is a group action because $1 \cdot \vec{v} = 1\vec{v} = \vec{v}$ and $\alpha \cdot \paren*{ \beta \cdot \vec{v} } = \alpha \cdot \paren*{ \beta\vec{v} } = \alpha\beta\vec{v} = \paren*{\alpha\beta} \vec{v} = \paren*{\alpha\beta} \cdot \vec{v}$.
\end{negg}

\begin{negg}
    Let $\Z$ be the additive group of integers which acts on itself by $z \cdot a = z + a$, for all $z, a \in \Z$. This is a group action because for all $a \in \Z$, $0 \cdot a = 0 + a = a$ and
    \begin{equation*}
        z_{1} \cdot (z_{2} \cdot a)
        = z_{1} \cdot (z_{2} + a)
        = z_{1} + (z_{2} + a)
        = (z_{1} + z_{2}) + a
        = (z_{1} + z_{2}) \cdot a
    \end{equation*}
    for all $z_{1}, z_{2} \in \Z$.
\end{negg}

\begin{negg}
    Let $\R$ be the additive group of real numbers, and define $\cdot : \R \times \R^{2} \to \R^{2}$ by
    \begin{equation*}
        r \cdot (x,y) = (x + ry, y).
    \end{equation*}
    Now, for all $(x,y) \in \R^{2}$ we have $0 \cdot (x,y) = (x + 0y, y) = (x, y)$. And, for all $r_{1}, r_{2} \in \R$,
    \begin{equation*}
        r_{1} \cdot (r_{2} \cdot (x,y))
        = r_{1} \cdot (x + r_{2}y, y)
        = (x + r_{2}y + r_{1}y, y)
        = (x + (r_{1} + r_{2})y, y)
        = (r_{1} + r_{2}) \cdot (x, y).
    \end{equation*}
    Therefore, this is a group action of $\R$ on $\R^{2}$.
\end{negg}

\begin{ndfn}
    Let $\cdot : G \times A \to A$ be a group action. The kernel of this group action, denoted by $K$, is the set of elements of $G$ that act trivially on every element of $A$; i.e.
    \begin{equation*}
        K = \set{g \in G \st g \cdot a = a, \, \forall a \in A}.
    \end{equation*}
    The stabiliser of $a \in A$ in $G$, denoted by $G_{a}$, is the set of elements of $G$ that fix $a$; i.e.
    \begin{equation*}
        G_{a} = \set{g \in G \st g \cdot a = a}.
    \end{equation*}
\end{ndfn}

\begin{ndfn}
    An action is said to be faithful if its kernel is the trivial subgroup of $G$.
\end{ndfn}

\begin{nprop}
    Let $G$ be a group acting on a non-empty set $A$. Then,
    \begin{enumerate}
        \item $K \nsubgroup G$.
        \item $G_{a} \subgroup G$.
    \end{enumerate}
\end{nprop}
\begin{proof}
    \phantom{.}
    \begin{enumerate}
        \item Note that $K$ is non-empty because $1 \in K$. Take $k_{1}, k_{2} \in K$ and $a \in A$. Then,
        \begin{equation*}
            k_{2}^{-1} \cdot a
            = k_{2}^{-1} \cdot (k_{2} \cdot a)
            = (k_{2}^{-1} k_{2}) \cdot a
            = 1 \cdot a
            = a.
        \end{equation*}
        So, $(k_{1}k_{2}^{-1}) \cdot a = k_{1} \cdot (k_{2}^{-1} \cdot a) = k_{1} \cdot a = a$.

        Therefore, $k_{1} k_{2}^{-1} \in K$, and we conclude that $K \subgroup G$. Next, for every $g \in G$ and $k \in K$,
        \begin{equation*}
            (gkg^{-1}) \cdot a
            = g \cdot (k \cdot (g^{-1} \cdot a))
            = g \cdot (g^{-1} \cdot a)
            = (gg^{-1}) \cdot a
            = 1 \cdot a
            = a.
        \end{equation*}
        Therefore, $gkg^{-1} \in K$, which means $K \nsubgroup G$.

        \item We have $1 \in G_{a}$ because $1 \cdot a = a$, so this is a non-empty subset of $G$. Now, take $x, y \in G_{a}$. Then, $x \cdot a = y \cdot a = a$. So, $y^{-1} \cdot a = y^{-1} \cdot (y \cdot a) = (y^{-1} y) \cdot a = 1 \cdot a = a$. As a result,
        \begin{equation*}
            (x y^{-1}) \cdot a = x \cdot (y^{-1} \cdot a) = x \cdot a = a.
        \end{equation*}
        Therefore, $x y^{-1} \in G_{a}$, which means $G_{a} \subgroup G$.
        \qedhere
    \end{enumerate}
\end{proof}

\begin{nprop}\label{thm:action-to-hom}
    Let $\cdot : G \times A \to A$ be a group action. For each fixed $g \in G$, define a map $\gamma_{g} : A \to A$ by $\gamma_{g}(a) = g \cdot a$. Then,
    \begin{enumerate}
        \item $\gamma_{g}$ is a permutation of $A$.
        \item The map $\varphi : G \to S_{A}$ given by $\varphi(g) = \gamma_{g}$ is a homomorphism.
    \end{enumerate}
\end{nprop}
\begin{proof}
    \phantom{.}
    \begin{enumerate}
        \item Given $\gamma_{g}$, consider $\gamma_{g^{-1}}$. This exists because $g \in G \implies g^{-1} \in G$. Then,
        \begin{gather*}
            (\gamma_{g^{-1}} \circ \gamma_{g})(a)
            = \gamma_{g^{-1}} (\gamma_{g}(a))
            = \gamma_{g^{-1}} (g \cdot a)
            = g^{-1} \cdot (g \cdot a)
            = (g^{-1}g) \cdot a
            = 1 \cdot a
            = a,\\
            (\gamma_{g} \circ \gamma_{g^{-1}})(a)
            = \gamma_{g} (\gamma_{g^{-1}}(a))
            = \gamma_{g} (g^{-1} \cdot a)
            = g \cdot (g^{-1} \cdot a)
            = (gg^{-1}) \cdot a
            = 1 \cdot a
            = a.
        \end{gather*}
        So, $\gamma_{g}$ has a two-sided inverse. Therefore, $\gamma_{g}$ is a bijection; i.e. a permutation of $A$.

        \item For every $g, h \in G$ and $a \in A$, we have
        \begin{equation*}
            (\gamma_{g} \circ \gamma_{b}) (a)
            = \gamma_{g} (\gamma_{h} (a))
            = g \cdot (h \cdot a)
            = (gh) \cdot a
            = \gamma_{gh}(a).
        \end{equation*}
        Therefore, $\varphi(gh) = \gamma_{gh} = \gamma_{g} \circ \gamma_{h} = \varphi(g)\varphi(h)$. Therefore, $\varphi$ is a homomorphism.
        \qedhere
    \end{enumerate}
\end{proof}

\begin{nprop}
    Let $G$ act on $A$. The relation $\sim$ on $A$ defined by $a \sim b$ if and only if $a = g \cdot b$ for some $g \in G$ is an equivalence relation. Moreover, for each $a \in A$, the number of elements in the equivalence class containing $a$ is $[G : G_{a}]$ the index of the stabilizer of $a$.
\end{nprop}
\begin{proof}
    This is an equivalence relation because
    \begin{itemize}
        \item For all $a \in A$, $a = 1 \cdot a =$. So, $a \sim a$.
        \item Let $a \sim b$. So, $a = g \cdot b$ for some $g \in G$. Then, $b = g^{-1} \cdot a = g^{-1} \cdot (g \cdot a) = (g^{-1}g) \cdot a = 1 \cdot a = a$. Thus, $b \sim a$.
        \item Let $a \sim b$ and $b \sim c$. So, for some $g, h \in G$ we have $a = g \cdot b$ and $b = h \cdot c$. Then, $a = g \cdot b = g \cdot (h \cdot c) = (gh) \cdot c$. Therefore, $a \sim c$.
    \end{itemize}
    Let $\mathrm{Cl}(a) = \set{g \cdot a \st g \in G}$ denote the equivalence class containing $a$.

    Next, consider the map $\varphi : \mathrm{Cl}(a) \to H$, where $H = \set{gG_{a} \st g \in G}$ is the set of left cosets of $G_{a}$ in $G$. Define it as $\varphi(g \cdot a) = g G_{a}$. To show that this map is well-defined note that
    \begin{equation*}
        g \cdot a = h \cdot a
        \iff h^{-1} \cdot (g \cdot a) = h^{-1} \cdot (h \cdot a)
        \iff (h^{-1} g) \cdot a = (h^{-1} h) \cdot a = a.
    \end{equation*}
    And, $(h^{-1} g) \cdot a = a \implies h^{-1} g \in G_{a} \implies gG_{a} = hG_{a} \implies \varphi(g \cdot a) = \varphi(h \cdot a)$.

    This map is clearly surjective, with the pre-image of $gG_{a}$ being $g \cdot a$. Also, it is injective because
    \begin{equation*}
        \varphi(g \cdot a) = \varphi(h \cdot a)
        \implies gG_{a} = hG_{a}
        \implies h^{-1}g \in G_{a}
        \implies (h^{-1}g) \cdot a = a
        \implies g \cdot a = h \cdot a.
    \end{equation*}

    Therefore, $\varphi$ is a bijection and we obtain that $\abs{\mathrm{Cl}(a)} = \abs{H} = [G : G_{a}]$.
\end{proof}

\begin{ndfn}
    Let $\cdot : G \times A \to A$ be a group action. Then,
    \begin{enumerate}
        \item The equivalence class $\set{g \cdot a \st g \in G}$ is called the orbit of $G$ containing $a$.
        \item The action of $G$ on $A$ is called transitive if there is only one orbit. In other words, for all $a, b \in A$, there is some $g \in G$ such that $a = g \cdot b$.
    \end{enumerate}
\end{ndfn}

Every group action $\cdot : G \times A \to A$ gives rise to a homomorphism $\varphi : G \to S_{A}$ with $\varphi(g) = \gamma_{g}$, such that $\gamma_{g}(a) = g \cdot a$. In other words, the group action assigns a permutation from $S_{A}$ to each element $g \in G$. Moreover, each homomorphism $G \to S_{A}$ comes from a group action. The following result formally proves this statement.

\begin{nprop}
    Let $A$ be a non-empty set. Then, there is a bijection between the actions of $G$ on $A$ and the homomorphisms of $G$ into $S_{A}$.
\end{nprop}
\begin{proof}
    Let $\varphi : G \to S_{A}$ be a group homomorphism, and consider the map $\cdot : G \times A \to A$ with $g \cdot a = \varphi(g)(a)$. Since, $\varphi$ is a homomorphism, the identity element $1$ maps to the identity map $I$. Therefore, $1 \cdot a = \varphi(1)(a) = I(a) = a$ for all $a \in A$. Similarly,
    \begin{equation*}
        g \cdot (h \cdot a)
        = g \cdot (\varphi(h)(a))
        = \varphi(g)\bigg( \varphi(h)(a) \bigg)
        = (\varphi(g) \circ \varphi(h))(a)
        = \varphi(gh)(a)
        = gh \cdot a,
    \end{equation*}
    for all $g, h \in G$ and $a \in A$. Therefore, $\cdot$ is a group action of $G$ on $A$; every homomorphism $G \to S_{A}$ induces an action of $G$ on $A$.

    In proposition~(\ref{thm:action-to-hom}), we have already proved that every group action induces a homomorphism $G \to S_{A}$. Therefore, there is a bijection between the actions of $G$ on $A$ and the homomorphisms of $G$ into $S_{A}$.
\end{proof}

\begin{ndfn}
    A permutation representation of $G$ on $A$ is a homomorphism of $G$ into $S_{A}$.
\end{ndfn}

Since each group action $\cdot : G \times A \to A$ induces a homomorphism $\varphi: G \to S_{A}$ with $\varphi(g)(a) = g \cdot a$, we say that $\varphi$ is a permutation representation of $G$ on $A$ induced by the action $g \cdot a$. The image $\varphi(G)$ is a subgroup of $S_{A}$, and by the 1st isomorphism theorem $G / \ker \varphi \isom \img \varphi$.

\section{Representation: Left Multiplication Action of \texorpdfstring{$G$}{G} on \texorpdfstring{$G$}{G}}

Let $\cdot : G \times G \to G$ be a group action with $g \cdot a = ga$, where $ga$ is the usual multiplication in $G$. This is a group action because $g \cdot (h \cdot a) = g \cdot (ha) = g(ha) = (gh)a = (gh) \cdot a$ and $1 \cdot a = 1a = a$. This action is known as the group acting on itself by left multiplication.

If the group is additive, then we write $g \cdot a = g + a$ and call this action left translation.

The permutation representation $\varphi : G \to S_{G}$ induced by this action is called the `permutation representation due to left multiplication action of $G$ on $G$.'

Suppose $\abs{G} = n$. Then, we can make the identification $\set{g_{1}, g_{2}, \dots, g_{n}} \leftrightarrow \set{1, \dots, n}$. For each $g \in G$ we have the permutation $\gamma_{g} : G \to G$ with $\gamma_{g}(a) = g \cdot a = ga$ for $a \in G$. In other words, $\gamma_{g}$ maps $i$ to $j$; note that $i \leftrightarrow g_{i}$ and $j \leftrightarrow g_{j}$. Note further, that a different labelling of the group elements will lead to a different (but equivalent) representation.

The following examples gives a clear method for finding the permutation representation due to left multiplication action of $G$ on $G$.

\begin{negg}
    Consider the Klein-4 group $V_{4} = \set{1,a,b,ab}$ with $a^{2} = b^{2} = (ab)^{2} = 1$. We need to find the permutation representation due to left multiplication action of $V_{4}$ on itself. Denote this representation by the homomorphism $\varphi : V_{4} \to S_{4}$ with $g \mapsto \gamma_{g}$, and make the identification $\set{1,a,b,ab} \leftrightarrow \set{1,2,3,4}$.

    Now,
    \begin{gather*}
        1 \cdot \set{1, a, b, ab}
        = \set{1, 1a, 1b, 1ab}
        = \set{1, a, b, ab}
        \to
        \begin{pmatrix*}
            1 & 2 & 3 & 4\\
            1 & 2 & 3 & 4\\
        \end{pmatrix*}
        = I,\\
        a \cdot \set{1, a, b, ab}
        = \set{a1, a^{2}, ab, a^{2}b}
        = \set{a, 1, ab, b}
        \to
        \begin{pmatrix*}
            1 & 2 & 3 & 4\\
            2 & 1 & 4 & 3\\
        \end{pmatrix*}
        = (1 2)(3 4),\\
        b \cdot \set{1, a, b, ab}
        = \set{b1, ba, b^{2}, bab}
        = \set{b, ab, 1, a}
        \to
        \begin{pmatrix*}
            1 & 2 & 3 & 4\\
            3 & 4 & 1 & 2\\
        \end{pmatrix*}
        = (1 3)(2 4),\\
        ab \cdot \set{1, a, b, ab}
        = \set{ab1, aba, ab^{2}, (ab)^{2}}
        = \set{ab, b, a, 1}
        \to
        \begin{pmatrix*}
            1 & 2 & 3 & 4\\
            4 & 3 & 2 & 1\\
        \end{pmatrix*}
        = (1 4)(2 3).
    \end{gather*}
    Therefore, a permutation representation of $V_{4}$ on itself due to left multiplication action is
    \begin{equation*}
        \gamma_{1} = I, \quad
        \gamma_{a} = (1 2)(3 4), \quad
        \gamma_{b} = (1 3)(2 4), \quad
        \gamma_{ab} = (1 4)(2 3).
    \end{equation*}
    (Note that we only needed to find the permutations $\gamma_{g}$ corresponding to the generators $a, b \in V_{4}$ because then $\gamma_{ab} = \gamma_{a} \circ \gamma_{b}$.)

    Under this representation $\varphi(V_{4}) = \set{ I, (1 2)(3 4), (1 3)(2 4), (1 4)(2 3) } \subgroup S_{4}$. Moreover, $V_{4} \isom \varphi(V_{4})$ because the homomorphism $\varphi$ is injective.
\end{negg}

\begin{nex}
    Show that the action of $G$ on itself by left multiplication is transitive and faithful, and find the stabiliser of every $g \in G$.
\end{nex}

\begin{negg}
    Consider the dihedral group $D_{3} = \set{1,a,a^{2}, b, ba, ba^{2}}$ with $a^{3} = b^{2} = 1$. We need to find the permutation representation due to left multiplication action of $D_{3}$ on itself. Denote this representation by the homomorphism $\varphi : D_{3} \to S_{6}$ with $g \mapsto \gamma_{g}$, and make the identification $\set{1,a,a^{2}, b, ba, ba^{2}} \leftrightarrow \set{1,2,3,4,5,6}$.

    It is enough to find the images of the generators $a, b$. So,
    \begin{gather*}
        a \cdot \set{1,a,a^{2}, b, ba, ba^{2}}
        = \set{a1,aa,aa^{2}, ab, aba, aba^{2}}
        = \set{a,a^{2},1, ba^{2}, b, ba}\\
        \therefore \gamma_{a}
        = \begin{pmatrix*}
            1 & 2 & 3 & 4 & 5 & 6\\
            2 & 3 & 1 & 6 & 4 & 5\\
        \end{pmatrix*}
        = (1 2 3)(4 6 5),
        \shortintertext{and}
        b \cdot \set{1,a,a^{2}, b, ba, ba^{2}}
        = \set{b1, ba, ba^{2}, b^{2}, b^{2}a, b^{2}a^{2}}
        = \set{b, ba, ba^{2}, 1, a, a^{2}}\\
        \therefore \gamma_{b}
        = \begin{pmatrix*}
            1 & 2 & 3 & 4 & 5 & 6\\
            4 & 5 & 6 & 1 & 2 & 3\\
        \end{pmatrix*}
        = (1 4)(2 5)(3 6).
    \end{gather*}
    Now, we can easily obtain the permutations corresponding to other elements. For example,
    \begin{gather*}
        \gamma_{a^{2}}
        = \gamma_{a} \circ \gamma_{a}
        = (1 2 3)(4 6 5) (1 2 3)(4 6 5)
        = (1 3 2)(4 5 6),\\
        \gamma_{ba}
        = \gamma_{b} \circ \gamma_{a}
        = (1 4)(2 5)(3 6) (1 2 3)(4 6 5)
        = (1 5)(2 6)(3 4),
    \end{gather*}
    and so on.
\end{negg}

\begin{nex}
    Find the permutation representation due to left multiplication of the groups $\Z_{4}$, $Q_{8}$, $D_{4}$ and $\Z_{2} \oplus \Z_{2}$.
\end{nex}

\section{Representation: Left Multiplication of \texorpdfstring{$G$}{G} on Cosets}

\textbf{Note:} This section is non-examinable.

Other permutation representations of $G$ can be obtained by considering other group actions of $G$. One particular choice, which generalises the previous construction, is to let the underlying set $A$ be a set of cosets of $H \subgroup G$. This construction proceeds as follows:

Let $H \subgroup G$ and $A = \set{aH \st a \in G}$ be the set of distinct left cosets of $H$ in $G$. If $G$ is finite, then $\abs{A} = m$ for some $m \in \Z_{+}$ with $m \leq \abs{G}$. This allows us to make the identification
\begin{equation*}
    A = \set{a_{1}H, a_{2}H, \dots, a_{m}H} \leftrightarrow \set{1, 2, \dots, m}.
\end{equation*}
Define a map $\cdot : G \times A \to A$ by $g \cdot (aH) = (ga)H$. This is well defined because
\begin{equation*}
    aH = bH
    \implies gaH = abH
    \implies g \cdot (aH) = g \cdot (bH).
\end{equation*}
Also, it satisfies the axioms of being a group actions because, $1 \cdot (aH) = (1a) H = aH$ and
\begin{equation*}
    g_{1} \cdot (g_{2} \cdot (aH))
    = g_{1} \cdot ((g_{2}a) H) 
    = (g_{1}g_{2}a) H
    = (g_{1}g_{2}) \cdot (aH)
\end{equation*}
for all $g_{1}, g_{2} \in G$ and $aH \in A$.

As a group action, this induces a permutation representation of $G$ on the set of cosets $A$, denoted by $\pi_{H} : G \to S_{\abs{A}}$ where $\pi_{H}(g) = \gamma_{g}$. Again, note that a different labelling of the cosets will lead to a different (but equivalent) representation.

\begin{negg}
    Consider the dihedral group $D_{3} = \set{1,a,a^{2}, b, ba, ba^{2}}$ with $a^{3} = b^{2} = 1$, and $H = \set{1, b} \subgroup G$. Let $A$ be the set of distinct left cosets of $H$. We need to find the permutation representation of $G$ on $A$. Denote this representation by the homomorphism $\pi_{H} : D_{3} \to S_{3}$ with $g \mapsto \gamma_{g}$, and make the identification $\set{H, aH, a^{2}H} \leftrightarrow \set{1,2,3}$. (Note: $\abs{A} = [G : H] = 3$.)

    It is enough to find the images of the generators $a, b$. So,
    \begin{gather*}
        a \cdot \set{H, aH, a^{2}H}
        = \set{aH, a^{2}H, a^{3}H}
        = \set{aH, a^{2}H, H}\\
        \therefore \gamma_{a}
        = \begin{pmatrix*}
            1 & 2 & 3\\
            2 & 3 & 1\\
        \end{pmatrix*}
        = (1 2 3),
        \shortintertext{and}
        b \cdot \set{H, aH, a^{2}H}
        = \set{bH, baH, ba^{2}H}
        = \set{bH, a^{2}bH, abH}
        = \set{H, a^{2}H, aH}\\
        \therefore \gamma_{b}
        = \begin{pmatrix*}
            1 & 2 & 3\\
            1 & 3 & 2\\
        \end{pmatrix*}
        = (2 3).
    \end{gather*}
    Now, we can easily obtain the permutations corresponding to the other elements as
    \begin{gather*}
        \gamma_{a^{2}}
        = \gamma_{a} \gamma_{a}
        = (1 2 3) (1 2 3)
        = (1 3 2),
    \end{gather*}
    and so on. Overall,
    \begin{equation*}
        \pi_{H}(G)
        = \pi_{H}(\set{1,a,a^{2},b,ba,ba^{2}})
        = \set{I, (1 2 3), (1 3 2), (2 3), (1 3), (1 2)} = S_{3}.
    \end{equation*}
    Clearly, this homomorphism is injective. So, $D_{3} \isom S_{3}$. \eggqed
\end{negg}

Note that if $H = \set{e}$, then $A = G$ and we obtain the previous case where $G$ acts on itself by left multiplication.

\begin{nex}
    Find the permutation representation of $D_{4}$ due to the left multiplication action on $H = \set{1, b} \subgroup G$.
\end{nex}

\section{Representation: Conjugation Action of \texorpdfstring{$G$}{G} on \texorpdfstring{$G$}{G}}

Consider $\cdot : G \times G \to G$ with $g \cdot a = gag^{-1}$. This is a group action because $1 \cdot a = 1a1^{-1} = a$ and $g \cdot (h \cdot a) = g \cdot (hah^{-1}) = g(ha)g^{-1} = (gh)a(gh)^{-1} = (gh) \cdot a$. This action is known as the group acting on itself by conjugation.

The orbit of $G$ acting on $G$ containing $a \in G$ is simply the conjugacy class of $a$.

\begin{remark}
    If $G$ is abelian, then this action is the trivial action because for all $g \in G$ and $a \in G$,
    \begin{equation*}
        g \cdot a = g a g^{-1} = g g^{-1} a = a.
    \end{equation*}
\end{remark}

Like other actions, this one can also be used to obtain the permutation representations of $G$. We use the same method as described in the previous sections.

\begin{negg}
    Consider the dihedral group $D_{3} = \set{1,a,a^{2}, b, ba, ba^{2}}$ with $a^{3} = b^{2} = 1$. We need to find the permutation representation due to conjugation action of $D_{3}$ on itself. Note that $\abs{D_{3}} = 6$. Denote this representation by the homomorphism $\varphi : D_{3} \to S_{6}$ with $g \mapsto \gamma_{g}$, and make the identification $\set{1,a,a^{2}, b, ba, ba^{2}} \leftrightarrow \set{1,2,3,4,5,6}$.

    It is enough to find the images of the generators $a, b$. So,
    \begin{gather*}
        a \cdot \set{1,a,a^{2}, b, ba, ba^{2}}
        = \set{a1a^{-1},aaa^{-1},aa^{2}a^{-1}, aba^{-1}, abaa^{-1}, aba^{2}a^{-1}}
        = \set{1,a,a^{2}, ba, ba^{2}, b}\\
        \therefore \gamma_{a}
        = \begin{pmatrix*}
            1 & 2 & 3 & 4 & 5 & 6\\
            1 & 2 & 3 & 5 & 6 & 4\\
        \end{pmatrix*}
        = (4 5 6),
        \shortintertext{and}
        b \cdot \set{1,a,a^{2}, b, ba, ba^{2}}
        = \set{b1b^{-1},bab^{-1},ba^{2}b^{-1},bbb^{-1},bbab^{-1},bba^{2}b^{-1}}
        = \set{1, a^{2}, a, b, ba^{2}, ba}\\
        \therefore \gamma_{b}
        = \begin{pmatrix*}
            1 & 2 & 3 & 4 & 5 & 6\\
            1 & 3 & 2 & 4 & 6 & 5\\
        \end{pmatrix*}
        = (2 3)(5 6).
    \end{gather*}
    Now, we can easily obtain the permutations corresponding to other elements. For example, $\gamma_{a^{2}} = \gamma_{a} \circ \gamma_{a}$, $\gamma_{ba} = \gamma_{b} \circ \gamma_{a}$ and so on.
\end{negg}

% \section{Free Abelian Groups}

% \textbf{Note:} This section is non-examinable.

% Throughout this section, all groups will be abelian, therefore we shall use additive notation. Here, $0$ is the identity element and $+$ is the symbol for the binary operation. Moreover,
% \begin{equation*}
%     n0 =
%     \begin{cases}
%         0 & n=0,\\
%         a + \dots + a & n>0,\\
%         (-a) + \dots + (-a) & n<0.\\
%     \end{cases}
% \end{equation*}
% Here the addition occurs $n$-times. Moreover, let $A = \set{x_{1}, \dots, x_{r}} \subset G$ be a non-empty set, then
% \begin{equation*}
%     \cyclic{A} = \set{n_{1}x_{1} + \dots + n_{r}x_{r} \st n_{i} \in \Z, x_{i} \in A, r \in \Z_{+}}.
% \end{equation*}
% in additive notation.

% \begin{nthm}
%     Let $G$ be a non-trivial free abelian group with a finite basis. Then, every basis of $G$ is finite, and all the bases of $G$ have the same number of elements.
% \end{nthm}
% \begin{proof}
%     Let $G$ have a basis $\set{x_{1}, \dots, x_{r}}$. Then, $G$ is isomorphic to $\Z \times \dots \times \Z$ ($r$-factors). Let $2G = \set{2g \st g \in G}$. Then, clearly $2G$ is a subgroup of $G$ (check).

%     Now, $G \isom \Z \times \dots \times \Z \implies G / 2G \isom \Z \times \dots \times \Z / 2\Z \times \dots \times 2\Z \isom \Z_{2} \times \dots \times \Z_{2}$ ($r$-factors).

%     Thus, $\abs{G/2G} = 2^{r}$. So, the number of elements in any finite basis $X$ is $r = \log_{2} \abs{G/2G}$.
% \end{proof}

% \begin{nthm}
%     Every subgroup of a free group is also a free group.
% \end{nthm}

\end{document}