\documentclass[11pt]{penrose}

\usepackage{mathsphystools}
\usepackage{thmstyles}

\setlist[enumerate]{label=(\roman*)}
% \geometry{left=24mm, right=24mm}
% \fancyhfoffset{0pt}

\title{MATH 421: Group Theory II}
\subtitle{Brief lecture notes}
\author{Rashid M. Talha}
\affiliation{School of Natural Sciences, NUST}
\date{\today}
\begin{document}

\maketitle

\textbf{Textbooks \& Reference books:}
\begin{itemize}
    \item \dots
    \item \dots
    \item \dots
\end{itemize}

\textbf{Disclaimer:} This document most likely contains some errors. Use with caution. I have rephrased or paraphrased the content in many of the sections. Some examples may be missing. The numbering that I have used for sections, definitions, theorems, etc will not match the numbering given in the lectures.

\section{Recap of Group Theory I}
\begin{ndfn}
    Let $G$ be a non-empty set, and $* : G \times G \to G$ be a binary operation. The pair $(G, *)$ is called a \keyword{group} if it satisfies all of the following
    \begin{enumerate}
        \item $\forall a, b, c \in G$, $(a * b) * c = a * (b * c)$ \hfill(\keyword{Associativity})
        \item $\exists e \in G$ such that $\forall a \in G$, $a * e = e * a = a$ \hfill(\keyword{Identity})
        \item $\forall a \in G$, $\exists a^{-1} \in G$ such that $a * a^{-1} = a^{-1} * a = e$ \hfill(\keyword{Inverse})
    \end{enumerate}
\end{ndfn}

Often we don't explicitly specify the binary operation and simply refer to $G$ as the group. Likewise, $a * b$ is abbreviated to $ab$. When the binary operation behaves like addition, we denote it by $+$, and the use $0$ as the identity element and $-a$ as the inverse of $a \in G$.

\begin{nthm}
    Let $G$ be a group. Then $G$ has a unique identity element. Each $a \in G$ has a unique inverse element.
\end{nthm}

\begin{nthm}
    Let $G$ be a group. For all $a, b \in G$, $(a^{-1})^{-1} = a$ and $(ab)^{-1} = b^{-1} a^{-1}$.
\end{nthm}

\begin{ndfn}
    The \keyword{order of a group} $G$, denoted $\abs{G}$ or $O(G)$, is the number of elements in $G$. If $G$ has infinitely many elements then $\abs{G} = \infty$. The \keyword{order of an element} $a \in G$ is the least positive integer $k$ such that $a^k = e$. We denote this as $O(a)$ or $\abs{a}$. If there is no such positive integer then $\abs{a} = \infty$.
\end{ndfn}

\begin{ndfn}
    A non-empty subset $H \subseteq G$ is called a \keyword{subgroup} of $G$ if it is a group under the same binary operation. We denote this as $H \subgroup G$. If $H \neq G$ then $H$ is called a \keyword{proper subgroup} of $G$. This is sometimes emphasised by writing $H < G$. A subgroup is called \keyword{trivial} if $H = \set{e}$, and non-trivial otherwise.
\end{ndfn}

\begin{nthm}
    Let $G$ be a group. A non-empty subset $H \subseteq G$ is a subgroup of $G$ if and only if for all $a, b \in H$, $ab^{-1} \in H$.
\end{nthm}


\begin{ndfn}
    A group $G$ is called \keyword{cyclic} if there is an element $a \in G$ such that all elements of $G$ can be written as powers of $a$. More precisely, $\forall g \in G$, $\exists m \in \Z$ such that $g = a^m$.

    Such an element $a$ is called a \keyword{generator} of $G$, and we say that $G$ is the group generated by $a$ and denote this as $G = \cyclic{a}$. Cyclic group of order $n$ is sometimes denoted as $C_n$.
\end{ndfn}

Generators are not unique. Indeed if $a \in G$ is a generator then so is $a^{-1}$.

\begin{nthm}
    Let $G$ be a group, and take $a \in G$ such that $a^n = e$. Then, the cyclic subgroup $\cyclic{a}$ has the form $\set{e, a, a^2, \dots, a^{n-1}}$.
\end{nthm}

\begin{nthm}
    Every cyclic group is abelian. Every subgroup of a cyclic group is cyclic. And, the order of a cyclic group is equal to the order of its generator.
\end{nthm}

\begin{nthm}
    Let $G = \cyclic{a}$ be a finite cyclic group of order $n$. Then an element $a^k$ is a generator of $G$ if and only if $\gcd(k,n) = 1$.
\end{nthm}

\begin{nthm}[Lagrange]
    Let $G$ be a finite group and $H$ be its subgroup. Then $|H|$ divides $|G|$. Moreover, the number of distinct left (right) cosets of $H$ in $G$ is $|G|/|H|$.
\end{nthm}

\begin{ncor}
    Let $G$ be a finite group and $a \in G$. Then $O(a)$ divides the order of $G$. Moreover, then $a^{|G|} = e$.
\end{ncor}

\begin{ncor}
    Every group of prime order is cyclic.
\end{ncor}

\begin{ndfn}
    Let $H \subgroup G$. Take some $a \in G$. The subset $aH = \set{ah \st h \in H}$ is called the left \keyword{cosets} of $H$ containing $a \in G$. The element $a$ is called a \keyword{representative} of the coset.
\end{ndfn}

We note that any element of $aH$ can act as its representative.

\begin{ndfn}
    Let $H \subgroup G$. The number of distinct left (right) cosets of $H$ in $G$ is called the index of $H$ in $G$. It is denoted as $[G:H]$.
\end{ndfn}

\begin{nthm}
    Let $H \subgroup G$ and $a, b \in G$.
    \begin{enumerate}
        \item $a \in aH$.
        \item $aH = H$ if and only if $a \in H$.
        \item $aH = bH$ if and only if $a \in bH$ if and only if $a^{-1} b \in H$.
        \item $aH = bH$ or $aH \cap bH = \emptyset$.
        \item $|aH| = |bH|$.
    \end{enumerate}
\end{nthm}

\begin{ndfn}
    Let $H \subgroup G$. If $aH = Ha$ for all $a \in G$ then $H$ is called a normal subgroup of $G$. We write this as $H \nsubgroup G$.
\end{ndfn}

\begin{nthm}
    Let $H \subgroup G$. Then, the following are equivalent:
    \begin{enumerate}
        \item $H \nsubgroup G$.
        \item For all $x \in H$ and $g \in G$, $g^{-1} x g \in H$.
        \item For all $g \in G$, $g^{-1} H g = H$.
    \end{enumerate}
\end{nthm}

\begin{nthm}
    Let $H \subgroup G$. If $[H:G] = 2$, then $H \nsubgroup G$.
\end{nthm}

\begin{nthm}
    Let $H \nsubgroup G$. Then, $G/H = \set{gH : g \in G}$ is a group with respect to $(aH)(bH) = (ab)H$, for $a, b \in G$. Here, the identity element is $H$ and $(aH)^{-1} = a^{-1} H$. This is called the quotient group or factor group of $H$ in $G$.
\end{nthm}


HOMOMORPHISMS AND ISOMORPHISMS.



KLEIN FOUR, DIHEDRAL, SYMMETRIC, QUATERNION, DIRECT PRODUCT.



\section{Subgroups generated by Subsets}

\begin{nthm}
    Let $G$ be a group and $S$ be a non-empty subset of $G$. Let
    \begin{equation*}
        H = \set*{ x_{1}^{a_{1}} x_{2}^{a_{2}} \dots x_{n}^{a_{n}} \st n \in \N, x_{i} \in S, a_{i} = \pm 1, \forall i}
    \end{equation*}
    be the set of all finite products of elements of $S$ and the inverses of elements of $S$. Then, $H \subgroup G$ and $S \subseteq H$. This is called the subgroup generated by $S$, and we write $H = \cyclic{S}$.
\end{nthm}
\begin{proof}
    Exercise.
\end{proof}

\begin{remark}\phantom{}
    \begin{enumerate}
        \item $H = \cyclic{S}$ is the smallest subgroup of $G$ that contains $S$. In other words, if $K \subgroup G$ with $S \subseteq K$ then $H \subseteq K$.
        \item If $S = \set{x_{1}, \dots, x_{n}}$ then we write $\cyclic{S} = \cyclic{x_{1}, \dots, x_{n}}$, instead of $\cyclic{\set{x_{1}, \dots, x_{n}}}$.
        \item If $S = \set{a}$ is a singleton set, then $H = \set{a}$ is a cyclic subgroup.
        \item More than one generating sets may give rise to the same subgroup.
        \item Every group $G$ has at least one generating set, namely $S = G$.
        \item If $G = \cyclic{S}$ and $S$ is finite, then we say that $G$ is finitely generated. If no such finite set exists then we say that $G$ is infinitely generated.
    \end{enumerate}
\end{remark}


\section{Commutator Subgroup}
\begin{ndfn}
    Let $G$ be a group and take $a,b \in G$. We define $\comm{a,b} = aba^{-1}b^{-1}$ as the commutator of $a$ and $b$.
\end{ndfn}

Note that $\comm{a,b}^{-1} = \paren*{aba^{-1}b^{-1}}^{-1} = bab^{-1}a^{-1} = \comm{b,a}$.

\begin{ndfn}
    Let $G$ be a group. Consider the set $S = \set*{ \comm{a,b} \st a,b \in G }$. The subgroup generated by $S$ is called the commutator subgroup of the derived subgroup of $G$. We denote this as $G'$ or $\comm{G,G}$.
\end{ndfn}

By definition, a general element of $\cyclic{S}$ is a finite product of commutators and their inverses. However, since the inverse of a commutator is again a commutator, we get that a general element of $\cyclic{S}$ is simply a product of commutators;
\begin{equation*}
    x \in \cyclic{S} = \comm{G,G}
    \implies
    x = \comm{a_{1}, b_{1}} \dots \comm{a_{k}, b_{k}}
    \quad\text{for}\quad a_{1}, \dots, a_{k}, b_{1}, \dots, b_{k} \in G.
\end{equation*}

\begin{nthm}
    A group $G$ is abelian if and only if $G'$ is trivial.
\end{nthm}
\begin{proof}
    If $G'$ is trivial, then for all $a, b \in G$, $\comm{a,b} = 1 \implies aba^{-1}b^{-1} = 1 \implies ab = ba$. Therefore, $G$ is abelian.

    Conversely, if $G$ is abelian, then for all $a,b \in G$, $\comm{a,b} = aba^{-1}b^{-1} = aa^{-1}bb^{-1} = 1$. Therefore, $G' = \cyclic{1} = \set{1}$.
\end{proof}

\begin{negg}
    It is easy to check that $V_{4}' = \set{e}$ and $D_{3}' = \set{1, a, a^{2}}$.
\end{negg}

\begin{nthm}
    Let $G$ be a group. Then,
    \begin{enumerate}
        \item $G' \nsubgroup G$ (i.e. $G'$ is normal in $G$).
        \item $G / G'$ is abelian.
        \item If $G / K$ is abelian for some $K \nsubgroup G$, then $G' \subseteq K$ (i.e. $G'$ is the smallest subgroup of $G$ such that $G/G'$ is abelian).
    \end{enumerate}
\end{nthm}
\begin{proof}
    Exercise.
\end{proof}



\end{document}