\documentclass[11pt]{penrose}

\usepackage{mathsphystools}

\title{Past Paper: Partial Differential Equations}
\subtitle{Midterm Exam, 2024}
\author{Rashid M. Talha}
\begin{document}

\newenvironment{problem}[2][Question]{\textbf{#1 #2.}\par}{}
\newcommand{\solution}{\textit{Solution.}\hspace{2mm}}

\maketitle

\textbf{Disclaimer:} Use at your own risk. Errors possible.

\begin{problem}{1}
    Classify the following equation and reduce it to the canonical form
    \begin{equation*}
        u_{xy} + y u_{yy} + \sin(x+y) = 0.
    \end{equation*}

    \solution We start by making a change of variables $x \leftrightarrow y$ to get the transformed PDE
    \begin{equation*}
        x u_{xx} + u_{xy} + \sin(x+y) = 0. \tag{\textasteriskcentered}
    \end{equation*}
    Now, $A = x$, $B = 1$, $C = D = E = 0$. Since, $B^2 - 4AC = 1 > 0$, this is a hyperbolic PDE. So, the corresponding characteristic equation
    \begin{equation*}
        \frac{dy}{dx} = \frac{B \pm \sqrt{B^2 - 4AC}}{2A} = \frac{1 \pm 1}{2x}
    \end{equation*}
    gives
    \begin{equation*}
        \frac{dy}{dx} = \frac{1}{x} \implies k + y = \ln x \implies x e^{-y} = c_1, \quad
        \frac{dy}{dx} = 0 \implies y = c_2.
    \end{equation*}
    These lead to $\xi = x e^{-y} = c_1$ and $\eta = y = c_2$. As a result
    \begin{align*}
        B^{*} &= EQQ = EQQ\\
        D^{*} &= EQQ = EQQ\\
        E^{*} &= EQQ = EQQ.
    \end{align*}
    (The coefficients $A^{*} = C^{*} = 0$ because the PDE is hyperbolic.)

    As a result, equation (\textasteriskcentered) implies
    \begin{equation*}
        e^{y} u_{\xi\eta} - e^{y} u_{\xi\xi} = \sin(x + y)
        \implies u_{\xi\eta} - u_{\xi\xi} = e^{-\eta}\sin\paren*{\eta + \xi e^\eta}.
    \end{equation*}
    Here, we used $y = \eta$ and $x = \xi e^y = \xi e^\eta$.

    Overall, the canonical form of the given PDE is
    \begin{equation*}
        u_{\xi\eta} = u_{\xi\xi} + e^{-\eta}\sin\paren*{\eta + \xi e^\eta}
    \end{equation*}
    with $\xi = y e^{-x}$ and $\eta = x$ (in the original variables).
\end{problem}

\begin{problem}{2}
    Apply a linear transformation $\xi = x+by$ and $\eta = x+dy$ to transform the Euler equation
    \begin{equation*}
        A u_{xx} + 2B u_{xy} + C u_{yy} = 0
    \end{equation*}
    into canonical form, where $b$, $d$ $A$, $B$ and $C$ are constants.

    \solution Solution
\end{problem}

\begin{problem}{3}
    Determine the solution of the equation
    \begin{equation*}
        x \paren{y^2 + u} u_x - y \paren{x^2 + u} u_y = \paren{x^2 - y^2} u
    \end{equation*}
    with the data $x + y = 0$ and $u = 1$.

    \solution Solution
\end{problem}

\begin{problem}{4}
    Show that the general solution of a first-order quasilinear partial differential equation
    \begin{equation*}
        a(x,y,u) u_x + b(x,y,u) u_y = c(x,y,u)
    \end{equation*}
    is $f(\varphi, \psi) = 0$, where $f$ is an arbitrary function of $\varphi(x,y,u)$ and $\psi(x,y,u)$, and $\varphi(x,y,u) = c_1$ and $\psi(x,y,u) = c_2$ are solution curves of the characteristic equations
    \begin{equation*}
        \frac{dx}{a} = \frac{dy}{b} = \frac{du}{c}.
    \end{equation*}

    \solution Solution.
\end{problem}

\begin{problem}{5}
    Verify that the function
    \begin{equation*}
        u = \varphi(xy) + x \psi\paren*{\frac{y}{x}}
    \end{equation*}
    is the general solution of the equation
    \begin{equation*}
        x^2 u_{xx} - y^2 u_{yy} = 0.
    \end{equation*}

    \solution We have
    \begin{align*}
        u_{x} &= y \varphi'(xy) + \psi\paren*{\frac{y}{x}} - x \paren*{\frac{y}{x^2}} \psi'\paren*{\frac{y}{x}} = y \varphi'(xy) + \psi\paren*{\frac{y}{x}} - \frac{y}{x} \psi'\paren*{\frac{y}{x}}\\
        u_{y} &= x \varphi'(xy) + x \paren*{\frac{1}{x}} \psi'\paren*{\frac{y}{x}} = x \varphi'(xy) + \psi'\paren*{\frac{y}{x}}\\
        u_{xx} &= y^2 \varphi''(xy) - \frac{y}{x^2} \psi'\paren*{\frac{y}{x}} + \frac{y}{x^2} \psi'\paren*{\frac{y}{x}} + \frac{y^2}{x^3} \psi''\paren*{\frac{y}{x}}
        = y^2 \varphi''(xy) + \frac{y^2}{x^3} \psi''\paren*{\frac{y}{x}}\\
        u_{yy} &= x^2 \varphi''(xy) + \frac{1}{x} \psi''\paren*{\frac{y}{x}}
    \end{align*}
    Therefore,
    \begin{equation*}
        x^2 u_{xx} - y^2 u_{yy} = x^2 y^2 \varphi''(xy) + \frac{y^2}{x} \psi''\paren*{\frac{y}{x}} - x^2 y^2 \varphi''(xy) + \frac{y^2}{x} \psi''\paren*{\frac{y}{x}} = 0
    \end{equation*}
\end{problem}

\clearpage
\begin{problem}{6}
    Determine the solution of the initial boundary value problem
    \begin{equation*}
        \begin{aligned}
            u_{tt} &= 4u_{xx}, & &0 < x < 1, \quad t > 0,\\
            u(x,0) &= 0, & &0 \leq x \leq 1,\\
            u_t(x,0) &= x(1-x), & &0 \leq x \leq 1,\\
            u(0,t) &= 0, \; u(1,t) = 0, & \quad &t \geq 0.
        \end{aligned}
    \end{equation*}
    
    \solution This is the wave equation ($c = 2$) on a bounded spatial interval $[0,1]$. The general solution is $u(x, t) = \varphi(x + 2t) + \psi(x - 2t)$ with
    \begin{equation*}
        \varphi(\eta) = \frac{1}{2} f(\eta) + \frac{1}{2c} \int_0^\eta g(\tau) \,d\tau + \frac{K}{2},
        \quad
        \psi(\eta) = \frac{1}{2} f(\eta) - \frac{1}{2c} \int_0^\eta g(\tau) \,d\tau - \frac{K}{2}.
    \end{equation*}
    for $0 \leq \eta \leq 1$. Using $f(\eta) = 0$ and $g(\eta) = \eta (1 - \eta)$ gives,
    \begin{gather}
        \varphi(\eta) = \frac{1}{4} \paren*{\frac{1}{2}\eta^2 - \frac{1}{3}\eta^3} + \frac{K}{2}, \quad 0 \leq \eta \leq 1, \label{eq:1-phi}\\
        \psi(\eta) = - \frac{1}{4} \paren*{\frac{1}{2}\eta^2 - \frac{1}{3}\eta^3} - \frac{K}{2}, \quad 0 \leq \eta \leq 1. \label{eq:1-psi}
    \end{gather}

    The boundary conditions give
    \begin{gather*}
        u(0,t) = 0 \implies \psi(-2t) = -\varphi(2t) \implies \psi(\eta) = -\varphi(-\eta), \tag{\textasteriskcentered}\\
        u(1,t) = 0 \implies \phi(1+2t) = -\varphi(1-2t) \implies \psi(\eta) = -\varphi(2-\eta). \tag{\textasteriskcentered\textasteriskcentered}
    \end{gather*}

    Now, using (\textasteriskcentered) in \eqref{eq:1-phi} gives,
    \begin{equation}
        \psi(\eta)
        = -\varphi(-\eta)
        = -\frac{1}{4} \paren*{\frac{1}{2}\eta^2 + \frac{1}{3}\eta^3} - \frac{K}{2}. \label{eq:2-psi}
    \end{equation}
    This is valid for $0 \leq -\eta \leq 1 \implies -1 \leq \eta \leq 0$. So, the domain of $\psi$ has been extended.

    Similarly, using (\textasteriskcentered\textasteriskcentered) in \eqref{eq:1-psi} gives,
    \begin{equation}
        \varphi(\eta)
        = -\psi(2-\eta)
        = \frac{1}{4} \paren*{\frac{1}{2}(2-\eta)^2 - \frac{1}{3}(2-\eta)^3} + \frac{K}{2}. \label{eq:2-phi}
    \end{equation}
    This is valid for $0 \leq 2-\eta \leq 1 \implies 1 \leq \eta \leq 2$. So, the domain of $\varphi$ has been extended.

    We repeat this process to extend the domain further: using (\textasteriskcentered) in \eqref{eq:2-psi} and (\textasteriskcentered\textasteriskcentered) in \eqref{eq:2-phi}.
\end{problem}

\end{document}