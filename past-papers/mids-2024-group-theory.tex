\documentclass[11pt]{penrose}


\usepackage{mathsphystools}

\author{Rashid M. Talha}

\newenvironment{problem}[2][Question]{\textbf{#1 #2.}\par}{}
\newcommand{\solution}{\textit{Solution.}\hspace{2mm}}

\newcommand{\warningtext}{\textbf{Disclaimer:} Use at your own risk. Errors possible.}

\title{Past Paper: Group Theory I}
\subtitle{Midterm Exam, 2024}
\begin{document}

\maketitle
\warningtext

\begin{problem}{1 (i)}
    Prove that the order of a cyclic group is equal to the order of its generator.

    \solution Let $G$ be a cyclic group generated by $a$; $G = \cyclic{a}$.

    If $O(a)$ is infinite then $a^n$ and $a^m$ are distinct for all $n \neq m$, because otherwise,
    \begin{equation*}
        a^n = a^m \implies a^{n - m} = e \implies O(a) \leq \abs{n-m}
    \end{equation*}
    which is a contradiction since $O(a) = +\infty$. Now, since each $a^n$ is different from $a^{m}$, we have infinitely many distinct elements in the set $\cyclic{a} = \set{a^k \st k \in \Z}$. Consequently, $G = \cyclic{a}$ has infinitely many elements; i.e. $|G|$ is also infinite.

    Next, consider the case when $O(a) = n$ is finite. By definition $\cyclic{a} = \set{a^k \st k \in \Z}$. We can write $k = qn + r$ for $q, r \in \Z$ and $0 \leq r < n$. So, for all $k \in \Z$,
    \begin{equation*}
        a^k = a^{qn+r} = a^{qn} a^r = (a^{n})^{q} = e a^r = a^r.
    \end{equation*}
    Therefore, $\cyclic{a} = \set{a^r \st 0 \leq r < n} = \set{e, a, a^2, \dots, a^{n-1}}$. Again, take any $a^p, a^q \in \cyclic{a}$ with $p \neq q$, then
    \begin{equation*}
        a^p = a^q \implies a^{p - q} = e \implies O(a) \leq \abs{p-q} \leq n-1
    \end{equation*}
    which is not possible because $O(a) = n$. Therefore, each element in this set is distinct, and we get $|G| = |\cyclic{a}| = n$.
\end{problem}

\begin{problem}{1 (ii)}
    Let $G = \cyclic{a}$ be a finite cyclic group of order $n$. Then prove that an element $a^k$ is a generator if and only if $\gcd(n, k) = 1$.

    \solution Suppose $a^k$ is a generator of $G$. Then, we can write $a$ as a power of $a^k$, say, $a = \paren{a^k}^m = a^{km}$ for some $m \in \Z$.

    Then, $a = a^{km} \implies a a^{-km} = e \implies a^{1-km} = e$. So, $n \mid 1 - km$. That is $\exists q \in \Z$ such that $1 - km = qn$. We can re-arrange this to get $qn + km = 1$. From number theory, this implies $\gcd(n,k)=1$ (Bezout's lemma).

    Conversely, suppose $\gcd(n,k)=1$. Then, there exist integers $x, y$ such that $xk + yn = 1$. So, $a = a^1 = a^{xk + yn} = a^{xk} a^{yn} = \paren{a^{k}}^{x} \paren{a^{n}}^{y} = \paren{a^{k}}^{x} e^{y} = \paren{a^{k}}^{x}$.

    Now, for all $b \in G=\cyclic{a}$, we have $b = a^r$ for some $r \in Z$. Therefore, we can write it as a power of $a^k$ as $b = a^r = \paren{\paren{a^{k}}^{x}}^{r} = \paren{a^{k}}^{xr}$. So, $a^k$ also generates $G$.
\end{problem}

\begin{problem}{2 (i)}
    Let $H$ be a subgroup of $G$. Let $\sim$ be a relation on $G$ defined by $a \sim b$ if and only if $a^{-1} b \in H$. Show that $\sim$ is an equivalence relation on $G$. Also prove $[a] = aH$.

    \solution We check the three conditions for being an equivalence relation.
    
    (Reflexive.) Take $a \in G$. Then, $a^{-1} a = e$ and $e \in H$ because $H \subgroup G$. So, $a \sim a$.

    (Symmetric.) If $a \sim b$, then $a^{-1} b \in H$. As $H \subgroup G$, the element $\paren*{a^{-1} b}^{-1} \in H$. And $\paren*{a^{-1} b}^{-1} = b^{-1} (a^{-1})^{-1} = b^{-1} a$. That is, $b^{-1} a \in H$. So, $b \sim a$.

    (Transitive.) If $a \sim b$ and $b \sim c$, then $a^{-1} b \in H$ and $b^{-1} c \in H$. As $H \subgroup G$, closure and associativity gives $\paren*{a^{-1} b} \paren*{b^{-1} c} = a^{-1} (b b^{-1}) c = a^{-1} c \in H$. So, $a \sim c$.

    Next, $[a] = \set{b \in G \st b \sim a} = \set{b \in G \st a \sim b} = \set{b \in G \st a^{-1} b \in H}$. And, $a^{-1} b \in H$ means $a^{-1} b = h$ for some $h \in H$. As a result, $b = ah$ and we can write
    \begin{equation*}
        [a] = \set{b \in G \st a^{-1} b \in H} = \set{ah \in G \st h \in H} = aH.
    \end{equation*}
\end{problem}

\begin{problem}{2 (ii)}
    Let $G$ be a group and $H$ be a subgroup of $G$. If $a, b \in G$, then prove that $aH = bH$ if and only if $a^{-1} b \in H$.

    \solution Suppose $aH = bH$. As $H \subgroup G$, we have $e \in H$. So, $b = be \in bH = aH$. Thus, we can write $b = ah$ for some $h \in H$. Consequently, $a^{-1} b = h \in H$.

    Conversely, let $a^{-1} b \in H$. This means $a^{-1} b = h$ for some $h \in H$. As a result, $b = ah$. Therefore,
    \begin{equation*}
        bH = \set{bk \st k \in H} = \set{(ah)k \st k \in H} = \set{ah' \st h' \in H} = aH.
    \end{equation*}
    Here we used the associativity and closure property because $H \subgroup G$.
\end{problem}

\begin{problem}{2 (iii)}
    Let $G$ be a group and $a \in G$ such that $O(a) = n$. If $m$ is an integer such that $a^m = e$, then prove that $n$ divides $m$.

    \solution By the division algorithm, $m = qn + r$ for $q, r \in \Z$ and $0 \leq r < n$. So,
    \begin{equation*}
        a^m = a^{nq + r} = \paren*{{a^n}^q} \paren*{a^r} = \paren*{({a^n})^q} \paren*{a^r} = \paren*{e^q} \paren*{a^r} = e \paren*{a^r} = a^r.
    \end{equation*}
    Now, $a^m = e \implies a^r = e$. However, $r < n$ and $n$ is the least positive integer for which $a^n = e$. Therefore, $r$ must be zero (if it was positive then it would contradict the minimality of $n$). Therefore, $m = qn$. That is $n \mid m$.
\end{problem}

\begin{problem}{3 (i)}
    Let $\Z_{12} = \set{0, 1, \dots, 11}$ be the group of integers modulo $12$. Then find the order of each element of $\Z_{12}$.

    \solution The order of each element in $\Z_{12}$ is
    \begin{center}
    \begin{tabularx}{0.8\textwidth}{XXXX}
        $O(0)  =  1$ & $O(1)  = 12$ & $O(2)  =  6$ & $O(3)  =  4$\\
        $O(4)  =  3$ & $O(5)  = 12$ & $O(6)  =  2$ & $O(7)  = 12$\\
        $O(8)  =  3$ & $O(9)  =  4$ & $O(10) =  6$ & $O(11) = 12$\\
    \end{tabularx}
    \end{center}
\end{problem}

\begin{problem}{3 (ii)}
    Find all subgroups of the cyclic group $C_{12} = \set{1, a, a^2, \dots, a^{11} \st a^{12}=1}$.

    \solution We arrange these by the GCD of $n = 12$ and the order of the generator.
    \begin{equation*}
    \begin{aligned}
        \gcd &= 1:  & & \cyclic{1}   = \set{1}\\
        \gcd &= 2:  & & \cyclic{a^6} = \set{1, a^6}\\
        \gcd &= 3:  & & \cyclic{a^4} = \cyclic{a^8} = \set{1, a^4, a^8}\\
        \gcd &= 4:  & & \cyclic{a^3} = \cyclic{a^9} = \set{1, a^3, a^6, a^9}\\
        \gcd &= 6:  & & \cyclic{a^2} = \cyclic{a^{10}} = \set{1, a^2, a^4, a^6, a^8, a^{10}}\\
        \gcd &= 12: & & \cyclic{a}   = \cyclic{a^{11}} = \cyclic{a^5} = \cyclic{a^7} = \set{1, a, a^2, a^3, a^4, a^5, a^6, a^7, a^8, a^9, a^{10}, a^{11}}
    \end{aligned}
    \end{equation*}
\end{problem}

\begin{problem}{3 (iii)}
    Let $G = \set{\begin{psmallmatrix*} a & b \\ c & d \end{psmallmatrix*} \st a, b, c, d \in \Z}$ be a group under addition of matrices, and take $H = \set{\begin{psmallmatrix*} a & b \\ c & d \end{psmallmatrix*} \st a + b + c + d = 1 \in \Z}$. Prove or disprove that $H$ is a subgroup of $G$.

    \solution $H$ is not a subgroup of $G$ because it doesn't contain the identity element. That is $\begin{psmallmatrix*} 0 & 0 \\ 0 & 0 \end{psmallmatrix*} \notin H$ since $0 + 0 + 0 + 0  \neq 1$.
\end{problem}

\begin{problem}{4 (i)}
    Let $*$ be a binary operation on $\Q^{+}$ defined by $a * b = \frac{ab}{2}$. Then show that $\Q^{+}$ is an Abelian group with respect to $*$. ($\Q^{+}$ denotes the set of positive rational numbers.)

    \solution We check the group axioms.

    (Closure.) Take $a, b \in \Q^+$. In particular, $a, b \in \Q$ and $a, b > 0$. So, $ab \in \Q$ and $\frac{ab}{2} \in \Q$. Moreover, $ab > 0$ and $\frac{ab}{2} > 0$. That means $\frac{ab}{2} \in \Q^+$. Therefore, $a * b = \frac{ab}{2} \in \Q^+$.

    (Associativity.) Take $a, b, c \in Q^+$. Then,
    \begin{align*}
        a * (b * c) &= a * \paren*{\frac{bc}{2}} = \frac{a(\frac{bc}{2})}{2} = \frac{abc}{4},\\
        (a * b) * c &= \paren*{\frac{ab}{2}} * c = \frac{(\frac{ab}{2})c}{2} = \frac{abc}{4}.
    \end{align*}
    As a result, $a * (b * c) = (a * b) * c$.

    (Identity.) The element $e = 2 \in \Q^+$ serves as the identity element because
    \begin{equation*}
        a * 2 = \frac{(a)(2)}{2} = a
        \quad\text{and}\quad
        2 * a = \frac{(2)(a)}{2} = a.
    \end{equation*}

    (Inverse.) For any $a \in \Q^+$ take $a^{-1} = 4/a$. Then, $a^{-1} \in \Q^+$ because $a > 0$, and
    \begin{equation*}
        a * \frac{4}{a} = \frac{(a)(\frac{4}{a})}{2} = 2 = e
        \quad\text{and}\quad
        \frac{4}{a} * a = \frac{(\frac{4}{a})(a)}{2} = 2 = e.
    \end{equation*}

    (Commutativity.) For any $a, b \in \Q^+$, $a * b = \frac{ab}{2} = \frac{ba}{2} = b * a$
    % \begin{equation*}
        % a * b = \frac{ab}{2} = \frac{ba}{2} = b * a
    % \end{equation*}
    because the usual multiplication of rational numbers is commutative, $ab = ba$.
\end{problem}

\begin{problem}{4 (ii)}
    Find all cyclic subgroups of $D_4 = \set{1, a, a^2, a^3, b, ba, ba^2, ba^3}$, $a^4 = b^2 = 1$, $ab = ba^3$.

    \solution The cyclic subgroups of $D_4$ are
    \begin{gather*}
        \cyclic{1} = \set{1}, \qquad
        \cyclic{a} = \cyclic{a^3} = \set{1, a, a^2, a^3}, \qquad
        \cyclic{a^2} = \set{1, a^2},\\
        \cyclic{b} = \set{1, b}, \qquad
        \cyclic{ba} = \set{1, ba}, \qquad
        \cyclic{ba^2} = \set{1, ba^2}, \qquad
        \cyclic{ba^3} = \set{1, ba^3}.
    \end{gather*}
\end{problem}

\end{document}