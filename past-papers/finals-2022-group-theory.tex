\documentclass[11pt]{penrose}


\usepackage{mathsphystools}

\author{Rashid M. Talha}

\newenvironment{problem}[2][Question]{\textbf{#1 #2.}\par}{}
\newcommand{\solution}{\textit{Solution.}\hspace{2mm}}

\newcommand{\warningtext}{\textbf{Disclaimer:} Use at your own risk. Errors possible.}

\title{Past Paper: Group Theory I}
\subtitle{Final Exam, 2022}
\begin{document}

\maketitle
\warningtext

\begin{problem}{1 (i)}
    % Prove that the order of any permutation of $\varphi \in S_n$ is equal to the least common multiple of the orders of the disjoint cycles in $\varphi$.

    % \solution
    We write $\varphi = \alpha_1 \dots \alpha_m$, where $\alpha_1, \dots, \alpha_m$ are pairwise disjoint cycles. Let $r_i$ be the order of $\alpha_i$. So, $\alpha_i^{r_i} = e$ for all $i$.

    Let $r = \lcm(r_1, \dots, r_m)$. Then, for each $i$, there is some $k_i \in \Z_+$ such that $r = r_i k_i$ because $r_i \mid r$. Therefore,
    \begin{equation*}
        \varphi^r
        = \smash[b]{\underbrace{\paren*{\alpha_1 \dots \alpha_m}^r = \alpha_1^r \dots \alpha_m^r}_{\mathclap{\text{disjoint, so commute}}}}
        = (\alpha_1^{r_1})^{k_1} \dots (\alpha_m^{r_m})^{k_m}
        = e^{k_1} \dots e^{k_m}
        = e.
    \end{equation*}

    As a result, $O(\varphi) \leq r$.

    Now, suppose $O(\varphi) = s$, i.e. $\varphi^s = e$ with $0 < s \leq r$. Since, $\varphi^s = \paren*{\alpha_1 \dots \alpha_m}^s = \alpha_1^s \dots \alpha_m^s$, we obtain $\alpha_1^s \dots \alpha_m^s = e$. This is only possible when $\alpha_i^s = e$ for each $i$ because $\alpha_i$ are all disjoint. So, $r_i \mid s$ for all $i$. Thus, $r \mid s$.
    \begin{equation*}
        0 < s \leq r \text{ and } r \mid s \implies s = r.
    \end{equation*}
    Consequently, $O(\varphi) = r \equiv \lcm(r_1, \dots, r_m)$.
\end{problem}

\begin{problem}{1 (ii)}
    % Let $G$ be a group and $H$ be a normal subgroup of $G$. Let $G/H = \set{aH \st a \in G}$ be the set of all left cosets of $H$ in $G$. Prove that $G/H$ forms a group under the binary operation $(aH)(bH) = abH$.

    % \solution
    (Closure.) Take any two cosets $aH, bH$. Now, since $H \nsubgroup G$, we have
    \begin{equation*}
        (aH)(bH) = a(Hb)H = a(bH)H = (ab)HH = (ab)H.
    \end{equation*}
    So, the prodcut of any two left cosets is again a left coset.

    (Associativity.) Take any three cosets $aH, bH$ and $cH$. Then,
    \begin{align*}
        \paren*{(aH)(bH)}(cH) &= ((ab)H)(cH) = (abc)H,\\
        (aH)\paren*{(bH)(cH)} &= (aH)((bc)H) = (abc).
    \end{align*}
    So, $\paren*{(aH)(bH)}(cH) = (aH)\paren*{(bH)(cH)}$.

    (Identity.) For any $aH$ we have
    \begin{equation*}
        (aH)(eH) = (ae)H = aH
        \quad\text{and}\quad
        (eH)(aH) = (ea)H = aH.
    \end{equation*}
    Thus, $(aH)(eH) = (eH)(aH) = aH$. So, $eH$ acts as the identity element.

    (Identity.) For any $aH$ we have
    \begin{equation*}
        (aH)(a^{-1}H) = (aa^{-1})H = eH
        \quad\text{and}\quad
        (a^{-1}H)(aH) = (a^{-1}a)H = eH.
    \end{equation*}
    Thus, $(aH)(a^{-1}H) = (a^{-1}H)(aH) = eH$. So, $a^{-1}H$ acts as the inverse of $aH$.
\end{problem}

\clearpage
\begin{problem}{1 (iii)}
    % Let $G = \set{\begin{psmallmatrix*} a & b \\ c & d \end{psmallmatrix*} \st a, b, c, d \in \Z}$ be a group under addition of matrices, and take $H = \set{\begin{psmallmatrix*} a & b \\ c & d \end{psmallmatrix*} \st a + b + c + d = 1 \in \Z}$. Prove or disprove that $H$ is a subgroup of $G$.

    % \solution
    $H$ is not a subgroup of $G$ because it doesn't contain the identity element. That is $\begin{psmallmatrix*} 0 & 0 \\ 0 & 0 \end{psmallmatrix*} \notin H$ since $0 + 0 + 0 + 0  \neq 1$.
\end{problem}

\begin{problem}{2 (i)}
    % Let $n \geq 3$. If $\alpha, \beta \in S_n$, then show that $\alpha\beta\alpha^{-1}\beta^{-1}$ is an even permutation.

    % \solution
    Suppose $\alpha$ can be written as a product of $p$ transpositions, and $\beta$ can be written as a product of $q$ transpositions: $\alpha = \tau_1 \dots \tau_p$ and $\beta = \tilde{\tau}_1 \dots \tilde{\tau}_q$.

    Then, $\alpha^{-1}$ is simply obtained by reversing the order in which the transpositions are multiplied $\alpha^{-1} = (\tau_1 \dots \tau_p)^{-1} = \tau_p^{-1} \dots \tau_1^{-1} = \tau_p \dots \tau_1$.
    
    So, $\alpha^{-1}$ can also be written as a product of $p$ transpositions. Similarly, $\beta^{-1}$ can be written as a product of $q$ transpositions.

    Consequently, $\alpha\beta\alpha^{-1}\beta^{-1}$ is a product of $p+q+p+q = 2(p+q)$ transpositions. That is, it is a product of an even number of transpositions. Therefore, it is an even permutation.
\end{problem}

\begin{problem}{2 (ii)}
    % Let $\alpha = (1\ 2\ 3\ 4\ 5\ 6)(2\ 5\ 6\ 7\ 8)(8\ 9\ 10) \in S_{10}$ be a permutation. Then,
    % \begin{enumerate}
    %     \item[(a)] Find the order of $\alpha$.
    %     \item[(b)] Check whether $\alpha$ is even or odd.
    % \end{enumerate}

    % \solution
    We start by writing $\alpha$ as a product of disjoint cycles. This is $\alpha = (1\ 2\ 6\ 7\ 8\ 9\ 10\ 3\ 4\ 5)$.    Therefore, $\alpha$ can be written as a single cycle.

    \begin{enumerate}
        \item[(a)] The order of $\alpha$ is equal to its length. So, $O(\alpha) = 10$.
        \item[(b)] Since, the length of $\alpha$ is 10, it can be written as a product of $10-1=9$ transpositions. So, $\alpha$ is an odd permutation.
    \end{enumerate}
\end{problem}

\begin{problem}{2 (iii)}
    % Let $S_n$ be the symmetric group, and let $\varphi: S_n \to \set{-1,1}$ be defined by
    % \begin{equation*}
    % \varphi(\alpha) =
    % \begin{cases}
    %     1, & \text{if } \alpha \text{ is an even permutation};\\
    %     0, & \text{if } \alpha \text{ is an odd permutation}.
    % \end{cases}
    % \end{equation*}
    % Show that $\varphi$ is a homomorphism.

    % \solution
    Take $\alpha, \beta \in S_n$. There are four cases to consider.
    \begin{itemize}
        \item If $\alpha$ is even and $\beta$ is even, then $\alpha\beta$ is even, and $\varphi(\alpha)\varphi(\beta) = (1)(1) = 1 = \varphi(\alpha\beta)$.
        \item If $\alpha$ is even and $\beta$ is odd, then $\alpha\beta$ is odd, and $\varphi(\alpha)\varphi(\beta) = (1)(-1) = -1 = \varphi(\alpha\beta)$.
        \item If $\alpha$ is odd and $\beta$ is even, then $\alpha\beta$ is odd, and $\varphi(\alpha)\varphi(\beta) = (-1)(1) = -1 = \varphi(\alpha\beta)$.
        \item If $\alpha$ is odd and $\beta$ is odd, then $\alpha\beta$ is even, and $\varphi(\alpha)\varphi(\beta) = (-1)(-1) = 1 = \varphi(\alpha\beta)$.
    \end{itemize}
    In every case, $\varphi(\alpha\beta) = \varphi(\alpha)\varphi(\beta)$. So, $\varphi$ is a homomorphism.
\end{problem}

\begin{problem}{3 (i)}
    % Let $H$ be a subgroup of $G$. If $[G:H]=2$, then prove that $H$ is a normal subgroup of $G$.

    % \solution
    As $H$ is a subgroup of $G$ with index $2$, there are exactly two distinct left (right) cosets of $H$ in $G$; and $G$ is a union of these two disjoint cosets.

    Take $g \in G$. If $g \in H$, then $gH = H$ and $Hg = H$. So, $gH = Hg$.

    Otherwise, if $g \notin H$, then $gH \cap H = \emptyset$ and $Hg \cap H = \emptyset$. Therefore,
    \begin{equation*}
        G = H \cup gH \quad\text{and}\quad G = H \cup Hg.
    \end{equation*}
    And,
    \begin{gather*}
        gH \cap H = \emptyset \text{ and } gH \subseteq G = H \cup Hg \implies gH \subseteq Hg,\\
        Hg \cap H = \emptyset \text{ and } Hg \subseteq G = H \cup gH \implies Hg \subseteq gH.
    \end{gather*}
    So, $gH = Hg$.

    Overall, $gH = Hg$ for all $g \in G$. Therefore, $H$ is a normal subgroup of $G$.
\end{problem}

\begin{problem}{3 (ii)}
    % Define the kernel of a group homomorphism. Give an example of a group homomorphism from a group $G$ to any group $G'$ whose kernel is non-trivial (that is the kernel contains at least 2 elements)

    % \solution\\
    \underline{Definition}: Let $\varphi:G \to H$ be a group homomorphism. The kernel of $\varphi$ is the set
    \begin{equation*}
        \ker\varphi = \set{g \in G \st \varphi(g) = e} \subseteq G.
    \end{equation*}

    \underline{Example}: Consider the map $\det: \GL_2(\R) \to \R^*$, where $\det(A)$ is the determinant of the matrix $A$. This is a homomorphism because $\det(AB) = \det(A) \det(B)$.

    Now, $\ker\det = \set{A \in \GL_2(\R) \st \det(A) = 1} = \SL_2(\R)$. This is non-trivial because, for example,
    \begin{equation*}
        \begin{pmatrix*} 1 & 0 \\ 0 & 1 \end{pmatrix*},
        \begin{pmatrix*} 1 & 1 \\ 0 & 1 \end{pmatrix*}
        \in \ker\det
    \end{equation*}
\end{problem}

\begin{problem}{3 (iii)}
    % Let $G$ be an abelian group. If $\varphi: G \to G$ be a map defined by $\varphi(g) = g^{-1}$, then prove that $\varphi$ is an isomorphism.

    % \solution
    Take any $a,b \in G$. Then, $\varphi(ab) = (ab)^{-1} = b^{-1} a^{-1} = a^{-1} b^{-1} = \varphi(a)\varphi(b)$. So, $\varphi$ is a homomorphism. Here, we used $b^{-1} a^{-1} = a^{-1} b^{-1}$ because $G$ is abelian.

    Next, $\varphi(a) = \varphi(b) \implies a^{-1} = b^{-1} \implies aa^{-1}b = ab^{-1}b \implies eb = ae \implies b = a$. So, $\varphi$ is injective. Here we multiplied by $a$ on the left and by $b$ on the right and used associativity and the properties of inverse.

    Now, every $g^{-1} \in G$ has an inverse $(g^{-1})^{-1} = g \in G$ because $G$ is a group. So, for all $g^{-1} \in G$, there is a $g \in G$ such that $\varphi(g) = g^{-1}$. This shows that $\varphi$ is surjective.

    Overall, $\varphi$ is an isomorphism (a bijective homomorphism).
\end{problem}

\begin{problem}{4 (i)}
    % Prove that a subgroup $H$ of $G$ is normal if and only if $g^{-1}hg \in H$ for all $h \in H$, $g \in G$.

    % \solution
    Suppose $H \nsubgroup G$, i.e. $Hg = gH$ for all $g \in G$. Then, for all $h \in H$ and $g \in G$,
    \begin{equation*}
         Hg = gH \implies hg \in gH \implies hg = gh' \implies g^{-1}hg = h' \in H.
    \end{equation*}
    Overall, for all $h \in H$ and $g \in H$ $g^{-1}hg \in H$.

    Conversely, fix any $g \in G$. Then, for all $h \in H$,
    \begin{equation*}
        g^{-1}hg \in H
        \implies g(g^{-1}hg) \in gH
        \implies (gg^{-1})hg \in gH
        \implies hg \in gH
        \implies Hg \subseteq gH.
    \end{equation*}
    Also, $x^{-1}hx \in H \implies ghg^{-1} \in H$ by taking $x = g^{-1}$. So,
    \begin{equation*}
        ghg^{-1} \in H
        \implies (ghg^{-1})g \in Hg
        \implies gh(g^{-1}g) \in Hg
        \implies gh \in Hg
        \implies gH \subseteq Hg.
    \end{equation*}
    Overall, $gH = Hg$ for all $g \in G$, and $H$ is a normal subgroup of $G$.
\end{problem}

\begin{problem}{4 (ii)}
    % Let $\sim$ be an equivalence relation on $S$. Prove that $\sim$ partitions $S$ into equivalence classes.

    % \solution
    The set of equivalence classes due to $\sim$ is $\set{[a] \st a \in S}$. We need to show that every equivalence class is non-empty, and that the union of all the equivalence classes form $S$, and that if $[a] \neq [b]$ then $[a]$ and $[b]$ are disjoint.
    \begin{enumerate}
        \item By reflexivity, $a \sim a$ we have $a \in [a]$. Therefore, $[a] \neq \emptyset$.

        \item By definition $[a] \subseteq S$, so $\bigcup_{a \in S} [a] \subseteq S$.

        Take any $a \in S$. Then, $a \in [a] \subseteq \bigcup_{a \in S} [a]$. So, $S = \bigcup_{a \in S} [a]$.

        \item Suppose $[a] \cap [b] \neq \emptyset$. So, there is some $c \in [a] \cap [b]$. By definition, this means $c \sim a$ and $c \sim b$. Then, by symmetry, $a \sim c$. So, the transitivity of $\sim$ gives
        \begin{equation*}
            a \sim c \quad\text{and}\quad c \sim b \implies a \sim b.
        \end{equation*}
        Also, by symmetry, $b \sim a$.

        Now, if $x \in [a]$, then $x \sim a$. By transitivity, $x \sim a$ and $a \sim b$ implies $x \sim b$. That is, $x \in [b]$. So, $[a] \subseteq [b]$.

        Similarly, if $y \in [b]$, then $y \sim b$. Again, by transitivity, $y \sim b$ and $b \sim a$ implies $y \sim a$. That is, $y \in [a]$. So, $[b] \subseteq [a]$. Overall, $[a] = [b]$.
    \end{enumerate}

    Overall, the set of equivalence classes partitions $S$.
\end{problem}

\begin{problem}{5 (i)}
    % Prove that a subgroup of a cyclic group is cyclic.

    % \solution
    Consider the cyclic group $G = \cyclic{a}$. Let $H$ be a subgroup of $G$. If $H = \set{e}$ then it is generated by $e$. So, suppose $H$ is not the trivial subgroup. Then, every element in $b \in H$ can be written as $b = a^k$ for some $k \in \Z$ because $b \in G$ also. Let $m$ be the least positive integer such that $a^m \in H$. Therefore, by the existence of inverses in the subgroup, $a^{-m} \in H$ also.

    Take any $a^t \in H$. Then, we can write $t = mq + r$ for some $q, r \in \Z$ with $0 \leq r < m$. Equivalently, $r = t - mq$. So,
    \begin{equation*}
        a^r = a^{t - mq} = a^{t} a^{-mq} = a^{t} (a^{-m})^{q} \in H
    \end{equation*}
    by closure. If $r \neq 0$, then this contradicts the requirement that $m$ is the least positive integer with $a^m \in H$ because $0 \leq r < m$ and we have shown that $a^r \in H$. So, $r = 0$. Therefore, $t = mq$, and every arbitrary element of $H$ has the form $a^t = a^{mq} = (a^{m})^{q}$. In other words, $H \subgroup G$ is cyclic and generated by $a^m$.
\end{problem}

\begin{problem}{5 (ii)}
    % Let $D_4 = \set{1, a, a^2, a^3, b, ba, ba^2, ba^3}$, $a^4 = b^2 = 1$ and $ab = ba^3$ be the dihedral group.
    % \begin{enumerate}
    %     \item[(a)] Find all cyclic subgroups of $D_4$.
    %     \item[(b)] If $H = \set{1,a^2}$ is a normal subgroup of $D_4$, then find $D_4/H$ and all the subgroups of $D_4/H$.
    % \end{enumerate}

    % \solution
    The cyclic subgroups of $D_4$ are
    \begin{gather*}
        \cyclic{1} = \set{1}, \qquad
        \cyclic{a} = \cyclic{a^3} = \set{1, a, a^2, a^3}, \qquad
        \cyclic{a^2} = \set{1, a^2},\\
        \cyclic{b} = \set{1, b}, \qquad
        \cyclic{ba} = \set{1, ba}, \qquad
        \cyclic{ba^2} = \set{1, ba^2}, \qquad
        \cyclic{ba^3} = \set{1, ba^3}.
    \end{gather*}

    Next, note that $H = \set{1,a^2}$ is indeed a normal subgroup of $D_4$ because the centre of $D_4$ is $Z(D_4) = \set{1,a^2}$, and the centre of a group is always a normal subgroup.

    Now, the elements of $D_4/H$ are simply the (left) cosets of $H$ is $D_4$. That is
    \begin{equation*}
        D_4/H = \set{H, aH, bH, baH}.
    \end{equation*}
    (This is the Klein-4 group because $\abs{D_4/H} = 4$ and each non-identity element has $2$.)

    Let $K$ be any non-trivial proper subgroup of $D_4/H$. By Lagrange's theorem, $\abs{K} = 2$. Therefore, all subgroups of $D_4/H$ are
    \begin{equation*}
        K_0 = \set{H}, \quad
        K_1 = \set{H, aH}, \quad
        K_2 = \set{H, bH}, \quad
        K_3 = \set{H, baH}, \quad
        K_4 = D_4/H.
    \end{equation*}
\end{problem}

\end{document}