\documentclass[11pt]{penrose}


\usepackage{mathsphystools}

\author{Rashid M. Talha}

\newenvironment{problem}[2][Question]{\textbf{#1 #2.}\par}{}
\newcommand{\solution}{\textit{Solution.}\hspace{2mm}}

\newcommand{\warningtext}{\textbf{Disclaimer:} Use at your own risk. Errors possible.}

\title{Past Paper: Group Theory I}
\subtitle{OHT-II, 2022}
\begin{document}

\maketitle
\warningtext

\begin{problem}{1}
    % Let $G = \cyclic{a}$ be a finite cyclic group of order $n$. Then, prove that $a^k$ is a generator of $G$ if and only if $\gcd(k,n) = 1$.

    % \solution
    Suppose $a^k$ is a generator of $G$. Then, we can write $a$ as a power of $a^k$, say, $a = \paren{a^k}^m = a^{km}$ for some $m \in \Z$.

    Then, $a = a^{km} \implies a a^{-km} = e \implies a^{1-km} = e$. So, $n \mid 1 - km$. That is $\exists q \in \Z$ such that $1 - km = qn$. We can re-arrange this to get $qn + km = 1$. From number theory, this implies $\gcd(n,k)=1$ (Bezout's lemma).

    Conversely, suppose $\gcd(n,k)=1$. Then, there exist integers $x, y$ such that $xk + yn = 1$. So, $a = a^1 = a^{xk + yn} = a^{xk} a^{yn} = \paren{a^{k}}^{x} \paren{a^{n}}^{y} = \paren{a^{k}}^{x} e^{y} = \paren{a^{k}}^{x}$.

    Now, for all $b \in G=\cyclic{a}$, we have $b = a^r$ for some $r \in Z$. Therefore, we can write it as a power of $a^k$ as $b = a^r = \paren{\paren{a^{k}}^{x}}^{r} = \paren{a^{k}}^{xr}$. So, $a^k$ also generates $G$.
\end{problem}

\begin{problem}{2 (i)}
    % Let $G$ be a finite group. If $H$ is a subgroup of $G$ then prove that $\abs{H}$ divides $\abs{G}$.

    % \solution
    Let $a_1 H, \dots, a_k H$ be all the distinct cosets of $H$ in $G$. Then, $G = \bigcup_{j=1}^{k} a_j H$ because for all $g \in G$, $g \in a_j H$ for some $j$.

    Moreover, $|aH| = |bH|$, for all $a, b \in G$. In particular, $|a_j H| = |e H| = |H|$ for all $j$.

    Now, since $G$ is written as a union of distinct sets, we have
    \begin{equation*}
        |G| = |a_1 H| \cup \dots \cup |a_k H| = |H| \cup \dots \cup |H| = k |H|.
    \end{equation*}
    Therefore, $|H|$ divides $|G|$.
\end{problem}

\begin{problem}{2 (ii)}
    % Let $H$ be a subgroup of $G$. If $a, b \in G$, then prove that $aH = bH$ if and only if $a^{-1}b \in H$.

    % \solution
    Suppose $aH = bH$. As $H \subgroup G$, we have $e \in H$. So, $b = be \in bH = aH$. Thus, we can write $b = ah$ for some $h \in H$. Consequently, $a^{-1} b = h \in H$.

    Conversely, let $a^{-1} b \in H$. This means $a^{-1} b = h$ for some $h \in H$. As a result, $b = ah$. Therefore,
    \begin{equation*}
        bH = \set{bk \st k \in H} = \set{(ah)k \st k \in H} = \set{ah' \st h' \in H} = aH.
    \end{equation*}
    Here we used the associativity and closure property because $H \subgroup G$.
\end{problem}

\begin{problem}{3}
    % Let $C_{24} = \set{1, a, a^2, \dots, a^{23} \st a^{24}=1}$ be the cyclic group of order $24$. If $H = \cyclic{a^4}$ then find $[C_{24} : H]$.

    % \solution
    We have $O(a^4) = 6$. So, $\abs{H} = O(a^4) = 6$. Also, $\abs{C_{24}} = 24$.
    \begin{equation*}
        [C_{24} : H] = \frac{\abs{C_{24}}}{\abs{H}} = \frac{24}{6} = 4.
    \end{equation*}
\end{problem}

\clearpage

\begin{problem}{4 (i)}
    % Let $\Z_{15} = {0,1,2,\dots,14}$ be the group of integers under addition modulo $15$. Let $H = \set{0,3,6,9,12}$ is a subgroup of $\Z_{15}$. Find the partition of $\Z_{15}$ into left cosets of $H$ in $\Z_{15}$.

    % \solution
    Firstly note that there are exactly $[Z_{15} : H] = \abs{\Z_{15}}/\abs{H} = 15/5 = 3$ distinct (left) cosets of $H$ in $\Z_{15}$. These are $\set{H, 1+H, 2+H}$. So,
    \begin{equation*}
        \Z_{15} = H \cup (1+H) \cup (2+H).
    \end{equation*}
    Here, $1+H = \set{1,4,7,10,13}$ and $2+H = \set{2,5,8,11,14}$.
\end{problem}

\begin{problem}{4 (ii)}
    Find all the subgroups of $\Z_{15}$.

    \solution $\Z_{15}$ is a cyclic group, so all of its subgroups are also cyclic. We arrange these by the GCD of $n = 15$ and the order of the generator.
    \begin{equation*}
    \begin{aligned}
        \gcd &= 1:  & & \cyclic{1}   = \set{1}\\
        \gcd &= 3:  & & \cyclic{a^5} = \cyclic{a^{10}} = \set{1, a^5, a^{10}}\\
        \gcd &= 5:  & & \cyclic{a^3} = \cyclic{a^6} = \cyclic{a^9} = \cyclic{a^{12}} = \set{1, a^3, a^6, a^9, a^{12}}\\
        \gcd &= 15: & & \cyclic{a} = \cyclic{a^2} = \cyclic{a^4} = \cyclic{a^7} = \cyclic{a^8} = \cyclic{a^{11}} = \cyclic{a^{13}} = \cyclic{a^{14}} = \Z_{15}
    \end{aligned}
    \end{equation*}
\end{problem}

\begin{problem}{5}
    % If $a, b \in \Z$, define $a R b$ if $ab \geq 0$. Prove or disprove that $R$ is an equivalence relation on $\Z$.

    % \solution
    This is not an equivalence relation because it is not transitive.

    We have $-1 R 0$ because $(-1)(0) = 0 \geq 0$, and $0 R 1$ because $(0)(1) = 0 \geq 0$, but $-1 \slashed{R} 1$ because $(-1)(1) = -1 \ngeq 0$.
\end{problem}

\end{document}