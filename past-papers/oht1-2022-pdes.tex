\documentclass[11pt]{penrose}


\usepackage{mathsphystools}

\author{Rashid M. Talha}

\newenvironment{problem}[2][Question]{\textbf{#1 #2.}\par}{}
\newcommand{\solution}{\textit{Solution.}\hspace{2mm}}

\newcommand{\warningtext}{\textbf{Disclaimer:} Use at your own risk. Errors possible.}

\title{Past Paper: Partial Differential Equations}
\subtitle{OHT-I, 2022}
\begin{document}

\maketitle
\warningtext

\begin{problem}{1}
    Reduce each of the following equations into canonical form and then find the general solution
    \begin{enumerate}
        \item[(a)] $u_x + 2xy u_y = x$.
        \item[(b)] $u_x - y u_y - u = 1$.
    \end{enumerate}
    
    \solution TBC
\end{problem}

\begin{problem}{2}
    Solve the following first-order quasi-linear equation
    \begin{equation*}
        (cy-bz) z_x + (az-cx) z_y = bx-ay
    \end{equation*}
    
    \solution TBC
\end{problem}

\begin{problem}{3}
    Reduce the following equation into canonical form
    \begin{equation*}
        u_{xx} - u_{yy} + 3u_{x} - 2u_{y} + u = 0.
    \end{equation*}
    
    \solution TBC
\end{problem}

\begin{problem}{4}
    Verify that the function
    \begin{equation*}
        u = \varphi(xy) + x \psi\paren*{\frac{y}{x}}
    \end{equation*}
    is the general solution of the equation
    \begin{equation*}
        x^2 u_{xx} - y^2 u_{yy} = 0.
    \end{equation*}

    \solution We have
    \begin{align*}
        u_{x} &= y \varphi'(xy) + \psi\paren*{\frac{y}{x}} - x \paren*{\frac{y}{x^2}} \psi'\paren*{\frac{y}{x}} = y \varphi'(xy) + \psi\paren*{\frac{y}{x}} - \frac{y}{x} \psi'\paren*{\frac{y}{x}}\\
        u_{y} &= x \varphi'(xy) + x \paren*{\frac{1}{x}} \psi'\paren*{\frac{y}{x}} = x \varphi'(xy) + \psi'\paren*{\frac{y}{x}}\\
        u_{xx} &= y^2 \varphi''(xy) - \frac{y}{x^2} \psi'\paren*{\frac{y}{x}} + \frac{y}{x^2} \psi'\paren*{\frac{y}{x}} + \frac{y^2}{x^3} \psi''\paren*{\frac{y}{x}}
        = y^2 \varphi''(xy) + \frac{y^2}{x^3} \psi''\paren*{\frac{y}{x}}\\
        u_{yy} &= x^2 \varphi''(xy) + \frac{1}{x} \psi''\paren*{\frac{y}{x}}
    \end{align*}
    Therefore,
    \begin{equation*}
        x^2 u_{xx} - y^2 u_{yy} = x^2 y^2 \varphi''(xy) + \frac{y^2}{x} \psi''\paren*{\frac{y}{x}} - x^2 y^2 \varphi''(xy) + \frac{y^2}{x} \psi''\paren*{\frac{y}{x}} = 0
    \end{equation*}
\end{problem}

\end{document}