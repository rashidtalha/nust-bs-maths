\documentclass[11pt]{penrose}


\usepackage{mathsphystools}

\author{Rashid M. Talha}

\newenvironment{problem}[2][Question]{\textbf{#1 #2.}\par}{}
\newcommand{\solution}{\textit{Solution.}\hspace{2mm}}

\newcommand{\warningtext}{\textbf{Disclaimer:} Use at your own risk. Errors possible.}

\title{Past Paper: Numerical Analysis-I}
\subtitle{Midterm Exam, 2024}
\begin{document}

\maketitle
\warningtext

\begin{problem}{1}
    Convert the following numbers to base $10$.
    \begin{enumerate}[label=(\alph*)]
        \item $x = (10 010 110)_2$
        \item $x = (1 011)_2$
        \item $x = (777)_8$
    \end{enumerate}

    \solution \phantom{}
    \begin{enumerate}[label=(\alph*)]
        \item $x = (10 010 110)_2 = 0(1) + 1(2) + 1(4) + 0(8) + 1(16) + 0(32) + 0(64) + 1(128) = 150$.
        \item $x = (1 011)_2 = 1(1) + 1(2) + 0(4) + 1(8) = 11$.
        \item $x = (777)_8 = 7(1) + 7(8) + 7(64) = 511$.
    \end{enumerate}
\end{problem}

\begin{problem}{2}
    Determine the LU factorization of the given matrix using Crout's method
    \begin{equation*}
        \begin{bmatrix*}[r] 2 & 1 & 4 \\ 8 & -3 & 2 \\ 4 & 11 & -1 \end{bmatrix*}
    \end{equation*}

    \solution We take
    \begin{equation*}
        L = \begin{bmatrix*} \ell_{11} & 0 & 0 \\ \ell_{21} & \ell_{22} & 0 \\ \ell_{31} & \ell_{32} & \ell_{33} \end{bmatrix*}
        \quad\text{and}\quad
        U = \begin{bmatrix*} 1 & u_{12} & u_{13} \\ 0 & 1 & u_{23} \\ 0 & 0 & 1 \end{bmatrix*}
    \end{equation*}
    Then, $A = LU$ gives
    \begin{gather*}
        \ell_{11} = 2, \quad \ell_{21} = 8, \quad \ell_{31} = 4,\\
        \ell_{11} u_{12} = 1 \implies u_{12} = 0.5, \quad \ell_{11} u_{13} = 4 \implies u_{13} = 2\\
        \ell_{21} u_{12} + \ell_{22} = -3 \implies \ell_{22} = -7, \quad \ell_{31} u_{12} + \ell_{32} = 11 \implies \ell_{32} = 9\\
        \ell_{21} u_{13} + \ell_{22} u_{23} = 2 \implies u_{23} = 2, \quad \ell_{31} u_{13} + \ell_{32} u_{23} + \ell_{33} = -1 \implies \ell_{33} = -27.
    \end{gather*}
    Overall,
    \begin{equation*}
        \begin{bmatrix*}[r] 2 & 1 & 4 \\ 8 & -3 & 2 \\ 4 & 11 & -1 \end{bmatrix*}
        = \begin{bmatrix*}[r] 2 & 0 & 0 \\ 8 & -7 & 0 \\ 4 & 9 & -27 \end{bmatrix*}
        \begin{bmatrix*}[r] 1 & 0.5 & 2 \\ 0 & 1 & 2 \\ 0 & 0 & 1 \end{bmatrix*}
    \end{equation*}
\end{problem}

\begin{problem}{3}
    Apply the Gauss-Seidel method to solve the following system of equations.
    \begin{equation*}
        4x - y + z = 12,
        \quad
        x - 2y + 4z = 5,
        \quad
        -x + 4y - 2z = -1.
    \end{equation*}
    Perform two iterations with $5$ decimal place digit calculations.

    \solution We arrange this system as
    \begin{equation*}
    \begin{gathered}
    \begin{aligned}
        4x - y + z &= 12\\
        -x + 4y - 2z &= -1\\
        x - 2y + 4z &= 5
    \end{aligned}
    \qquad\to\qquad
    \begin{aligned}
        \underbrace{
        \begin{bmatrix*}[r]
            4 & -1 & 1 \\ -1 & 4 & -2 \\ 1 & -2 & 4
        \end{bmatrix*}
        }_{A}
        \underbrace{
        \begin{bmatrix}
            x \\ y \\ z
        \end{bmatrix}
        }_{X}
        =
        \underbrace{
        \begin{bmatrix}
            12 \\ -1 \\ 5
        \end{bmatrix}
        }_{B}
    \end{aligned}
    \end{gathered}
    \end{equation*}
    to ensure $\abs{a_{11}} > \abs{a_{12}} + \abs{a_{13}}$, $\abs{a_{22}} > \abs{a_{21}} + \abs{a_{23}}$ and $\abs{a_{33}} > \abs{a_{31}} + \abs{a_{32}}$.

    The Gauss-Seidel method for this system gives
    \begin{align*}
        x_{n+1} &= \frac{1}{a_{11}} \paren*{b_{1} - a_{12}y_{n} - a_{13}z_{n}} = \frac{1}{4} \paren*{12 + y_{n} - z_{n}},\\
        y_{n+1} &= \frac{1}{a_{22}} \paren*{b_{2} - a_{21}x_{n+1} - a_{23}z_{n}} = \frac{1}{4} \paren*{-1 + x_{n+1} + 2 z_{n}},\\
        z_{n+1} &= \frac{1}{a_{33}} \paren*{b_{3} - a_{31}x_{n+1} - a_{32}y_{n+1}} = \frac{1}{4} \paren*{5 - x_{n+1} + 2 y_{n+1}}.
    \end{align*}

    So, starting with $x_0 = y_0 = z_0 = 0$, we have:

    \underline{1st Iteration:}
    \begin{align*}
        x_{1} &= \frac{1}{4} \paren*{12 + 0 - 0} = 3.00000,\\
        y_{1} &= \frac{1}{4} \paren*{-1 + 3 + 0} = 0.50000,\\
        z_{1} &= \frac{1}{4} \paren*{5 - 3 + 2 (0.5)} = 0.75000.
    \end{align*}
    
    \underline{2nd Iteration:}
    \begin{align*}
        x_{2} &= \frac{1}{4} \paren*{12 + 0.5 - 0.75} = 2.93750,\\
        y_{2} &= \frac{1}{4} \paren*{-1 + 2.9375 + 2 (0.75)} = 0.85938,\\
        z_{2} &= \frac{1}{4} \paren*{5 - 2.9375 + 2 (0.85938)} = 0.94532.
    \end{align*}
\end{problem}

\begin{problem}{4 (a)}
    Derive the formula for the false-position method.

    \solution Suppose the equation $f(x) = 0$ has a root $x_r$ in the interval $[x_\ell, x_u]$. That means $f(x_r) = 0$. In order for the root to lie in the interval $(x_\ell, x_u)$, we must have $f(x_\ell)f(x_u) < 0$. So, without loss of generality, we take $f(x_\ell) < 0$ and $f(x_u) > 0$.

    By the method of similar triangles, we obtain
    \begin{equation*}
        \frac{f(x_u)}{x_u - x_r} = \frac{-f(x_\ell)}{x_r - x_\ell}
    \end{equation*}
    which can be re-expressed as
    \begin{equation*}
        (x_r - x_\ell) f(x_u) = - (x_u - x_r) f(x_\ell)
        \implies
        x_r (f(x_u) - f(x_\ell)) = x_\ell f(x_u) - x_u f(x_\ell).
    \end{equation*}

    Therefore, if the root $x_r$ lies within the interval $[x_\ell, x_u]$, then it can be iteratively estimated
    \begin{equation*}
        x_r = \frac{x_\ell f(x_u) - x_u f(x_\ell)}{f(x_u) - f(x_\ell)}.
    \end{equation*}
\end{problem}

\begin{problem}{4 (b)}
    Apply the method of false position to find a real root of the equation $x^4 - 11x + 8 = 0$ within the interval $[1,2]$. Perform five decimal place digit calculations. Apply two iterations.

    \solution Let $f(x) = x^4 - 11x + 8$.

    \underline{1st Iteration:}
    We set $x_\ell = 1$ and $x_u = 2$. This gives $f(x_\ell) = -2$ and $f(x_u) = 2$. So, $f(x_\ell)f(x_u) < 0$ and a root lies in the interval $[1,2]$. Then,
    \begin{equation*}
        x_r = \frac{x_\ell f(x_u) - x_u f(x_\ell)}{f(x_u) - f(x_\ell)} = \frac{(1)(2) - (2)(-2)}{2 + 2} = 1.50000.
    \end{equation*}
    
    \underline{2nd Iteration:}
    Next, $f(x_r) = -3.4375 < 0$. Therefore, $f(x_r)f(x_u) < 0$ and the root lies in the interval $[1.5, 2]$. So, we set $x_\ell = 1.5$ and $x_u = 2$. Then,
    \begin{align*}
        x_r = \frac{x_\ell f(x_u) - x_u f(x_\ell)}{f(x_u) - f(x_\ell)} = \frac{(1.5)(2) - (2)(-3.4375)}{2 + 3.4375} = 1.81609.
    \end{align*}
\end{problem}

\begin{problem}{5 (a)}
    Derive Newton-Raphson method.

    \solution Suppose the equation $f(x) = 0$ has a root at $x = x_r$. That means $f(x_r) = 0$. We can Taylor expand $f(x)$ near $x_r$ to get
    \begin{equation*}
        f(x_r) = f(x) + (x_r - x) f'(x) + \BigO(2).
    \end{equation*}
    Now, working up to linear order and using $f(x_r) = 0$ gives $f(x) + (x_r - x) f'(x) = 0$, which can be re-arranged as
    \begin{equation*}
        x_r = x - \frac{f(x)}{f'(x)}.
    \end{equation*}
    This gives an iterative formula for root estimation, with an initial guess $x = x_0$,
    \begin{equation*}
        x_{n+1} = x_n - \frac{f(x_n)}{f'(x_n)}.
    \end{equation*}
\end{problem}

\begin{problem}{5 (b)}
    Evaluate $\sqrt{29}$ by Newton-Raphson method. Carry out five decimal place digit calculations. Stop the calculations when five decimal digits match with each other. Take an initial guess as $x_0 = 3.3$.

    \solution Take $f(x) = x^2 - 29$. Then, $f(x) = 0 \implies x = \pm \sqrt{29}$. So, we apply the Newton-Raphson method to estimate $f(x) = 0$. The corresponding formula is
    \begin{equation*}
        x_{n+1} = x_{n} - \frac{f(x_n)}{f'(x_n)} = x_{n} - \frac{x_n^2 - 29}{2 x_n} = \frac{x_n}{2} - \frac{29}{2 x_n}.
    \end{equation*}
    As a result, we have
    \begin{align*}
        x_0 &= 3.3\\
        x_1 &= 6.0439394\dots\\
        x_2 &= 5.4210672\dots\\
        x_3 &= 5.3852837\dots\\
        x_4 &= 5.3851648\dots\\
        x_5 &= 5.3851648\dots
    \end{align*}
    Therefore, $\sqrt{29} = 5.38516$ to 5 decimal digits.
\end{problem}

\begin{problem}{6}
    Convert the system into an iterative form suitable for Newton-Raphson method then solve the following
    \begin{equation*}
        x - y^3 - 1 = 0,
        \quad
        \sin(x) + y - 1 = 0.
    \end{equation*}
    Use the initial guess as $(x_0, y_0) = (0,0)$. Apply one iteration only.

    \solution We have $f_1 (x,y) = x - y^3 - 1$ and $f_2 (x,y) = \sin(x) + y - 1$. The iterative formula for the Newton-Raphson method is
    \begin{align*}
        x_{n+1} &= x_{n} + \left.\frac{\displaystyle -f_1 \frac{\pd f_2}{\pd y} + f_2 \frac{\pd f_1}{\pd y}}{J( f_1, f_2 )}\right|_{(x_n, y_n)}\\
        y_{n+1} &= y_{n} + \left.\frac{\displaystyle -f_2 \frac{\pd f_1}{\pd x} + f_1 \frac{\pd f_2}{\pd x}}{J( f_1, f_2 )}\right|_{(x_n, y_n)}
    \end{align*}
    where
    \begin{equation}
        J(f_1, f_2)
        = \begin{vmatrix}
        \frac{\pd f_1}{\pd x} & \frac{\pd f_1}{\pd y}\\
        \frac{\pd f_2}{\pd x} & \frac{\pd f_2}{\pd y}
        \end{vmatrix}
        = \frac{\pd f_1}{\pd x} \frac{\pd f_2}{\pd y} - \frac{\pd f_1}{\pd y} \frac{\pd f_2}{\pd x}.
    \end{equation}

    For the given $f_1$ and $f_2$ we have
    \begin{equation}
        \frac{\pd f_1}{\pd x} = 1, \quad
        \frac{\pd f_1}{\pd y} = -3y^2, \quad
        \frac{\pd f_2}{\pd x} = \cos x, \quad
        \frac{\pd f_2}{\pd y} = 1, \quad
        J(f_1, f_2) = 1 + 3y^2 \cos x.
    \end{equation}
    Starting with $(x_0, y_0) = (0,0)$ we get
    \begin{align*}
        x_{1} &= \left.\frac{\displaystyle -(x - y^3 - 1)(1) + (\sin(x) + y - 1)(-3y^2)}{1 + 3y^2 \cos x}\right|_{(0,0)} = 1,\\
        y_{1} &= \left.\frac{\displaystyle -(\sin(x) + y - 1)(1) + (x - y^3 - 1)(\cos x)}{1 + 3y^2 \cos x}\right|_{(0,0)} = 0.
    \end{align*}
\end{problem}

\end{document}