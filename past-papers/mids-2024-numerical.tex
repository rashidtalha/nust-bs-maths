\documentclass[11pt]{penrose}


\usepackage{mathsphystools}

\author{Rashid M. Talha}

\newenvironment{problem}[2][Question]{\textbf{#1 #2.}\par}{}
\newcommand{\solution}{\textit{Solution.}\hspace{2mm}}

\newcommand{\warningtext}{\textbf{Disclaimer:} Use at your own risk. Errors possible.}

\title{Past Paper: Numerical Analysis-I}
\subtitle{Midterm Exam, 2024}
\begin{document}

\maketitle
\warningtext

\begin{problem}{1}
    Convert the following numbers to base $10$.
    \begin{enumerate}[label=(\alph*)]
        \item $x = (10 010 110)_2$
        \item $x = (1 011)_2$
        \item $x = (777)_8$
    \end{enumerate}

    \solution \phantom{}
    \begin{enumerate}[label=(\alph*)]
        \item $x = (10 010 110)_2 = 0(1) + 1(2) + 1(4) + 0(8) + 1(16) + 0(32) + 0(64) + 1(128) = 999$.
        \item $x = (1 011)_2 = 1(1) + 1(2) + 0(4) + 1(8) = 11$.
        \item $x = (777)_8 = 7(1) + 7(8) + 7(64) = 999$.
    \end{enumerate}
\end{problem}

\begin{problem}{2}
    Determine the LU factorization of the given matrix using Crout's method
    \begin{equation*}
        \begin{bmatrix*} 2 & 1 & 4 \\ 8 & -3 & 2 \\ 4 & 11 & -1 \end{bmatrix*}
    \end{equation*}

    \solution We take
    \begin{equation*}
        L = \begin{bmatrix*} \ell_{11} & 0 & 0 \\ \ell_{21} & \ell_{22} & 0 \\ \ell_{31} & \ell_{32} & \ell_{33} \end{bmatrix*}
        \quad\text{and}\quad
        U = \begin{bmatrix*} 1 & u_{12} & u_{13} \\ 0 & 1 & u_{23} \\ 0 & 0 & 1 \end{bmatrix*}
    \end{equation*}
    Then, $A = LU$ gives
    \begin{gather*}
        \ell_{11} = 2, \quad \ell_{21} = 8, \quad \ell_{31} = 4,\\
        \ell_{11} u_{12} = 1 \implies u_{12} = 0.5, \quad \ell_{11} u_{13} = 4 \implies u_{13} = 2\\
        \ell_{21} u_{12} + \ell_{22} = -3 \implies \ell_{22} = -7, \quad \ell_{31} u_{12} + \ell_{32} = 11 \implies \ell_{32} = 9\\
        \ell_{21} u_{13} + \ell_{22} u_{23} = 2 \implies u_{23} = 2, \quad \ell_{31} u_{13} + \ell_{32} u_{23} + \ell_{33} = -1 \implies \ell_{33} = -27.
    \end{gather*}
    Overall,
    \begin{equation*}
        \begin{bmatrix*} 2 & 1 & 4 \\ 8 & -3 & 2 \\ 4 & 11 & -1 \end{bmatrix*}
        = \begin{bmatrix*} 2 & 0 & 0 \\ 8 & -7 & 0 \\ 4 & 9 & -27 \end{bmatrix*}
        \begin{bmatrix*} 1 & 0.5 & 2 \\ 0 & 1 & 2 \\ 0 & 0 & 1 \end{bmatrix*}
    \end{equation*}
\end{problem}

\begin{problem}{3}
    Apply the Gauss-Seidel method to solve the following system of equations.
    \begin{equation*}
        4x - y + z = 12,
        \quad
        x - 2y + 4z = 5
        \quad
        -x + 4y - 2z = -1.
    \end{equation*}
    Perform two iterations with $5$ decimal place digit calculations.

    \solution Solution
\end{problem}

\begin{problem}{4 (a)}
    Derive the formula for the false-position method.

    \solution Solution
\end{problem}

\begin{problem}{4 (b)}
    Apply the method of false position to find a real root of the equation $x^4 - 11x + 8 = 0$ within the interval $[1,2]$. Perform five decimal place digit calculations. Apply two iterations.

    \solution Solution
\end{problem}

\begin{problem}{5 (a)}
    Derive Newton-Raphson method.

    \solution Solution
\end{problem}

\begin{problem}{5 (b)}
    Evaluate $\sqrt{29}$ by Newton-Raphson method. Carry out five decimal place digit calculations. Stop the calculations when five decimal digits match with each other. Take an initial guess as $x_0 = 3.3$.

    \solution Take $f(x) = x^2 - 29$. Then, $f(x) = 0 \implies x = \pm \sqrt{29}$. So, we apply the Newton-Raphson method to estimate $f(x) = 0$. The corresponding formula is
    \begin{equation*}
        x_{n+1} = x_{n} - \frac{f(x_n)}{f'(x_n)}.
    \end{equation*}
    with $f'(x) = 2x$. As a result, we have
    \begin{align*}
        x_0 &= 3.3\\
        x_1 &= SOMETHING\\
        x_2 &= SOMETHING\\
        x_3 &= SOMETHING\\
        x_4 &= SOMETHING\\
        x_5 &= SOMETHING.
    \end{align*}
    Therefore, $\sqrt{29} = 5.ABCDE$ to 5 decimal digits.
\end{problem}

\begin{problem}{6}
    Convert the system into an iterative form suitable for Newton-Raphson method then solve the following
    \begin{equation*}
        x - y^3 - 1 = 0,
        \quad
        \sin(x) + y - 1 = 0.
    \end{equation*}
    Use the initial guess as $(x_0, y_0) = (0,0)$. Apply one iteration only.

    \solution Solution
\end{problem}

\end{document}