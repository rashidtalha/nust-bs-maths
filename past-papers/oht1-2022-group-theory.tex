\documentclass[11pt]{penrose}


\usepackage{mathsphystools}

\author{Rashid M. Talha}

\newenvironment{problem}[2][Question]{\textbf{#1 #2.}\par}{}
\newcommand{\solution}{\textit{Solution.}\hspace{2mm}}

\newcommand{\warningtext}{\textbf{Disclaimer:} Use at your own risk. Errors possible.}

\title{Past Paper: Group Theory}
\subtitle{OHT-I, 2022}
\begin{document}

\maketitle
\warningtext

\begin{problem}{1 (a)}
    \underline{Definition}: A group $G$ is called cyclic if there exists $g \in G$ such that every element of $G$ can be written in the form $g^n$ for some $n \in \Z$.

    \underline{Infinite cyclic}: The group $\Z$ of integers with the usual addition is an infinite cyclic group generated by $1$.
    
    \underline{Finite cyclic}: The group $(\Z_3, +_3)$ of integers with addition modulo $3$ is a finite cyclic group generated by $[1]$. The group $(\Z_5, +_5)$ of integers with addition modulo $5$ is a finite cyclic group generated by $[1]$.
\end{problem}

\begin{problem}{1 (b)}
    Consider the group $\Z$ with the usual addition. Both $2\Z$ and $3\Z$ are subgroups of $\Z$. 

    Let, $F = 2\Z \cup 3\Z$. Then, $2, 3 \in F$ but $2+3 = 5 \notin F$. So, $F$ is not closed under the group operation. Therefore, $2\Z \cup 3\Z$ is not a subgroup of $\Z$.
\end{problem}

\begin{problem}{2 (a)}
    Firstly, the set $H$ is non-empty because $a^0 = e \in H$.

    Next, take any $x, y \in H$. Then, we can write $x = a^p$ and $y = a^q$ for some $p, q \in \Z$. So,
    \begin{equation*}
        x y^{-1} = a^p a^{-q} = a^{p-q} \in H
    \end{equation*}
    because $p-q \in \Z$. Therefore, by the subgroup criteria, $H \subgroup G$.
\end{problem}

\begin{problem}{2 (b)}
    Let $G = (\C, +)$ and $H = \set{a+bi \st a,b \in \R, ab \geq 0}$.

    Take $z_1 = -2 + 0i$ and $z_2 = 1 + 1i$. Now, $z_1 \in H$ because $-2, 0 \in \R$ and $(-2)(0) = 0 \geq 0$. Also, $z_2 \in H$ because $1 \in \R$ and $(1)(1) = 1 \geq 0$.

    However, $z_1 + z_2 = -1 + 1i \notin H$ because $(-1)(1) = -1 \ngeq 0$.

    So, $H$ is not closed under the group operation. Therefore, $H$ is not a subgroup of $G$.
\end{problem}

\begin{problem}{3}
    Let $H$ be a non-empty subset of $G$.

    Suppose $H \subgroup G$. Take any $a, b \in H$. Then, $b^{-1} \in H$ by the existence of inverses, and $a b^{-1} \in H$ by the closure property.

    Conversely, suppose $a b^{-1} \in H$ for all $a, b \in H$.
    \begin{itemize}
        \item (Associativity.) $H$ has the same binary operator as $G$, so it inherits associativity.
        \item (Identity.) As $H \neq \emptyset$, we have some $a \in H$. Now, $a a^{-1} = e$ because $a \in G$. Therefore, $a a^{-1} \in H \implies e \in H$.
        \item (Inverses.) Take any $a \in H$. So $e, a \in H \implies e a^{-1} \in H \implies a^{-1} \in H$.
        \item (Closure.) Take any $a, b \in H$. Then, $b^{-1} \in H$. So, $a (b^{-1})^{-1} = ab \in H$.
    \end{itemize}
    Therefore, $H$ is a subgroup of $G$.
\end{problem}

\begin{problem}{4}
    \underline{Definition}: Let $G$ be a group and $g \in G$. The order of $g$ is the smallest positive integer $n$ such that $g^n = e$, the identity element of $G$.

    The order of each element in $\Z_{12}$ is
    \begin{center}
    \begin{tabularx}{0.8\textwidth}{XXXX}
        $O(0)  =  1$ & $O(1)  = 12$ & $O(2)  =  6$ & $O(3)  =  4$\\
        $O(4)  =  3$ & $O(5)  = 12$ & $O(6)  =  2$ & $O(7)  = 12$\\
        $O(8)  =  3$ & $O(9)  =  4$ & $O(10) =  6$ & $O(11) = 12$\\
    \end{tabularx}
    \end{center}
\end{problem}

\begin{problem}{5 (a)}
    Let $G$ be a group and $a \in G$. By the existence of inverses, $a^{-1} \in G$ such that $a a^{-1} = e$. Again, by the existence of inverses, $(a^{-1})^{-1} \in G$ such that $(a^{-1}) (a^{-1})^{-1} = e$. So,
    \begin{equation*}
        (a^{-1})^{-1}
        = e (a^{-1})^{-1}
        = (a a^{-1}) (a^{-1})^{-1}
        = a (a^{-1} (a^{-1})^{-1})
        = a e = a.
    \end{equation*}
    We used the properties of $e$ and associativity in the last step.
\end{problem}

\begin{problem}{5 (a)}
    Let $G$ be a group and $a, b \in G$. Then, $ab \in G$ by closure. By the existence of inverses $(ab)^{-1} \in G$ such that $(ab) (ab)^{-1} = (ab)^{-1} (ab) = e$.

    Now, $(ab)(b^{-1}a^{-1}) = (a(bb^{-1}))a^{-1} = (ae)a^{-1} =aa^{-1} = e$.
    
    And, $(b^{-1}a^{-1})(ab) = (b^{-1}(a^{-1}a))b = (b^{-1}e)b =b^{-1}b = e$.

    So, $(b^{-1}a^{-1})$ is also an inverse of $ab$. By the uniqueness of inverses $(b^{-1}a^{-1}) = (ab)^{-1}$.
\end{problem}

\begin{problem}{5 (c)}
    Firstly, by definition, $(a^{-1} b a)^5 = (a^{-1} b a) (a^{-1} b a) (a^{-1} b a) (a^{-1} b a) (a^{-1} b a)$.

    So, by associativity, we can write
    \begin{align*}
        (a^{-1} b a)^5
        &= (a^{-1} b a) (a^{-1} b a) (a^{-1} b a) (a^{-1} b a) (a^{-1} b a)\\
        &= a^{-1} b a a^{-1} b a a^{-1} b a a^{-1} b a a^{-1} b a\\
        &= a^{-1} (b (a a^{-1})) (b (a a^{-1})) (b (a a^{-1})) (b (a a^{-1})) b a\\
        &= a^{-1} (b e) (b e) (b e) (b e) b a\\
        &= a^{-1} b b b b b a\\
        &= a^{-1} b^5 a.
    \end{align*}
\end{problem}

\end{document}