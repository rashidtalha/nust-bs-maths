\documentclass[11pt]{penrose}


\usepackage{mathsphystools}

\author{Rashid M. Talha}

\newenvironment{problem}[2][Question]{\textbf{#1 #2.}\par}{}
\newcommand{\solution}{\textit{Solution.}\hspace{2mm}}

\newcommand{\warningtext}{\textbf{Disclaimer:} Use at your own risk. Errors possible.}

\title{Past Paper: Differential Geometry}
\subtitle{Midterm Exam, 2024}
\begin{document}

\maketitle
\warningtext

\begin{problem}{1}
    Let $\alpha(t) = (t, t^2, t^3)$ be a curve in the Euclidean space. Calculate the following parameters for the given curve: unit tangent vector, unit normal vector, unit binormal vector, curvature, and torsion.

    \solution Firstly, note that
    \begin{equation*}
        \kappa = \frac{\abs{\alpha' \times \alpha''}}{\abs{\alpha'}^3},\quad
        \tau = \frac{\alpha' \times \alpha'' \cdot \alpha'''}{\abs{\alpha' \times \alpha''}^2},\quad
        \vec{T} = \frac{\alpha'}{\abs{\alpha'}},\quad
        \vec{B} = \frac{\alpha' \times \alpha''}{\abs{\alpha' \times \alpha''}},\quad
        \vec{N} = \vec{B} \times \vec{T}.
    \end{equation*}

    Now, $\alpha'(t) = (1, 2t, 3t^2)$, $\alpha''(t) = (0, 2, 6t)$ and $\alpha'''(t) = (0, 0, 6)$. Therefore,
    \begin{equation*}
        \alpha' \times \alpha'' = (6t^2, -6t, 2),\quad
        \abs{\alpha' \times \alpha''} = \sqrt{4 + 36t^2 + 36t^4} = 2\sqrt{1 + 9t^2 + 9t^4},
    \end{equation*}
    and $\abs{\alpha'} = \sqrt{1 + 4t^2 + 9t^4}$. As a result,
    \begin{align*}
        \kappa &= \frac{2\sqrt{1 + 9t^2 + 9t^4}}{\paren*{\sqrt{1 + 4t^2 + 9t^4}}^3} = 2 \paren*{\frac{1 + 9t^2 + 9t^4}{\paren*{1 + 4t^2 + 9t^4}^3}}^{1/2}\\
        \tau &= \frac{(6t^2, -6t, 2) \cdot (0,0,6)}{\paren*{\sqrt{4 + 36t^2 + 36t^4}}^2} = \frac{12}{4 + 36t^2 + 36t^4} = \frac{3}{1 + 9t^2 + 9t^4}\\
        \vec{T} &= \frac{(1, 2t, 3t^2)}{\sqrt{1 + 4t^2 + 9t^4}}\\
        \vec{B} &= \frac{(6t^2, -6t, 2)}{2\sqrt{1 + 9t^2 + 9t^4}} = \frac{(3t^2, -3t, 1)}{\sqrt{1 + 9t^2 + 9t^4}}.
    \end{align*}
    Lastly, $(3t^2, -3t, 1) \times (1, 2t, 3t^2) = (-2t-9t^3, 1-9t^4, 3t+6t^3)$ gives
    \begin{equation*}
        \vec{N} = \frac{(-2t-9t^3, 1-9t^4, 3t+6t^3)}{\sqrt{(1 + 9t^2 + 9t^4) (1 + 4t^2 + 9t^4)}}.
    \end{equation*}
\end{problem}

\begin{problem}{2 (i)}
    State and prove the Wirtinger's inequality.

    \solution\\
    \underline{Statement}: Let $F : [0, \pi] \to \R$ be a smooth function with $F(0) = F(\pi) = 0$. Then,
        \begin{equation*}
            \int_{0}^{\pi} \paren*{\frac{dF}{dt}}^2 \,dt \geq \int_{0}^{\pi} F^2 \,dt
        \end{equation*}
        and the equality holds if and only if $F(t) = K \sin t$ for all $t \in [0, \pi]$ and $K$ is a constant.
    
    \underline{Proof}: Let $G(t) = F(t) / \sin(t)$. This is well-defined because $\sin(t)$ is non-zero for $t \in (0, \pi)$, and smooth because it is a quotient of two smooth functions. Writing $F(t) = G(t) \sin(t)$ we get,
    \begin{align*}
        \int_{0}^{\pi} \paren*{\frac{dF}{dt}}^2 \,dt
        &= \int_{0}^{\pi} \paren*{ G' \sin t + G \cos t }^2 \,dt\\
        &= \int_{0}^{\pi} \cparen*{G'^2 \sin^2 t + G^2 \cos^2 t + 2 G G' \sin t \cos t} \,dt\\
        &= \int_{0}^{\pi} \cparen*{G'^2 \sin^2 t + G^2 \cos^2 t} + \int_{0}^{\pi} \cparen*{2 G G' \sin t \cos t} \,dt.
    \end{align*}

    Writing $2 G G' = (G^2)'$ and $\sin t \cos t = \frac{1}{2} \sin 2t$ gives
    \begin{equation*}
        \int_{0}^{\pi} \cparen*{2 G G' \sin t \cos t} \,dt
        = \int_{0}^{\pi} \cparen*{(G^2)' \times \frac{1}{2} \sin 2t} \,dt.
        = \frac{1}{2} \int_{0}^{\pi} \cparen*{(G^2)' \sin 2t} \,dt.
    \end{equation*}
    Integrating this by parts, leads to
    \begin{equation*}
        \frac{1}{2} \int_{0}^{\pi} \cparen*{(G^2)' \sin 2t} \,dt
        = \bigg. \frac{1}{2} G(t)^2  \sin (2t) \bigg|_{0}^{\pi} - \int_{0}^{\pi} G^2(t) \cos 2t \,dt
        = - \int_{0}^{\pi} G^2(t) \cos 2t \,dt.
    \end{equation*}
    By using $\cos(2t) = \cos^2 t - \sin^2 t$, we obtain
    \begin{equation*}
        \frac{1}{2} \int_{0}^{\pi} \cparen*{(G^2)' \sin 2t} \,dt
        = -\int_{0}^{\pi} G^2(t) \paren*{\cos^2 t - \sin^2 t} \,dt.
    \end{equation*}
    Consequently,
    \begin{align*}
        \int_{0}^{\pi} \paren*{\frac{dF}{dt}}^2 \,dt
        &= \int_{0}^{\pi} \cparen*{G'^2 \sin^2 t + G^2 \cos^2 t} + \int_{0}^{\pi} \cparen*{2 G G' \sin t \cos t} \,dt\\
        &= \int_{0}^{\pi} \cparen*{G'^2 \sin^2 t + G^2 \cos^2 t} - \int_{0}^{\pi} G^2(t) \paren*{\cos^2 t - \sin^2 t} \,dt\\
        &= \int_{0}^{\pi} \cparen*{G'^2 \sin^2 t + G^2 \sin^2 t} \,dt\\
        &= \int_{0}^{\pi} \paren*{G' \sin t}^2 \,dt + \int_{0}^{\pi} F^2 \,dt.
    \end{align*}

    Therefore,
    \begin{equation*}
        \int_{0}^{\pi} \paren*{\frac{dF}{dt}}^2 \,dt
        - \int_{0}^{\pi} F^2 \,dt
        = \int_{0}^{\pi} \paren*{G' \sin t}^2 \,dt \geq 0
    \end{equation*}
    since $\paren*{G' \sin t}^2 \geq 0$. We find that the equality holds if and only if $G' \equiv 0$ (because $\sin^2 t > 0$ for $t \in (0, \pi)$). That is, the equality holds if and only if $G(t) = K$ for some constant $K$ and $F(t) = K \sin t$.
\end{problem}

\begin{problem}{2 (ii)}
    Using the Wirtinger's inequality, prove the isoperimetric inequality.

    \solution\\
    \underline{Statement}: Let $\gamma$ be a simple closed curve in the plane with length $\ell(\gamma)$, and enclosing an area $A(\gamma)$. Then,
        \begin{equation*}
            A(\gamma) \leq \frac{1}{4\pi} \ell(\gamma)^2
        \end{equation*}
        and the equality holds if and only if $\gamma$ is a circle.
    
    \underline{Proof}: Without loss of generality, we assume that the curve is parametrised by arc-length, so $\gamma : [0, \ell] \to \R^2$ with $\gamma = \gamma(s)$. Also, we translate the curve so that $\gamma(0) = 0$; this doesn't effect the value of $\ell(\gamma)$ and $A(\gamma)$. Also note that $\gamma(\ell) = \gamma(0)$ because it is a closed curve.

    We reparametrize the curve using the parameter $t = \pi s / \ell$. Then, $t \in [0, \pi]$ and $ds/dt = \ell/\pi$. We can then write the curve in coordinates as $\gamma(t) = \paren*{x(t), y(t)}$. Switching to polar coordinates $x = r \cos\theta$ and $y = r \sin\theta$ we find $\dot{x}^2 + \dot{y}^2 = \dot{r}^2 + r^2 \dot{\theta}^2$ and $x\dot{y} - y\dot{x} = r^2 \dot{\theta}$ because $\dot{x} = \dot{r} \cos\theta - r \dot{\theta} \sin\theta$ and $\dot{y} = \dot{r} \sin\theta + r \dot{\theta} \cos\theta$.

    Therefore,
    \begin{equation*}
        \dot{r}^2 + r^2 \dot{\theta}^2
        = \paren*{ \paren*{\frac{dx}{dt}}^2 + \paren*{\frac{dy}{dt}}^2 }
        = \paren*{ \paren*{\frac{dx}{ds}}^2 + \paren*{\frac{dy}{ds}}^2 } \paren*{\frac{ds}{dt}}^2
        = \frac{\ell^2}{\pi^2}
    \end{equation*}
    as $(dx/ds)^2 + (dy/ds)^2 = 1$. This allows us to write
    \begin{equation*}
        \frac{\ell^2}{4\pi^2} = \frac{1}{4} \int_0^\pi \paren*{\dot{r}^2 + r^2 \dot{\theta}^2} \,dt.
    \end{equation*}
    Also, $A = \frac{1}{2} \int_0^\pi \paren*{x\dot{y} - y\dot{x}} \,dt = \frac{1}{2} \int_0^\pi r^2 \dot{\theta} \,dt$. Consequently,
    \begin{align*}
        \frac{\ell^2}{4\pi} - A
        &= \frac{1}{4} \int_0^\pi \paren*{\dot{r}^2 + r^2 \dot{\theta}^2 - 2r^2\dot{\theta}} \,dt\\
        &= \frac{1}{4} \int_0^\pi \paren*{\dot{r}^2 + r^2 \dot{\theta}^2 - 2r^2\dot{\theta} + r^2 - r^2} \,dt\\
        &= \frac{1}{4} \int_0^\pi \paren*{r^2 (\dot{\theta}^2 - 2\dot{\theta} + 1) + \dot{r}^2 - r^2} \,dt\\
        &= \frac{1}{4} \int_0^\pi r^2 \paren{\dot{\theta}-1}^2 \,dt + \frac{1}{4}\int_0^\pi \paren*{\dot{r}^2 - r^2} \,dt. \tag{\textasteriskcentered}
        \label{eq:isoperi-proof-a}
    \end{align*}
    In equation~\eqref{eq:isoperi-proof-a}, the first term is non-negative because the integrand is non-negative, and the second is non-negative by Wirtinger's inequality. Consequently, $\frac{\ell^2}{4\pi} - A \geq 0$. Or equivalently,
    \begin{equation*}
        A \leq \frac{1}{4\pi} \ell^2.
    \end{equation*}
    Also, the first term in equation~\eqref{eq:isoperi-proof-a} is zero precisely when $\dot{\theta} = 1$ (i.e. $\theta = t + \theta_0$ with $\theta_0$ constant), while the second term becomes zero when $r = K \sin\theta$ for some constant $K$. Collectively, both of these terms vanish when $r = K \sin(t + \theta_0)$. This is the equation of a circle with diameter $K$ and passing through the origin. Therefore, $A = \ell^2 / 4\pi$ if and only if the curve is a circle.
\end{problem}

\begin{problem}{3 (i)}
    For a unit speed curve $\beta(s)$, show that $\beta'' \cdot \beta''' \times \beta'''' = \kappa^5 \frac{d}{ds}\paren*{\frac{\tau}{\kappa}}$, where $\kappa$ and $\tau$ are curvature and torsion of the curve $\beta(s)$.

    \solution For a unit speed curve $\beta'' = \kappa \vec{N}$ and $\beta''' = -\kappa^2 \vec{T} + \kappa' \vec{N} + \tau\kappa \vec{B}$. So,
    \begin{equation*}
        \beta^{(4)}
        = -3\kappa\dot{\kappa} \vec{T}
        + \paren*{\dot{\tau}\kappa + 2\tau\dot{\kappa}} \vec{B}
        + \paren*{\ddot{\kappa} - \kappa^3 - \tau^2 \kappa} \vec{N}.
    \end{equation*}
    Frenet-Serret equations were used to simplify the previous expression. Now, by the cyclic property of the triple scalar product, $\beta'' \cdot \beta''' \times \beta^{(4)} = \beta^{(4)} \cdot \beta'' \times \beta'''$. And
    \begin{equation*}
        \beta'' \times \beta'''
        = \kappa\vec{N} \times \paren*{-\kappa^2 \vec{T} + \kappa' \vec{N} + \tau\kappa \vec{B}}
        = \tau \kappa^2 \vec{T} + \kappa^3 \vec{B}
    \end{equation*}
    where we used $\vec{N} \times \vec{T} = - \vec{B}$ and $\vec{N} \times \vec{B} = \vec{T}$. Consequently,
    \begin{equation*}
        \beta^{(4)} \cdot \beta'' \times \beta'''
        = \paren*{-3 \kappa^3 \tau \dot{\kappa}} + \paren*{\kappa^4 \dot{\tau} + 2 \kappa^3 \tau \dot{\kappa}}
        = \kappa^4 \dot{\tau} - \kappa^3 \tau \dot{\kappa}.
    \end{equation*}
    We can re-write this as
    \begin{equation*}
        \beta'' \cdot \beta''' \times \beta^{(4)}
        = \beta^{(4)} \cdot \beta'' \times \beta'''
        = \kappa^4 \dot{\tau} - \kappa^3 \tau \dot{\kappa}
        = \kappa^5 \paren*{ \frac{\dot{\tau} \kappa}{\kappa^2} - \frac{\tau \dot{\kappa}}{\kappa^2} }
        = \kappa^5 \frac{d}{ds}\paren*{\frac{\tau}{\kappa}}.
    \end{equation*}
\end{problem}

\begin{problem}{3 (ii)}
    Reparametrize the curve $\alpha(t) = e^{t}\paren*{\cos t, \sin t, 1}$ by its arc-length.

    \solution Differentiating $\alpha(t) = \paren*{e^t \cos t, e^t \sin t, e^t}$ gives,
    \begin{equation*}
        \alpha'(t) = (e^t \cos t - e^t \sin t, e^t \sin t + e^t \cos t, e^t) = e^t \paren*{\cos t - \sin t, \cos t + \sin t, 1}.
    \end{equation*}
    So, $\abs{\alpha'}^2 = e^{2t} \paren*{\cos^2 t + \sin^2 t - 2\cos t \sin t + \cos^2 t + \sin^2 t + 2\cos t \sin t + 1} = 3 e^{2t}$. Therefore, $\abs{\alpha'(t)} = \sqrt{3} e^t$ and,
    \begin{equation*}
        s(t)
        = \int_0^t \abs{\alpha'(u)} \,du
        = \sqrt{3} \int_0^t e^u \,du
        = \sqrt{3} e^u \bigg|_0^t
        = \sqrt{3} (e^t - 1).
    \end{equation*}
    Inverting this to make $t$ the subject,
    \begin{equation*}
        s(t) = \sqrt{3} (e^t - 1)
        \implies
        e^t = 1 + \frac{s}{\sqrt{3}}
        \implies
        t = \ln \paren*{1 + \frac{s}{\sqrt{3}}}
    \end{equation*}

    So, the arc-length paremetrisation of $\alpha$ is
    \begin{equation*}
        \alpha(s) = \paren*{1 + \frac{s}{\sqrt{3}}}\cos \paren*{\ln \paren*{1 + \frac{s}{\sqrt{3}}}} \uvec{i} + \paren*{1 + \frac{s}{\sqrt{3}}}\sin \paren*{\ln \paren*{1 + \frac{s}{\sqrt{3}}}} \uvec{j} + \paren*{1 + \frac{s}{\sqrt{3}}} \uvec{k}.
    \end{equation*}
\end{problem}

\begin{problem}{4 (i)}
    Show that the curve $\beta(s) = \frac{1}{2}\paren*{s + \sqrt{s^2 + 1}, \paren*{s + \sqrt{s^2 + 1}}^{-1}, \sqrt{2}\ln\paren*{s + \sqrt{s^2 + 1}}}$ has unit speed.

    \solution Differentiating $\beta(s)$ gives,
    \begin{align*}
        \beta'(s)
        &= \frac{1}{2}
        \paren*{
            1 + \frac{s}{\sqrt{s^2 + 1}},
            -\frac{1 + \frac{s}{\sqrt{s^2 + 1}}}{\paren*{s + \sqrt{s^2 + 1}}^2},
            \frac{ \sqrt{2} \paren*{1 + \frac{s}{\sqrt{s^2 + 1}}} }{ s + \sqrt{s^2 + 1} }
        }\\
        &= \frac{1}{2}
        \paren*{
            \frac{ s + \sqrt{s^2 + 1} }{ \sqrt{s^2 + 1} },
            -\frac{1}{\sqrt{s^2 + 1} \paren*{s + \sqrt{s^2 + 1}}},
            \frac{ \sqrt{2} }{ \sqrt{s^2 + 1} }
        }\\
        &= \frac{1}{2} \frac{1}{\sqrt{s^2 + 1}}
        \paren*{s + \sqrt{s^2 + 1}, -\frac{1}{\paren*{s + \sqrt{s^2 + 1}}}, \sqrt{2} }\\
        &= \frac{1}{2\sqrt{s^2 + 1}}
        \paren*{s + \sqrt{s^2 + 1}, s - \sqrt{s^2 + 1}, \sqrt{2} }.
    \end{align*}
    Here we used
    \begin{equation*}
        1 + \frac{s}{\sqrt{s^2 + 1}} = \frac{ s + \sqrt{s^2 + 1} }{ \sqrt{s^2 + 1} }.
    \end{equation*}
    in the second line, and $-\paren*{s + \sqrt{s^2 + 1}}^{-1} = s - \sqrt{s^2 + 1}$ in the last line.

    Now, since $\paren*{s + \sqrt{s^2 + 1}}^2 + \paren*{s - \sqrt{s^2 + 1}}^2 = 4s^2 + 2$, we obtain
    \begin{equation*}
        \abs{\beta'(s)}^2
        = \frac{ 1 }{4(s^2 + 1)} \paren*{4s^2 + 2 + 2 }
        = \frac{ 4(s^2 + 1) }{4(s^2 + 1)}
        = 1.
    \end{equation*}
    Thus, $\abs{\beta'(s)} = 1$. In other words, $\beta(s)$ is a unit-speed parametrised curve.
\end{problem}

\begin{problem}{4 (ii)}
    Calculate the curvature of the curve $\beta(s)$ given in Q4(i).

    \solution Since $\beta = \beta(s)$ is unit-speed parameterised, its curvature is $\kappa = \abs{\beta''(s)}$. Differentiating
    \begin{equation*}
        \beta'(s)
        = \frac{1}{2\sqrt{s^2 + 1}}
        \paren*{s + \sqrt{s^2 + 1}, s - \sqrt{s^2 + 1}, \sqrt{2} }
        = \frac{1}{2}
        \paren*{\frac{s}{\sqrt{s^2 + 1}} + 1, \frac{s}{\sqrt{s^2 + 1}} - 1, \frac{\sqrt{2}}{\sqrt{s^2 + 1}} }.
    \end{equation*}
    with respect to $s$ gives, (carefully using the product rule)
    \begin{align*}
        \beta''(s)
        &= \frac{1}{2}
        \paren*{
            \frac{\sqrt{s^2 + 1} - s^2 (s^2 + 1)^{-1/2}}{s^2 + 1},
            \frac{\sqrt{s^2 + 1} - s^2 (s^2 + 1)^{-1/2}}{s^2 + 1},
            \frac{- \sqrt{2} s (s^2 + 1)^{-1/2}}{s^2 + 1}
        }\\
        &= \frac{1}{2}
        \paren*{
            \frac{1}{\paren*{s^2 + 1}^{3/2}},
            \frac{1}{\paren*{s^2 + 1}^{3/2}},
            -\frac{s\sqrt{2}}{\paren*{s^2 + 1}^{3/2}}
        }\\
        &= \frac{1}{2 \paren*{s^2 + 1}^{3/2}} \paren*{ 1, 1, -s\sqrt{2}}
    \end{align*}
    Now, $\abs*{\paren*{ 1, 1, -s\sqrt{2}}}^2 = 1 + 1 + 2s^2 = 2 (1 + s^2)$. Therefore,
    \begin{equation*}
        \kappa
        = \abs*{\beta''}
        = \frac{\sqrt{2} \paren*{ 1 + s^2 }^{1/2}}{2 \paren*{s^2 + 1}^{3/2}}
        = \frac{1}{\sqrt{2} \paren*{s^2 + 1}}
    \end{equation*}
\end{problem}

\end{document}
