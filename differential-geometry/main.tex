\documentclass[11pt]{penrose}

\usepackage{mathsphystools}
\usepackage{thmstyles}

\newcommand{\vT}{\vec{T}}
\newcommand{\vN}{\vec{N}}
\newcommand{\vB}{\vec{B}}

\renewcommand{\interior}[1]{\mathrm{int}(#1)}
\newcommand{\exterior}[1]{\mathrm{ext}(#1)}

\newcommand{\missing}[1]{{\color{red}#1}}
\newcommand{\keyword}[1]{\textsf{#1}}

\title{MATH 382: Differential Geometry}
\subtitle{Brief lecture notes}
\author{Rashid M. Talha}
\affiliation{School of Natural Sciences, NUST}
\date{\today}
\begin{document}

\maketitle

\textbf{Textbook:} Elementary Differential Geometry, Andrew Pressley.

\textbf{Disclaimer:} This document most likely contains some errors --- use with caution. I have rephrased or paraphrased the content in most of the sections. Some examples may be missing. The numbering that I have used for sections, definitions, theorems, etc will not match the numbering given in the lectures.

\section{Introduction}
A \keyword{curve} in $\R^n$ is most precisely defined as a one dimensional smooth submanifold. In this course, we take a more relaxed approach and instead identify a curve as the image of a suitable map.

\begin{ndfn}
    A \keyword{parametrised curve} in $\R^n$ is a map $\gamma : (\alpha, \beta) \to \R^n$. We allow the case where $\alpha = -\infty$ or $\beta = \infty$. A parametrised curve whose image is $C$ is called a \keyword{parametrisation} of $C$.
\end{ndfn}

Parametrisations are not unique.

We shall focus only on smooth parametrised curves. Smoothness refers to the fact that each component of $\gamma \in \R^n$ is assumed to be sufficiently continuously differentiable; usually, twice continuously differentiable.

All curves in the following sections will be parametrised and smooth.

\begin{ndfn}
    The \keyword{tangent vector} to a curve $\gamma$ at $\gamma(t)$ is $\vT(t) = \dot{\gamma}(t) \equiv d\gamma/dt$.
\end{ndfn}

\begin{nprop}
    Let $\gamma : (a, b) \to \R^n$ be a curve with $\vT(t) = \vT_0$ for all $t$. Then, the curve is a straight line.
\end{nprop}
\begin{proof}
    Integrating the equation $d\gamma/dt = \vT_0$ with respect to $t$ gives $\gamma(t) = t \vT_0 + \vec{k}$, for some arbitrary vector $\vec{k}$. This is the equation of a straight line.
\end{proof}

\section{Arc-Length}
\begin{ndfn}
    The \keyword{arc-length} of a curve $\gamma : (\alpha, \beta) \to \R^n$ is the function
    \begin{equation}
        s(t) = \int_{t_0}^{t} \abs*{\dot{\gamma}(u)} \,du.
        \label{eq:arclength-dfn}
    \end{equation}
\end{ndfn}

Here, $t_0 \in (\alpha, \beta)$ is an arbitrary value of the parameter that is used as the starting point of the length measurement. Note that $s(t_0) = 0$ and that $s$ is a monotonically increasing function. From the fundamental theorem of calculus,
\begin{equation}
    \frac{ds}{dt} = \abs{\dot{\gamma}(t)}
    \label{eq:speed-dfn}
\end{equation}
Since $\gamma(t)$ is the \keyword{position} vector to a point on the curve, its derivative $\dot{\gamma}(t)$ can be interpreted as a \keyword{velocity} vector. As a result, $\abs{\dot{\gamma}(t)}$ is called the \keyword{speed} (at which the curve is traversed under the given parametrisation).

\begin{ndfn}
    We say that $\gamma = \gamma(t)$ is a \keyword{unit-speed curve} if $\abs{\dot{\gamma}(t)} \equiv 1$.
\end{ndfn}

\begin{nprop}\label{prop:unit-vector-perp}
    Let $\vec{a}(t)$ be a unit vector that depends smoothly on $t$. Then,
    \begin{equation}
        \vec{a}(t) \cdot \dot{\vec{a}}(t) = 0
    \end{equation}
    for all $t$; i.e. $\dot{\vec{a}}(t)$ is either zero or perpendicular to $\vec{a}(t)$.
\end{nprop}
\begin{proof}
    Since $\vec{a}(t)$ is a unit vector, we have $\vec{a}(t) \cdot \vec{a}(t) = 1$. Differentiating this with respect to $t$ gives $\vec{a}(t) \cdot \dot{\vec{a}}(t) + \dot{\vec{a}}(t) \cdot \vec{a}(t) = 0$. This simplifies to give
    \begin{equation}
        2 \vec{a}(t) \cdot \dot{\vec{a}}(t) = 0
        \implies
        \vec{a}(t) \cdot \dot{\vec{a}}(t) = 0.\qedhere
    \end{equation}
\end{proof}

As a consequence, if $\gamma$ is a unit speed curve then $\dot{\gamma}$ is perpendicular to $\gamma$ (indeed coinciding with the terminology `tangent' vector). Note that, the magnitude of $\dot{\gamma}$ may or may not be constant; so there is no guarantee that $\ddot{\gamma}$ and $\dot{\gamma}$ are orthogonal.

\section{Reparametrisations and Regular Curves}
\begin{ndfn}
    A curve $\gamma_1 : (\alpha, \beta) \to \R^n$ is called a \keyword{reparametrisation} of another curve $\gamma_2 : (a, b) \to \R^n$ if there is a smooth bijective map $\varphi : (\alpha, \beta) \to (a, b)$ with a smooth inverse such that $\gamma_1(t) = \gamma_2( \varphi(t) )$ for all $t \in (\alpha, \beta)$.
\end{ndfn}

The function $\varphi$ is called the \keyword{reparametrisation map}.

If $\gamma_1$ is a reparametrisation of $\gamma_2$, then $\gamma_2$ is a reparametrisation of $\gamma_1$. Likewise, if $\gamma_1$ is a reparametrisation of $\gamma_2$, then $\gamma_1$ has the same image as $\gamma_2$.

\begin{ndfn}
    Let $\gamma = \gamma(t)$ be a parametrised curve. We say that $\gamma(t)$ is a \keyword{regular} point of $\gamma$ if $\dot{\gamma}(t) \neq 0$. Otherwise, $\gamma(t)$ is called a \keyword{singular} point. The curve $\gamma$ is said to be regular if it is regular at all points.
\end{ndfn}

Notice that the condition that a given point $\gamma(t)$ is a regular is equivalent to the condition that the tangent vector to $\gamma$ does not vanish at the given point.

\begin{nthm}
    Any reparametrisation of a regular curve is regular.
\end{nthm}
\begin{proof}
    Let $\gamma_1 : (\alpha, \beta) \to \R^n$ be the reparametrisation of a regular curve $\gamma_2 : (a,b) \to \R^n$ through the map $\varphi : (\alpha, \beta) \to (a, b)$. So, $\gamma_1 (t) = \gamma_2 (\varphi(t))$. Then, by chain rule,
    \begin{equation}
        \gamma'_1 (t) = \gamma'_2 (\varphi(t)) \varphi'(t).
    \end{equation}

    Now, $\gamma'_2 (\varphi(t)) \neq 0$ for all $t$ because it is a regular curve. Also, differentiating both sides of $\varphi (\varphi^{-1} (t)) = t$ with respect to $t$ leads to
    \begin{equation}
        \varphi' (\varphi^{-1} (t)) \cdot \paren*{\varphi^{-1} (t)}' = 1, \quad\forall t.
    \end{equation}
    Consequently, $\varphi' (u) \neq 0$ for all $u$.

    Therefore, $\gamma'_1 (t) \neq 0$ for all $t$, because both $\gamma'_2 (\varphi(t))$ and $\varphi'(t)$ are always non-zero. So, $\gamma_1$ is a regular curve.
\end{proof}

\begin{nthm}
    If $\gamma = \gamma(t)$ is a regular curve, then its arc-length function $s = s(t)$ is smooth.
\end{nthm}
\begin{proof}
    The arc-length function for $\gamma$ is $s(t) = \int_{t_0}^{t} \abs*{\dot{\gamma}(u)} \,du$. By the fundamental theorem of calculus, $ds/dt = \abs*{\dot{\gamma}}$. So, $s = s(t)$ is differentiable. Now,
    \begin{equation}
        \abs*{\dot{\gamma}} = \sqrt{\dot{\gamma}_1^2 + \dots + \dot{\gamma}_n^2}.
    \end{equation}
    And $x \mapsto \sqrt{x}$ is smooth for all $x > 0$. So, if $\gamma$ is regular then for all $t$, $\dot{\gamma}(t) \neq 0$. Therefore, $\dot{\gamma}_1^2 + \dots + \dot{\gamma}_n^2 > 0$. As a result, $ds/dt \equiv \abs*{\dot{\gamma}}$ is a smooth function. In other words, $s = s(t)$ is smooth.
\end{proof}

\begin{nthm}
    A parametrised curve has a unit-speed reparametrisation if and only if it is regular.
\end{nthm}
\begin{proof}
    TBC.
\end{proof}

\section{Closed Curves}
\begin{ndfn}
    A curve $\gamma : \R \to \R^n$ is said to be \keyword{$T$-periodic} for $T \in R$ if
    \begin{equation}
        \gamma(t + T) = \gamma(t)
        \quad\forall t \in \R.
    \end{equation}
\end{ndfn}

Clearly, every curve is $0$-periodic. Moreover, if a curve is $T$-periodic then it is also $(-T)$-periodic as $\gamma(t - T) = \gamma((t-T) + T) = \gamma(t)$. Consequently, if $\gamma$ is $T$-periodic for $T \neq 0$, then we can always find a $T' > 0$ such that $\gamma$ is $T'$-periodic.

\begin{ndfn}
    The \keyword{period} of a curve $\gamma$ is the smallest positive number $T > 0$ such that $\gamma$ is $T$-periodic.
\end{ndfn}

The existence of a period (a least positive real number) is not obvious because not every set of positive real numbers has a smallest element.

\begin{ndfn}
    If $\gamma$ is $T$-periodic for $T \neq 0$ then $\gamma$ is called a \keyword{closed curve}.
\end{ndfn}

Closed curves (of periodicity $T$) are usually defined only over the interval $[0, T]$.

\section{Curvature of Curves}
Let $\gamma = \gamma(t)$ be a regular parametrised curve. Intuitively we know that $\gamma$ can bend or curve in different directions for different values of $t$. To quantify this behaviour, we introduce the concept of \keyword{curvature}, which measures the degree to which a curve deviates from its tangent line.

Taylor expansion gives
\begin{equation}
    \gamma(t + \Delta t) = \gamma(t) + \dot{\gamma}(t) \Delta t + \frac{1}{2} \ddot{\gamma}(t) \paren*{\Delta t}^2 + O(\paren*{\Delta t}^3)
\end{equation}
The second order term gives the deviation from the tangent line, and its strength is determined by the value of $\abs*{\ddot{\gamma}(t)}$. Therefore, we introduce the following definition.

\begin{ndfn}
    Let $\gamma = \gamma(t)$ be a unit speed parametrisation of a curve. Its \keyword{curvature} at the point $\gamma(t)$ is
    \begin{equation}
        \kappa(t) = \abs{\ddot{\gamma}(t)}.
        \label{eq:curvature-dfn}
    \end{equation}
\end{ndfn}

\begin{egg}
    A unit speed parametrisation $\gamma : [0, 2\pi] \to \R^2$ of a circle centred at $(x_0, y_0)$ with radius $R$ is
    \begin{equation}
        \gamma(t) = \paren*{ x_0 + R \cos\paren*{ \frac{t}{R} }, y_0 + R \sin\paren*{ \frac{t}{R} } }.
    \end{equation}
    We find that,
    \begin{equation}
        \dot{\gamma} = \paren*{ -\sin\paren*{ \frac{t}{R} }, \cos\paren*{ \frac{t}{R} } }
        \quad\text{and}\quad
        \ddot{\gamma} = \paren*{ -\frac{1}{R} \cos\paren*{ \frac{t}{R} }, -\frac{1}{R} \sin\paren*{ \frac{t}{R} } }.
    \end{equation}
    So, $\abs{\dot{\gamma}(t)} = 1$ for all $t$, showing that it is indeed a unit speed parametrisation.

    The curvature at $\gamma(t)$ is $\kappa(t) = \abs{\ddot{\gamma}(t)} = \frac{1}{R}$, for all $t$.
\end{egg}

\begin{egg}
    A unit speed parametrisation of a helix around the $z$-axis is
    \begin{equation}
        \beta(t) = \paren*{ a \cos\paren*{\frac{s}{c}}, a \sin\paren*{\frac{s}{c}}, b\paren*{\frac{s}{c}} }
    \end{equation}
    with $c = \sqrt{a^2 + b^2}$ and $a, b \in \R$. By calculating $\abs{\ddot{\gamma}(t)}$, we find that $\kappa = a / (a^2 + b^2)$ for all $t$.
\end{egg}

For a general regular curve (not necessarily unit speed), we have the following result.
\begin{nprop}
    Let $\gamma(t)$ be a regular curve in $\R^3$, then its curvature is
    \begin{equation}
        \kappa(t) = \frac{\abs{ \ddot{\gamma} \times \dot{\gamma} }}{ \abs{\dot{\gamma}}^3 }.
        \label{eq:curvature-general}
    \end{equation}
\end{nprop}
\begin{proof}
    By definition $\kappa(t) = \abs*{ \frac{d^2}{ds^2} \gamma(t) }$. So,
    \begin{equation}
        \kappa(t)
        = \abs*{ \frac{d^2 \gamma}{ds^2} }
        = \abs*{ \frac{d}{ds} \paren*{\frac{d\gamma}{ds}} }
        = \abs*{ \frac{dt}{ds} \frac{d}{dt} \paren*{\frac{d\gamma}{dt} \frac{dt}{ds}} }
        = \abs*{ \frac{\frac{d}{dt} \paren*{ \frac{ d\gamma/dt }{ ds/dt } }}{\frac{ds}{dt}} }.
    \end{equation}
    Using the quotient rule
    \begin{equation}
        \frac{d}{dt} \paren*{ \frac{ d\gamma/dt }{ ds/dt } }
        = \dfrac{ \frac{ds}{dt} \frac{d^2 \gamma}{dt^2} - \frac{d^2 s}{dt^2} \frac{d\gamma}{dt} }{ \paren*{\frac{ds}{dt}}^2 }.
    \end{equation}
    Therefore,
    \begin{equation}
        \kappa(t)
        = \abs*{ \dfrac{ \frac{ds}{dt} \frac{d^2 \gamma}{dt^2} - \frac{d^2 s}{dt^2} \frac{d\gamma}{dt} }{ \paren*{\frac{ds}{dt}}^3 } }
        \label{eq:curvature-deriv-a}
    \end{equation}

    Now, $\paren*{\frac{ds}{dt}}^2 = \abs*{ \frac{d\gamma}{dt} }^2 = \dot{\gamma} \cdot \dot{\gamma}$. Moreover, differentiating this identity with respect to $t$ gives $\frac{ds}{dt} \frac{d^2 s}{dt^2} = \dot{\gamma} \cdot \ddot{\gamma}$.

    Multiplying the numerator and the denominator of equation~\eqref{eq:curvature-deriv-a} by $ds/dt$ and using these expressions gives
    \begin{equation}
        \kappa(t)
        = \abs*{ \dfrac{ \frac{ds}{dt} \frac{d^2 \gamma}{dt^2} - \frac{d^2 s}{dt^2} \frac{d\gamma}{dt} }{ \paren*{\frac{ds}{dt}}^3 } }
        = \abs*{ \dfrac{ \paren*{\frac{ds}{dt}}^2 \frac{d^2 \gamma}{dt^2} - \frac{ds}{dt} \frac{d^2 s}{dt^2} \frac{d\gamma}{dt} }{ \paren*{\frac{ds}{dt}}^4 } }
        = \abs*{ \dfrac{ \paren*{\dot{\gamma} \cdot \dot{\gamma}} \ddot{\gamma} - \paren*{\dot{\gamma} \cdot \ddot{\gamma}} \dot{\gamma} }{ \paren*{\dot{\gamma} \cdot \dot{\gamma}}^2 } }
    \end{equation}

    Using the vector identity $\vec{a} \times (\vec{b} \times \vec{c}) = (\vec{a} \cdot \vec{c})\vec{b} - (\vec{a} \cdot \vec{b})\vec{c}$, we get
    \begin{equation}
        \abs*{ \paren*{\dot{\gamma} \cdot \dot{\gamma}} \ddot{\gamma} - \paren*{\dot{\gamma} \cdot \ddot{\gamma}} \dot{\gamma} }
        = \abs*{ \dot{\gamma} \times \paren*{\ddot{\gamma} \times \dot{\gamma}} }
        = \abs*{ \dot{\gamma} } \abs*{ \ddot{\gamma} \times \dot{\gamma} }.
    \end{equation}
    In the final step we used the fact that $\ddot{\gamma} \times \dot{\gamma}$ is perpendicular to $\dot{\gamma}$. Therefore, we don't pick up a factor of $\sin\theta$.

    Also, $\dot{\gamma} \cdot \dot{\gamma} = \abs*{\dot{\gamma}}^2$ implies $\abs*{ \paren*{\dot{\gamma} \cdot \dot{\gamma}}^2 } = \abs*{\dot{\gamma}}^4$. Therefore,
    \begin{equation}
        \kappa
        = \abs*{ \dfrac{ \paren*{\dot{\gamma} \cdot \dot{\gamma}} \ddot{\gamma} - \paren*{\dot{\gamma} \cdot \ddot{\gamma}} \dot{\gamma} }{ \paren*{\dot{\gamma} \cdot \dot{\gamma}}^2 } }
        = \frac{\abs*{ \dot{\gamma} } \abs*{ \ddot{\gamma} \times \dot{\gamma} }}{\abs*{\dot{\gamma}}^4}
        = \frac{\abs*{ \ddot{\gamma} \times \dot{\gamma} }}{\abs*{\dot{\gamma}}^3}.\qedhere
    \end{equation}
\end{proof}

By expanding $\gamma(t)$ as $\paren*{x(t), y(t), z(t)}$, the expression for $\kappa$ can be written as
\begin{equation}
    \kappa
    = \sqrt{ \frac{ \paren*{\dot{z}\ddot{y}-\dot{y}\ddot{z}}^2 + \paren*{\dot{x}\ddot{z}-\dot{z}\ddot{x}}^2 + \paren*{\dot{y}\ddot{x}-\dot{x}\ddot{y}}^2 }{\paren*{\dot{x}^2 + \dot{y}^2 + \dot{z}^2}^3} }
    \label{eq:curvature-space-full}
\end{equation}
Consequently, for the curvature of a regular plane curve with the parametrisation $x = x(t)$, $y = y(t)$, equation~\eqref{eq:curvature-space-full} can be reduced to a more practical form
\begin{equation}
    \kappa = \frac{\abs*{\dot{x}\ddot{y} - \ddot{x}\dot{y}}}{\phantom{1}\paren*{\dot{x}^2+\dot{y}^2}^{3/2}}.
    \label{eq:curvature-plane-full}
\end{equation}
If, instead, the plane curve is given in the Cartesian form $y = y(x)$, then equation~\eqref{eq:curvature-plane-full} becomes
\begin{equation}
    \kappa = \frac{\abs*{y''}}{\paren*{1+y'^2}^{3/2}}
    \label{eq:curvature-cartesian}
\end{equation}
by introducing the parametrisation $x = t$, $y =y(t)$. Furthermore, if the plane curve is given in the polar form $r = r(\theta)$, then a tedious calculation shows that
\begin{equation}
    \kappa = \frac{r^2 + 2 r_{\theta}^{2} - r r_{\theta\theta}}{\phantom{1}\paren*{r^{2}+r_{\theta}^{2}}^{3/2}}.
    \label{eq:curvature-polar}
\end{equation}

\section{Planar Curves: Curvature}

For planar curves, the notion of curvature can be refined further. In this section, we focus entirely on curves in $\R^2$.

Let $\gamma(s)$ be a unit speed parametrised curve. Its tangent vector was defined to be
\begin{equation}
    \vT(s) = \dot{\gamma} \equiv \frac{d\gamma}{ds}.
    \label{eq:tangent-dfn}
\end{equation}
As $\gamma$ is unit-speed, $\abs{\vT(s)} = \abs*{\dot{\gamma}(s)} = 1$. So, $\vT$ is a unit vector in this case.

There are now two choices for the \keyword{normal} vector: $\pi/2$ radians in the clockwise direction, or the counter clockwise direction.

We define a unit vector, $\vN_s$ at $\gamma(s)$ by rotating $\vT(s)$ counter clockwise by $\pi/2$ radians. This is called the \keyword{signed unit normal}; it is a unit vector because it has the same magnitude as $\vT$, and it is a normal vector because by construction it is orthogonal to the tangent vector.

Since $\abs{\vT} = 1$, we have $\vT \cdot \dot{\vT} = 0$ by proposition~(\ref{prop:unit-vector-perp}). That is, $\dot{\gamma} \cdot \ddot{\gamma} = 0$. So, $\ddot{\gamma}$ is perpendicular to $\vT$, and therefore, parallel (or anti-parallel) to $\vN_s$. This can be written as
\begin{equation}
    \frac{d\vT}{ds} \equiv \ddot{\gamma} = \kappa_s \vN_s.
\end{equation}
The factor $\kappa_s$ is called the \keyword{signed curvature} of $\gamma$. It is important to note that this is only defined for regular plane curves.

We observe that $\kappa = \abs*{\ddot{\gamma}} = \abs*{\kappa_s \vN_s} = \abs{\kappa_s}$ because $\abs*{\vN_s} = 1$.

% Geometrically, the the curvature $\kappa$ always points ... THE FOUR-CURVE DIAGRAM.

We can generalise this analysis to a general regular (plane) curve, $\gamma = \gamma(t)$, by considering its arc-length reparametrisation.
\begin{equation}
    \vT(t)
    = \frac{d\gamma(t)}{ds}
    = \frac{dt}{ds} \frac{d\gamma(t)}{dt}
    = \frac{d\gamma/dt}{ds/dt}
    = \frac{d\gamma/dt}{\abs{d\gamma/dt}}
    \equiv \frac{\dot{\gamma}(t)}{\abs{\dot{\gamma}(t)}}.
\end{equation}
Also,
\begin{equation*}
    \frac{d\vT}{dt}
    = \frac{d\vT}{ds} \frac{ds}{dt}
    = \kappa_s \vN_s \abs*{\frac{d\gamma}{dt}}.
\end{equation*}

\begin{ndfn}
    Let $\gamma(s)$ be a unit-speed curve. Let $\varphi(s)$ denote the angle that the tangent vector $\vT(s) \equiv \dot{\gamma}(s)$ makes with the positive $x$-axis. That is
    \begin{equation}
        \vT \equiv \dot{\gamma}(s) = \paren*{\cos\varphi(s), \sin\varphi(s)}.
    \end{equation}
    We call $\varphi$ the \keyword{turning angle} of $\gamma$.
\end{ndfn}

This quantity allows us to calculate the signed curvature $\kappa_s$ directly.

\begin{nprop}
    Let $\gamma(s)$ be a unit speed parametrised plane curve, and let $\varphi(s)$ be its turning angle. Then,
    \begin{equation*}
        \kappa_s = \frac{d\varphi}{ds}.
    \end{equation*}
    (This states that the signed curvature is the rate at which the tangent vector of the curve rotates. It is positive if the tangent vector rotates counter clockwise, and positive if it rotates clockwise.)
\end{nprop}
\begin{proof}
    By definition, $\vT(s) \equiv \dot{\gamma}(s) = \paren*{\cos\varphi(s), \sin\varphi(s)}$. Then,
    \begin{equation}
        \vN_s(s) = \paren*{-\sin\varphi(s), \cos\varphi(s)}
    \end{equation}
    because it is obtained by rotating $\vT$ counter clockwise by $\pi/2$. Therefore,
    \begin{equation}
        \frac{d\vT}{ds} \equiv \ddot{\gamma}
        = \paren*{-\sin\varphi(s), \cos\varphi(s)} \frac{d\varphi}{ds}.
        = \vN_s \frac{d\varphi}{ds}.
    \end{equation}
    Comparison with $\ddot{\gamma} = \kappa_s \vN_s$ gives $\kappa_s(s) = d\varphi/ds$.
\end{proof}

It is useful to note that $\vT = \paren*{\cos\varphi(s), \sin\varphi(s)}$ and $\vT = \paren*{\dot{x}, \dot{y}}$ collectively implies
\begin{equation}
    \tan\varphi(s) = \frac{\dot{y}}{\dot{x}} \equiv \frac{dy/dt}{dx/dt},
\end{equation}
where $t$ is any regular parameter. However, since $\varphi = \varphi(s)$, the resulting quotient must be re-expressed in terms of a unit speed parameter $s$.

\begin{negg}
    Consider a catenary $\gamma(t) = (t, \cosh(t))$. In order to calculate its signed curvature $\kappa_s(s)$, we start by noting its arc-length function
    \begin{equation}
        s = \int_0^t \abs*{\dot{\gamma}} du
        = \int_0^t \sqrt{1 + \sinh^2(u)} du
        = \sinh(t).
    \end{equation}

    Next,
    \begin{equation}
        \tan\varphi(s) = \frac{dy/dt}{dx/dt}
        = \frac{\sinh(t)}{1} = s.
    \end{equation}
    Equivalently, $\varphi(s) = \atan{s}$. Therefore,
    \begin{equation}
        \kappa_s(s) = \frac{d\varphi}{ds} = \frac{1}{1+s^2}
    \end{equation}
\end{negg}

\begin{negg}
    Consider a circle of radius $a$ centred at the origin parametrised by
    \begin{equation}
        \gamma(t) = \paren*{a \cos t, a \sin t}.
    \end{equation}
    We note that its arc-length function is $s = \int_0^t \abs*{\dot{\gamma}} du = \int_0^t a du = at$.

    As a result,
    \begin{equation}
        \tan\varphi(s) = \frac{dy/dt}{dx/dt} = -\frac{\cos(t)}{\sin(t)} = -\cot(t) = -\cot\paren*{\frac{s}{a}}.
    \end{equation}
    Equivalently, $\varphi(s) = \atan{ \paren*{-\cot\paren*{s/a}} }$. Therefore,
    \begin{align}
        \kappa_s
        = \frac{d\varphi}{ds}
        = \frac{1}{1+\cot^2\paren*{ \frac{s}{a} }} \cdot \cosec^2\paren*{ \frac{s}{a} } \cdot \frac{1}{a}
        = \frac{1}{\cosec^2\paren*{ \frac{s}{a} }} \cdot \cosec^2\paren*{ \frac{s}{a} } \cdot \frac{1}{a}
        = \frac{1}{a}
    \end{align}
    Therefore, the signed curvature for this parametrisation of the circle is $\kappa_s = 1/a$.
\end{negg}

% \begin{remark}
%     A simple calculation show that
%     \begin{equation}
%         \kappa_s = \frac{d\varphi}{ds} = \frac{\dot{x}\ddot{y} - \ddot{x}\dot{y}}{\paren*{\dot{x}^2+\dot{y}^2}^{3/2}}.
%     \end{equation}
%     This follows from differentiating $\tan\varphi(s) = f(t)$ with respect to $s$ using the chain rule, and setting $f(t) = \frac{dy/dt}{dx/dt}$.
% \end{remark}

An interesting quantity derived from $\kappa_s(s)$ is the \keyword{total signed curvature}
\begin{equation}
    \int_0^\ell \kappa_s (s) \,ds
\end{equation}
where $\ell$ is the total length of the curve. It can be shown that the total signed curvature for a closed planar curve is always a multiple of $2\pi$. Its value gives information about how many times a curve winds around before returning to its initial point.

\begin{nprop}
    Let $\gamma : [0, \ell] \to \R^2$ be a regular closed plane curve. Then,
    \begin{equation}
        \int_0^\ell \kappa_s (s) \,ds = 2\pi n,
        \quad n \in \Z.
    \end{equation}
\end{nprop}
\begin{proof}
    We start by noting that
    \begin{equation}
        \int_0^\ell \kappa_s (s) \,ds
        = \int_0^\ell \frac{d\phi}{ds} \,ds
        = \phi(\ell) - \phi(0).
    \end{equation}

    Now, since $\gamma$ is a closed curve with length $\ell$, $\gamma(t + \ell) = \gamma(t)$ for all $0 \leq t \leq \ell$. As a result, $\dot{\gamma}(t + \ell) = \dot{\gamma}(t)$ for all $0 \leq t \leq \ell$. In particular, $\dot{\gamma}(\ell) = \dot{\gamma}(0)$.

    From the definition of the turning angle, $\phi(t)$, we have $\dot{\gamma}(t) = \paren*{ \cos\phi(t), \sin\phi(t) }$. Consequently,
    \begin{align}
        \dot{\gamma}(\ell) = \dot{\gamma}(0)
        &\implies \paren*{ \cos\phi(\ell), \sin\phi(\ell) } = \paren*{ \cos\phi(0), \sin\phi(0) }\\
        &\implies \phi(\ell) = \phi(0) + 2\pi n, \quad\text{for } x \in \Z.
    \end{align}

    As a result, $\int_0^\ell \kappa_s (s) ds = \phi(\ell) - \phi(0) = 2\pi n$ for $n \in \Z$.
\end{proof}

Analogously, we can define the \keyword{total curvature} for a regular closed curve $\gamma = \gamma(t)$. In this case, we don't need to restrict $\gamma$ to be a plane curve since curvature $\kappa$ is defined for all regular curves. After several lines of algebraic manipulations we find that
\begin{equation}
    \int_0^\ell \kappa(t) \,dt
    = \int_{0}^{\ell} \paren*{ \frac{ \paren*{\dot{\gamma}\cdot\dot{\gamma}}\paren*{\ddot{\gamma}\cdot\ddot{\gamma}} - \paren*{\dot{\gamma}\cdot\ddot{\gamma}}^2 }{ \dot{\gamma}\cdot\dot{\gamma} } }^{1/2} \,dt.
\end{equation}

\begin{negg}
    Consider $\gamma(t) = \paren*{ \frac{4}{5} \cos t, 1 - \sin t, -\frac{3}{5} \cos t }$ with $0 \leq t \leq 2\pi$. This is a regular closed space curve with period $2\pi$.

    We find that $\dot{\gamma}\cdot\dot{\gamma} = 1$, $\ddot{\gamma}\cdot\ddot{\gamma} = 1$ and $\dot{\gamma}\cdot\ddot{\gamma} = 0$ for all $t$. As a result,
    \begin{equation}
        \int_0^{2\pi} \kappa(t) \,dt
        = \int_{0}^{2\pi} \paren*{ \frac{ \paren*{1}\paren*{1} - \paren*{0}^2 }{ 1 } }^{1/2} \,dt
        = 2\pi.
    \end{equation}
\end{negg}

% \missing{\begin{nthm}
%     THE SIGNED CURVATURE DETERMINES THE CURVE UNIQUELY UPTO ISOMETRIES. (NO PROOF)
% \end{nthm}}

\section{Osculating Circle for Planar Curves}
Let $\gamma = \gamma(t)$ be a plane curve, and consider the point $\gamma(a)$ on this curve where the curvature $\kappa \neq 0$. Near this point, the curve can be approximated as (an arc of) a circle with radius $1/\kappa$; the curvature of a circle with radius $\rho$ is $1/\rho$. We can extend this arc to form a full circle, $C$. This is known as the \keyword{osculating circle} of $\gamma$ at $\gamma(a)$.

The osculating circle $C$ at $\gamma(a)$ has the following defining properties:
\begin{itemize}
    \item It is tangent to $\gamma$ at $\gamma(a)$.
    \item Its curvature is the same as that of $\gamma$ at $\gamma(a)$; equivalently, its radius is $1/\kappa$.
    \item Its centre is on the concave side of the curve. In other words, the centre is located at $\gamma(a) + \rho\vN$, where $\rho = 1/\kappa$ is the radius, and $\vN$ is the unit normal vector at $\gamma(a)$.
\end{itemize}

\begin{negg}
    Consider a parametrisation of the parabola $\gamma(t) = (t, t^2)$. We compute its osculating circle at an arbitrary value of the parameter $t$.

    We start by calculating the curvature at $\gamma(t)$. Firstly, $\dot{\gamma} = (1, 2t)$. So,
    \begin{equation}
        \vT = \frac{d\gamma/dt}{\abs*{ d\gamma/dt }} = \paren*{ \frac{1}{\sqrt{1 + 4t^2}}, \frac{2t}{\sqrt{1 + 4t^2}} }
    \end{equation}
    and
    \begin{equation}
        \frac{d\vT}{dt} = \paren*{ -\frac{4t}{\paren*{1+4t^2}^{3/2}}, \frac{2}{\paren*{1+4t^2}^{3/2}} }
    \end{equation}
    Therefore,
    \begin{equation}
        \kappa
        = \frac{\abs*{ d\vT/dt }}{\abs*{ d\gamma/dt }}
        = \frac{2}{1+4t^2} \cdot \frac{1}{\sqrt{1 + 4t^2}}
        = \frac{2}{ \paren*{1+4t^2}^{3/2} }
    \end{equation}

    So, the radius of the osculating circle is $r = \frac{1}{\kappa} = \frac{1}{2}\paren*{1+4t^2}^{3/2}$.

    The unit normal vector at $\gamma(t)$ is
    \begin{equation}
        \vN
        = \frac{ d\vT/dt }{\abs*{ d\vT/dt }}
        = \paren*{ -\frac{2t}{\paren*{1+4t^2}^{1/2}}, \frac{1}{\paren*{1+4t^2}^{1/2}} }.
    \end{equation}
    So, the centre of the osculating circle is at $\vec{a} = \gamma(t) + r\vN = \paren*{ -4 t^3, \frac{1}{2} + 3t^2 }$.

    In the particular case $t=0$, we have $r = 1/2$ and $\vec{a} = (0, 1/2)$.
\end{negg}

\section{Space Curves: Curvature and Torsion}
Consider a space curve $\gamma : (a, b) \subseteq \R \to \R^3$ with the unit-speed parametrisation. Let $\vT(s) \equiv \dot{\gamma}(s)$ denote its tangent vector at $\gamma(s)$. If the curvature $\kappa(s) \neq 0$ then we can define the \keyword{principal normal} vector of $\gamma$ at $\gamma(s)$ as
\begin{equation}
    \vN(s) = \frac{1}{k(s)} \dot{\vT} \equiv \frac{\dot{\vT}}{\abs{\dot{\vT}}}
    \label{eq:principal-normal-dfn}
\end{equation}

This is indeed perpendicular to $\vT$ because $\abs*{\vT} = 1$ implies that $\vT \cdot \dot{\vT} = 0$. Moreover, since $\kappa(s) = \abs*{\ddot{\gamma}(s)} \equiv \abs*{\dot{\vT}(s)}$, the vector $\vN$ is a unit vector.

Since $\gamma$ is a curve in $\R^3$, we can define another vector that is perpendicular to both $\vT$ and $\vN$ through
\begin{equation}
    \vB(s) = \vT(s) \times \vN(s).
    \label{eq:bivector-dfn}
\end{equation}
The vector $\vB$ is called the \keyword{binormal} vector of $\gamma$ at $\gamma(s)$. By construction, this is perpendicular to both $\vN$ and $\vT$ and $\abs*{\vB} = 1$. As a result, $\set{\vT, \vN, \vB}$ forms a right-handed orthonormal basis of $\R^3$ at each point $\gamma(s)$ when $\kappa \neq 0$. Such a basis is called a Frenet (or comoving) frame.

Since $\abs*{\vB} = 1$, we conclude that $\vB \cdot \dot{\vB} = 0$. That is, $\dot{\vB}$ is perpendicular to $\vB$. Also,
\begin{equation}
    \vB = \vT \times \vN
    \implies
    \dot{\vB} = \vT \times \dot{\vN}
\end{equation}
using $\dot{\vT} \times \vN = \kappa \vN \times \vN = 0$. So, $\vB$ is also perpendicular to $\vT$. Therefore, it must be parallel to $\vN$, and we write
\begin{equation}
    \dot{\vB}(s) = - \tau(s) \vN(s)
\end{equation}
where the scalar $\tau(s)$ is called the torsion of $\gamma$ at $\gamma(s)$.

Just as the curvature $\kappa(s)$ measures the deviation of the curve from its tangent line, the torsion measure the deviation of the curve from the plane spanned by $\vT$ and $\vN$.

So far we have obtained expressions for $\dot{\vT}$ and $\dot{\vB}$ in terms of the basis set $\set{\vT, \vN, \vB}$. We can also write $\dot{\vN}$ as a linear combination of these orthonormal basis vectors
\begin{equation}
    \dot{\vN} = \paren*{\dot{\vN}\cdot\vT} \vT + \paren*{\dot{\vN}\cdot\vN} \vN + \paren*{\dot{\vN}\cdot\vB} \vB,
\end{equation}
and note that
\begin{gather}
    \abs*{\dot{\vN}} = 1 \implies \dot{\vN}\cdot\vN = 0\\
    \vN \cdot \vT = 0 \implies \dot{\vN}\cdot\vT = -\vN \cdot \dot{\vT} = -\kappa\\
    \vN \cdot \vB = 0 \implies \dot{\vN}\cdot\vB = -\vN \cdot \dot{\vB} = \tau,
\end{gather}
to obtain
\begin{equation}
    \dot{\vN}(s) = -\kappa(s) \vT(s) + \tau(s) \vB(s).
\end{equation}

\subsection*{Summary of the results:}
For a unit speed parametrised space curve $\gamma = \gamma(s)$ with $\kappa(s) \neq 0$, we have
\begin{gather*}
    \vT(s) \coloneq \dot{\gamma},
    \quad
    \vN(s) \coloneq \frac{\dot{\vT}}{\kappa(s)} \equiv \frac{\ddot{\gamma}}{\abs*{\ddot{\gamma}(s)}},
    \quad
    \vB(s) = \vT(s) \times \vN(s) \equiv \frac{\dot{\gamma} \times \ddot{\gamma}}{\abs*{\ddot{\gamma}(s)}},\\
    \dot{\vT}(s) = \kappa(s) \vN,
    \quad
    \dot{\vN}(s) = -\kappa(s) \vT(s) + \tau(s) \vB(s),
    \quad
    \dot{\vB}(s) = - \tau(s) \vN(s),\\
    \kappa(s) = \abs*{\dot{\vT}(s)} \equiv \abs*{\ddot{\gamma}(s)},
    \quad
    \tau(s) = \abs*{\dot{\vB}(s)}.
\end{gather*}

These quantities can also be generalised to other regular parametrisations of the curve in a well-defined manner. (We have seen the associated formulae for $\vT(t)$ and $\kappa(t)$.)

\begin{negg}
    Consider the arc-length parametrised helix
    \begin{equation}
        \gamma(s) = \paren*{ a \cos\paren*{\frac{s}{c}}, a\sin\paren*{\frac{s}{c}}, \paren*{\frac{bs}{c}} }
    \end{equation}
    where $c = \sqrt{a^2 + b^2}$ and $a > 0$. Then,
    \begin{equation}
        \dot{\gamma} = \paren*{ -\frac{a}{c} \sin\paren*{\frac{s}{c}}, \frac{a}{c} \cos\paren*{\frac{s}{c}}, \frac{b}{c} }
        \quad\text{and}\quad
        \ddot{\gamma} = \paren*{ -\frac{a}{c^2} \cos\paren*{\frac{s}{c}}, -\frac{a}{c^2} \sin\paren*{\frac{s}{c}}, 0 }.
    \end{equation}
    We easily verify that indeed $\abs*{\dot{\gamma}} = 1$. Next,
    \begin{equation}
        \vT
        = \dot{\gamma}
        = \paren*{ -\frac{a}{c} \sin\paren*{\frac{s}{c}}, \frac{a}{c} \cos\paren*{\frac{s}{c}}, \frac{b}{c} }
        \quad\text{and}\quad
        \kappa = \abs*{\ddot{\gamma}} = \frac{a}{c^2} = \frac{a}{a^2 + b^2}.
    \end{equation}
    Leading to
    \begin{equation}
        \vN = \frac{\dot{\vT}}{\kappa} = \frac{\ddot{\gamma}}{\kappa} = \paren*{ -\cos\paren*{\frac{s}{c}}, -\sin\paren*{\frac{s}{c}}, 0 }.
    \end{equation}
    Also then,
    \begin{equation}
        \vB
        = \vT \times \vN
        = \paren*{ \frac{b}{c}\sin\paren*{\frac{s}{c}}, -\frac{b}{c}\cos\paren*{\frac{s}{c}}, \frac{a}{c} }.
    \end{equation}
    And,
    \begin{equation}
        \tau = \abs*{\dot{\vB}} = \frac{b}{a^2 + b^2}.
    \end{equation}
    Again, it is easy to check that $\abs*{\vT} = \abs*{\vN} = \abs*{\vB} = 1$.
\end{negg}

\section{Torsion}

\begin{nlemma}
    If $\alpha$ is a regular space curve with speed $v \equiv \abs*{\alpha'}$, then the velocity and the acceleration of $\alpha$ is given by
    \begin{gather}
        \alpha' = \frac{ds}{dt} \vT \equiv v \vT\\
        \alpha'' = \frac{d^2 s}{dt^2} \vT + \kappa \paren*{\frac{ds}{dt}}^2 \vN \equiv v' \vT + \kappa v^2 \vN
    \end{gather}
    where $\vT$ and $\vN$ are the unit tangent and normal vectors, respectively, and $\kappa$ is the curvature of $\alpha$.
\end{nlemma}
\begin{proof}
    TBC.
\end{proof}

\begin{nthm}
    Let $\gamma = \gamma(t)$ be a regular curve with nowhere-vanishing curvature. Then, its torsion is given by
    \begin{equation}
        \tau
        = \frac{\paren*{\dot{\gamma} \times \ddot{\gamma}} \cdot \dddot{\gamma}}{\abs*{\dot{\gamma} \times \ddot{\gamma}}^2}
        \label{eq:torsion-formula}
    \end{equation}
\end{nthm}
\begin{proof}
    Since, $\gamma' = v \vT$ and $\gamma'' = v' \vT + \kappa v^2 \vN$ we have that $\gamma' \times \gamma'' = \kappa v^3 \vB$. Moreover,
    \begin{align}
        \gamma'''
        &= v'' \vT + v' \vT' + \kappa' v^2 \vN + 2 \kappa v v' \vN + k v^2 \vN'\\
        &= v'' \vT + v' \paren*{\kappa v \vN} + \kappa' v^2 \vN + 2 \kappa v v' \vN + k v^2 \paren*{-\kappa v \vT + \tau v \vB}.
    \end{align}
    Now, since $\vB \cdot \vT = 0$, $\vB \cdot \vN = 0$ and $\vB \cdot \vB = 1$, we get
    \begin{equation}
        \paren*{\dot{\gamma} \times \ddot{\gamma}} \cdot \dddot{\gamma}
        = \tau k v^3 \vB \cdot \kappa v^3 \vB
        = \paren{\kappa v^3}^2 \tau
    \end{equation}

    Noting that $\kappa v^3 = \abs*{\dot{\gamma} \times \ddot{\gamma}}$ we obtain,
    \begin{equation}
        \frac{\paren*{\dot{\gamma} \times \ddot{\gamma}} \cdot \dddot{\gamma}}{\abs*{\dot{\gamma} \times \ddot{\gamma}}^2}
        = \frac{\paren{\kappa v^3}^2 \tau}{\paren{\kappa v^3}^2}
        = \tau.\qedhere
    \end{equation}
\end{proof}

\begin{negg}
    Use formula~\eqref{eq:torsion-formula} to calculate the torsion of the curve
    \begin{equation}
        \alpha(t)
        = \paren*{ t, 1 + \frac{1}{t}, \frac{1 - t^2}{t} }.
    \end{equation}

    Firstly,
    \begin{equation}
        \alpha'(t) = \paren*{0, -\frac{1}{t^2}, -\frac{t^2 + 1}{t^2}}, \quad
        \alpha''(t) = \paren*{0, \frac{2}{t^3}, \frac{2}{t^3}}, \quad
        \alpha'''(t) = \paren*{0, -\frac{6}{t^4}, -\frac{6}{t^4}}.
    \end{equation}
    So, $\alpha' \times \alpha'' = \paren*{\frac{2}{t^3}, 0, 0}$ and $\paren*{\alpha' \times \alpha''} \cdot \alpha''' = 0$. Therefore, $\tau(s) = 0$ for all $s$. (This shows that $\alpha(s)$ is a planar curve.)
\end{negg}


\section{Frenet-Serret Equations}

\begin{ndfn}
    Let $\gamma = \gamma(s)$ be a unit-speed curve with non-vanishing curvature, $\kappa(s) \neq 0$. The equations
    \begin{align}
        \dot{\vT}(s) &= \kappa(s) \vN(s),\\
        \dot{\vN}(s) &= -\kappa(s) \vT(s) + \tau(s) \vB(s),\\
        \dot{\vB}(s) &= -\tau(s) \vN(s)
    \end{align}
    are called the \keyword{Frenet-Serret equations} for $\gamma$.
\end{ndfn}

The Frenet-Serret equations are a system of ODEs. Results on initial-value problems for ODEs that given smooth $\kappa(s)$ and $\tau(s)$ and some initial-conditions, we can always find a unique solution for $\vT$, $\vN$ and $\vB$ for some range of the parameter $s$. In other words, suitable $\kappa(s)$ and $\tau(s)$ determine a space curve uniquely upto Euclidean isometries (translations, rotations, and reflections).

\begin{negg}
    Find a space curve $\alpha(s)$ whose curvature function is $\kappa(s) = 1/s$ and torsion is $\tau(s) = 0$.

    Using the Frenet-Serret equations, we get $\dot{\vT} = \kappa \vN$, $\dot{\vN} = - \kappa \vT$, and $\dot{\vB} = 0$. This leads to the second-order Cauchy-Euler equation
    \begin{equation}
        \dot{\vT} = \frac{1}{s} \vN
        \implies
        \ddot{\vT} = -\frac{1}{s^2} \vN + \frac{1}{s} \dot{\vN}
        \implies
        s^2 \ddot{\vT} + s \dot{\vT} + \vT = 0,
    \end{equation}
    with the solution $\vT(s) = \vec{c_1} \cos\paren*{\ln s} + \vec{c_2} \sin\paren*{\ln s}$, where $\vec{c_1}$ and $\vec{c_2}$ are arbitrary constant vectors.

    Finally, since $\alpha'(s) = \vT(s)$, we integrate $\vT(s)$ to obtain
    \begin{equation}
        \alpha(s) = \vec{c_0}
        + \frac{s}{2} \paren[\bigg]{\sin\paren*{\ln s} + \cos\paren*{\ln s}} \vec{c_1}
        + \frac{s}{2} \paren[\bigg]{\sin\paren*{\ln s} - \cos\paren*{\ln s}} \vec{c_2}.
    \end{equation}
    where $\vec{c_0}$, $\vec{c_1}$ and $\vec{c_2}$ are arbitrary constant vectors.
\end{negg}

% \missing{
\begin{nex}
    Show that $\alpha''' = -\kappa^2 \vT + \kappa' \vN + \tau\kappa \vB$.
\end{nex}
% }

\begin{nex}
    For a unit speed space curve $\alpha = \alpha(s)$, show that
    \begin{equation}
        \alpha'' \cdot \alpha''' \times \alpha^{(4)} = \kappa^5 \frac{d}{ds}\paren*{\frac{\tau}{\kappa}},
    \end{equation}
    and that if $\alpha$ is a helix then $\alpha'' \cdot \alpha''' \times \alpha^{(4)} = 0$. Is the coverse also true?
\end{nex}

% APPROXIMATING ALPHA(S) NEAR ALPHA(0) USING FRENET-SERRET

\begin{nthm}
    Let $\gamma = \gamma(s)$, $a \leq s \leq b$ be a unit speed curve with $\kappa > 0$. This is a plane curve if and only if $\tau = 0$ for all $s$.
\end{nthm}
\begin{proof}
    Suppose $\gamma(s)$ is a plane curve. This means there are constant vectors $\vec{p}, \vec{q}$ such that $\paren*{\gamma(s) - \vec{p}} \cdot \vec{q} = 0$ for all $s$. Differentiating this with respect to the parameter $s$ gives
    \begin{equation}
        \paren*{\gamma(s) - \vec{p}} \cdot \vec{q} = 0
        \implies
        \dot{\gamma}(s) \cdot \vec{q} = 0
        \implies
        \ddot{\gamma}(s) \cdot \vec{q} = 0.
    \end{equation}
    Thus, both $\dot{\gamma} \equiv \vT$ and $\ddot{\gamma} \equiv \vN$ are perpendicular to a constant vector $\vec{q}$. So,
    \begin{equation}
         \vec{q} \times \vB
        = \vec{q} \times \paren*{\vT \times \vN}
        = \paren*{\vec{q} \cdot \vN} \vT - \paren*{\vec{q} \cdot \vT} \vN
        = 0
    \end{equation}
    Therefore, the unit vector $\vB$ is always parallel to $\vec{q}$. So, $\vB$ is a constant vector and $\dot{\vB} = 0$. Thus, $\tau = \abs{\dot{\vB}} = 0$.

    Conversely, if $\tau = 0$, then the Frenet-Serret equations give $\dot{\vB} = 0$. That is $\vB$ is a constant unit vector. Let $\vec{A} = \gamma(c)$ for any $c \in [a, b]$ and consider the function
    \begin{equation}
        f(s) = \paren*{\gamma(s) - \vec{A}} \cdot \vB.
    \end{equation}
    Note that $f'(s) = \dot{\gamma} \cdot \vB = \vT \cdot \vB = 0$, so $f$ is a constant function. Also, $f(c) = 0$. Thus, $f(s) = 0$ for all $s$. So, $\paren*{\gamma(s) - \vec{A}} \cdot \vB \equiv 0$. This means $\gamma(s)$ lies in a plane perpendicular to $\vB$ and passing through the point $\vec{A}$. Therefore, $\gamma$ is a plane curve.
\end{proof}

\begin{nthm}
    If $\alpha = \alpha(s)$ is a unit speed curve that lies on a sphere of radius $a$ and has $\tau \neq 0$ for all $s$, then its curvature is $\kappa \geq 1/a$.

    (A curve that lies on a sphere is also called a spherical curve.)
\end{nthm}
\begin{proof}
    TBC. (Differentiate the defining equation and then use the Schwarz inequality.)
\end{proof}

\begin{nthm}
    Let $\alpha = \alpha(s)$ be a unit speed curve with $\kappa > 0$ and $\tau \neq 0$ for all $s$. If $\alpha$ is a spherical curve then
    \begin{equation}
        \frac{\tau}{\kappa} = \frac{d}{ds}\paren*{\frac{\kappa'}{\tau \kappa^2}}.
    \end{equation}
\end{nthm}
\begin{proof}
    TBC. (Differentiate the defining equation four times.)
\end{proof}

\begin{nthm}
    If $\alpha = \alpha(s)$ is a unit speed curve with constant positive curvature $\kappa$ and $\tau = 0$ for all $s$, then $\alpha$ is a part of the circle.
\end{nthm}
\begin{proof}
    TBC. (Construct the center as a curve, $\alpha + \vN/\kappa$, show that it is constant. Show that $\alpha$ lies a distance $1/\kappa$ from this centre.)
\end{proof}

\section{Global Properties of Curves}
\subsection{Jordan Curve Theorem}

\begin{ndfn}
    A simple closed planar curve is called a \keyword{Jordan curve}.
\end{ndfn}

\begin{nthm}
    Let $C$ be a Jordan curve in $\R^2$. Then its complement $\R^2 - C$ consists of exactly two connected components. One the of the components is bounded, denoted as $\interior{C}$, and the other is unbounded, denoted as $\exterior{C}$.
\end{nthm}

$C$ is the boundary curve of both $\interior{C}$ and $\exterior{C}$.

\begin{ndfn}
    A curve $\gamma$ is said to be positivvely oriented if the signed unit normal $\vN_s$ of $\gamma$ always points into $\interior{C}$.
\end{ndfn}

\subsection{Area Enclosed by a Jordan Curve}
The area of the region enclosed by a Jordan curve $\gamma$ in the $\R^2$ plane is
\begin{equation}
    A(\gamma) = \iint_{\interior{\gamma}} dxdy
\end{equation}

\begin{nthm}
    Let $\gamma(t) = (x(t), y(t))$ be a Jordan curve in the $xy$-plane, with $a \leq t \leq b$. The area enclosed by this curve is
    \begin{equation}
        A(\gamma) = \frac{1}{2} \int_a^b \paren*{x\dot{y} - y\dot{x}} \,dt
    \end{equation}
\end{nthm}
\begin{proof}
    Let $f(x,y) = -y/2$ and $g(x,y) = x/2$. Then,
    \begin{align}
        \frac{1}{2} \int_a^b \paren*{x\dot{y} - y\dot{x}} \,dt
        &= \frac{1}{2} \int_a^b  -y \,dx + x \,dy\\
        &= \int_a^b  f(x,y) \,dx + g(x,y) \,dy\\
        &= \iint_{\interior{\gamma}} \paren*{ \frac{\pd g}{\pd x} - \frac{\pd f}{\pd y} } \,dxdy\\
        &= \iint_{\interior{\gamma}} dxdy
    \end{align}
    Here we used the Green's theorem to convert the line integral into an area integral. Overall,
    \begin{equation}
        \frac{1}{2} \int_a^b \paren*{x\dot{y} - y\dot{x}} \,dt = \iint_{\interior{\gamma}} dxdy = A(\gamma)
        \qedhere 
    \end{equation}
\end{proof}

\subsection{Isoperimetric Inequality}

\begin{nthm}[Wirtinger's Inequality]
    Let $F : [0, \pi] \to \R$ be a smooth function with $F(0) = F(\pi) = 0$. Then,
    \begin{equation}
        \int_{0}^{\pi} \paren*{\frac{dF}{dt}}^2 \,dt \geq \int_{0}^{\pi} F^2 \,dt
    \end{equation}
    and the equality holds if and only if $F(t) = K \sin t$ for all $t \in [0, \pi]$ and $K$ is a constant.
\end{nthm}
\begin{proof}
    Let $G(t) = F(t) / \sin(t)$. This is well-defined because $\sin(t)$ is non-zero for $t \in (0, \pi)$, and smooth because it is a quotient of two smooth functions. Writing $F(t) = G(t) \sin(t)$ we get,
    \begin{align}
        \int_{0}^{\pi} \paren*{\frac{dF}{dt}}^2 \,dt
        &= \int_{0}^{\pi} \paren*{ G' \sin t + G \cos t }^2 \,dt\\
        &= \int_{0}^{\pi} \cparen*{G'^2 \sin^2 t + G^2 \cos^2 t + 2 G G' \sin t \cos t} \,dt.
    \end{align}
    Writing $2 G G' = (G^2)'$, $\sin t \cos t = \frac{1}{2} \sin 2t$ and integrating the last term by parts gives
    \begin{align}
        \frac{1}{2} \int_{0}^{\pi} \cparen*{(G^2)' \sin 2t} \,dt
        &= \bigg. \frac{1}{2} \sin (2t) G(t)^2 \bigg|_{0}^{\pi} - \int_{0}^{\pi} G^2(t) \cos 2t \,dt\\
        &= \int_{0}^{\pi} G^2(t) \paren*{\cos 2t - \sin 2t} \,dt.
    \end{align}
    Consequently,
    \begin{align}
        \int_{0}^{\pi} \paren*{\frac{dF}{dt}}^2 \,dt
        &= \int_{0}^{\pi} \cparen*{G'^2 \sin^2 t + G^2 \cos^2 t - G^2 \paren*{\cos^2 t - \sin^2 t}} \,dt\\
        &= \int_{0}^{\pi} \cparen*{G'^2 \sin^2 t + G^2 \sin^2 t} \,dt\\
        &= \int_{0}^{\pi} \paren*{G' \sin t}^2 \,dt + \int_{0}^{\pi} F^2 \,dt.
    \end{align}

    Therefore,
    \begin{equation}
        \int_{0}^{\pi} \paren*{\frac{dF}{dt}}^2 \,dt
        - \int_{0}^{\pi} F^2 \,dt
        = \int_{0}^{\pi} \paren*{G' \sin t}^2 \,dt \geq 0
    \end{equation}
    since $\paren*{G' \sin t}^2 \geq 0$. We find that the equality holds if and only if $G' \equiv 0$ (because $\sin^2 t > 0$ for $t \in (0, \pi)$). That is, the equality holds if and only if $G(t) = K$ for some constant $K$ and $F(t) = K \sin t$.
\end{proof}

\begin{nthm}[Isoperimetric inequality]
    Let $\gamma$ be a simple closed curve. Let $\ell(\gamma)$ be its length and $A(\gamma)$ be the area enclosed by $\gamma$. Then,
    \begin{equation}
        A(\gamma) \leq \frac{1}{4\pi} \ell(\gamma)^2
    \end{equation}
    and the equality holds if and only if $\gamma$ is a circle.
\end{nthm}
\begin{proof}
    TBC
\end{proof}

\section{Surfaces}
A \keyword{surface} in $\R^3$ is a two dimensional smooth submanifold. In this course, we will take a more relaxed approach and not use the concept of a manifold.

We say that the set of points $S = \set{(x,y,z) \in \R^3 \st f(x,y,z) = 0}$ defines a surface in $\R^3$ if $f(x,y,z) = 0$ is a smooth function with $\grad f \neq 0$ for all $(x,y,z) \in S$.

We will primarily deal with parametrised surfaces. Let $x = x(u,v)$, $y = y(u,v)$ and $z = z(u,v)$ with a \keyword{Jacobian}
\begin{equation}
    J = \begin{vmatrix} x_u & y_u & z_u\\ x_v & y_v & z_v \end{vmatrix}
\end{equation}
of rank $2$ everywhere. The (parametrised) surface $S$ is the set $\set{(x,y,z) \in \R^3 \st (u,v) \in D} \subseteq \R^3$ where the parameters $u$ and $v$ take values in some suitable domain. Here $x = x(u,v)$, $y = y(u,v)$ and $z = z(u,v)$ is called a parametrisation of the surface.

A point where the rank of the Jacobian, $J$ is less than $2$ is called a \keyword{singular point}.

\end{document}